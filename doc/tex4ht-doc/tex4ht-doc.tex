\documentclass{book}
\usepackage{url}
\usepackage{xcolor}
\usepackage{tex4ht-styles}

\usepackage{glossaries}
\title{TeX4ht Documentation}

\author{by TeX4ht Project}

\begin{document}

\maketitle

\tableofcontents

\chapter{Introduction}

\chapter{Basic tutorial}
\chapter{Basic usage}
\section{Calling commands}
\section{Output Formats}
\section{Command Line Options}


\chapter{Configurations}
\section{tex4ht commands}
\subsection{Low-level \texfourht\ Commands}

\texcommand{\HCode}

This command allows only for the expansion of macros, before sending its content to the output. The instruction \texcommand{\Hnewline} may be introduced there for requesting line breaks, and the command \texcommand{\#} may be used for the sharp symbol ‘\#’.

\begin{texsource}
 Two lines of text      \HCode{<br />}
 separated by a horizontal line.

 You probably don't want a '<br>'.
\end{texsource}

\texcommand{\Tg}

\subsection{Hyperlinks}
\texcommand{\Link[target-file arguments]{target-loc}{cur-loc}anchor\EndLink}

This command requests an anchor that links to \verb|‘target-file#target-loc’|, and marks the current location with the name \texttt{‘cur-loc’}.

The component \texttt{‘[...]’} is optional when it is empty, and the target file need not be mentioned if it is created from the current source file.


\texcommand{\LinkCommand}


\subsection{Paragraph Handling}

% https://tex.stackexchange.com/a/66172/2891

\texcommand{\Configure{HtmlPar}}
\texcommand{\IgnorePar}
\texcommand{\EndP}
etc.

\subsection{Logical Document Structure Commands}
I've created some alternative commands to \texcommand{\HCode} or \texcommand{\Tg}. The idea is to define
semantic names for logical elements of the document, such as titles, authors,
sections etc. It is possible to assign HTML elements and attributes to these
logical elements. There are commands for inline and block level elements,
which should eliminate the need for constructs like \texcommand{\ifvmode\IgnorePar\fi\EndP}
etc.

I think it will be best to show some concrete examples:


\begin{texsource}
\NewLogicalBlock{textit}
\SetBlockProperty{textit}{class}{textit}

\NewLogicalBlock{maketitle}
\SetTag{maketitle}{header}

\NewLogicalBlock{titlehead}
\SetTag{titlehead}{h1}
\SetBlockProperty{titlehead}{class}{titleHead}

\Configure{textit}
{\NoFonts\InlineElementStart{textit}{}}
{\InlineElementEnd{textit}\EndNoFonts}

\Configure{maketitle}{%
{\Configure{maketitle}{}{}{}{}%
\Tag{TITLE+}{\@title}}
\BlockElementStart{maketitle}{}}
{\BlockElementEnd{maketitle}}
{\NoFonts\BlockElementStart{titlehead}{}}
{\BlockElementEnd{titlehead}\EndNoFonts}
\end{texsource}



The \texcommand{\NewLogicalBlock} creates a new logical element. The used tag is configured
using \texcommand{\SetTag}, the attributes are set using \texcommand{\SetBlockProperty}. The blocks are
inserted to the document using 

\begin{texsource}
\InlineElementStart ... \InlineElementEnd
\end{texsource}

\noindent or

\begin{texsource}
\BlockElementStart ... \BlockElementEnd 
\end{texsource}

\noindent pairs. The start commands take two
parameters, first is the logical block name, the second can be local
parameters which shouldn't be used for the given logical block globally.

The main idea behind this mechanism is to allow easy work with new HTML5
elements and attributes for WAI-ARIA or Schema.org properties. I hope that
this should allow us to make somehow more clear configurations for basic
document building blocks.

\section{Configuration files}
\section{Styling the Document}

\section{Use Webfonts}
\section{Use JavaScript}
\section{Document Navigation}
\section{Tables of Contents}

\section{Sections}
\section{Lists}
\section{Tables}

\section{Fonts}
\subsection{Basic font commands}

Information about the \option{fonts} option and \term{MathML} issues. Example configuration:
\url{https://tex.stackexchange.com/a/416613/2891}
\section{Colors}

\section{Graphics}
\section{TikZ }

Animations using Animate package: \url{https://tex.stackexchange.com/a/404600/2891}
\section{Pstricks}

\section{Math}
\section{MathML}
\section{MathJax}

\section{Bibliographies}
\section{Indexing}

\chapter{Make4ht Build Files}
\section{Calling commands}
\section{Filters}

Some samples:

\begin{itemize}
  \item Render math by Mathjax during tex4ht compilation \url{https://tex.stackexchange.com/a/402159/2891}
\end{itemize}
\section{Image conversion}

\chapter{For developers}

\section{Writing basic support for a new package}
\begin{itemize}
  \item \url{https://tex.stackexchange.com/a/402283/2891}
\end{itemize}

\section{Two types of .4ht files}

\subsection{Inserting configurable hooks for packages}

\subsection{Configure the hooks in output format configuration files}

\section{How to add support for a package to the \texfourht\ literate sources}

To add a proper support for a new package, it is necessary to edit the 
\texfourht\ literate sources. The configurable hooks need to be placed in the \file{tex4ht-4ht.tex},
the configuration of these hooks must be added to the output format configuration files.
The most common output format is \HTML, which can be configured in \file{tex4ht-html4.tex}, or 
\file{tex4ht-html5.tex} if \HTMLV\ features are used. It is also necessary to update the
\file{mktex4ht-cnf.tex}.

\subsection{Example}

Given following package \file{sample.sty}:

\begin{texsource}
\ProvidesPackage{sample}
\newcommand\hello{world}
\endinput
\end{texsource}

This simple package defines command \texcommand{\hello}, which simply prints the word \textit{hello} when used in a document.

Let's say that we want to insert some \HTML\ tags before and after the text content printed by the command.

Basic template for \file{tex4ht-4ht.tex}

% examples/basicpackage/sample.4ht
\begin{texsource}
\<sample.4ht\><<<
% sample.4ht (|version), generated from |jobname.tex
% Copyright 2017 TeX Users Group
|<TeX4ht license text|>
\NewConfigure{hello}{2}
\pend:def\hello{\a:hello}
\append:def\hello{\b:hello}
\Hinput{sample}
>>> \AddFile{9}{sample}
\end{texsource}

Configuration for each package must follow this basic template. The \ProTeX\ system is used as system for literate programming.

The \texcommand{\<name\><<<code>>>} block defines new macro which can be then called using \texcommand{|<name|>}. The license text
is included in the example this way. In order to generate the \file{sample.4ht} file, we need to use \texttt{sample.4ht} as a name
in the code block and command \texcommand{\AddFile{9}{sample}} after the block definition\footnote{I have no idea what the number
in the first parameter means.}.

Each package configuration  must include \texcommand{\Hinput{packagename}}, in order to load the configurations for the package.

The command \texcommand{\NewConfigure{hello}{2}} declares new configuration \texttt{hello}, with two configurable hooks. 
These hooks are named  \texcommand{\a:hello} and \texcommand{\b:hello}. The hooks must be inserted into the 
\texcommand{\hello}, which can be easily done using the \texcommand{\pend:def} and \texcommand{\append:def} commands. These
commands can insert code  at the beginning, respective at the end of the redefined command.

The configuration for \HTML\ must be placed in the \file{tex4ht-html4.tex} file:


% examples/basicpackage/config.cfg
\begin{texsource}
\<configure html4 sample\><<<
\Configure{hello}{\HCode{<span class="hello">}}{\HCode{</span>}}
\Css{.hello{color:red;}}
>>>
\end{texsource}

The configuration for a package must be placed in \texcommand{\<configure html4 packagename\>} block.
% ToDo: write more info


The package name must be also included in \file{mktex4ht-cnf.tex}:

\begin{texsource}
\AddFile{9}{sample}
\end{texsource}

The \file{.4ht} files can be generated simply using 

\begin{shellcommand}
make
\end{shellcommand}

command.

The following sample \TeX\ file:


% examples/basicpackage/hello.tex
\begin{texsource}
\documentclass{article}
\usepackage{sample}
\begin{document}
  \hello\ world.
\end{document}
\end{texsource}

Produces a following \HTML\ code:

\begin{htmlsource}
<!--l. 4--><p class="noindent" >
<span class="hello">world</span> world. 
</p> 
\end{htmlsource}
\chapter{Glossary}
\chapter{Bibliography}
\chapter{Index}

\end{document}
