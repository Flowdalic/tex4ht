\documentclass{book}
\usepackage{url}
\usepackage{xcolor}
\usepackage{array}
\usepackage{graphicx}

% \usepackage[T1]{fontenc}
\usepackage{upquote}
\usepackage{textcomp}
\usepackage{hyperref}
\usepackage{csquotes}
\usepackage{tex4ht-styles}



\usepackage{glossaries}
\title{TeX4ht Documentation}

\author{by TeX4ht Project}

\begin{document}

\maketitle

% Don't introduce table of contents in the HTML mode, as it introduces another page
\ifdefined\HCode\else\tableofcontents\fi


\chapter{Introduction}


\begin{acknowledgements}
This work was supported with a financial support from \href{https://cstug.cz/}{CSTUG}.
\end{acknowledgements}

\chapter{Basic Tutorial}
\chapter{How to}

\section{Change design}
\subsection{Basics}

By default, \texfourht\ separates paragraphs by spaces. If you want to use text indenting instead, try the \option{p-indent} option.

\subsection{CSS}
\subsection{Web fonts}

\section{Math}

\subsection{MathJax}


\subsection{MathML}
\subsection{Subscripts and superscripts}
\input{howto/subscripts}
\subsection{MathJax Node}


\section{Graphics}
\subsection{Include graphics (svg,pdf)}
\subsection{Change image size and resolution}

\section{Blogging}

\section{Work with external commands}
\subsection{Indexing}
\subsection{Bibliographies}
\subsection{R}
\subsection{Markdown}
\subsection{PythonTeX}


\chapter{Calling Commands}
\input{sections/calling-commands}
\label{sec:calling-commands}
\chapter{Output Formats}
\chapter{\texfourht\ Options}

% $Id$
% Copyright 2009-2013 CV Radhakrishnan.
% Released under LPPLv1.3+.
% 
% TeX4ht options.

\ifx\texhtstandalonedoc1
\documentclass[a4paper]{article}
\usepackage{xspace,graphicx,blog}
\makeatletter
 \@ifundefined{Css}{\let\Css\@gobble}{}
 \@ifundefined{svg}{\def\svg#1{\textsc{svg}\xspace}}{}
\makeatother
\begin{document}
\Css{dt.description{font-family: monospace; color: olive;}}
\section{Various Options}
\else
\chapter{Various Options for \TeX4ht}
\fi

\catcode`\#=11 \catcode`\^=11 \catcode`\_=11

\begin{description}

\item[-css] to ignore \css code, use command line option \verb=-css=.

\item[-xtpipes] to avoid \verb=xtpipes= post-processing the
  output. This might be useful for docbook \xml output.

  % \item[/bib]
  % \item[/obeylines]
  % \item[0.0]

\item[0] pagination shall be obtained through the option \verb=0= or
  \verb=1=, at locations marked with \verb=\PageBreak=.

\item[1, 2, 3, 4, 5, 6, 7] for automatic sectioning pagination (to
  break at various section levels), use the appropriate command line
  option \verb=1, 2, 3, 4, 5, 6, 7=.

\item[DOCTYPE] to request a \verb=DOCTYPE= declaration, use the
  command line option \verb=DOCTYPE=.

\item[Gin-dim] for key dimensions of the graphic, try this option.

\item[Gin-dim+] for key dimensions when the bounding box is not
  available.

\item[NoFonts] to ignore \css font decoration.

\item[PMath] Option to choose positioned math. Example: Example:
  \Verb=\def\({\PMath$}=; \Verb=\def\){$\EndPMath}=;
  \Verb=\def\[{\PMath$$}=; \Verb=\def\]{$$\EndPMath}=.

\item[RL2LR] to reverse the direction of RL sentences.

%\item[ShowFont]

\item[TocLink] option to request links from the tables of contents.
  
\item[\tt ^13] option for active superscript character.

\item[\tt _13] option for active subscript character.

%\item[accent-]

%\item[base]

\item[bib-] for degraded bibliography friendlier for conversion to
  \verb=.doc=.

\item[bibtex2] Option \verb=bibtex2= requires compilation of
  \verb=\jobname j.aux= with bibtex.

%\item[broken-index]

\item[charset] for alternate character set, use the command line
  option \verb+charset="..."+ (e.g., \verb+charset="utf8"+).

%\item[core]

\item[css-in] the inline \css code will be extracted from the input of
  the previous compilation, so an extra compilation might be needed for
  this option to make it effective.

\item[css2] for \css2 code.

% \item[css]
% \item[debug-]
% \item[debug]
% \item[draw]
% \item[dtd]

\item[\tt early^] for default catcode of superscript in the
  \verb=\Preamble=.

\item[\tt early_] for default catcode of subscript in the
  \verb=\Preamble=.

%\item[edit]

\item[endnotes] for end notes instead of footnotes, use this option.

%\item[enum]

\item[enumerate+] for enumerated list elements with valued data. This
  will use the description list like \verb=<dt>...</dt>= for the list
  counter.

\item[enumerate-] for enumerated list element's \verb=<li>='s with
  value attributes, use this command line option. This will be an
  ordered list with the value of list counter provided as an attribute
  namely, \verb=value= of the \verb=<li>= element.

%\item[family]

\item[fn-in] for inline footnotes use this option.

\item[fn-out] for offline footnotes.

\item[fonts] for tracing \latex font commands, use this command line
  option.

\item[fonts+] for marking of the base font, use this option.

\item[font] for adjusted font size, use the command line option
  \verb+font=...+ (e.g., font=-2).

\item[frames-] for frames support. \verb=frames= is also valid option
  for frames support.

\item[frames-fn] for content, \chfont{TOC}\xspace and footnotes in
  three frames.

\item[frames] for \chfont{TOC}\space and content in two frames.

%\item[fussy]

\item[gif] for bitmaps of pictures in \verb=.gif= format, use this
  option.

\item[graphics-] if the included graphics are of degraded quality, try
  the command line options \verb=graphics-num= or \verb=graphics-=.
  The \verb=num= should provide the density of pixels in the bitmaps
  (e.g., 110).

%\item[graphics-dim]

\item[hidden-ref] option to hide clickable index and bibliography
  references.

% \item[hooks++]
% \item[hooks+]
% \item[hooks]
% \item[hshow]
% \item[htm3]
% \item[htm4]
% \item[htm5]
% \item[htm]

\item[html+] for stricter \html code.

%\item[html]

\item[imgdir] for addressing images in a subdirectory, use the option
  \verb=\imgdir:.../=.

\item[image-maps] for \verb=image-maps= support.

\item[index] for \emph{n}-column index, use the command line option,
  \verb+index=n+ (e.g., index=2).

\item[info-oo] for extra tracing information while generating open
  office output.

\item[info] for extra information in the \verb=\jobname.log= file.

\item[java] for \verb=java=support.

\item[javahelp] for \verb=JavaHelp= output format, use this command
  line option.

\item[javascript] for \verb=javascript= support.

\item[jh-] for sources failing to produce \xml versions of \html, try
  this command line option.

%\item[jh1.0]

\item[jpg] for bitmaps of pictures in \verb=.jpg= format, use this
  option.

\item[li-] for enumerated list elements li's with value attributes.

\item[math-] option to use when sources fail to produce clean math
  code.

%\item[mathaccent-]

\item[mathltx-] option to use when sources fail to produce clean
  \verb=mathltx= code.

\item[mathml-] option to use when sources fail to produce clean
  \mathml code.

\item[mathplayer] for \mathml on Internet Explorer + MathPlayer.

\item[minitoc<] for mini tocs immediately after the header use the
  command line option, \verb=minitoc<=.

\item[mouseover] for pop ups on mouse over.

%\item[new-accents]

\item[next] for linear cross-links of pages, use this option.

\item[nikud] for Hebrew vowels, use the command line option,
  \verb=nikud=.

\item[no-DOCTYPE] to remove \chfont{DOCTYPE}\space declaration from
  the output.

\item[no-VERSION] to remove \verb+<?xml version="..."?>+ processing
  instruction from the output.

% \item[no-align]
% \item[no-array]
% \item[no-bib]
% \item[no-cases]
% \item[no-halign]
% \item[no-matrix]
% \item[no-pmatrix]

\item[\tt no^] for non-active \verb=^= (superscript), use the option
  \verb=no^=.

\item[\tt no_] for non-active \verb=_= (subscript command), use the
  command line option, \verb=no_=.

\item[\tt no_^] for both non-active superscript and subscript, use the
  option \verb=no_^=.

\item[nolayers] to remove overlays of slides, use this option.

\item[nominitoc] this will eliminate mini tables of contents from the
  output.

\item[notoc*] for tocs without \verb=*= entries, use this option. The
  \verb=notoc*= option is applicable only to pages that are
  automatically decomposed into separate web pages along section
  divides. It shall be used when \verb=\addcontentsline= instructions
  are present in the sources.

\item[obj-toc] for frames-like object based table of contents, use the
  command line option \verb=obj-toc=.

%\item[old-longtable]

\item[p-width] for width specifications of tabular \verb=p= entries,
  use this option.

\item[pic-RL] for pictorial RL.

\item[pic-align] for pictorial align environment.

\item[pic-array] for pictorial array.

\item[pic-cases] for pictorial cases environment.

\item[pic-eqalign] for pictorial eqalign environment.

\item[pic-eqnarray] for pictorial eqnarray.

\item[pic-equation] for pictorial equations.

\item[pic-fbox] for pictorial or bitmapped fbox'es.

\item[pic-framebox] for bitmap frameboxes.

\item[pic-longtable] for bitmapped longtable.

\item[pic-m+] for pictorial \verb=$...$= and \verb=$$...$$=
  environments with \latex alt, use the command line option
  \verb=pic-m+= (not safe).

\item[pic-m] for pictorial \verb=$...$= environments, use the command
  line option \verb=pic-m= (not recommended).

\item[pic-matrix] for pictorial matrix.

% \item[pic-tabbing']

% \item[pic-tabbing]

% \item[pic-table]

\item[pic-tabular] use this option for pictorial tabular.

\item[plain-] for scaled down implementation.

% \item[pmathml-css]

% \item[pmathml]

% \item[postscript]

\item[prog-ref] for pointers to code files from root fragments, use
  the command line option \verb=prof-ref=. This is for debugging.

\item[refcaption] for links into captions, instead of flat heads, use
  this option.

\item[rl2lr] to reverse the direction of Hebrew words, use this
  option.

\item[sec-filename] for file names derived from section titles, use
  the command line option \verb=sec-filename=.

\item[sections+] for back links to table of contents, use this option.

% \item[sections-]
% \item[settabs-]
% \item[stackrel-]

\item[svg-] for external \svg files, try this option.

\item[svg-obj] same as above.

\item[svg] for dvi pictures in \verb=svg= format.

\item[tab-eq] for tab-based layout of equation environment, use this
  option.

%\item[th4]

\item[trace-onmo] for mouseover tracing of compilation, use the
  command line option, \verb=trace-onmo=.

% \item[uni-emacspeak]
% \item[uni-html4]
% \item[uniaccents]
% \item[unicode]

% \item[url-]

\item[url-enc] for \chfont{URL}\space encoding within href, use this
  option.  \verb=\Configure{url-encoder}= can be used to fine tune
  encoding.

\item[url-il2-pl] for il2-pl \chfont{URL} encoding.

\item[ver] for vertically stacked frames. Effective when \verb=frames=
  option is requested.

% \item[verify+]
% \item[verify]

\item[xht] for file name extension, \verb=.xht=, use this command line
  option.

\item[xhtml] for \xml code, use the command line option, \verb=xml= or
  \verb=xhtml=.

\item[xml] See previous entry.

% \item[xmldtd]

\end{description}

\ifx\texhtstandalonedoc1 \end{document}\fi


\chapter{Configurations}
\input{sections/configuration-files}
\section{tex4ht Commands}
\subsection{Low-level \texfourht\ Commands}

\texcommand{\HCode}

This command allows only for the expansion of macros, before sending its content to the output. The instruction \texcommand{\Hnewline} may be introduced there for requesting line breaks, and the command \texcommand{\#} may be used for the sharp symbol ‘\#’.

\begin{texsource}
 Two lines of text      \HCode{<br />}
 separated by a horizontal line.

 You probably don't want a '<br>'.
\end{texsource}

\texcommand{\Tg}

\subsection{Hyperlinks}
\texcommand{\Link[target-file arguments]{target-loc}{cur-loc}anchor\EndLink}

This command requests an anchor that links to \verb|‘target-file#target-loc’|, and marks the current location with the name \texttt{‘cur-loc’}.

The component \texttt{‘[...]’} is optional when it is empty, and the target file need not be mentioned if it is created from the current source file.


\texcommand{\LinkCommand}


\subsection{Paragraph Handling}

% https://tex.stackexchange.com/a/66172/2891

\texcommand{\Configure{HtmlPar}}
\texcommand{\IgnorePar}
\texcommand{\EndP}
etc.

\subsection{Logical Document Structure Commands}
I've created some alternative commands to \texcommand{\HCode} or \texcommand{\Tg}. The idea is to define
semantic names for logical elements of the document, such as titles, authors,
sections etc. It is possible to assign HTML elements and attributes to these
logical elements. There are commands for inline and block level elements,
which should eliminate the need for constructs like \texcommand{\ifvmode\IgnorePar\fi\EndP}
etc.

I think it will be best to show some concrete examples:


\begin{texsource}
\NewLogicalBlock{textit}
\SetBlockProperty{textit}{class}{textit}

\NewLogicalBlock{maketitle}
\SetTag{maketitle}{header}

\NewLogicalBlock{titlehead}
\SetTag{titlehead}{h1}
\SetBlockProperty{titlehead}{class}{titleHead}

\Configure{textit}
{\NoFonts\InlineElementStart{textit}{}}
{\InlineElementEnd{textit}\EndNoFonts}

\Configure{maketitle}{%
{\Configure{maketitle}{}{}{}{}%
\Tag{TITLE+}{\@title}}
\BlockElementStart{maketitle}{}}
{\BlockElementEnd{maketitle}}
{\NoFonts\BlockElementStart{titlehead}{}}
{\BlockElementEnd{titlehead}\EndNoFonts}
\end{texsource}



The \texcommand{\NewLogicalBlock} creates a new logical element. The used tag is configured
using \texcommand{\SetTag}, the attributes are set using \texcommand{\SetBlockProperty}. The blocks are
inserted to the document using 

\begin{texsource}
\InlineElementStart ... \InlineElementEnd
\end{texsource}

\noindent or

\begin{texsource}
\BlockElementStart ... \BlockElementEnd 
\end{texsource}

\noindent pairs. The start commands take two
parameters, first is the logical block name, the second can be local
parameters which shouldn't be used for the given logical block globally.

The main idea behind this mechanism is to allow easy work with new HTML5
elements and attributes for WAI-ARIA or Schema.org properties. I hope that
this should allow us to make somehow more clear configurations for basic
document building blocks.

\section{Styling the Document}

\texfourht\ provides several commands that can be used for changing of the
document appearance using Cascading Style Sheets (\css). Only basic styling for
the document is provided by default. Additional styles are added by configurations for the
fonts, packages and commands used in the document. Full control of the document
styling can be achieved using following commands and configurations.


\texcommand{\Css{content}}

This command sends its content to the CSS file of the document. 

\texcommand{\CssFile[list-of-css-files]content\EndCssFile}

The CSS file \texfourht\ used by default initially consists just
a single line,  \texcommand{/* css.sty */}. This line is later
replaced with the code submitted by the \texcommand{\Css{...}} commands.

The \texcommand{\CssFile} command allows to specify an alternative to the initial CSS file.
The alternative file consists of the code loaded from listed files, and of the
content explicitly specified in its body.

\begin{texsource}
\ConfigureList{mylist} 
{\HCode{<div class="mylist">}} {\HCode{</div>}} {* }{} 
       
\begin{document} 
       
\CssFile 
/* css.sty */ 
.mylist { color : red; } 
\EndCssFile 
\end{texsource}

The names in the list of files should be separated by commas, and the rectangular brackets are optional when the list is empty.

The file should include a line having the content of \verb|/* css.sty */|. If
more than one such line is included, the content of the \texcommand{\Css{...}} commands
replace the first occurrence of this line. Arbitrary many space characters may
appear around the substrings ‘/*’ and ‘*/’. 

\DocConfigure{AddCss} {CSS file name}\EndDoc

Require external CSS file.

\section{Webfonts}
\input{sections/webfonts}
\section{Use JavaScript}
\section{Hyperlinks}
% https://tex.stackexchange.com/a/521497/2891
% https://tex.stackexchange.com/a/521905/2891
\section{Document Navigation}

\subsection{Cross-links}

The following configurations modify behaviour of cross-links between pages in a multi page document.

\DocConfigure{crosslinks} {left-delimiter} {right-delimiter} {next} {prev} {prev-tail} {front} {tail} {up}\EndDoc

This command configures the appearance of the cross-links between hypertext pages obtained for sectioning commands.

\begin{texsource}
 \Configure{crosslinks}
   {}{}{$\scriptstyle\Rightarrow$}
   {$\scriptstyle\Leftarrow$}
   {}{}{}{$\scriptstyle\Uparrow$}
\end{texsource}

\DocConfigure{crosslinks*} {1--7 arguments}\EndDoc

  Links to be included and their order. Available
  options: next, prev, prevtail, tail, front, up.
  The last argument must be empty.

  Default:

\begin{texsource}
\Configure{crosslinks*}{next}
   {prev}{prevtail}
   {tail}{front}
   {up}{}
\end{texsource}

\DocConfigure{crosslinks+} {before-top-links} {after-top-links} {before-bottom-links} {after-bottob-links}\EndDoc

The top cross links are omitted, if both \verb|#1| and \verb|#2| are empty.
The bottom cross links are omitted, if both \verb|#3| and \verb|#4| are empty.

\DocConfigure{next} {the anchor of the next button of the front page}\EndDoc

Default: The value provided in \texcommand{\Configure{crosslinks}}

\DocConfigure{next+}{before} {after}\EndDoc

\begin{description}
  \item[\#1]  before the next button of the front page, when the `next'
       option is active.
  \item[\#2]  after the button
\end{description}

    Default: The values provided in \texcommand{\Configure{crosslinks}}

\begin{texsource}
\Configure{crosslinks:next}..................1
\Configure{crosslinks:prev}..................1
\Configure{crosslinks:prevtail}..............1
\Configure{crosslinks:tail}..................1
\Configure{crosslinks:front}.................1
\Configure{crosslinks:up}....................1
\end{texsource}

  \verb|#1| local configurations for the delimiters and hooks

\DocConfigure{crosslinks-}{before} {after}\EndDoc

Asks to show linkless buttons with the following insertions.

The default values are used, if both \verb|#1| and \verb|#2| are empty

   Examples:

\begin{texsource}
\Configure{crosslinks-}{}{}

\Configure{crosslinks-}
    {\HCode{<span class="hidden">}[}
    {]\HCode{</span>} }
\Css{span.hidden {visibility:hidden;}}
\end{texsource}

\section{Tables of Contents}

\section{Sections}
\section{Lists}
\section{Tables}

\section{Fonts}
\subsection{Basic font commands}

Information about the \option{fonts} option and \term{MathML} issues. 
Example configuration:
\url{https://tex.stackexchange.com/a/416613/2891}

\section{Multi-lingual support}

RTL support in the ODT output: \url{https://tex.stackexchange.com/a/470434/2891}.

\subsection{Right-to-left text}

There is a difference in the RTL support for HTML and ODT output formats. In HTML, RTL can be requested using:

\DocConfigure{LRdir} { value for the dir attribute}\EndDoc

Example:

\begin{texsource}
\ConfigureEnv{arab}
{\Configure{LRdir}{ dir="rtl"}}
{\Configure{LRdir}{}}{}{}
\end{texsource}

This configuration sets the direction to \term{rtl} inside the \term{arab} environment and resets it after the environment end.

In the ODT output, different mechanism is used:

\begin{texsource}
\ConfigureEnv{arab}{\@rltrue}{\@rlfalse}{}{}
\end{texsource}

\subsection{Unicode}

Generally speaking, \texfourht\ supports \term{Unicode}, but there are some
issues to be aware of. Most complete support exists for Lua\LaTeX, thanks to
special Lua script which is automatically loaded during the compilation. No
additional packages are necessary.

PDF\LaTeX\ doesn't support nativelly, but it is possible to emulate it using the
\package{inputenc} and \package{fontenc} packages:

\begin{texsource}
\documentclass{article}
\usepackage[utf8]{inputenc}
\usepackage[T1]{fontenc}
\begin{document}
Unicode text
\end{document}
\end{texsource}

Xe\LaTeX\ is an Unicode format, similarly to Lua\LaTeX. The supporting
mechanism for \texfourht\ is different in this case and full Unicode range is
not supported out of the box. By default, only most Latin based characters are
supported. For other scripts, such as Greek or Cyrillic, two ways to enable
support exists. 

First option is to define new font family using \package{fontspec} \texcommand{\newfontfamily} with the \texttt{Script} option.

\begin{texsource}
\newfontfamily\greekfont{Linux Libertine O}[Script=Greek]
\end{texsource}


The second option is to declare load support for a script in the custom config
file using the \texcommand{\xeuniuseblock}:


\begin{texsource}
\xeuniuseblock{Greek}
\end{texsource}

The block names are based on \href{https://en.wikipedia.org/wiki/Unicode_block}{Unicode blocks}.

It is also possible to declare all characters in an Unicode range. The command
\texcommand{\xeuniregisterblockhex} takes two hexadecimal parameters with
Unicode range to be declared.

\begin{texsource}
\xeuniregisterblockhex{0100}{017F}
\end{texsource}

Individual character can be declared using the \texcommand{\xeuniregisterchar} command:

\begin{texsource}
\xeuniregisterchar{"1F00}
\end{texsource}

In contrast to \texcommand{\xeuniregisterblockhex}, it uses decimal numbers by
default, so it is necessary to use the \texttt{"} character in front of
a hexadecimal number.

\begin{warning}
It is possible to run into issues because of the way how Xe\LaTeX\ Unicode
support works. Common problem is filename support, for example in included
graphics. In general, it is better to avoid such filenames. If it is not possible, try to use the \texcommand{\detokenize} command.
\begin{texsource}
  \includegraphics{\detokenize{háček.jpg}}
\end{texsource}
\end{warning}

\section{Colors}

Information about the \texcommand{\color} command:
\url{https://tex.stackexchange.com/a/195677/2891}.
Example of possible configuration for the \texcommand{\color} command: 
\url{https://tex.stackexchange.com/q/470179/2891}.

Example of extracting color information to the CSS and custom color environment support:
\url{https://tex.stackexchange.com/a/422281/2891}. Extracting of color information to the HTML attributes:
\url{https://tex.stackexchange.com/a/390151/2891}.



\section{Graphics and Pictures}

\input{sections/graphics}
\section{TikZ }

Animations using Animate package: \url{https://tex.stackexchange.com/a/404600/2891}

Issues with drivers: \url{https://tex.stackexchange.com/a/471460/2891}.
\section{Pstricks}

\section{Math}
\subsection{Default math handling}
\subsection{MathML}

\mathml\ is a XML markup for the math encoding. It is supported in many
aplications including OpenOffice Writer or Firefox web browser. 
The advantage over use of images % Todo: write about advantages of MathML.

The \mathml\ code produced by \texfourht\ may contain some issues. For example,
one common issue may happen when the math contain unmatched delimiters:

\begin{texsource}
 Mail address: $\lparen$hello@world.com$\rparen$
\end{texsource}

In such cases, the \option{matml-} may help. 

It is also advisable to always use \extension{common\_domfilters}
\term{make4ht} extension (see section \ref{sec:make4ht-extensions} for more
information about \term{make4ht} extensions), as it fixes some common \mathml\
errors that cannot be easily fixed on the \TeX\ level.


Add information about the \url{https://github.com/pshihn/math-ml} - it adds
support for MathML in all modern web browsers with HTML 5.


\subsection{MathJax}
\texfourht\ supports MathJax, library for math rendering in HTML documents. 
 It supports two modes -- \LaTeX\ math and \mathml.

The \term{MathJax} processing can be required using the \option{mathjax} option.

The address of the \term{MathJax} script and its configuration string can be
specified in the \configuration{MathjaxSource} configuration. Default value of this configuration is:

\begin{texsource}
\Configure{MathjaxSource}
{https://cdn.jsdelivr.net/npm/mathjax@3/es5/tex-chtml-full.js}
\end{texsource}

\paragraph{\LaTeX\ mode}

In the \LaTeX math mode, \TeX\ macros used in the math mode are preserved in
the output HTML document. They are parsed and rendered by MathJax when the
document is displayed by a web browser. The downside of this mode is that
commands unknown to MathJax must be configured in a special configuration for
MathJax. Cross-references to equations and other numbered math environments
don't work out of the box.

By default, inline and display math, as well as math environments, are kept as
raw LaTeX code in the \HTML\ output. 

The additional configuration for \term{MathJax} can be provided in special
script environment in the \HTML\ page header. The following example provides
support for some custom \LaTeX\ macros.

\begin{texsource}
\Preamble{xhtml}
\Configure{@HEAD}{\HCode{
<script>
window.MathJax = {
  tex: {
    macros: {
      \unexpanded{
        sc: "\\small\\rm",
        sl: "\\it",
      }
    }
  }
};
</script>
}}
\begin{document}
\EndPreamble

\end{texsource}

The \texcommand{\detokenize} macro is used to avoid issues with backslash
characters used in the macro definitions. Backslashes must be doubled in the
JavaScript strings.


\paragraph{\mathml\ mode}

Math is converted to \mathml\ by \texfourht, MathJax then renders it. Custom
commands and cross-references work in this mode.

The \mathml\ MathJax mode can be required using the \option{mathml,mathjax} option.

\paragraph{Table of contents issues}

Some math commands may cause issues when they are used in section titles in the MathJax mode. 
This can be fixed using the \texcommand{\fixmathjaxtoc} command:

\begin{texsource}
\fixmathjaxtoc\int
\end{texsource}


\section{Bibliographies}
\section{Indexing}

\section{OpenDocument Format}
\input{sections/odt}

\chapter{Make4ht Build Files}
\section{Commands execution}
\section{Filters}

Some samples:

\begin{itemize}
  \item Render math by Mathjax during tex4ht compilation \url{https://tex.stackexchange.com/a/402159/2891}
\end{itemize}
\section{Image conversion}

\chapter{FAQ}
\input{sections/faq}
\chapter{For developers}
\input{sections/tex4ht-development}

\chapter{Glossary}
\chapter{Bibliography}
\chapter{Index}

\end{document}
