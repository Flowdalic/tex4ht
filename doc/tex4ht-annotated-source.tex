% $Id$
% Copyright 2011-2013 CV Radhakrishnan.
% Released under LPPLv1.3+.
% 
% TeX4ht annotated source example.

\ifx\texhtstandalonedoc1
\documentclass[a4paper]{article}
\documentclass[doc,sspro]{stm}
%\documentclass[html]{stm}
\usepackage{blog}
\begin{document}
\title{An Annotated Document Source}
\author{C.\,V.\,Radhakrishnan}
\footeraddress{%
  \hlink{http://www.cvr.cc?cat=14}{\textsm{WWW.CVR.CC}}, residing at some
  server  in the data\\ center of
  \hlink{http://hetzner.de}{\textsm{HETZNER.DE}} 
  somewhere in Germany.}
\contact{cvr@cvr.cc}
\maketitle
\advance\baselineskip by 1.5pt
\else
\section{Annotated source of lettrine example}
\fi

\noindent
Mahesh wanted to see the annotated source of the \tex document that
generated the output
\hlink{http://download.river-valley.com/tex4ht/lettrine/ltrn-test.html}
{shown here}. 

There are two components to the source:
\begin{enumerate}
\item document source, let us call it as \verb'test.tex'.
\item a custom configuration for \texht, let us call it \verb'my.cfg'.

\end{enumerate}

\subsection*{Document Source}


\begin{verbatim}
\documentclass[a4paper]{article}

\usepackage{lettrine}
\usepackage{lipsum,shortvrb}
\usepackage[utf8]{inputenc}
  
\MakeShortVerb{\|}
   
\def\hlink#1#2{\Tg<a href="#1">#2\Tg</a>}
   
\begin{document}
   
\title{Taming \TeX4ht to Support Lettrine Package}
   
\author{\hlink{http://www.cvr.cc}{\textsc{cv radhakrishnan}}, 
        \Tg<small>\hlink{http://www.river-valley.com}{River Valley},
        Trivandrum, \textsc{india}\Tg</small>} 
   
\maketitle
   
\lettrine{T}{his code} has kindly been contributed by
\hlink{http://www.raphink.info/}{Rapha\"el Pinson} and is
\hlink{https://github.com/cc-translators/sagesse/blob/master/lettrine.4ht}
{available here}. The original code was slightly modified to fit
into the scheme of \TeX4ht.  Rapha\"el had provided eight hooks to
reconfigure |\lettrine| which I'd reduced to seven. All the hooks are
provided below:
 
\begin{enumerate}
 \item Before |\lettrine|
 \item After  |\lettrine|
 \item Before lettrine initial character
 \item Between lettrine initial character and lettrine line
 \item After lettrine line
 \item Before lettrine ante
 \item After lettrine ante
\end{enumerate}
 
\lettrine{F}{urther} hooks have been defined to access initial
character and lettrine line in case we want to configure
|\lettrine|. |\HlettrineChar| and |\HlettrineString| can be used to
access the character and line respectively. The screenshot of this
pages as rendered in Firefox is \hlink{lettrine.png}{available here}.
 
Here is a bit of dummy text to fill in the page. \lipsum[5-9]
 
\end{document}
\end{verbatim}

There is nothing special to explain in the above part. \verb'\hlink' is
a custom macro with two arguments to cross-link something to a
\textsm{URL}.  The packages are popular ones and needs no introduction.  

\subsection*{Custom Configuration}

The custom configuration, \verb'my.cfg' has two parts --- a preamble
part and a post preamble part. Preamble area shall consist of three
important commands:
\begin{enumerate}
\item \verb'\Preamble' --- beginning of preamble area.
\item \verb'\begin{document}' --- do not confuse with the
    \verb'\begin{document}' in your document source.
\item \verb'\EndPreamble' --- end of preamble area.

\end{enumerate}
\verb'\Preamble' shall contain all the
options that are to be passed on to \texht. Options can be provided as
comma separated strings.  If any \tex packages are needed, those
should be loaded before \verb'\Preamble' command as shown below, where
we have loaded \verb'\xspace' package.

\begin{verbatim}
\RequirePackage{xspace}

\Preamble{xhtml,NoFonts,ext=html,charset="utf-8"}
\end{verbatim}
For more details about options, please see
\hlink{http://www.cvr.cc/tex4ht-options/}{this post}.

\begin{verbatim}
\Configure{PROLOG}{DOCTYPE}
\Configure{VERSION}{}
\Configure{DOCTYPE}{\HCode{<!DOCTYPE html>\Hnewline}}
\Configure{HTML}{\HCode{<html>\Hnewline}}{\HCode{\Hnewline</html>}}
\Configure{@HEAD}{}
\Configure{@HEAD}{\HCode{<meta
         \expandafter\csname a:charset\endcsname/>\Hnewline}}
\Configure{@HEAD}{\HCode{<meta name="generator" content="TeX4ht
         (http://www.tug.org/tex4ht/)" />\Hnewline}}
\Configure{@HEAD}{\HCode{<link
         rel="stylesheet" type="text/css"
         href="\expandafter\csname aa:CssFile\endcsname" />\Hnewline
    }
}
\Configure{BODY}{\Tg<body>}{\Tg</body>}

\begin{document}

\def\hlink#1#2{\Tg<a href="#1">#2\Tg</a>\xspace}

\EndPreamble
\end{verbatim}
All the configuration items that are processed by \texht at
\verb'\begin{document}' shall be provided before the
  \verb'\begin{document}' command that appears in the preamble area.
    All the elements that form part of prolog and \verb'<head>'
    element of \html are the ideal candidates.  Thus,
    \verb'PROLOG', \verb'VERSION', \verb'DOCTYPE', \verb'HTML',
    \verb'@HEAD' and \verb'BODY' are re-configured as we want them in
    a different manner.  In \html5, we do not need \textsm{DTD}, \xml
    version and further several meta part are more simplified, we
    reconfigure these items in this part of the configuration.  You
    might note that \verb'@HEAD' is  additive, each
    \verb'\Configure{@HEAD}' has one argument which will be added to
    the \verb'<head>...</head>' part of the \html.  We have inserted
    character set, \verb'.css', meta information relating to generator
    which is \texht, by re-configuring \verb'@HEAD'.

\begin{verbatim}
\Configure{textsc}{\Tg<span class="sc">}{\Tg</span>}
\Configure{texttt}{\Tg<span class="tt">}{\Tg</span>}
\Configure{textbf}{\Tg<span class="bf">}{\Tg</span>}

\Css{.sc{font-variant: small-caps;}}
\Css{.bf{font-weight: bold;}}
\Css{small{font-size: .8em;}}
\Css{.tt,.verb{font-family: monospace; white-space: nowrap;
    color: \#222; font-size: 85\%;}}
\end{verbatim}

    The above lines show a few font attributes configuration where we
    have configured commands like \verb'\textsc', \verb'\textbf', and
    \verb'\texttt'.  The related \verb'.css' declarations have been
    provided following the font attributes.
    
\begin{verbatim}
\Css{body{margin-top: 2em; margin-bottom: 1em; width: 60\%;
    margin-left: auto; margin-right: auto; padding: 3em;
    color: \#444; box-shadow: 2px 2px 20px -5px \#bfbfbf;
    -webkit-box-shadow: 2px 2px 20px -8px \#bfbfbf;
    -moz-box-shadow: 2px 2px 20px -5px \#bfbfbf;
    font-family: "EB Garamond";
    font-style: normal;
    font-size: 1em; line-height: 1.3em;
    border: 1px solid \#fefeef; }}

\Css{p.indent{ text-indent: 0em }}
\end{verbatim}
The above provides \verb'.css' declarations for \verb'<body>' elements
with a soft gray shadow, 60\% width of the window, slightly grayed
text color, slightly increased line height, 3em padding on all sides,
etc. Not a big deal. The text indent has been suppressed through
another \verb'.css' declaration, as the indent can interfere with
lettrine spacing. 
    
Our custom configuation ends here if you do not want to make use of
the free web font, namely, EB Garamond created by
\hlink{http://www.georgduffner.at/ebgaramond/impressum.html}{Georg
  Duffner} and distributed under
\hlink{http://scripts.sil.org/cms/scripts/page.php?site_id=nrsi&id=OFL}{\textsm{OFL}}
font licence.  The font can be had from Google or from
\hlink{http://www.georgduffner.at/ebgaramond/}{here}.  The succeeding
lines are \verb'.css' declarations to make use of Garamond and
CheshireInitials fonts in your document. 

%
% Needed only if you want custom font (Garamond) for your page
% 
% http://www.georgduffner.at/ebgaramond/
% \hlink{http://www.georgduffner.at/ebgaramond/impressum.html}{Georg Duffner}

\begin{verbatim}
\Css{@font-face{font-family: InitialsFont; font-style: normal;
    src: url(/fonts/CheshireInitials.ttf);}}
\Css{@font-face{font-family: "EB Garamond SC"; font-variant: small-caps;
    src: url(/fonts/EBGaramond12-SC.ttf);}}
\Css{@font-face{font-family: "EB Garamond"; font-style: normal;
    font-weight: normal; font-variant: normal;
    src: url(/fonts/EBGaramond12-Regular.ttf);}}
\Css{@font-face{font-family: "EB Garamond Italic"; font-style: italic; 
    font-weight: normal;font-variant: normal;
    src: url(/fonts/EBGaramond12-Italic.ttf);}} 
\Css{@font-face{font-family: "EB Garamond Bold"; font-style: normal;
    font-weight: bold; font-variant: normal;
    src: url(/fonts/EBGaramond12-Bold.ttf);}}
\Css{em, i, .ecti, .ecti1, .thanks, .signature, .dvquote-text, 
.dvprayer-text{font-family: "EB Garamond Italic", serif;}}
%
\Css{strong, b, .partToc a, .partToc, .likepartToc a, .likepartToc,
.chapterToc a, .chapterToc, .likechapterToc a, .likechapterToc,
.appendixToc a, .appendixToc, .addchapToc, 
.paragraphHead, .likeparagraphHead,
.subparagraphHead, .likesubparagraphHead,
.caption td.id{font-family: "EB Garamond Bold", serif;}}
%
\Css{.small-caps, .booksubtitle, .dvday, .sectionHead,
.dvquote-ref, .dvbox .lettrine-line{font-family: "EB Garamond SC", serif;}}
%
% CheshireInitials for Lettrine character
%
\Css{.lettrine{float: left; font-family: InitialsFont;
    line-height: 1.15; margin-left: -0.1em; padding-bottom: -0.6em;
    margin-bottom: -1em; margin-right: 0.2em;}}

\end{verbatim}


\subsection*{How to run \texht}

Save the above sources as \verb=test.tex= and \verb=my.cfg=
respectively.  Following command can be invoked from terminal:

\begin{verbatim}
  mzlatex test my ' -cunihtf'
\end{verbatim}

The synatax is 
\begin{verbatim}
  script <tex-file> <conf-file> '<options-for-t4ht>'
\end{verbatim}
Please note the space between the opening single quote character and
\verb'-cunihtf' which is very important.

Enjoy the magic!

\ifx\texhtstandalonedoc1 \end{document}\fi
