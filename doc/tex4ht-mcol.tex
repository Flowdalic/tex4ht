% $Id$
% Copyright 2013 CV Radhakrishnan.
% Released under LPPLv1.3+.
% 
% TeX4ht and multicolumn support.

\ifx\texhtstandalonedoc1
\documentclass{article}
\usepackage{blog}
\usepackage[colorlinks]{hyperref}
\author{CV Radhakrishnan}
\date{\today}
\title{Multicolumn Support in TeX4ht}
\begin{document}
\maketitle
\tableofcontents
\else
\chapter{Multicolumn Support in \TeX4ht}
\fi

\section*{Prolog}
\label{sec-1}

\href{http://www.12000.org/}{Nasser M. Abbassi} had filed a feature
request at \href{http://puszcza.gnu.org.ua/bugs/?188}{TeX4ht} project
for multicolumn support in TeX4ht which is lacking at the
moment. \href{http://www.w3.org/TR/css3-multicol/}{This page} provides
the latest specifications of \href{http://www.w3.org}{W3C} relating to
multicolumn layout in HTML which some of the mainstream browsers support
in varying degrees.  Our initial efforts are pointed towards
\texttt{multicol} package so that TeX4ht provides enough hooks to
generate output with multicolumn layout as envisaged in \LaTeX{} by
\texttt{multicol} package.  Support for
\texttt{\textbackslash{}twocolumn} still remains to be done, though.

\section*{Approach to the Problem}
\label{sec-2}

W3C provides different kinds of CSS properties to bring about
multicolumn layout in HTML. These are briefly listed below:

\begin{enumerate}
\item \texttt{column-count}: number of columns
\item \texttt{column-width}: width of the columns
\item \texttt{column-gap}: gap between columns
\item \texttt{column-rule-color}: color of separator rule
\item \texttt{column-rule-width}: width of separator rule
\item \texttt{column-rule-style}: style of separator rule
\item \texttt{column-span}: should the entry span all columns
\item \texttt{column-fill}: should the multicolumn layout be balanced or ragged
\item \texttt{column-break}: directive for breaks in print media
\end{enumerate}

We have more or less equivalent commands in \texttt{multicol} package and
are listed below:

\begin{enumerate}
\item \texttt{\textbackslash{}columnbreak}: forced break in a column
\item \texttt{\textbackslash{}columnsep}: gap between columns
\item \texttt{\textbackslash{}columnseprule}: rule width
\item \texttt{\textbackslash{}columnseprulecolor}: rule color
\item \texttt{\textbackslash{}flushcolumns}: balanced columns
\item \texttt{\textbackslash{}raggedcolumns}: unbalanced columns
\item \texttt{\textbackslash{}begin\{multicols\}\{<num>\}}: beginning of
        multicolumn enviromnemt
\item \texttt{\textbackslash{}end\{multicols\}}: end of multicolumn environment
\item \texttt{\textbackslash{}multicoltolerance}: no effect on HTML output
\end{enumerate}

Some of the CSS properties do not have equivalent commands in
\texttt{multicol} package. For instance, \texttt{column-width} can be
specified in CSS as either \texttt{auto} or as a legal dimension, but
in \texttt{multicol}, it is always automatically computed. Hence, we
can safely ignore such properties. Conversely, certain
\texttt{multicol} commands like
\texttt{\textbackslash{}multicoltolerance} has no effect in HTML, so
we can ignore such commands in the source documents.

We have written the \texttt{multicol.4ht} to redefine the related
\texttt{multicol} commands to generate the equivalent CSS properties
for HTML output so that multicolumn layout is visually similar to PDF
output.

\section*{Usage}
\label{sec-3}

The multicolumn layout can be tailored using
\texttt{\textbackslash{}Configure} command on various items as given
below:

\begin{verbatim}
\Configure{columngap}{\the\columnsep}
\end{verbatim}
\texttt{columngap} will default to the dimension provided by
\texttt{multicol}'s \texttt{\textbackslash{}columsep}
command. However, if you want to override, you can make use of the
above command. The legal values admissible are any valid
\texttt{dimen} or simply \texttt{normal} which is considered to be
1em.

\begin{verbatim}
\Configure{columnrulewidth}{\the\columnseprule}
\end{verbatim}
\texttt{columnrulewidth} defaults to \LaTeX{} dimension provided by
\texttt{\textbackslash{}columnseprule} which can be overridden by the
above command. The legal values are \texttt{thin | medium | thick} or
any valid dimension.

\begin{verbatim}
\Configure{columnrulecolor}{\#555;}
\end{verbatim}
You can specify any valid color for \texttt{columnrulecolor}. Default
is provided above.

\begin{verbatim}
\Configure{columnrulestyle}{outset}
\end{verbatim}
\texttt{columnrulestyle} has no equivalent in \texttt{multicol}. Specify
rule style with the above hook. Valid values are:
 \texttt{none}, \texttt{hidden}, \texttt{dotted}, \texttt{dashed},
 \texttt{solid}, \texttt{double},  \texttt{groove},
 \texttt{ridge}, \texttt{inset},  \texttt{outset}.

\begin{verbatim}
\Configure{columnspan}{none}
\end{verbatim}
     \texttt{columnspan} has no equivalent command in
     \texttt{multicol}. This is to provide CSS directive of any column
     spanning all the columns, the default value being
     \texttt{none}. Alternate value is \texttt{all}.

\begin{verbatim}
\Configure{columnfill}{balance}
\end{verbatim}
     \texttt{columnfill} does the job what
     \texttt{\textbackslash{}raggedcolumns} and
     \texttt{\textbackslash{}flushcolumns} do in
     \texttt{multicol}. \texttt{balance}, meaning, balanced columns is
     the default value, the other being \texttt{auto}.

\begin{verbatim}
\Configure{multicols}{columns}
\end{verbatim}
This command provides the name of class attribute of the multicolumn
chunk of text. You can assign any string of your choice, the default
is \texttt{columns}. 
\section*{Example}
\label{sec-4}
Here is an example of a working document that can be used to test
these newly introduced features.

\begin{verbatim}
\documentclass[a4paper]{article}
\usepackage{xspace}
\usepackage{lipsum}
\usepackage[margin=1cm,ignoreall]{geometry}

\usepackage{multicol}
\columnseprule=.1pt
\columnsep=1em

\begin{document}

\parindent=0pt

% For webkit browsers (Chrome, for example),
% kindly remove comment marks from two lines below:
%
% \Configure{columnrulecolor}{\#000;}
% \Configure{columnrulewidth}{1px}

\title{Support for Multicolumn Layout in \TeX4ht}
\author{\TeX4ht Project Team}

\maketitle

\lipsum[1]

\begin{multicols}{2}[\section{Testing the Section Heading}]

\lipsum[2-8]

\end{multicols}

\lipsum[1]

\begin{multicols}{3}[\section{Three Column Layout}]

\lipsum[2-8]

\end{multicols}

\lipsum[1]

\begin{multicols}{4}[\section{Four Column Layout}]

\lipsum[2-8]

\end{multicols}

\lipsum[1]
\ifx\HCode\undefined\raggedright\fi

\begin{multicols}{11}[\section{More Than Ten Columns (limit of
    multicol package)}]

\lipsum[2-8]

\end{multicols}

\end{document}
\end{verbatim}

The HTML output can be viewed \href{http://localhost/mcoltest.html}{here}.
\section*{W3C References}
\label{sec-5}

More detailed information is available at
\href{http://www.w3.org/TR/css3-multicol/#property-index}{Property
  Index} page provided at Word Wide Web Consortium web site.


(Originally generated by Org mode 8.0.6 in Emacs 24.3.1.)

\ifx\texhtstandalonedoc1 \end{document}\fi
