% $Id$
% htlatex tex4ht-javahelp-xtpipes "xhtml,next,3" "" "-d./"
%
% Copyright 2009-2020 TeX Users Group
% Copyright 2002-2009 Eitan M. Gurari
% Released under LPPL 1.3c+.
% See tex4ht-cpright.tex for license text.

\documentclass{article}
    \Configure{ProTex}{log,<<<>>>,title,list,`,[[]]}
    \usepackage{url}
\begin{document}

% $Id$
% A few common TeX definitions for literate sources.  Not installed in runtime.
% 
% Copyright 2009-2017 TeX Users Group
% Copyright 1996-2009 Eitan M. Gurari
%
% This work may be distributed and/or modified under the
% conditions of the LaTeX Project Public License, either
% version 1.3c of this license or (at your option) any
% later version. The latest version of this license is in
%   http://www.latex-project.org/lppl.txt
% and version 1.3c or later is part of all distributions
% of LaTeX version 2005/12/01 or later.
%
% This work has the LPPL maintenance status "maintained".
%
% The Current Maintainer of this work
% is the TeX4ht Project <http://tug.org/tex4ht>.
% 
% If you modify this program, changing the 
% version identification would be appreciated.

\newcount\tmpcnt  \tmpcnt\time  \divide\tmpcnt  60
\edef\temp{\the\tmpcnt}
\multiply\tmpcnt  -60 \advance\tmpcnt  \time

\edef\version{\the\year-\ifnum \month<10 0\fi
  \the\month-\ifnum \day<10 0\fi\the\day
   -\ifnum \temp<10 0\fi \temp
   :\ifnum \tmpcnt<10 0\fi\the\tmpcnt}

% a fixed-string version that can be enabled for debugging.
%\edef\versionDebug{000-00-00-00:00}
%\let\version\versionDebug

% #1 is the first year for Eitan.  The last year is always 2009.  RIP.
\def\CopyYear.#1.{#1-2009}

% command for write to terminal and the log file
% this version is used in the .4ht files build
% identical command is defined also in tex4ht-sty.tex, 
% it is used in TeX document compilation
\def\writesixteen#1{\immediate\write1616{#1}}

\<TeX4ht copyright\><<<
%
% This work may be distributed and/or modified under the
% conditions of the LaTeX Project Public License, either
% version 1.3c of this license or (at your option) any
% later version. The latest version of this license is in
%   http://www.latex-project.org/lppl.txt
% and version 1.3c or later is part of all distributions
% of LaTeX version 2005/12/01 or later.
%
% This work has the LPPL maintenance status "maintained".
%
% The Current Maintainer of this work
% is the TeX4ht Project <http://tug.org/tex4ht>.
% 
% If you modify this program, changing the 
% version identification would be appreciated.
>>>

% public domain.
\def\HOME{./tex4ht.dir/}
\def\DTDS{./dtd.dir/}           


%%%%%%%%%%%%%%%%%%
\part{Post Processing for Html Output Mode}
%%%%%%%%%%%%%%%%%%


%%%%%%%%%%%%%%%%%%
\section{Outline}
%%%%%%%%%%%%%%%%%%

\AtEndDocument{\OutputCodE\<javahelp.4xt\>}

\Needs{"xmllint --valid --noout javahelp.4xt"} 

\<javahelp.4xt\><<<
<?xml version="1.0" encoding="UTF-8" ?>
<!DOCTYPE xtpipes SYSTEM "xtpipes.dtd" >
<!-- jsml.4xt (`version), generated from `jobname.tex
     Copyright (C) 2009-2010 TeX Users Group
     Copyright (C) `CopyYear.2002. Eitan M. Gurari
`<TeX4ht copyright`> -->
<xtpipes preamble="yes" signature="javahelp.4xt (`version)">
   <sax content-handler="xtpipes.util.ScriptsManager" 
        lexical-handler="xtpipes.util.ScriptsManagerLH" >
     `<tabular script`>
     `<longtable script`>
     `<email scriptNO`>
     `<empty html element scriptNO`>
   </sax>
</xtpipes>
>>>


%%%%%%%%%%%%%%%%%%
\section{email script}
%%%%%%%%%%%%%%%%%%

\<email script\><<<
<script element="span::email" >
   `<set htm email links`>
</script> 
>>>

   <set name="email" >
      `<open xslt script`>
      `<email templates`>
      `<close xslt script`>
   </set>
   <xslt name="." xml="." xsl="email" />


\<email templates\><<<
<xsl:template match=" text()[contains(.,'@')] " >
<span class="set-emails">
      <xsl:copy>
      </xsl:copy> 
</span>     
</xsl:template> 
>>>


\<set htm email links\><<<
<sax name="." xml="." content-handler="tex4ht.XhtmlEmails" />
>>>


% \AtEndDocument{ 
%    \OutputCodE\<XhtmlEmails.java\> 
% } 
% \Needs{"
%     javac -classpath /home/4/gurari/tex4ht.dir/texmf/tex4ht/bin/tex4ht.jar
%           XhtmlEmails.java
%           -d /home/4/gurari/xtpipes.dir/. 
% "} 
 
\<XhtmlEmails.java\><<< 
package tex4ht;
/* XhtmlEmails.java (`version), generated from `jobname.tex
   Copyright (C) 2009-2010 TeX Users Group
   Copyright (C) `CopyYear.2002. Eitan M. Gurari
`<TeX4ht copyright`> */
import xtpipes.XtpipesUni;
import org.xml.sax.*;
import org.xml.sax.helpers.DefaultHandler;
import java.io.*;
import java.lang.reflect.*;
import java.util.HashMap;

public class XhtmlEmails extends DefaultHandler {
        PrintWriter out = null;
        String data = "";
  public XhtmlEmails(PrintWriter out, 
                       HashMap<String,Object> scripts,
                       Method method, PrintWriter log, boolean trace) {
    this.out = out;
  }
  public void characters(char[] ch, int start, int length) {
    data += new String(ch, start, length);

//XtpipesUni.toUni(ch, start, length, "<>&");
  }

  public void startElement(String ns, String sName,
                                      String qName,
                                      Attributes atts) {
      String s = "<" + qName + "\n";
      for (int i = 0; i < atts.getLength(); i++) {
        String name = atts.getQName(i);
        s += (" " + name + "=\"" 
            + XtpipesUni.toUni(atts.getValue(i), "<>&\"")
            + "\"");
      } 
      s += ">";
      `<process email data`>
      out.print(XtpipesUni.toUni(data, "&") + s);
      data = "";
  }
  public void endElement(String ns, String sName, String qName) {
      String s = "</" + qName + ">";
      `<process email data`>
      out.print( XtpipesUni.toUni(data, "&") + s);
      data = "";
  }  

}
>>>


\<process email data\><<<
while( data.indexOf('@') >0 ){
  `<clean prefix`>
  `<clean postfix`>
}
data = XtpipesUni.toUni(data, "<>&");
>>>




\<clean prefix\><<<
String [] pre = data.split(
                   "[`<email name`>]*@"
                   , 2);
if( pre[0].endsWith("}") ){
  if( pre[0].indexOf("{") != -1 ){
     pre[0] = pre[0].substring( 0, pre[0].lastIndexOf("{") );        
} }
int len = pre[0] . length();
if( len > 0 ){
   out.print( XtpipesUni.toUni(pre[0], "<>&") );
   data = data.substring(len);
}
>>>

\<clean postfix\><<<
String [] post = data.split(
                   "@[`<email name`>]*"
                   , 2);
if( post[1] . length() > 0 ){
  data = data.substring(0, data.length() - post[1] . length());
}
`<set a link to email`>
data = post[1];
>>>


\<set a link to email\><<<
if( data.indexOf("{") == -1 ){
   out.print(
      "<a href=\"mailto:" + XtpipesUni.toUni(data, "&") + "\">"
      + XtpipesUni.toUni(data, "<>&")
      + "</a>"
   );
} else {
  int idx = data.indexOf('@');
  String ext = data.substring(idx);
  data = data.substring(0,idx);
  `<clean group prefix`>
}
>>>



\<clean group prefix\><<<
while( true ){
   pre = data.split(
                "[`<email name`>]+"
                , 2);    
   if( pre.length < 2 ){
      out.print( XtpipesUni.toUni(data + ext, "<>&") );
      data = "";
      break;
   }
   len = pre[0] . length();
   if( len > 0 ){
      out.print( XtpipesUni.toUni(pre[0], "<>&") );
      data = data.substring(len);
   }
   data = data.substring(0, data.length() - pre[1].length());   
   out.print(
      "<a href=\"mailto:" + XtpipesUni.toUni(data + ext, "&") + "\">"
      + XtpipesUni.toUni(data, "<>&")
      + "</a>"
   );
   data = pre[1];
}
>>>

\<email name\><<<
\\p{javaLowerCase}\\p{javaUpperCase}\\d\\-_\\./&>>>


\url{http://tools.ietf.org/html/rfc2368}

\begin{verbatim}
Alice Smith" <alice@somewhere.com>
{alice, bob}@somewhere.com
alice@somewhere.com, bob@somewhere.com
\end{verbatim}

%%%%%%%%%%%%%%%%%%
\section{Long Tables}
%%%%%%%%%%%%%%%%%%


 
\<longtable script\><<<
<script element="table::longtable" >
   <set name="longtbl" >
      `<open xslt script`>
      `<longtable templates`>
      `<close xslt script`>
   </set>
   <xslt name="." xml="." xsl="longtbl" />
</script> 
>>>


\<longtable templates\><<<
<xsl:template match=" table[ @class='longtable' ]
                    / tr[ (normalize-space(.)='') ] " >
   <xsl:if test=" normalize-space(following-sibling::*) != '' ">
      <xsl:copy>
         <xsl:apply-templates select="*|@*|text()|comment()" />
      </xsl:copy>      
   </xsl:if>
</xsl:template> 
>>>




%%%%%%%%%%%%%%%%%%
\section{Tabular}
%%%%%%%%%%%%%%%%%%


 
\<tabular script\><<<
<script element="table::tabular" >
   <set name="tabular" >
      `<open xslt script`>
      `<tabular templates`>
      `<close xslt script`>
   </set>
   <xslt name="." xml="." xsl="tabular" />
</script> 
>>>


\<tabular templates\><<<
<xsl:template match=" table[ @class='tabular' ]
                    / tr[ (normalize-space(.)='') ] " >

<xsl:message terminate="no">
OK 1
</xsl:message>

   <xsl:if test=" normalize-space(following-sibling::*) != '' ">
      <xsl:copy>
         <xsl:apply-templates select="*|@*|text()|comment()" />
<xsl:message terminate="no">
OK 2
</xsl:message>
      </xsl:copy>      
   </xsl:if>
</xsl:template> 
>>>





\<tabular templates\><<<
<xsl:template match=" tr[ (normalize-space(.)='') 
                          and
                          (parent::table[ @class='tabular' ])
                        ] " >

<xsl:message terminate="no">
OK 1
</xsl:message>

   <xsl:if test=" normalize-space(following-sibling::*) != '' ">
      <xsl:copy>
         <xsl:apply-templates select="*|@*|text()|comment()" />
<xsl:message terminate="no">
OK 2
</xsl:message>
      </xsl:copy>      
   </xsl:if>
</xsl:template> 
>>>


%%%%%%%%%%%%%%%%%%
\section{Empty HTML Element Script}
%%%%%%%%%%%%%%%%%%

\<empty html element script\><<<
<script element="meta" >
   <set name="meta" >
      `<open xslt script`>
      <xsl:template match="meta" >
         <xsl:text disable-output-escaping="yes">&lt;meta</xsl:text>
             <xsl:apply-templates select="@*" mode="attrs" />
         <xsl:text disable-output-escaping="yes">/&gt;</xsl:text>
      </xsl:template> 
      `<copy attributes`>
      `<close xslt script`>
   </set>
   <xslt name="." xml="." xsl="meta" />
</script> 
>>>


\<empty html element script\><<< 
<script element="area" > 
   <set name="area" > 
      `<open xslt script`> 
      <xsl:template match="area" > 
         <xsl:text disable-output-escaping="yes">&lt;area</xsl:text> 
             <xsl:apply-templates select="@*" mode="attrs" /> 
         <xsl:text disable-output-escaping="yes">/&gt;</xsl:text> 
      </xsl:template>  
      `<copy attributes`> 
      `<close xslt script`> 
   </set> 
   <xslt name="." xml="." xsl="area" /> 
</script>  
>>> 




\<empty html element script\><<< 
<script element="base" > 
   <set name="base" > 
      `<open xslt script`> 
      <xsl:template match="base" > 
         <xsl:text disable-output-escaping="yes">&lt;base</xsl:text> 
             <xsl:apply-templates select="@*" mode="attrs" /> 
         <xsl:text disable-output-escaping="yes">/&gt;</xsl:text> 
      </xsl:template>  
      `<copy attributes`> 
      `<close xslt script`> 
   </set> 
   <xslt name="." xml="." xsl="base" /> 
</script>  
>>> 




 \<empty html element script\><<< 
<script element="basefont" > 
   <set name="basefont" > 
      `<open xslt script`> 
      <xsl:template match="basefont" > 
         <xsl:text disable-output-escaping="yes">&lt;basefont</xsl:text> 
             <xsl:apply-templates select="@*" mode="attrs" /> 
         <xsl:text disable-output-escaping="yes">/&gt;</xsl:text> 
      </xsl:template>  
      `<copy attributes`> 
      `<close xslt script`> 
   </set> 
   <xslt name="." xml="." xsl="basefont" /> 
</script>  
>>> 




 \<empty html element script\><<< 
<script element="br" > 
   <set name="br" > 
      `<open xslt script`> 
      <xsl:template match="br" > 
         <xsl:text disable-output-escaping="yes">&lt;br</xsl:text> 
             <xsl:apply-templates select="@*" mode="attrs" /> 
         <xsl:text disable-output-escaping="yes">/&gt;</xsl:text> 
      </xsl:template>  
      `<copy attributes`> 
      `<close xslt script`> 
   </set> 
   <xslt name="." xml="." xsl="br" /> 
</script>  
>>> 




 \<empty html element script\><<< 
<script element="col" > 
   <set name="col" > 
      `<open xslt script`> 
      <xsl:template match="col" > 
         <xsl:text disable-output-escaping="yes">&lt;col</xsl:text> 
             <xsl:apply-templates select="@*" mode="attrs" /> 
         <xsl:text disable-output-escaping="yes">/&gt;</xsl:text> 
      </xsl:template>  
      `<copy attributes`> 
      `<close xslt script`> 
   </set> 
   <xslt name="." xml="." xsl="col" /> 
</script>  
>>> 




 \<empty html element script\><<< 
<script element="frame" > 
   <set name="frame" > 
      `<open xslt script`> 
      <xsl:template match="frame" > 
         <xsl:text disable-output-escaping="yes">&lt;frame</xsl:text> 
             <xsl:apply-templates select="@*" mode="attrs" /> 
         <xsl:text disable-output-escaping="yes">/&gt;</xsl:text> 
      </xsl:template>  
      `<copy attributes`> 
      `<close xslt script`> 
   </set> 
   <xslt name="." xml="." xsl="frame" /> 
</script>  
>>> 




 \<empty html element script\><<< 
<script element="hr" > 
   <set name="hr" > 
      `<open xslt script`> 
      <xsl:template match="hr" > 
         <xsl:text disable-output-escaping="yes">&lt;hr</xsl:text> 
             <xsl:apply-templates select="@*" mode="attrs" /> 
         <xsl:text disable-output-escaping="yes">/&gt;</xsl:text> 
      </xsl:template>  
      `<copy attributes`> 
      `<close xslt script`> 
   </set> 
   <xslt name="." xml="." xsl="hr" /> 
</script>  
>>> 




 \<empty html element script\><<< 
<script element="img" > 
   <set name="img" > 
      `<open xslt script`> 
      <xsl:template match="img" > 
         <xsl:text disable-output-escaping="yes">&lt;img</xsl:text> 
             <xsl:apply-templates select="@*" mode="attrs" /> 
         <xsl:text disable-output-escaping="yes">/&gt;</xsl:text> 
      </xsl:template>  
      `<copy attributes`> 
      `<close xslt script`> 
   </set> 
   <xslt name="." xml="." xsl="img" /> 
</script>  
>>> 




 \<empty html element script\><<< 
<script element="input" > 
   <set name="input" > 
      `<open xslt script`> 
      <xsl:template match="input" > 
         <xsl:text disable-output-escaping="yes">&lt;input</xsl:text> 
             <xsl:apply-templates select="@*" mode="attrs" /> 
         <xsl:text disable-output-escaping="yes">/&gt;</xsl:text> 
      </xsl:template>  
      `<copy attributes`> 
      `<close xslt script`> 
   </set> 
   <xslt name="." xml="." xsl="input" /> 
</script>  
>>> 




 \<empty html element script\><<< 
<script element="isindex" > 
   <set name="isindex" > 
      `<open xslt script`> 
      <xsl:template match="isindex" > 
         <xsl:text disable-output-escaping="yes">&lt;isindex</xsl:text> 
             <xsl:apply-templates select="@*" mode="attrs" /> 
         <xsl:text disable-output-escaping="yes">/&gt;</xsl:text> 
      </xsl:template>  
      `<copy attributes`> 
      `<close xslt script`> 
   </set> 
   <xslt name="." xml="." xsl="isindex" /> 
</script>  
>>> 




 \<empty html element script\><<< 
<script element="link" > 
   <set name="link" > 
      `<open xslt script`> 
      <xsl:template match="link" > 
         <xsl:text disable-output-escaping="yes">&lt;link</xsl:text> 
             <xsl:apply-templates select="@*" mode="attrs" /> 
         <xsl:text disable-output-escaping="yes">/&gt;</xsl:text> 
      </xsl:template>  
      `<copy attributes`> 
      `<close xslt script`> 
   </set> 
   <xslt name="." xml="." xsl="link" /> 
</script>  
>>> 




 \<empty html element script\><<< 
<script element="param" > 
   <set name="param" > 
      `<open xslt script`> 
      <xsl:template match="param" > 
         <xsl:text disable-output-escaping="yes">&lt;param</xsl:text> 
             <xsl:apply-templates select="@*" mode="attrs" /> 
         <xsl:text disable-output-escaping="yes">/&gt;</xsl:text> 
      </xsl:template>  
      `<copy attributes`> 
      `<close xslt script`> 
   </set> 
   <xslt name="." xml="." xsl="param" /> 
</script>  
>>> 










\<copy attributes\><<<
<xsl:template match="@*" mode="attrs" >
   <xsl:text> </xsl:text>
   <xsl:value-of select="name()" />
   <xsl:text>="</xsl:text>
   <xsl:value-of select="." />
   <xsl:text>"</xsl:text>
</xsl:template>
>>>




%%%%%%%%%%%%%%%%%%
\section{Shared}
%%%%%%%%%%%%%%%%%%



\<open xslt script\><<<
<![CDATA[ 
   <xsl:stylesheet version="1.0"
      xmlns:xsl="http://www.w3.org/1999/XSL/Transform"
   >
      <xsl:output omit-xml-declaration = "yes" method="xml" />
>>>

\<close xslt script\><<<
      <xsl:template match="*|@*|text()|comment()" >
        <xsl:copy>
          <xsl:apply-templates select="*|@*|text()|comment()" />
        </xsl:copy>
      </xsl:template>
   </xsl:stylesheet> 
]]>
>>>





\AtEndDocument{\Needs{%
    "pushd \XTPIPES || exit 1
     ;  
     jar cf tex4ht.jar *
     ;
     popd
     ;
     mv tex4ht.jar \TEXMFTEXivBIN
     ;
     if [ ! -d \TEXMFTEXivXTPIPES\space]; then exit 1; fi
"}}

\end{document}
