% $Id$
% compile 3 times:     latex   tex4ht-tei   
%                  or  xhlatex tex4ht-tei "html,3,sections+"
%
% Copyright (C) 2009-2016 TeX Users Group
% Copyright (C) 2000-2009 Eitan M. Gurari
% Released under LPPL 1.3c+.
% See tex4ht-cpright.tex for license text.

%%%%%%%%%%%%%%%%%% load style files %%%%%%%%%%%%%%%%%%%%%%%%%%

\ifx \HTML\UnDef
   \def\HTML{tei,tei-mml,tei-mmltei,tei-math}                             
   \def\CONFIG{\jobname}
   \def\MAKETITLE{\title{Source for TEI}%     
        \author{Eitan M. Gurari \& Sebastian Rahtz}}   
   \def\next{\input mktex4ht.4ht  \endinput}
   \expandafter\next
\fi

% $Id$
% A few common TeX definitions for literate sources.  Not installed in runtime.
% 
% Copyright 2009-2017 TeX Users Group
% Copyright 1996-2009 Eitan M. Gurari
%
% This work may be distributed and/or modified under the
% conditions of the LaTeX Project Public License, either
% version 1.3c of this license or (at your option) any
% later version. The latest version of this license is in
%   http://www.latex-project.org/lppl.txt
% and version 1.3c or later is part of all distributions
% of LaTeX version 2005/12/01 or later.
%
% This work has the LPPL maintenance status "maintained".
%
% The Current Maintainer of this work
% is the TeX4ht Project <http://tug.org/tex4ht>.
% 
% If you modify this program, changing the 
% version identification would be appreciated.

\newcount\tmpcnt  \tmpcnt\time  \divide\tmpcnt  60
\edef\temp{\the\tmpcnt}
\multiply\tmpcnt  -60 \advance\tmpcnt  \time

\edef\version{\the\year-\ifnum \month<10 0\fi
  \the\month-\ifnum \day<10 0\fi\the\day
   -\ifnum \temp<10 0\fi \temp
   :\ifnum \tmpcnt<10 0\fi\the\tmpcnt}

% a fixed-string version that can be enabled for debugging.
%\edef\versionDebug{000-00-00-00:00}
%\let\version\versionDebug

% #1 is the first year for Eitan.  The last year is always 2009.  RIP.
\def\CopyYear.#1.{#1-2009}

% command for write to terminal and the log file
% this version is used in the .4ht files build
% identical command is defined also in tex4ht-sty.tex, 
% it is used in TeX document compilation
\def\writesixteen#1{\immediate\write1616{#1}}

\<TeX4ht copyright\><<<
%
% This work may be distributed and/or modified under the
% conditions of the LaTeX Project Public License, either
% version 1.3c of this license or (at your option) any
% later version. The latest version of this license is in
%   http://www.latex-project.org/lppl.txt
% and version 1.3c or later is part of all distributions
% of LaTeX version 2005/12/01 or later.
%
% This work has the LPPL maintenance status "maintained".
%
% The Current Maintainer of this work
% is the TeX4ht Project <http://tug.org/tex4ht>.
% 
% If you modify this program, changing the 
% version identification would be appreciated.
>>>


%%%%%%%%%%%%%%%%%%%%%%%%%%%%%%%%%%%%%%%%%%%%%%%%%%%%%%%%%%%%%%%%%%%%%%%%
\chapter{File Organization}
%%%%%%%%%%%%%%%%%%%%%%%%%%%%%%%%%%%%%%%%%%%%%%%%%%%%%%%%%%%%%%%%%%%%%%%%

\<tei\><<<
% tei.4ht (|version), generated from |jobname.tex
% Copyright 2009-2016 TeX Users Group
% Copyright (C) |CopyYear.2000. Eitan M. Gurari
|<TeX4ht copywrite|>
>>>

\<configure tei tex4ht\><<<
\if:latex  |<Hinclude latex|>
\else      |<Hinclude plain|>  \fi
>>>

\<configure tei plain\><<<
|<Hinclude plain lib|>
|<Hinclude plain + latex lib|>
>>>

\<configure tei latex\><<<
|<Hinclude latex lib|>
|<Hinclude plain + latex lib|>
>>>

\<configure tei Preamble\><<<    
\Configure{ext}{xml}
\edef\Preamble{\Preamble,refcaption}
>>>

\<tei-mml\><<<
%%%%%%%%%%%%%%%%%%%%%%%%%%%%%%%%%%%%%%%%%%%%%%%%%%%%%%%%%%  
% tei-mml.4ht (|version), generated from |jobname.tex
% Copyright (C) |CopyYear.2000. Eitan M. Gurari
|<TeX4ht copywrite|>
>>>


\<tei-math\><<<
%%%%%%%%%%%%%%%%%%%%%%%%%%%%%%%%%%%%%%%%%%%%%%%%%%%%%%%%%%  
% tei-math.4ht (|version), generated from |jobname.tex
% Copyright (C) |CopyYear.2000. Eitan M. Gurari
|<TeX4ht copywrite|>
>>>


\<tei-mmltei\><<<
%%%%%%%%%%%%%%%%%%%%%%%%%%%%%%%%%%%%%%%%%%%%%%%%%%%%%%%%%%  
% tei-mmltei.4ht (|version), generated from |jobname.tex
% Copyright (C) |CopyYear.2000. Eitan M. Gurari
|<TeX4ht copywrite|>
>>>



%%%%%%%%%%%%%%%%%%%%%%%%%%%%%%%%%%%%%%%%%%%%%%%%%%%%%%%%%%%%%%%%%%%%%%%%
\chapter{Origin of Hooks}
%%%%%%%%%%%%%%%%%%%%%%%%%%%%%%%%%%%%%%%%%%%%%%%%%%%%%%%%%%%%%%%%%%%%%%%%

%%%%%%%%%%%%%%%%%%
\section{tex4ht}
%%%%%%%%%%%%%%%%%%

\<configure tei tex4ht\><<<    
|<tex4ht wrapper|>
|<tex4ht paragraphs|>
|<tex4ht cross-linking|>
|<tex4ht math|>
>>>



%%%%%%%%%%%%%%%
\section{latex}
%%%%%%%%%%%%%%%

\<configure tei latex\><<< 
|<special latex environments|>
|<revised TITLE for latex|>
|<latex environments|>
|<latex lists|>
|<latex tables|>
|<latex env|>
|<latex,plain accents|>
|<latex accents|>
|<latex math|>
|<obey lines|>
|<latex etc|>
>>>




%%%%%%%%%%%%%%%
\section{plain}
%%%%%%%%%%%%%%%

\<configure tei plain\><<< 
|<latex,plain accents|>
>>>


%%%%%%%%%%%%%%%%%%%%%%%%%%%%%%%
\section{article, report, book}
%%%%%%%%%%%%%%%%%%%%%%%%%%%%%%%

\<configure tei article\><<< 
|<article,report,book utilities|>
|<article,report maketitle|>
|<article sections|>
|<article,report,book lists|>
|<article,report,book figures|>
|<article tocs|>
|<article,report|>
>>>


\<configure tei report\><<< 
|<article,report,book utilities|>
|<article,report maketitle|>
|<article,report,book lists|>
|<article,report,book figures|>
|<report,book tocs|>
|<article,report|>
>>>



\<configure tei book\><<< 
|<article,report,book utilities|>
|<book maketitle|>
|<article,report,book lists|>
|<article,report,book figures|>
|<report,book tocs|>
>>>


\<configure tei amsart\><<< 
|<sectioning utilities|>
|<article,report maketitle|>
|<ams article,report maketitle|>
|<ams art,proc,book|>
>>>


%%%%%%%%%%%%%%%%%%%%%%%%%%%%%%%
\section{TUG Boat/Proc}
%%%%%%%%%%%%%%%%%%%%%%%%%%%%%%%


\<configure tei ltugboat\><<< 
\Configure{address}
   {\HCode{<byline rend="address">}}
   {\HCode{</byline>}}
\Configure{netaddress}
   {\HCode{<byline rend="netaddress">}}
   {\HCode{</byline>}}
\Configure{PersonalURL}
   {\HCode{<byline rend="PersonalURL">}}
   {\HCode{</byline>}}
>>>


\<configure tei ltugproc\><<< 
|<ltugproc maketitle|>
\Configure{thanks author date and}{}{}
   {|<tag proc author|>\HCode{<docAuthor>}} 
   {\HCode{</docAuthor>}}  
   {\HCode{<docDate>}}      {\HCode{</docDate>}}  
   {\HCode{<!--and-->}}
   {}
>>>

\<tag proc author\><<<
{\Configure{thanks author date and}{}{}{}{}{}{}{}{}%
|<disable latex fonts|>|<proc title @author|>\Tag{)author)}{\@author}}%
>>>

\<proc title @author\><<<
\count@\authornumber
\@tempcnta0
\let\:temp\relax
\let\@author\empty
\loop
  \ifnum\count@>0
    \advance\@tempcnta by \@ne      
    \edef\@author{\@author  \ifx\@author\empty\else, \fi
                 \:temp{\number\@tempcnta}}%
    \advance\count@ by \m@ne
\repeat
\let\ignorespaces\empty
\let\unskip\empty
\def\:temp##1{\csname theauthor##1\endcsname}%
>>>


\<configure tei ltugproc\><<< 
\ConfigureEnv{abstract}
   {}{} {}{}
\Configure{abstract}
   {\bgroup
    \HCode{<epigraph rend="abstract"><p>}\IgnoreIndent\par}
   {\ifvmode\IgnorePar\fi \EndP\HCode{</p></epigraph>}\egroup}
>>>





%%%%%%%%%%%%%%%%%%%%%%%%%%%%%%%%%%%%%%%%%%%%%%%%%%%%%%%%%%%%%%%%%%%%%%%%
\chapter{Configurations}
%%%%%%%%%%%%%%%%%%%%%%%%%%%%%%%%%%%%%%%%%%%%%%%%%%%%%%%%%%%%%%%%%%%%%%%%


%%%%%%%%%%%%%%%%%%%%%%%%%%%%%%%
\section{Front and Back Matter}
%%%%%%%%%%%%%%%%%%%%%%%%%%%%%%%



\begin{verbatim}
<TEI.2>
    <teiHeader> [ TEI Header information ]  </teiHeader>
    <text>
        <front> [ front matter ... ]   </front>
        <body>  [ body of text ... ]   </body>
        <back>  [ back matter ...  ]   </back>
    </text>
</TEI.2>
\end{verbatim}

The front and back elements are optional. 

\<tex4ht wrapper\><<<
\Configure{HTML}
   {\HCode{<TEI.2>}}
   {\HCode{</TEI.2>}}
\Configure{HEAD}
   {\HCode{<teiHeader type="text" status="new"><fileDesc>}}
   {\HCode{</fileDesc>\Hnewline
      <revisionDesc><list 
         type="simple"><item></item></list></revisionDesc>\Hnewline
      </teiHeader>}}
\Configure{TITLE} 
   {\HCode{<titleStmt><title></title></titleStmt>\Hnewline}}
   {}
\Configure{TITLE+} 
   {\HCode{<publicationStmt> <p></p> </publicationStmt>\Hnewline
           <sourceDesc><p>\FileName\space from \jobname.tex
                          (TeX4ht)</p></sourceDesc>\Hnewline
   }}
>>>



\<configure tei-mmltei tex4ht\><<<
\Configure{HTML}
   {\HCode{<TEI.2>}|<xsl elements|>}
   {\HCode{</TEI.2>}}
>>>


The  \verb+<!--@Basename=\jobname-->+ is for clean-tei.xsl.



\<revised TITLE for latex\><<<
\Configure{TITLE} 
   {\HCode{<titleStmt>\Hnewline
       <title>\ifTag{TITLE+}{\LikeRef{TITLE+}}{}</title>\Hnewline
       <author>\ifTag{)author)}{\LikeRef{)author)}}{}</author>\Hnewline
       </titleStmt>\Hnewline
   }}
   {}%
>>>

\<tag author\><<<
{\Configure{thanks author date and}{}{}{}{}{}{}{}{}%
|<disable latex fonts|>\Tag{)author)}{\@author}}%
>>>


\<tag title\><<<
{\Configure{maketitle}{}{}{}{}{}{}{}{}%
|<disable latex fonts|>\Tag{TITLE+}{\@title}}%
>>>

% \let\thanks=\:gobble 
% 
% \def\TeX{TeX}%
%\def\gobble:font##1##2{##2}\:TITLE:

\<disable latex fonts\><<<
\let\leavevmode=\empty \let\not@math@alphabet=\:gobbleII
\def\text@command##1{##1}\let\selectfont=\empty \let\\=\empty  
\no:fonts  \def\check@icl ##1\check@icr{}%
\:TITLE:  \a:NoSection
>>>

\<configure tei latex\><<<
\Configure{@TITLE}{\def\LaTeX{LaTeX}\let\fontencoding=\:gobble
  \let\fontfamily=\:gobble \let\fontseries=\:gobble
  \let\fontshape=\:gobble  \let\usefont=\:gobbleIV  }
>>>








\<special latex environments\><<<
\Configure{InsertTitle}{\let\label=\lb:l
   \let\ref=\o:ref \Configure{ref}{}{}{}}
\Configure{AfterTitle}{\let\index=\:index
       \let\ref=\:ref  \let\label=\lb:l }
\Configure{NoSection}
  {\let\sv:index=\index \let\sv:label=\label \let\sv:ref=\ref
   \let\sv:newline=\newline \def\newline{ }%
   \let\sv:setfontsize=\@setfontsize  \let\@setfontsize=\:gobbleIII
   \let\index=\@gobble  \let\label=\@gobble  \let\ref=\@gobble
  }
  {\let\index=\sv:index \let\label=\sv:label \let\ref=\sv:ref
   \let\newline=\sv:newline    \let\@setfontsize=\sv:setfontsize 
  }
>>>



\<book maketitle\><<<
\Configure{maketitle}
   {|<tag title|>\HCode{<titlePage>}} 
   {\HCode{</titlePage>}}
   {\HCode{<title>}} 
   {\HCode{</title>}}
\Configure{thanks author date and}{}{}
   {|<tag author|>\HCode{<author>}} 
   {\HCode{</author>}}  
   {\HCode{<date>}}      {\HCode{</date>}}  
   {\HCode{<!--and-->}}
   {}
>>>

\<article,report maketitle\><<<
\Configure{maketitle}
   {|<tag title|>\HCode{<titlePage>}} 
   {\HCode{</titlePage>}}
   {\HCode{<docTitle><titlePart  type="main">}} 
   {\HCode{</titlePart></docTitle>}}
\Configure{thanks author date and}{}{}
   {|<tag author|>\HCode{<docAuthor>}} 
   {\HCode{</docAuthor>}}  
   {\HCode{<docDate>}}      {\HCode{</docDate>}}  
   {\HCode{<!--and-->}}
   {}
>>>


\<tex4ht wrapper\><<<
\Configure{BODY}
   {\ifTag{TITLE+}%
     {\HCode{<text><front>}\PushStack\text:stack{front}}%
     {\HCode{<text><body>}\PushStack\text:stack{body}\ShowPar
      \csname par\endcsname}%
   }
   {\IgnorePar\EndP \PopStack\text:stack\:temp  \def\:tempa{front}%
    \HCode{</\:temp>\ifx\:temp\:tempa <body><p></p></body>\fi</text>}}
>>>

\<ams article,report maketitle\><<<
\Configure{authors}
   {\HCode{<docAuthor>}}  {\HCode{</docAuthor>}}      
\Configure{setdate}
     {\HCode{<docDate>}}      {\HCode{</docDate>}}   
\Configure{thanks author date and}{}{}
   {|<tag author|>\HCode{<docAuthor>}} 
   {\HCode{</docAuthor>}}  
   {\HCode{<date>}}      {\HCode{</date>}}  
   {\HCode{<!--and-->}}
   {}
>>>




\<ltugproc maketitle\><<<
\Configure{maketitle}
   {|<tag title|>\HCode{<titlePage>}%
    \bgroup \Configure{centerline}
              {\HCode{<byline rend="centerline">}}
              {\HCode{</byline>}}
   } 
   {\egroup \HCode{</titlePage>}}
   {\HCode{<docTitle>}} 
   {\HCode{</docTitle>}}
>>>

%%%%%%%%%%%%%%%%%%
\subsection{DTD's}
%%%%%%%%%%%%%%%%%%





\<configure tei Preamble\><<<  
\Configure{PROLOG}{VERSION,DOCTYPE,*XML-STYLESHEET}
\:CheckOption{no-DOCTYPE} \if:Option
   \Configure{PROLOG}{VERSION,*XML-STYLESHEET}
\fi
>>>



\<configure tei tex4ht\><<<
\Configure{VERSION}
   {\HCode{<?xml version="1.0"?>\Hnewline}}     
\Configure{VERSION}
   {\HCode{<?xml version="1.0"?>\Hnewline}}      
\Configure{DOCTYPE}
   {\HCode{<!DOCTYPE TEI.2 [|<tei dtd tex4ht|>]>}}
>>>

\<configure tei-mml tex4ht\><<<
\Configure{VERSION}
   {\HCode{<?xml version="1.0"?>\Hnewline}}     
\Configure{VERSION}
   {\HCode{<?xml version="1.0"?>\Hnewline}}      
\Configure{DOCTYPE}
   {\HCode{<!DOCTYPE TEI.2 [\Hnewline|<tei-mml dtd tex4ht|>]>\Hnewline}}
>>>

\<configure tei-mmltei tex4ht\><<<
\Configure{VERSION}
   {\HCode{<?xml version="1.0"?>\Hnewline}}     
\Configure{VERSION}
   {\HCode{<?xml version="1.0"?>\Hnewline}}      
\Configure{DOCTYPE}
   {\HCode{<!DOCTYPE TEI.2 |<tei-mmltei dtd tex4ht|> >\Hnewline}}
>>>


\<tei dtd tex4ht\><<<
|<tei dtd|>%
}\writesixteen{--- file \jobname.ent ---}\HCode{%
|<tei entities dtd|>%
>>>

\<tei-mml dtd tex4ht\><<<
|<tei-mml dtd|>%
|<tei dtd tex4ht|>%
|<mathml dtd|>%
>>>


\<tei-mmltei dtd tex4ht\><<<
SYSTEM "teimml.dtd" [\Hnewline}
|<xsl entities|>\HCode{%
]
>>>

\<tei-mml dtd\><<<
<!ELEMENT formula (math)>\Hnewline
>>>

% <!ENTITY \% list.qname "mathml-list">\Hnewline

\<tei dtd\><<<
<!ENTITY \% TEI.dtd PUBLIC\Hnewline
    "-//TEI//DTD TEI Lite XML ver. 1.3//|<dtd lang|>"\Hnewline
    "teixlite.dtd">
\Hnewline \%TEI.dtd; \Hnewline
>>>

\<mathml dtd\><<<
<!ENTITY \% mathml.dtd PUBLIC
    "-//W3C//DTD MathML 2.0//|<dtd lang|>"\Hnewline\space\space
    "mathml2.dtd">  
    \Hnewline \%mathml.dtd;\Hnewline
>>>


\<xsl elements\><<<
\HCode{<!--@entities=}|<xsl entities|>\HCode{-->}%
>>>

\<tei entities dtd\><<<
<!ENTITY \% ENT-\jobname.ent SYSTEM "\jobname.ent">
       \Hnewline \%ENT-\jobname.ent; \Hnewline
>>>



\<xsl entities\><<<
\bgroup
   \def\toctei##1##2##3{\HCode{##2\Hnewline}}%
   \no:catcodes{34}{36}{12}\no:catcodes{94}{95}{12}\catcode`\~=12
   \Configure{tableofcontents}  {}   {}   {}{}{}%
   \:TableOfContents[tei]%
\egroup
>>>



Instead of \verb=SYSTEM= earlier we had
\verb=PUBLIC\Hnewline "-//TEI U5-1995//DTD TEI Lite 1.0 Extensions//|<dtd lang|>"\Hnewline=.


\<early Decl\><<<
\let\:NOTATIONs=\empty 
\def\Decl#1{\edef\:NOTATIONs{\:NOTATIONs#1\Hnewline}}
>>>


\<early xslt Decl\><<<
\def\Decl#1{\edef\:temp{\the\:tokwrite{\string\doTocEntry
   \string\toc tei{}{#1}{}}}\:temp
}
>>>

\<main Decl\><<<
   \def\Decl##1{{\everypar{}%
      \ht:special{t4ht>\jobname.ent}\hbox{\HCode{##1\Hnewline}}%
      \ht:special{t4ht>\FileName}}}
   \Decl{\:NOTATIONs}\let\:NOTATIONs=\:UnDef
>>>


\begin{verbatim}
> oh yes. I copied the entity declaration into a comment after <TEI.2>,
> because xsl has no access to the document subset...

Consider this:

<?xml version="1.0"?>
<!DOCTYPE TEI.2 [
<!ELEMENT formula (math)>
<!ENTITY % TEI.dtd PUBLIC
"-//TEI//DTD TEI Lite XML ver. 1.3//|<dtd lang|>"
"teixlite.dtd">
%TEI.dtd;
<!ENTITY % ENT-1-3.ent SYSTEM "1-3.ent">
%ENT-1-3.ent;
<!ENTITY % list.qname "mathml-list"> <!ENTITY % mathml.dtd PUBLIC "-//W3C//DTD \
MathML 2.0//|<dtd lang|>"
  "mathml2.dtd">
%mathml.dtd;
]>


the entities are read in from an external file via the %ENT-1-3.ent;
unfortunately such parameter entities do not have to be expanded by
XML processors, so the file can not get read. xmltex is an example of
a system which fails to do so...

we need to textually include the entity file, not reference it.
\end{verbatim}


\<configure tei tex4ht\><<<
\HAssign\ent:cnt = 0
\Configure{Preamble}
   {}
   {{\ifdim \lastskip>\z@ \unskip\fi
     \IgnorePar\parindent\z@ \leavevmode}%
     |<main Decl|>%     
   } 
|<early Decl|>
>>>


\<configure tei-mmltei tex4ht\><<<
\Configure{Preamble}
   {}
   {{\ifdim \lastskip>\z@ \unskip\fi
     \IgnorePar\parindent\z@ \leavevmode}%
   } 
|<early xslt Decl|>
>>>


%%%%%%%%%%%%%%%%%%
\section{Sections}
%%%%%%%%%%%%%%%%%%

\<article sections\><<<
\Configure{part}
   {\OpenDiv{part}}   {\CloseDiv{part}}
   {\HCode{<head>}} {\CloseHead}
\Configure{section}
   {\OpenDiv{section}}   {\CloseDiv{section}}
   {\HCode{<head>}}   {\CloseHead}
\Configure{subsection}
   {\OpenDiv{subsection}}   {\CloseDiv{subsection}}
   {\HCode{<head>}}   {\CloseHead}
\Configure{subsubsection}
   {\OpenDiv{subsubsection}}   {\CloseDiv{subsubsection}}
   {\HCode{<head>}}   {\CloseHead}
\Configure{paragraph}
   {\OpenDiv{paragraph}}   {\CloseDiv{paragraph}}
   {\HCode{<head>}}   {\CloseHead}
\Configure{subparagraph}
   {\OpenDiv{subparagraph}}   {\CloseDiv{subparagraph}}
   {\HCode{<head>}}   {\CloseHead}
>>>




\<article sections\><<<
\Configure{likepart}
   {\OpenDiv{likepart}}   {\CloseDiv{likepart}}
   {\HCode{<head>}} {\CloseHead}
\Configure{likesection}
   {\OpenDiv{likesection}}   {\CloseDiv{likesection}}
   {\HCode{<head>}}   {\CloseHead}
\Configure{likesubsection}
   {\OpenDiv{likesubsection}}   {\CloseDiv{likesubsection}}
   {\HCode{<head>}}   {\CloseHead}
\Configure{likesubsubsection}
   {\OpenDiv{likesubsubsection}}   {\CloseDiv{likesubsubsection}}
   {\HCode{<head>}}   {\CloseHead}
\Configure{likeparagraph}
   {\OpenDiv{likeparagraph}}   {\CloseDiv{likeparagraph}}
   {\HCode{<head>}}   {\CloseHead}
\Configure{likesubparagraph}
   {\OpenDiv{likesubparagraph}}   {\CloseDiv{likesubparagraph}}
   {\HCode{<head>}}   {\CloseHead}
>>>



\<ams art,proc,book\><<<
|<article sections|>
>>>


%\let\contentsname|=\empty

\<article tocs\><<<
\Configure{tableofcontents}
   {\ifvmode\IgnorePar\fi \EndP 
    \HCode{<divGen type="toc"/>}} 
   {} {}{}{}
>>>



\<report,book tocs\><<<
|<article tocs|>
\ConfigureToc{appendix}{}{}{}{}
\ConfigureToc{chapter}{}{}{}{}
\ConfigureToc{likechapter}{}{}{}{}
>>>

\<article tocs\><<<
\ConfigureToc{likeparagraph}{}{}{}{}
\ConfigureToc{likepart}{}{}{}{}
\ConfigureToc{likesection}{}{}{}{}
\ConfigureToc{likesubparagraph}{}{}{}{}
\ConfigureToc{likesubsection}{}{}{}{}
\ConfigureToc{likesubsubsection}{}{}{}{}
\ConfigureToc{paragraph}{}{}{}{}
\ConfigureToc{part}{}{}{}{}
\ConfigureToc{section}{}{}{}{}
\ConfigureToc{subparagraph}{}{}{}{}
\ConfigureToc{subsection}{}{}{}{}
\ConfigureToc{subsubsection}{}{}{}{}
>>>







\<configure tei tex4ht-OUT\><<<
\Configure{TocEntry}{\Link{#2}{#3}#4\EndLink}   
>>>



\<article tocs-OUT\><<<
\Configure{tableofcontents}
  {\HCode{<list rend="toc">}}
  {\HCode{</list>}}
  {}{}{}
>>>


\<report,book tocs-OUT\><<<
|<article tocs|>
\ConfigureToc{appendix}
   {\HCode{\Hnewline<item rend="appendixToc">}}
   {\ }{} {\HCode{</item>}}
\ConfigureToc{chapter}
   {\HCode{\Hnewline<item rend="chapterToc">}}
   {\ }{} {\HCode{</item>}}
\ConfigureToc{likechapter}
   {} {\HCode{\Hnewline<item rend="likechapterToc">}}
   {} {\HCode{</item>}}
>>>

\<article tocs-OUT\><<<
\ConfigureToc{likeparagraph}
   {} {\HCode{\Hnewline<item rend="likeparagraphToc">}}
   {} {\HCode{</item>}}
\ConfigureToc{likepart}
   {} {\HCode{\Hnewline<item rend="likepartToc">}}
   {} {\HCode{</item>}}
\ConfigureToc{likesection}
   {} {\HCode{\Hnewline<item rend="likesectionToc">}}
   {} {\HCode{</item>}}
\ConfigureToc{likesubparagraph}
   {} {\HCode{\Hnewline<item rend="likesubparagraphToc">}}
   {} {\HCode{</item>}}
\ConfigureToc{likesubsection}
   {} {\HCode{\Hnewline<item rend="likesubsectionToc">}}
   {} {\HCode{</item>}}
\ConfigureToc{likesubsubsection}
   {} {\HCode{\Hnewline<item rend="likesubsubsectionToc">}}
   {} {\HCode{</item>}}
\ConfigureToc{paragraph}
   {\HCode{\Hnewline<item rend="paragraphToc">}}
   {\ }{} {\HCode{</item>}}
\ConfigureToc{part}
   {\HCode{\Hnewline<item rend="partToc">}}
   {\ }{} {\HCode{</item>}}
\ConfigureToc{section}
   {\HCode{\Hnewline<item rend="sectionToc">}}
   {\ }{} {\HCode{</item>}}
\ConfigureToc{subparagraph}
   {\HCode{\Hnewline<item rend="subparagraphToc">}}
   {\ }{} {\HCode{</item>}}
\ConfigureToc{subsection}
   {\HCode{\Hnewline<item rend="subsectionToc">}}
   {\ }{} {\HCode{</item>}}
\ConfigureToc{subsubsection}
   {\HCode{\Hnewline<item rend="subsubsectionToc">}}
   {\ }{} {\HCode{</item>}}
>>>











\subsection{Mapping Utilities}


\begin{verbatim}
                                  \section{...}
<div1 type="section">        
  <div2 type="preamble">                    
    .............................
  </div2>                                
                                  \subsection{...}
  <div2 type="subsection">                    
    .............................
  </div2>                    
    .............................
</div1>                    
                                  \section{...}
<div1 type="section">                    
    .............................
</div1>                    
\end{verbatim}










\<article,report,book utilities\><<<
|<sectioning utilities|>
>>>

\<sectioning utilities\><<<
\HAssign\DivLevel=0
\HAssign\DivCount=0
\def\OpenDiv#1{%
   \IgnorePar\EndP \PopStack\text:stack\:temp \def\:tempa{body}%
   \ifx \:temp\:tempa \else \HCode{</\:temp><body>}\fi
   \PushStack\text:stack{body}%
   |<close div preamble|>%
   \gHAdvance\DivLevel by 1
   \HCode{<!--l. \the\inputlineno--><div\Hnewline
       type="#1">}\gHAdvance\P:level by 1 }
\def\CloseDiv#1{%
   \IgnorePar\EndP \gHAdvance\P:level by -1 \HCode{</div><!--end #1-->}%
   \gHAdvance\DivLevel by -1
}
\def\CloseHead{\HCode{</head>}%
   |<open div preamble|>\ShowPar\par}
>>>


\<close div preamble\><<<
\ifnum \DivLevel>0  \ifnum \DivLevel=0\LikeRef{dv)\DivCount}\relax
   \gHAdvance\P:level by 1 
   \HCode{</div>}%
\fi\fi
>>>

\<open div preamble\><<<
\gHAdvance\DivCount by 1
\Tag{dv)\DivCount}{\DivLevel}%
\HAdvance\DivLevel by 1
\HAdvance\DivCount by 1
\ifnum \DivLevel=0\LikeRef{dv)\DivCount}\relax
   \HCode{<div type="preamble"><p></p>}%
   \gHAdvance\P:level by 1 
\fi
\HAdvance\DivLevel by -1
\HAdvance\DivCount by -1
>>>


The \verb+<p></p>+ is to ensure at least one paragaph in the preamble, e.g.,
between the section heads of
\begin{verbatim}
.................
\section{Distributions}%
\subsection{Definitions}
...............
\end{verbatim}


\<latex etc\><<<
\gHAssign\P:level=0 
>>>













\subsection{End Points}

\<article sections\><<<
\Configure{endsection}
     {likesection,chapter,likechapter,appendix,part,likepart}
\Configure{endlikesection}
     {section,chapter,likechapter,appendix,part,likepart}
\Configure{endsubsection}
   {likesubsection,section,likesection,chapter,%
      likechapter,appendix,part,likpart}
\Configure{endlikesubsection}
   {subsection,section,likesection,chapter,%
      likechapter,appendix,part,likpart}
\Configure{endsubsubsection}
   {likesubsubsection,subsection,likesubsection,section,%
      likesection,chapter,likechapter,appendix,part,likpart}
\Configure{endlikesubsubsection}
   {subsubsection,subsection,likesubsection,section,%
      likesection,chapter,likechapter,appendix,part,likpart}
\Configure{endparagraph}
   {likeparagraph,subsubsection,likesubsubsection,subsection,%
    likesubsection,section,%
    likesection,chapter,likechapter,appendix,part,likpart}
\Configure{endlikeparagraph}
   {paragraph,subsubsection,likesubsubsection,subsection,%
    likesubsection,section,%
    likesection,chapter,likechapter,appendix,part,likpart}
\Configure{endsubparagraph}
   {likesubparagraph,likeparagraph,subsubsection,likesubsubsection,%
    subsection,likesubsection,section,%
    likesection,chapter,likechapter,appendix,part,likpart}
\Configure{endlikesubparagraph}
   {subparagraph,likeparagraph,subsubsection,likesubsubsection,%
    subsection,likesubsection,section,%
    likesection,chapter,likechapter,appendix,part,likpart}
\ifx \part\:UnDef \else
   \Configure{endpart}{likepart}
   \Configure{endlikepart}{endpart}
\fi
>>>





%%%%%%%%%%%%%%%
\section{Lists}
%%%%%%%%%%%%%%%


\<latex lists\><<<
\ConfigureList{trivlist}%
   {\EndP\HCode{<list type="gloss">}\let\end:itm=\empty}
   {\EndP\HCode{</item></list>}\ShowPar}
   {\end:itm\def\end:itm{\EndP\Tg</item>}\HCode{<label>}\bgroup \bf}
   {\egroup\HCode{</label><item\Hnewline>}}
\ConfigureList{list}%
   {\EndP\HCode{<list type="gloss">}\let\end:itm=\empty}
   {\EndP\HCode{</item></list>}\ShowPar}
   {\end:itm\def\end:itm{\EndP\Tg</item>}\HCode{<label>}\bgroup \bf}
   {\egroup\HCode{</label><item\Hnewline>}}
\ConfigureList{itemize}%
   {\EndP\HCode{<list type="bulleted">}\let\end:itm=\empty}
   {\EndP\HCode{</item></list>}\ShowPar}
   {\end:itm\def\end:itm{\EndP\Tg</item>}\DeleteMark}
   {\HCode{<item>}}
\ConfigureList{enumerate}%
   {\EndP\HCode{<list type="ordered"\Hnewline>}\let\end:itm=\empty}
   {\EndP\HCode{</item></list>}\ShowPar}
   {\end:itm\def\end:itm{\EndP\Tg</item>}\DeleteMark}
   {\HCode{<item>}\AnchorLabel}
>>>


\<article,report,book lists\><<<
\ConfigureList{thebibliography}%
   {\IgnorePar\EndP
    \HCode{<div rend="thebibliography"><listBibl>}\let\en:bib=\empty}
   {\en:bib\HCode{</listBibl></div>}}
   {\en:bib\def\en:bib{\HCode{</bibl>}}%
      \HCode{<!--l. \the\inputlineno--><!--}|<par within bib|>} 
   {\HCode{--><bibl>}}
>>>


\<par within bib\><<<
\Configure{HtmlPar}
   {\HCode{<!--l. \the\inputlineno--><gap desc="bib p"/>}} 
   {\HCode{<!--l. \the\inputlineno--><gap desc="bib p"/>}}    
   {} {}
>>>





\<latex lists\><<<
\Configure{cite}{}{}{\Link}{\EndLink}
\Configure{bibitem}{\Link}{\EndLink}
>>>



%%%%%%%%%%%%%%%%
\section{Tables}
%%%%%%%%%%%%%%%%


\<latex tables\><<<
\Configure{tabular}
   {|<start count rows,cols|>|<set row count|>%
    \HCode{<table\Hnewline |<insert row count|> >}}
   {\HCode{</table>}%
     |<end count rows,cols|>}
   {|<modify rows count|>\HCode{<row role="data">}}
   {\HCode{</row>}}
   {|<modify cols count|>%
    \HCode{<cell\Hnewline cols="\HMultispan" rows="1" role="data">}}
   {\HCode{</cell>}}
\HAssign\tabularN=0
>>>

\<start count rows,cols\><<<
\gHAdvance\tabularN by 1  \PushStack\tabular:N\tabularN
\let\prevR:max=\R:max  \let\prevC:max=\C:max  \gHAssign\C:max=0
>>>

\<end count rows,cols\><<<
\PopStack\tabular:N\i:var |<toks Clr|>%
\Tag{rc-\i:var}{rend="|<get Clr|>" rows="\R:max" cols="\C:max"}
\global\let\R:max=\prevR:max  \global\let\C:max=\prevC:max 
>>>



\<toks Clr\><<<
\expandafter\tmp:toks\expandafter{\Clr}%
>>>

\<get Clr\><<<
\the\tmp:toks
>>>





\<modify cols count\><<<
\HAssign\i:var=\HCol \HAdvance\i:var by \HMultispan
\HAdvance\i:var by -1  \ifnum \C:max<\i:var \gHAssign\C:max=\i:var \fi
>>>

\<modify rows count\><<<
\gHAssign\R:max=\HRow
>>>

\<set row count\><<<
\expandafter\ifx \csname  cw:rc-\tabularN\endcsname \relax 
        \tmp:toks{ }%
\else
   \expandafter\expandafter\expandafter\tmp:toks
      \expandafter\expandafter\expandafter{\csname
                                              cw:rc-\tabularN\endcsname}%
\fi
>>>

\<insert row count\><<<
\the\tmp:toks
>>>

\<latex tables\><<<
\Configure{hline}{}
\Configure{//[]}{}
>>>


\<obey lines\><<<
\Configure{verb}{}{}
\Configure{obeylines}
   {\ifvmode \EndP
       \HCode{<div rend="obeylines-v">}%
       \gdef\ob:ln{div}%
    \else \HCode{<emph rend="obeylines-h">}%
       \gdef\ob:ln{emph}%
    \fi
   }
   {\ht:special{t4ht=</\ob:ln>}\ShowPar} {\hbox{\HCode{<br />}}}
>>>


\<ams art,proc,book\><<<
\ifx \deferred@thm@head\:UnDef\else   
  \Configure{newtheorem}
     {\par\HCode{<title rend="newtheorem-ams">}\IgnorePar\par}
     {\HCode{</title>}}
     {}
\fi
>>>




%%%%%%%%%%%%%%%%%
\section{Figures}
%%%%%%%%%%%%%%%%%

\<article,report,book figures\><<<
\ConfigureEnv{figure}
   {\ifvmode \IgnorePar\fi\EndP \HCode{<note rend="figure">}}
   {\ifvmode \IgnorePar\fi\EndP \HCode{</note>}}
   {}{}
\Configure{caption}
   {}{}
   {\HCode{<title>}}   {\HCode{</title>}}
>>>


%%%%%%%%%%%%%%%%%
\section{Accents}
%%%%%%%%%%%%%%%%%


\<latex env\><<<
\Configure{centerline}
    {\ifvmode \IgnorePar \fi \EndP\HCode{<p \Hnewline
      rend="centerline">}}
    {\EndP\HCode{</p>}\ShowPar}
\Configure{leftline}
    {\ifvmode \IgnorePar \fi \EndP\HCode{<p \Hnewline
      rend="leftline">}}
    {\EndP\HCode{</p>}\ShowPar}
\Configure{rightline}
    {\ifvmode \IgnorePar \fi \EndP\HCode{<p \Hnewline
      rend="rightline">}}
    {\EndP\HCode{</p>}\ShowPar}
>>>


%%%%%%%%%%%%%%%%%
\section{Accents}
%%%%%%%%%%%%%%%%%

\<latex,plain accents\><<<
\:CheckOption{new-accents}     \if:Option
   |<new accents|>
\else
   |<old accents|>
\fi
\let\^^_=\v
>>>


\<latex accents\><<<
\let\@acci=\' \let\@accii=\` \let\@acciii=\=       
>>>

\<new accents\><<<
\:CheckOption{accent-}     \if:Option
  \Configure{HAccent}\acute{AEIOUYaeiouy{}}{\Picture+{}}{\EndPicture}
  \Configure{HAccent}\bar{}{\Picture+{}}{\EndPicture}
  \Configure{HAccent}\breve{}{\Picture+{}}{\EndPicture}
  \Configure{HAccent}\check{}{\Picture+{}}{\EndPicture}
  \Configure{HAccent}\ddot{AEIOUYaeiouy{}}{\Picture+{}}{\EndPicture}
  \Configure{HAccent}\dot{}{\Picture+{}}{\EndPicture}
  \Configure{HAccent}\grave{AEIOUaeiou{}}{\Picture+{}}{\EndPicture}
  \Configure{HAccent}\hat{AEIOUaeiou{}}{\Picture+{}}{\EndPicture}
  \Configure{HAccent}\tilde{AOaoNn{}}{\Picture+{}}{\EndPicture}
  \Configure{HAccent}\vec{}{\Picture+{}}{\EndPicture}
  \Configure{HAccent}\widehat{}{\Picture+{}}{\EndPicture}
  \Configure{HAccent}\widetilde{}{\Picture+{}}{\EndPicture}
\fi
\:CheckOption{mathaccent-}     \if:Option
  \Configure{HAccent}\"{AEIOUYaeiouy{}}{\Picture+{}}{\EndPicture}
  \Configure{HAccent}\'{AEIOUYaeiouy{}}{\Picture+{}}{\EndPicture}
  \Configure{HAccent}\.{}{\Picture+{}}{\EndPicture}
  \Configure{HAccent}\={}{\Picture+{}}{\EndPicture}
  \Configure{HAccent}\H{}{\Picture+{}}{\EndPicture}
  \Configure{HAccent}\^{AEIOUaeiou{}}{\Picture+{}}{\EndPicture}
  \Configure{HAccent}\`{AEIOUaeiou{}}{\Picture+{}}{\EndPicture}
  \Configure{HAccent}\b{}{\Picture+{}}{\EndPicture}
  \Configure{HAccent}\c{Cc{}}{\Picture+{}}{\EndPicture}
  \Configure{HAccent}\d{}{\Picture+{}}{\EndPicture}
  \Configure{HAccent}\t{}{\Picture+{}}{\EndPicture}
  \Configure{HAccent}\u{}{\Picture+{}}{\EndPicture}
  \Configure{HAccent}\v{}{\Picture+{}}{\EndPicture}
  \Configure{HAccent}\~{AOaoNn{}}{\Picture+{}}{\EndPicture}
\fi
>>>


\<new accents\><<<
\Configure{accent}{*}
   {<!--tex4ht:accent\Hnewline font="}{" char="}{" type="}{"-->}
   {<!--tex4ht:end accent-->}
\Configure{mathaccent}{*}
   {<!--tex4ht:mathaccent\Hnewline font="}{" char="}{" type="}{"-->}
   {<!--tex4ht:end mathaccent-->}
\Configure{accented}{*}
   {<!--tex4ht:accented\Hnewline font="}{" char="}{" type="}{"-->}
   {<!--tex4ht:end accented-->}
\Configure{accenting}{*}
   {<!--tex4ht:accenting\Hnewline-->}
   {<!--tex4ht:end accenting-->}
>>>




\<old accents\><<<
\Configure{accent}\`\grave{A{A}E{E}I{I}O{O}U{U}%
                   a{a}e{e}i{i}\i{i}o{o}u{u}{}{}}
   {\a:accents{grave}{#1}}   {\b:accents{grave}{#1}{#2}}
\Configure{accent}\'\acute{A{A}E{E}I{I}O{O}U{U}Y%
           {Y}a{a}e{e}i{i}\i{i}o{o}u{u}y{y}{}{}}
   {\a:accents{acute}{#1}}   {\b:accents{acute}{#1}{#2}}
\Configure{accent}\^\hat{A{A}E{E}I{I}O{O}U{U}a{a}%
                      e{e}i{i}\i{i}o{o}u{u}{}{}}
   {\a:accents{circ}{#1}}   {\b:accents{hat}{#1}{#2}}
\Configure{accent}\~\tilde{A{A}O{O}a{a}o{o}N{N}n{n}{}{}}
   {\a:accents{tilde}{#1}}   {\b:accents{tilde}{#1}{#2}}
\Configure{accent}\"\ddot{A{A}E{E}I{I}O{O}U{U}Y%
           {Y}a{a}e{e}i{i}\i{i}o{o}u{u}y{y}{}{34}}
   {\a:accents{uml}{#1}}     {\b:accents{uml}{#1}{#2}}
>>>





The following are also placed under accents configuration.

\<old accents\><<<
\Configure{accent}\c\c{C{C}c{c}{}{}}
   {\a:accents{cedil}{#1}}     {\b:accents{cedil}{#1}{#2}}
\Configure{accent}\t\t{{}{}}
   {\a:accents{udot}{#1}}     {\b:accents{udot}{#1}{#2}}
\Configure{accent}\H\H{{}{}} {}{\b:accents{Huml}{#1}{#2}}
>>>

The following originally have been defined to be parameter-less.



\<old accents\><<<
\Configure{accent}\.\dot{{}{}}  {}{\b:accents{dot}{#1}{#2}}
\Configure{accent}\u\breve{{}{}}{}{\b:accents{breve}{#1}{#2}}
\Configure{accent}\vec\vec{{}{}}{}{\b:accents{vec}{#1}{#2}}
\Configure{accent}\v\check{{}{}}{}{\b:accents{check}{#1}{#2}} 
\Configure{accent}\=\bar{{}{}}  {}{\b:accents{bar}{#1}{#2}}
>>>


%  \= macron

\<old accents\><<<
\Configure{accent}\widetilde\widetilde{{}{}} 
   {}{\b:accents{widetilde}{#1}{#2}}
\Configure{accent}\widehat\widehat{{}{}} 
   {}{\b:accents{widehat}{#1}{#2}}
>>>


\verb'\vec', \verb'\widetilde', and \verb'\widehat' are for math mode.
\verb'\b', \verb'\c', \verb'\d', \verb'\t', and \verb'\H' are for text mode.




\<configure tei-math plain\><<<    
|<shared latex,plain accents|>
>>>

\<configure tei-math latex\><<<    
|<shared latex,plain accents|>
>>>

\<shared latex,plain accents\><<<
\:CheckOption{new-accents}     \if:Option \else
   \Configure{accents}
     {\ifmmode \expandafter\mathord\fi
      {|<temp hcode accents|>%
      \ht:special{t4ht@[}#2\ht:special{t4ht@]}}}
     {\ifmmode \expandafter\mathord\fi
       {\HCode{<term type='accent#1'>}#3\HCode{</term>}}}
\fi
>>>



\<temp hcode accents\><<<
\HCode{&\#x\expandafter \ifx\csname U#2#1\endcsname\relax
                 #2#1\else \csname U#2#1\endcsname\fi;}%
>>>


%\HCode{&#2#1;}%

%%%%%%%%%%%%%%%%%%%%
\section{Math}
%%%%%%%%%%%%%%%%%%%%




\<latex math\><<<
\Configure{equation}
   {}    {\HCode{<title>}}   {\HCode{</title>}}
\ConfigureEnv{equation}
   {\IgnorePar\EndP
    \HCode{<div rend="equation"><p rend="equation">}}
   {\HCode{</p></div>}\ShowPar\par}
   {}{}
>>>

\<configure tei-mml tex4ht\><<<
\Configure{$$}
   {\Configure{@math}{display="block"}\DviMath}
   {\EndDviMath{\csname HCondtrue\endcsname\noindent}}{}
\Configure{DviMath}
    {\mathmltrue \NoFonts\IgnoreRule\HCode{<!--l.
       \the\inputlineno--><formula notation="mathml"\Hnewline
         \expandafter\:rend \a:@math==//><math 
           xmlns="http://www.w3.org/1998/Math/MathML"\Hnewline
           \expandafter\m:rend\a:@math==//>}\a:math
       |<sv dvimath par|>\IgnorePar}
    {\HCode{</math></formula>}\EndIgnoreRule\EndNoFonts
     |<recall dvimath par|>\mathmlfalse}
\def\:rend#1=#2=#3//{\if !#1!\else rend=#2\fi}
\def\m:rend#1=#2=#3//{\if !#1!\else display=#2\fi}
>>>

\<recall dvimath par\><<<
\sv:ignore
>>>

\<sv dvimath par\><<<
\edef\sv:ignore{\if:nopar  
    \noexpand\IgnorePar\else \noexpand\ShowPar\fi}%
>>>



\<configure tei-mml amsmath\><<<  
\ConfigureEnv{gather}
   {\ShowPar\par}   {\ShowPar\par}
   {}{}
>>>






\<configure tei-mml amsmath\><<<
\ConfigureEnv{align*} 
   {\ifmmode\else \IgnorePar\EndP \HCode{<note rend="align-star">}\fi}
   {\ifmmode\else \HCode{</note>}\par\ShowPar\fi} {}{}    
\ConfigureEnv{alignat*}
   {\ifmmode\else \IgnorePar\EndP \HCode{<note rend="alignat-star">}\fi}
   {\ifmmode\else \HCode{</note>}\par\ShowPar\fi}{}{}
\ConfigureEnv{alignat}
   {\ifmmode\else \IgnorePar\EndP \HCode{<note rend="alignat">}\fi}
   {\ifmmode\else \HCode{</note>}\par\ShowPar\fi}{}{}   
\ConfigureEnv{align}
   {\ifmmode\else \IgnorePar\EndP \HCode{<note rend="align">}\fi}
   {\ifmmode\else \HCode{</note>}\par\ShowPar\fi}{}{}   
\ConfigureEnv{eqxample}
   {\ifmmode\else \IgnorePar\EndP \HCode{<note rend="eqxample">}\fi}
   {\ifmmode\else \HCode{</note>}\par\ShowPar\fi}{}{}
\ConfigureEnv{flalign*}
   {\ifmmode\else \IgnorePar\EndP \HCode{<note rend="flalign*">}\fi}
   {\ifmmode\else \HCode{</note>}\par\ShowPar\fi}{}{}
\ConfigureEnv{flalign}
   {\ifmmode\else \IgnorePar\EndP \HCode{<note rend="flalign">}\fi}
   {\ifmmode\else \HCode{</note>}\par\ShowPar\fi}{}{}
\ConfigureEnv{gather}
   {\ifmmode\else \IgnorePar\EndP \HCode{<note rend="gather">}\fi}
   {\ifmmode\else \HCode{</note>}\par\ShowPar\fi}{}{}
\ConfigureEnv{gather*}
   {\ifmmode\else \IgnorePar\EndP \HCode{<note rend="gather-star">}\fi}
   {\ifmmode\else \HCode{</note>}\par\ShowPar\fi}{}{}
\ConfigureEnv{gathered}
   {\ifmmode\else \IgnorePar\EndP \HCode{<note rend="gathered">}\fi}
   {\ifmmode\else \HCode{</note>}\par\ShowPar\fi}{}{}
\ConfigureEnv{matrix}
   {\ifmmode\else \IgnorePar\EndP \HCode{<note rend="matrix">}\fi}
   {\ifmmode\else \HCode{</note>}\par\ShowPar\fi}{}{}
\ConfigureEnv{xalignat*}
   {\ifmmode\else \IgnorePar\EndP \HCode{<note rend="xalignat-star">}\fi}
   {\ifmmode\else \HCode{</note>}\par\ShowPar\fi}{}{}
\ConfigureEnv{xalignat}
   {\ifmmode\else \IgnorePar\EndP \HCode{<note rend="xalignat">}\fi}
   {\ifmmode\else \HCode{</note>}\par\ShowPar\fi}{}{}
\ConfigureEnv{xxalignat}
   {\ifmmode\else \IgnorePar\EndP \HCode{<note rend="xxalignat">}\fi}
   {\ifmmode\else \HCode{</note>}\par\ShowPar\fi}{}{}
>>>





%%%%%%%%%%%%%%%%%%%%
\section{Loose Ends}
%%%%%%%%%%%%%%%%%%%%





\<latex environments\><<<
\ConfigureEnv{verbatim}
  {\ifinner\else \vfill\break\fi
    \IgnorePar\EndP\HCode{<p rend="verbatim">}\EndNoFonts}
  {\NoFonts \ht:special{t4ht=</p>}\ShowPar\par}
  {} {}
\ConfigureEnv{verbatim*}
  {\ifinner\else \vfill\break\fi
    \IgnorePar\EndP\HCode{<p rend="verbatim-star">}\EndNoFonts}
  {\NoFonts \HCode{</note><p>}\ShowPar\par}
  {} {}
\Configure{verbatim}{\:nbsp}{\a:sp}
>>>


\<latex environments\><<<
\ConfigureEnv{flushleft}
   {\IgnorePar\EndP\HCode{<div type="flushleft">}%
    \gHAdvance\P:level by 1 }
   {\IgnorePar\EndP\gHAdvance\P:level by -1 \HCode{</div>}}
   {}{}
\ConfigureEnv{flushright}
   {\IgnorePar\EndP\HCode{<div type="flushright">}%
    \gHAdvance\P:level by 1 }
   {\IgnorePar\EndP\gHAdvance\P:level by -1 \HCode{</div>}}
   {}{}
\ConfigureList{flushleft}
  {}{}
  {} {\par\ShowPar}
\ConfigureList{flushright}
  {}{}
  {} {\par\ShowPar}
>>>


\<article,report\><<<
\ConfigureEnv{abstract}
   {\IgnorePar\EndP\HCode{<div type="abstract">}%
    \gHAdvance\P:level by 1 }
   {\IgnorePar\EndP\gHAdvance\P:level by -1 \HCode{</div>}}
   {\HCode{<head>}}
   {\HCode{</head>}}
>>>


\<tex4ht paragraphs\><<<
\Configure{HtmlPar}
   {\EndP\HCode{<!--l. \the\inputlineno-->%
      \ifnum\P:level=0 <div type="p">\fi 
      <p>}\gHAdvance\P:level by 1 } 
   {\EndP\HCode{<!--l. \the\inputlineno-->%
      \ifnum\P:level=0 <div type="p">\fi 
      <p>}\gHAdvance\P:level by 1 }    
   {\gHAdvance\P:level by -1 \HCode{</p>\ifnum\P:level=0 </div>\fi }}
   {\gHAdvance\P:level by -1 \HCode{</p>\ifnum\P:level=0 </div>\fi }}    
>>>






\<latex etc\><<<
\Configure{newline}{\HCode{<!--br/-->}}             
>>>

\<configure tei-mml latex\><<< 
\Configure{newline}
  {\ifmtext \HCode{</mtext><mtext class="newline">}%
   \else \HCode{<!--br role="newline" /-->}\fi}
>>>



\<latex etc\><<<
\ConfigureEnv{minipage}
   {\IgnorePar\EndP\HCode{<note rend="minipage">}}
   {\IgnorePar\EndP\HCode{</note>}\ShowPar}
   {}{}
\Configure{newtheorem}
   {\IgnorePar\EndP\HCode{<note rend="newtheorem"><title>}}
   {\HCode{</title>}\ShowPar\par}
   {\IgnorePar\EndP\HCode{</note>}\ShowPar\par}
>>>




\<ams art,proc,book\><<<
\ifx \deferred@thm@head\:UnDef\else   
   |<shared ams thm/cls|>
\fi
>>>

\<configure tei amsthm\><<<
|<shared ams thm/cls|>
>>>

\<shared ams thm/cls\><<<
\Configure{proof}
   {\IgnorePar\EndP\HCode{<note rend="proof"><title>}}
   {\HCode{</title>}\ShowPar\par}
   {\IgnorePar\EndP\HCode{</note>}\ShowPar\par}
>>>




\<tex4ht cross-linking\><<<
\Configure{Link}{ref}{ target=}{ id=}{\empty}
>>>

\<configure tei url\><<<
\Decl{<!NOTATION URL SYSTEM "" >}
\Configure{url}{%
  \urlLink[xr-ent\ent:cnt]{}{}#1\EndurlLink
    \Decl{<!ENTITY xr-ent\ent:cnt\space  SYSTEM "#1" NDATA URL>}%
    \gHAdvance\ent:cnt by 1
}
\LinkCommand\urlLink{xref,doc}
>>>

 xr-ent has to be a notation-type entity to fit with TEI's demand
 of `\verb+doc ENTITY #IMPLIED+'. For instance,

\begin{verbatim}
<!NOTATION URL SYSTEM "">
<!ENTITY xr-ent0 SYSTEM
   "http://wwwinfo.cern.ch/asdoc/geant/geantall.html" NDATA URL>
\end{verbatim}

% \Decl{<!ATTLIST figure width CDATA \#IMPLIED>}
% \Decl{<!ATTLIST figure height CDATA \#IMPLIED>}


\<configure tei-mmltei graphics\><<<
\def\:tempc#1{%
   \Configure{graphics*}
      {#1}
      {\HCode{<figure entity="fg-ent\ent:cnt" 
                  width="\the\Gin@req@width" 
                 height="\the\Gin@req@height" />}%
       \Decl{<!ENTITY fg-ent\ent:cnt\space  SYSTEM 
             "\csname Gin@base\endcsname" NDATA #1>}%
       \gHAdvance\ent:cnt by 1
      }}
|<graphics types|>
>>>


\<configure tei graphics\><<<
\def\:tempc#1{%
   \Decl{<!NOTATION #1 SYSTEM "" >}
   \Configure{graphics*}
      {#1}
      {\HCode{<figure entity="fg-ent\ent:cnt" />%
              <!--width="\the\Gin@req@width" 
                 height="\the\Gin@req@height"-->}%
       \Decl{<!ENTITY fg-ent\ent:cnt\space  SYSTEM 
             "\csname Gin@base\endcsname.#1" NDATA #1>}%
       \gHAdvance\ent:cnt by 1
      }}
|<graphics types|>
>>>

\<graphics types\><<<
\:tempc{eps}
\:tempc{jpg}
\:tempc{pdf}
\:tempc{png}
\:tempc{tif}
\:tempc{gif}
>>>


\<configure tei fontmath\><<<
\Configure{mathit}{\ifmathml\else \HCode{<hi rend="it">}\fi}
                  {\ifmathml\else \HCode{</hi>}\fi}
\Configure{mathbf}{\ifmathml\else \HCode{<hi rend="bf">}\fi}
                  {\ifmathml\else \HCode{</hi>}\fi}
\Configure{mathtt}{\ifmathml\else \HCode{<hi rend="tt">}\fi}
                  {\ifmathml\else \HCode{</hi>}\fi}
\Configure{mathsf}{\ifmathml\else \HCode{<hi rend="sf">}\fi}
                  {\ifmathml\else \HCode{</hi>}\fi}
\Configure{mathrm}{\ifmathml\else \HCode{<hi rend="rm">}\fi}
                  {\ifmathml\else \HCode{</hi>}\fi}
>>>

\<configure tei fontmath\><<<
\Configure{textbf}{\ifmathml\else
                     |<par within font|>\HCode{<hi rend="bf">}\fi}
                  {\ifmathml\else 
                    \HCode{</hi>}|<end par within font|>\fi}
\Configure{textit}{\ifmathml\else
                     |<par within font|>\HCode{<hi rend="it">}\fi}
                  {\ifmathml\else 
                    \HCode{</hi>}|<end par within font|>\fi}
\Configure{textrm}{\ifmathml\else
                     |<par within font|>\HCode{<hi rend="rm">}\fi}
                  {\ifmathml\else 
                    \HCode{</hi>}|<end par within font|>\fi}
\Configure{textup}{\ifmathml\else
                     |<par within font|>\HCode{<hi rend="up">}\fi}
                  {\ifmathml\else 
                    \HCode{</hi>}|<end par within font|>\fi}
\Configure{textsc}{\ifmathml\else
                     |<par within font|>\HCode{<hi rend="sc">}\fi}
                  {\ifmathml\else 
                    \HCode{</hi>}|<end par within font|>\fi}
\Configure{textsf}{\ifmathml\else
                     |<par within font|>\HCode{<hi rend="sf">}\fi}
                  {\ifmathml\else 
                    \HCode{</hi>}|<end par within font|>\fi}
\Configure{textsl}{\ifmathml\else
                     |<par within font|>\HCode{<hi rend="sl">}\fi}
                  {\ifmathml\else 
                    \HCode{</hi>}|<end par within font|>\fi}
\Configure{texttt}{\ifmathml\else
                     |<par within font|>\HCode{<hi rend="tt">}\fi}
                  {\ifmathml\else 
                    \HCode{</hi>}|<end par within font|>\fi}
\Configure{emph}{\ifmathml\else
                     |<par within font|>\HCode{<emph>}\fi}
                  {\ifmathml\else 
                    \HCode{</emph>}|<end par within font|>\fi}
>>>



\<par within font\><<<
\leavevmode\SaveEndP\bgroup 
\Configure{HtmlPar}
   {\HCode{<!--br role="newline" /-->}}
   {\HCode{<!--br role="newline" /-->}}
   {}{}%
>>>                    

\<end par within font\><<<
\egroup \RecallEndP
>>>

\endinput
