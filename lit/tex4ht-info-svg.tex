% $Id$
% compile 3 times: latex tex4ht-info-svg   
%             or htlatex tex4ht-info-svg "html,sections+"
%            or ht latex tex4ht-info
%
% Copyright 2009-2017 TeX Users Group
% Copyright 2000-2009 Eitan M. Gurari
% Released under LPPL 1.3c+.
% See tex4ht-cpright.tex for license text.

\ifx \HTML\UnDef
   \def\HTML{infosvg}
   \def\CONFIG{\jobname}
   \def\MAKETITLE{\author{Eitan M. Gurari}}
   \def\next{\input mktex4ht.4ht   \endinput}
   \expandafter\next
\fi

% $Id$
% Common TeX definitions used only in the *-info.tex literate sources.
% Not installed.
% 
% Copyright 2009-2017 TeX Users Group
% Copyright 1996-2009 Eitan M. Gurari
% Released under LPPL 1.3c+.
% See tex4ht-cpright.tex for license text.

\expandafter\ifx \csname YES\HTML\endcsname\relax
    % begin comment. 21/07/2016 (dg)
    %   on first run \infoIVht expects \ConfigureHinput
    %   ( \def\infoIVht#1\ConfigureHinput{..} )
    %   so we feed it with "\ConfigureHinput" (no expansion here, merely
    %   a delimiter); the rest is slurped until the "//".
    % end
    \def\CleanComment{[0]\ConfigureHinput\id:gobble}
\else
    \let\saveCd=\<
    \def\<{\edef\FIRST{\the\inputlineno}\let\<\saveCd \saveCd}
    %
    % Eitan's commented-out definition started like this (and doesn't work):
    %\def\CleanComment#1tex4ht-info#2#3#4.#5>#6//{[\eatIV#4%     <--jobname
    %
    % Eitan's active definition started like this:
    %\def\CleanComment#1tex4ht-info#2#3#4.#5>#6//{[#4%     <--jobname
    %
    % However, that didn't work either.  #4 is not the jobname. 
    % The arguments when running htlatex tex4ht-info-mml.tex are these:
    % %#1<-
    %\CleanComment #1tex4ht-info#2#3#4.#5>#6//->
    %[\if ,\ifnum \FIRST =#6 .\else ,0\fi
    %#1<-
    %#2<--
    %#3<-m
    %#4<-ml
    %#5<-html#QPrTx1"\<infomml\
    %#6<-92\ifx \CodeId \:gobbleii \else ...\fi 
    %
    % As a result, when running  mzlatex hello.tex xhtml,info  there was
    % an error on the first line of infomml.4ht, which looked like this:
    % \ifx\infoIVht\UnDeF\def\infoIVht#1//{}\fi\infoIVht[ml0]28...//
    % That "ml" is not a number, so \ifnum fails.  This only happens
    % mzlatex and the info option, not htlatex.  We don't understand.
    %
    % This version avoids the spurious "ml" but mzlatex hello.tex still
    % fails, trying to process the \ConfigureHinput blocks as text.
    % Changing the bracketed number in infomml.4ht to small values seems
    % to make it pass, but can't see how to generate it.  The number
    % after the brackets (#6) changes also.
    % 
    % Since all this is only about the info option with mzlatex,
    % just leaving it failing for now.  Other things to do.
    \def\CleanComment#1tex4ht-info#2#3#4.#5>#6//{[1\empty %
                  \if,\ifnum \FIRST=#6 .\else ,0\fi\fi]#6//}
    \def\eatIV#1#2#3#4{}
\fi

\Comment{

\string\ifx\string\infoIVht\string\UnDeF\string\def\string\infoIVht#1//{}\string\fi\string\infoIVht\CleanComment}{//

}

\def\>>>#1<<<{\bgroup\csname no:catcodes\endcsname0{255}{12}%
   \csname no:catcodes\endcsname{13}{13}{13}% ^^M
   \def\temp##1>>>{\egroup
      \expandafter \def\csname #1\endcsname{##1}}\temp}

% $Id$
% A few common TeX definitions for literate sources.  Not installed in runtime.
% 
% Copyright 2009-2017 TeX Users Group
% Copyright 1996-2009 Eitan M. Gurari
%
% This work may be distributed and/or modified under the
% conditions of the LaTeX Project Public License, either
% version 1.3c of this license or (at your option) any
% later version. The latest version of this license is in
%   http://www.latex-project.org/lppl.txt
% and version 1.3c or later is part of all distributions
% of LaTeX version 2005/12/01 or later.
%
% This work has the LPPL maintenance status "maintained".
%
% The Current Maintainer of this work
% is the TeX4ht Project <http://tug.org/tex4ht>.
% 
% If you modify this program, changing the 
% version identification would be appreciated.

\newcount\tmpcnt  \tmpcnt\time  \divide\tmpcnt  60
\edef\temp{\the\tmpcnt}
\multiply\tmpcnt  -60 \advance\tmpcnt  \time

\edef\version{\the\year-\ifnum \month<10 0\fi
  \the\month-\ifnum \day<10 0\fi\the\day
   -\ifnum \temp<10 0\fi \temp
   :\ifnum \tmpcnt<10 0\fi\the\tmpcnt}

% a fixed-string version that can be enabled for debugging.
%\edef\versionDebug{000-00-00-00:00}
%\let\version\versionDebug

% #1 is the first year for Eitan.  The last year is always 2009.  RIP.
\def\CopyYear.#1.{#1-2009}

% command for write to terminal and the log file
% this version is used in the .4ht files build
% identical command is defined also in tex4ht-sty.tex, 
% it is used in TeX document compilation
\def\writesixteen#1{\immediate\write1616{#1}}

\expandafter\ifx \csname YES\HTML\endcsname\relax
\else
    % \def\CleanComment#1tex4ht-info-svg#2#3#4.#5>#6//{[#4%     <--jobname
    %               \if,\ifnum \FIRST=#6 .\else ,0\fi\fi]#6//}
    % Michal 05/10/2017
    % The following code works in other info files, but not in this one
    % I don't really don't know why
    \def\<{\edef\FIRST{\the\inputlineno}\let\<\saveCd \saveCd}
    \def\CleanComment#1tex4ht-info-svg#2#3#4.#5>#6//{[#4%     <--jobname
                  \if,\ifnum \FIRST=#6 .\else ,0\fi\fi]#6//}
    % \def\eatIV#1#2#3#4{}
\fi






%%%%%%%%%%%%%%%%%%%%%%%%%%%%%%%%%%%%%%%%%%%%%%%%%%%%%%%%%%%%%%%%%%%%%%%%
\chapter{Info SVG}
%%%%%%%%%%%%%%%%%%%%%%%%%%%%%%%%%%%%%%%%%%%%%%%%%%%%%%%%%%%%%%%%%%%%%%%%


\<infosvg\><<<
% infosvg.4ht (|version), generated from |jobname.tex
% Copyright 2009-2017 TeX Users Group
% Copyright |CopyYear.2000. Eitan M. Gurari
%
% This work may be distributed and/or modified under the
% conditions of the LaTeX Project Public License, either
% version 1.3c of this license or (at your option) any
% later version. The latest version of this license is in
%   http://www.latex-project.org/lppl.txt
% and version 1.3c or later is part of all distributions
% of LaTeX version 2005/12/01 or later.
%
% This work has the LPPL maintenance status "maintained".
%
% The Current Maintainer of this work
% is the TeX4ht Project <http://tug.org/tex4ht>.
% 
% If you modify this program, changing the 
% version identification would be appreciated.
\immediate\write-1{version |version}
{          \catcode`\@=0 \catcode`\\=11 @relax
  @gdef@infoIVht[#1]#2//{%
    @ifnum #1>1
      @def@infoIVht[##1]##2//{%
        @ifnum ##1>1 @ifnum ##1<#1
           @bgroup 
             @no:catcodes0{255}{11}%
             @no:catcodes{91}{91}{12}% [
             @no:catcodes{47}{47}{12}% /
             @newlinechar13 %   
             @long@def@infoIVht####1\ifx\infoIVht####2infoIVht[####3//{%
               @def@infoIVht{******************************************}%
               @immediate@write-1{@infoIVht}%
               @immediate@write-1{****** @csname :Hin@endcsname.4ht}%
               @immediate@write-1{@infoIVht}%
               @bgroup
                @def@infoIVht{~~~~~~~~~~~~~~~~~~~~~~~~~~~~~~~~~~~~~~~~~*}%
                @let~=@space   @immediate@write-1{@infoIVht}%
               @egroup   
               @immediate@write-1{####1}%
               @bgroup
                @def@infoIVht{~~~~~~~~~~~~~~~~~~~~~~~~~~~~~~~~~~~~~~~~~*}%
                @let~=@space   @immediate@write-1{@infoIVht}%
               @egroup
               @immediate@write-1{@infoIVht}%
             @egroup}%
           @expandafter@expandafter@expandafter@infoIVht
     @fi@fi }%
  @fi }
}
>>>


\chapter{Options}

\section{tex4ht}


\<configure infosvg tex4ht\><<<

The output or pictures in SVG format can be requested using following command
line options:

svg
---

Include pictures as external images. Existing configuration for SVG images is 
used. This is preffered way in modern documents.

svg-object
----------

Include pictures using <object> elements.

svg-
----

This configuration is simillar to the previous one

svg-inline
----------

Include the pictures directly to the XML document.

>>>



\chapter{The Code}

\section{tex4ht}

\<configure infosvg tex4ht\><<<
SVG
---

 Configured through IMG

  Example:
    \Configure{IMG}
      {\special{t4ht=<object type="image/svg+xml"  data="}}
      {\special{t4ht=" name="}}
      {" }
      {\special{t4ht=" }}
      {\special{t4ht=></object>}}
>>>




 
 
\endinput


