% $Id$
% compile 3 times: latex tex4ht-jats
%           or   xhlatex tex4ht-jats "html,3,sections+"
%
% Copyright 2009-2024 TeX Users Group
% Copyright 2000-2009 Eitan M. Gurari
% Released under LPPL 1.3c+.
% See tex4ht-cpright.tex for license text.

%%%%%%%%%%%%%%%%%% load style files %%%%%%%%%%%%%%%%%%%%%%%%%%

\ifx \HTML\UnDef
   \def\HTML{jats}                   
   \def\CONFIG{\jobname}
   \def\MAKETITLE{\author{Eitan M. Gurari}}         
   \def\next{\input mktex4ht.4ht  \endinput}
   \expandafter\next
\fi

% $Id$
% A few common TeX definitions for literate sources.  Not installed in runtime.
% 
% Copyright 2009-2017 TeX Users Group
% Copyright 1996-2009 Eitan M. Gurari
%
% This work may be distributed and/or modified under the
% conditions of the LaTeX Project Public License, either
% version 1.3c of this license or (at your option) any
% later version. The latest version of this license is in
%   http://www.latex-project.org/lppl.txt
% and version 1.3c or later is part of all distributions
% of LaTeX version 2005/12/01 or later.
%
% This work has the LPPL maintenance status "maintained".
%
% The Current Maintainer of this work
% is the TeX4ht Project <http://tug.org/tex4ht>.
% 
% If you modify this program, changing the 
% version identification would be appreciated.

\newcount\tmpcnt  \tmpcnt\time  \divide\tmpcnt  60
\edef\temp{\the\tmpcnt}
\multiply\tmpcnt  -60 \advance\tmpcnt  \time

\edef\version{\the\year-\ifnum \month<10 0\fi
  \the\month-\ifnum \day<10 0\fi\the\day
   -\ifnum \temp<10 0\fi \temp
   :\ifnum \tmpcnt<10 0\fi\the\tmpcnt}

% a fixed-string version that can be enabled for debugging.
%\edef\versionDebug{000-00-00-00:00}
%\let\version\versionDebug

% #1 is the first year for Eitan.  The last year is always 2009.  RIP.
\def\CopyYear.#1.{#1-2009}

% command for write to terminal and the log file
% this version is used in the .4ht files build
% identical command is defined also in tex4ht-sty.tex, 
% it is used in TeX document compilation
\def\writesixteen#1{\immediate\write1616{#1}}

\<TeX4ht copyright\><<<
%
% This work may be distributed and/or modified under the
% conditions of the LaTeX Project Public License, either
% version 1.3c of this license or (at your option) any
% later version. The latest version of this license is in
%   http://www.latex-project.org/lppl.txt
% and version 1.3c or later is part of all distributions
% of LaTeX version 2005/12/01 or later.
%
% This work has the LPPL maintenance status "maintained".
%
% The Current Maintainer of this work
% is the TeX4ht Project <http://tug.org/tex4ht>.
% 
% If you modify this program, changing the 
% version identification would be appreciated.
>>>


%%%%%%%%%%%%%%%%%%%%%%%%%%%%%%%%%%%%%%%%%%%%%%%%%%%%%%%%%%%%%%%%%%%%%%%%
\chapter{Preamble}
%%%%%%%%%%%%%%%%%%%%%%%%%%%%%%%%%%%%%%%%%%%%%%%%%%%%%%%%%%%%%%%%%%%%%%%%

\<jats\><<<
% jats.4ht (|version), generated from |jobname.tex
% Copyright 2022-2024 TeX Users Group
|<TeX4ht copywrite|>
>>>

  
%%%%%%%%%%%%%%%%%%%%%%%%%%%%%%%%%%%%%%%%%%%%%%%%%%%%%%%%%%%%%%%%%%%%%%%%
\chapter{Basic information}
%%%%%%%%%%%%%%%%%%%%%%%%%%%%%%%%%%%%%%%%%%%%%%%%%%%%%%%%%%%%%%%%%%%%%%%%

JATS support in TeX4ht is based on HTML and MathML configurations. 
Unsupported elements will be converted using make4ht DOM filters. 
Filters will be used also for the document structure, as JATS wants some
elements in the back (bibliography, etc.)

Only basic structures will be configured by hand in this configuration 
file.

We support the Journal Archiving and Interchange Tag Library NISO JATS 
Version 1.3. Documentation for tags is available at:

\Link{https://jats.nlm.nih.gov/archiving/tag-library/1.3/chapter/getting-started.html}{}
https://jats.nlm.nih.gov/archiving/tag-library/1.3/chapter/getting-started.html
\EndLink




%%%%%%%%%%%%%%%%%%%%%%%%%%%%%%%%%%%%%%%%%%%%%%%%%%%%%%%%%%%%%%%%%%%%%%%%
\chapter{Package configurations}
%%%%%%%%%%%%%%%%%%%%%%%%%%%%%%%%%%%%%%%%%%%%%%%%%%%%%%%%%%%%%%%%%%%%%%%%

\section{Basic system}

\<configure jats tex4ht\><<<
\special{t4ht.xml}
\Configure{ext}{xml}
|<xml structure|>
|<document structure|>
|<xml:lang declarations|>
|<basic hyperlinks|>
|<basic pictures|>
|<basic fonts|>
|<basic jats mathml|>
>>>

\<configure jats latex\><<<
|<latex fonts|>
% We must override MathML's version of basic font commands.
\AtBeginDocument{%
|<mathml fonts|>
}
>>>



%%%%%%%%%%%%%%%%%%%%%%%%%%%%%%%%%%%
\section{Classes}
%%%%%%%%%%%%%%%%%%%%%%%%%%%%%%%%%%%


\<configure jats article\><<<
|<article,report,book|>
|<article,report|>
>>>

\<configure jats book\><<<
|<article,report,book|>
|<book,report|>
>>>

\<configure jats report\><<<
|<article,report,book|>
|<article,report|>
|<book,report|>
>>>

Shared configurations for base LaTeX classes.

\<article,report,book\><<<
|<shared latex classes|>
>>>

Configurations that are shared only for article and report

\<article,report\><<<
|<shared article,report|>
>>>

\<book,report\><<<
|<shared book,report|>
>>>

\section{Configurations for Koma Script}

\<configure jats scrartcl\><<<
|<scr article,report,book|>
|<scr article,report|>
>>>

\<configure jats scrbook\><<<
|<scr article,report,book|>
|<scr report,book|>
>>>

\<configure jats scrreprt\><<<
|<scr article,report,book|>
|<scr article,report|>
|<scr report,book|>
>>>

\<scr article,report,book\><<<
|<shared latex classes|>
>>>

\<scr article,report\><<<
|<shared article,report|>
>>>

\<scr report,book\><<<
|<shared book,report|>
>>>

\section{Configurations for AMS classes}

\<configure jats amsart\><<<
|<ams art,proc,book|>
|<ams art,proc|>
>>>

\<configure jats amsproc\><<<
|<ams art,proc,book|>
|<ams proc,book|>
|<ams art,proc|>
>>>

\<configure jats amsbook\><<<
|<ams art,proc,book|>
|<ams proc,book|>
>>>

|<ams art,proc|><<<
>>>

\<ams proc,book\><<<
|<latex numbered chapters|>
>>>

\<ams art,proc,book\><<<
|<latex maketitle|>
|<latex numbered sections|>
|<latex floats|>
|<abstract|>
|<latex tables|>
|<latex footnotes|>
|<latex quotes|>
|<latex lists|>
>>>

\section{Configurations for Memoir}

\<configure jats memoir\><<<
|<shared latex classes|>
|<shared article,report|>
|<shared book,report|>
>>>

\section{Shared class configurations}
\<shared latex classes\><<<
|<latex maketitle|>
|<latex sections|>
|<latex floats|>
|<latex tables|>
|<latex footnotes|>
|<latex quotes|>
|<latex lists|>
>>>

\<shared article,report\><<<
|<abstract|>
>>>

\<shared book,report\><<<
|<latex chapters|>
>>>


%%%%%%%%%%%%%%%%%%%%%%%%%%%%%%%%%%%%%
\chapter{Basic system configurations}
%%%%%%%%%%%%%%%%%%%%%%%%%%%%%%%%%%%%%

\<xml structure\><<<
\Configure{VERSION}{\HCode{<?xml version="1.0" encoding="UTF-8"?>\Hnewline}}

% select JATS version
\NewConfigure{DtdVersion}{1}
\Configure{DtdVersion}{1.3}

\def\:DTD{jats} % this is necessary to prevent html-mml from overriding DOCTYPE
% 
\Configure{DOCTYPE}{\HCode{<!DOCTYPE article PUBLIC "-//NLM//DTD JATS (Z39.96) Journal Archiving and Interchange DTD v\a:DtdVersion\space 20210610//EN" "JATS-archivearticle1-3.dtd">\Hnewline}}

% XML configuratins
\edef\html:xmlns{}
\NewConfigure{xmlns}[2]{\concat:config\html:xmlns{xmlns:#1="#2"\Hnewline}}
\Configure{xmlns}{xlink}{http://www.w3.org/1999/xlink}
\Configure{xmlns}{mml}{http://www.w3.org/1998/Math/MathML}

\Configure{HTML}{\HCode{<article dtd-version="\a:DtdVersion" \html:xmlns xml:lang="\Get:Language">\Hnewline}}{\HCode{\Hnewline</article>}}
\Configure{@HEAD}{}
\Configure{HEAD}{\HCode{<front>}}{\HCode{</front>}}
% Add header structure
% Journal info in <journal-meta>. It needs to be provided fully from the user configuration
\NewConfigure{JournalMeta}{1}
% basic article meta can be reconstructed from \title and \author
% 
% Title needs to be placed at the correct place by DOM filter
\Configure{TITLE}{\HCode{<article-title>}}{\HCode{</article-title>}}
\NewConfigure{ArticleMeta}{1}
  \Configure{ArticleMeta}{
  \HCode{<article-meta>}%
    % title from \title command is placed by make4ht DOM filter. If you want more complex title, 
    % use \Configure{TitleGroup}
    \a:TitleGroup%
    \JATS:Contributors%
  \HCode{</article-meta>}%
}
% enable to configure contributors
\def\JATS:Contributors{}
\NewConfigure{Contributor}[1]{\concat:config\JATS:Contributors{#1}}
\NewConfigure{TitleGroup}{1}

% insert metas into document header
\Configure{@HEAD}{\a:JournalMeta}
\Configure{@HEAD}{\a:ArticleMeta}
>>>


\<xml:lang declarations\><<<
\def\Declare:Language#1#2{%
\expandafter\gdef\csname rfclang#1\endcsname{#2}%
}
\def\Get:Language{%
\ifx\bbl@main@language\@undefined en-US%  
\else%
\expandafter\ifdefined\csname rfclang\bbl@main@language\endcsname \csname rfclang\bbl@main@language\endcsname\fi%
\fi%
}

\Declare:Language{UKenglish}{en}
\Declare:Language{USenglish}{en}
\Declare:Language{latex}{en}
\Declare:Language{acadian}{fr}
\Declare:Language{albanian}{sq}
\Declare:Language{american}{en}
\Declare:Language{amharic}{am}
\Declare:Language{arabic}{ar}
\Declare:Language{armenian}{hy}
\Declare:Language{australian}{en}
\Declare:Language{austrian}{de}
\Declare:Language{basque}{eu}
\Declare:Language{bengali}{bn}
\Declare:Language{brazilian}{pt}
\Declare:Language{brazil}{pt}
\Declare:Language{breton}{br}
\Declare:Language{british}{en}
\Declare:Language{bulgarian}{bg}
\Declare:Language{canadian}{en}
\Declare:Language{canadien}{fr}
\Declare:Language{catalan}{ca}
\Declare:Language{croatian}{hr}
\Declare:Language{czech}{cs}
\Declare:Language{danish}{da}
\Declare:Language{divehi}{dv}
\Declare:Language{dutch}{nl}
\Declare:Language{english}{en}
\Declare:Language{esperanto}{eo}
\Declare:Language{estonian}{et}
\Declare:Language{finnish}{f\/i}
\Declare:Language{francais}{fr}
\Declare:Language{french}{fr}
\Declare:Language{galician}{gl}
\Declare:Language{germanb}{de}
\Declare:Language{german}{de}
\Declare:Language{greek}{el}
\Declare:Language{hebrew}{he}
\Declare:Language{hindi}{hi}
\Declare:Language{hungarian}{hu}
\Declare:Language{icelandic}{is}
\Declare:Language{interlingua}{ia}
\Declare:Language{irish}{ga}
\Declare:Language{italian}{it}
\Declare:Language{kannada}{kn}
\Declare:Language{khmer}{km}
\Declare:Language{korean}{ko}
\Declare:Language{lao}{lo}
\Declare:Language{latin}{la}
\Declare:Language{latvian}{lv}
\Declare:Language{lithuanian}{lt}
\Declare:Language{lowersorbian}{dsb}
\Declare:Language{magyar}{hu}
\Declare:Language{malayalam}{ml}
\Declare:Language{marathi}{mr}
\Declare:Language{naustrian}{de}
\Declare:Language{newzealand}{en}
\Declare:Language{ngerman}{de}
\Declare:Language{norsk}{no}
\Declare:Language{norwegiannynorsk}{nn}
\Declare:Language{nynorsk}{no}
\Declare:Language{occitan}{oc}
\Declare:Language{oldchurchslavonic}{cu}
\Declare:Language{persian}{fa}
\Declare:Language{polish}{pl}
\Declare:Language{polutonikogreek}{el}
\Declare:Language{portuges}{pt}
\Declare:Language{portuguese}{pt}
\Declare:Language{romanian}{ro}
\Declare:Language{romansh}{rm}
\Declare:Language{russian}{ru}
\Declare:Language{samin}{se}
\Declare:Language{sanskrit}{sa}
\Declare:Language{scottish}{gd}
\Declare:Language{serbian}{sr}
\Declare:Language{serbo-croatian}{sh}
\Declare:Language{slovak}{sk}
\Declare:Language{slovene}{sl}
\Declare:Language{slovenian}{sl}
\Declare:Language{spanish}{es}
\Declare:Language{swedish}{sv}
\Declare:Language{tamil}{ta}
\Declare:Language{telugu}{te}
\Declare:Language{thai}{th}
\Declare:Language{tibetan}{bo}
\Declare:Language{turkish}{tr}
\Declare:Language{turkmen}{tk}
\Declare:Language{ukrainian}{uk}
\Declare:Language{uppersorbian}{hsb}
\Declare:Language{urdu}{ur}
\Declare:Language{vietnamese}{vi}
\Declare:Language{welsh}{cy}
>>>

\<document structure\><<<
\Configure{HtmlPar}
  {\EndP\HCode{|<show input line no|><p \csname a:!P\endcsname>}}
  {\EndP\HCode{|<show input line no|><p \csname a:!P\endcsname>}}
  {\Tg</p>}%
  {\Tg</p>}%
>>>

\<show input line no\><<<
<!--l. \the\inputlineno-->%
>>>

%%%%%%%%%%%%%%%%%%%%%%%%%%%%%%%%%%%%%%%%%
\chapter{Hyperlinks}
%%%%%%%%%%%%%%%%%%%%%%%%%%%%%%%%%%%%%%%%%

\<basic hyperlinks\><<<
\Configure{Link}{xref}{rid=}{id=}{}   
\LinkCommand\ExternalLink{ext-link,xlink:href,id}
>>>


\<url links\><<<
\Configure{url}%
     {\tmp:toks{#1}\ExternalLink[\noexpand\the\tmp:toks]{}{}{\let\UrlBigBreaks\empty
                         \let\UrlBreaks\empty #1}\EndExternalLink}
>>>

\<hyperref links\><<<
>>>

%%%%%%%%%%%%%%%%%%%%%%%%%%%%%%%%%%%%%%%%%
\chapter{Pictures}
%%%%%%%%%%%%%%%%%%%%%%%%%%%%%%%%%%%%%%%%%

\<basic pictures\><<<
\Configure{IMG}
  {\ht:special{t4ht=<graphic\Hnewline xlink:href="\a:imgdir}}
  {\ht:special{t4ht=" xlink:title="}}
  {" }
  {\ht:special{t4ht=" }}
  {\ht:special{t4ht=\xml:empty>}}
>>>

%%%%%%%%%%%%%%%%%%%%%%%%%%%%%%%%%%%%%%%%%
\chapter{Math}
%%%%%%%%%%%%%%%%%%%%%%%%%%%%%%%%%%%%%%%%%

MathML in JATS must use the mml: prefix

\<basic jats mathml\><<<
\Configure{mathml}{mml:}
>>>

Inline and display math should use different elements for their
inclusion. We need to keep stack of the opened elements, as 
math can be nested.

For some reason, this configuration is called too early, resulting
in compilation error. We thus use AtBeginDocument to postpone it.

\<basic jats mathml\><<<
\edef\math:type:inline{display="inline" }

\ifdefined\AtBeginDocument
\AtBeginDocument{
  \Configure{DviMathML}{%
    \ifx\a:@math\math:type:inline%
    \PushStack\math:types{inline-formula}%
    \HCode{<inline-formula>}%
    \else%
    \HCode{<disp-formula>}%
    \PushStack\math:types{disp-formula}%
    \fi%
  }{\PopStack\math:types\:temp\HCode{</\:temp>}}%
}%
\fi

>>>

%%%%%%%%%%%%%%%%%%%%%%%%%%%%%%%%%%%%%
\chapter{Fonts}
%%%%%%%%%%%%%%%%%%%%%%%%%%%%%%%%%%%%%

JATS doesn't support any attributes for fonts, so the information
about font name and size is lost.

\<basic fonts\><<<
\Configure{htf}{0}{+}{<italic}{}{}{}{}{>}{</italic>}
\Configure{htf}{4}{+}{<sc}{}{}{}{}{>}{</sc>}
\Configure{htf}{6}{+}{<underline}{}{}{}{}{>}{</underline>}
>>>

\<latex fonts\><<<
\Configure{underline}
   {\HCode{<underline>}\:gobble}
   {\HCode{</underline>}}

\Configure{textsuperscript}
  {\HCode{<sup>}}
  {\HCode{</sup>}}

\Configure{textsubscript}
  {\HCode{<sub>}}
  {\HCode{</sub>}}
>>>

The basic font commands are redefined in mathml.4ht, so we need to redefine them again 

\<mathml fonts\><<<
\Configure{texttt}
   {\ifmathml\providemtextclass{\mml:class="texttt"
         mathvariant="monospace" }\else\Protect\HCode{<monospace>}\NoFonts\fi}
   {\ifmathml\else\EndNoFonts\Protect\HCode{</monospace>}\fi}%
\Configure{textit}
   {\ifmathml \providemtextclass{\mml:class="textit"
         mathvariant="italic" }\else\Protect\HCode{<italic>}\NoFonts\fi}
   {\ifmathml\else\EndNoFonts\Protect\HCode{</italic>}\fi}%
\Configure{textrm}
   {\ifmathml \providemtextclass{\mml:class="textrm"
         mathvariant="normal" }\else\Protect\HCode{<roman>}\NoFonts\fi}
   {\ifmathml\else\EndNoFonts\Protect\HCode{</roman>}\fi}%
\Configure{textup}
   {\ifmathml \providemtextclass{\mml:class="textup"
         mathvariant="normal" }\else\Protect\HCode{<roman>}\NoFonts\fi}
   {\ifmathml\else\EndNoFonts\Protect\HCode{</roman>}\fi}%
\Configure{textsl}
   {\ifmathml \providemtextclass{\mml:class="textsl"
         mathvariant="italic" }\else\Protect\HCode{<italic>}\NoFonts\fi}
   {\ifmathml\else\EndNoFonts\Protect\HCode{</italic>}\fi}%
\Configure{textsf}
   {\ifmathml \providemtextclass{\mml:class="textsf"
         mathvariant="sans-serif" }\else\Protect\HCode{<sans-serif>}\NoFonts\fi}
   {\ifmathml\else\EndNoFonts\Protect\HCode{</sans-serif>}\fi}%
\Configure{textbf}
   {\ifmathml \providemtextclass{\mml:class="textbf"
         mathvariant="bold" }\else\Protect\HCode{<bold>}\NoFonts\fi}
   {\ifmathml\else\EndNoFonts\Protect\HCode{</bold>}\fi}%
\Configure{textsc}
   {\ifmathml \providemtextclass{\mml:class="textsc"
         mathvariant="normal" }\else\Protect\HCode{<sc>}\NoFonts\fi}
   {\ifmathml\else\EndNoFonts\Protect\HCode{</sc>}\fi}%
\Configure{emph}
   {\ifmathml \providemtextclass{\mml:class="emph"
         mathvariant="italic" }\else\Protect\HCode{<italic>}\NoFonts\fi}
   {\ifmathml\else\EndNoFonts\Protect\HCode{</italic>}\fi}%
>>>


%%%%%%%%%%%%%%%%%%%%%%%%%%%%%
\chapter{Document structure}
%%%%%%%%%%%%%%%%%%%%%%%%%%%%%

%%%%%%%%%%%%%%%%%%%%%%%%%%%%%
\section{Maketitle}
%%%%%%%%%%%%%%%%%%%%%%%%%%%%%

JATS expects specific elements in the document metadata. We produce some custom
elements, which are expected to be removed in the make4ht post-processing. Contents
of maketitle should be moved to the metadata block and removed from the document 
body.

\<latex maketitle\><<<
\Configure{maketitle}
   {\ifvmode \IgnorePar\fi \EndP |<title for TITLE|>%
    \HCode{<maketitle>}}
   {\ifvmode \IgnorePar\fi \EndP \HCode{</maketitle>}}
   {\NoFonts\IgnorePar\HCode{<article-title>}\IgnorePar}
   {\HCode{</article-title>}\IgnoreIndent\EndNoFonts}
>>>

This macro is used to print zeropaded days and months in the ISO date attribute

\<latex maketitle\><<<
\def\date:zeropad#1{\ifnum #1<10 0\fi#1}
>>>


\<latex maketitle\><<<
\Configure{thanks author date and} 
   {\ifvmode \IgnorePar\fi\EndP \HCode{<aff>}}
   {\ifvmode \IgnorePar\fi\EndP \HCode{</aff>}}
   {\ifvmode \IgnorePar\fi\EndP \HCode{<contrib contrib-type="author"><name><string-name>}}
   {\ifvmode \IgnorePar\fi\EndP \HCode{</string-name></name></contrib>}}
   {\ifvmode \IgnorePar\fi\EndP \HCode{<date iso-8601-date="\the\year-\date:zeropad{\the\month}-\date:zeropad{\the\day}"><string-date>}}
   {\ifvmode \IgnorePar\fi\EndP \HCode{</string-date></date>}}
   {\HCode{</string-name></name></contrib><contrib contrib-type="author"><name><string-name>}}
   {\HCode{}}

\Configure{thank}
{\HCode{<affref rid="\the\c@footnote">}} 
{\HCode{</affref>}}
{\HCode{<aff id="\the\c@footnote"><affmark>}}
{\HCode{</affmark>}} {\HCode{</aff>}}
>>>

\<title for TITLE\><<<
{\Configure{maketitle}{}{}{}{}%
% \let\thanks|=\:gobble
\def\TeX{TeX}%
\def\mbox{\hbox}%
\def\gobble:font##1##2{##2}\:TITLE: \no:fonts
\def\footnotemark[##1]{}%
\def\:hashintitle{\protect\symbol{35}}
\let\#\:hashintitle
\NoFonts\Tag{TITLE+}{\@title}\EndNoFonts}
>>>

%%%%%%%%%%%%%%%%%%%%%%%%%%%%%
\section{Sections}
%%%%%%%%%%%%%%%%%%%%%%%%%%%%%

\<latex chapters\><<<
|<latex numbered chapters|>
|<latex unnumbered chapters|>
>>>


\<latex numbered chapters\><<<
\Configure{chapter}
{\ifvmode\IgnorePar\fi\EndP\IgnorePar\HCode{<sec>\Hnewline}}
{\ifvmode\IgnorePar\fi\EndP\IgnorePar\HCode{</sec>\Hnewline}}
   {\TitleMark\space\HCode{<title>}\HtmlParOff}
   {\HCode{</title>}\HtmlParOn \ShowPar\par}

\ConfigureMark{chapter}
   {\if@mainmatter
       \HCode{<label>}\chaptername\ \thechapter\HCode{</label>}\fi}


\Configure{appendix}
{\ifvmode\IgnorePar\fi\EndP\IgnorePar\HCode{<sec>\Hnewline}}
{\ifvmode\IgnorePar\fi\EndP\IgnorePar\HCode{</sec>\Hnewline}}
   {\TitleMark\space\HCode{<title>}\HtmlParOff}
   {\HCode{</title>}\HtmlParOn \ShowPar\par}

\ConfigureMark{appendix}{\HCode{<label>}%
   \appendixname\ \thechapter\HCode{</label>}}
>>>

\<latex unnumbered chapters\><<<
\Configure{likechapter}
{\ifvmode\IgnorePar\fi\EndP\IgnorePar\HCode{<sec>\Hnewline}}
{\ifvmode\IgnorePar\fi\EndP\IgnorePar\HCode{</sec>\Hnewline}}
   {\HCode{<title>}\HtmlParOff}
   {\HCode{</title>}\HtmlParOn \ShowPar\par}
>>>

\<latex sections\><<<
|<latex numbered sections|>
|<latex unnumbered sections|>
>>>

\<latex numbered sections\><<<
\Configure{section}
{\ifvmode\IgnorePar\fi\EndP\IgnorePar\HCode{<sec>\Hnewline}}
{\ifvmode\IgnorePar\fi\EndP\IgnorePar\HCode{</sec>\Hnewline}}
   {\TitleMark\space\HCode{<title>}\HtmlParOff}
   {\HCode{</title>}\HtmlParOn \ShowPar\par}

\ConfigureMark{section}
   {\ifnum \c:secnumdepth>\c@secnumdepth \expandafter\:gobble
    \else
       \HCode{<label>}\@seccntformat{section}%
       \HCode{</label>}\fi }


\Configure{subsection}
{\ifvmode\IgnorePar\fi\EndP\IgnorePar\HCode{<sec>\Hnewline}}
{\ifvmode\IgnorePar\fi\EndP\IgnorePar\HCode{</sec>\Hnewline}}
   {\TitleMark\space\HCode{<title>}\HtmlParOff}
   {\HCode{</title>}\HtmlParOn \ShowPar\par}

\ConfigureMark{subsection}
   {\ifnum \c:secnumdepth>\c@secnumdepth \expandafter\:gobble
    \else
       \HCode{<label>}\@seccntformat{section}%
       \HCode{</label>}\fi }


\Configure{subsubsection}
{\ifvmode\IgnorePar\fi\EndP\IgnorePar\HCode{<sec>\Hnewline}}
{\ifvmode\IgnorePar\fi\EndP\IgnorePar\HCode{</sec>\Hnewline}}
   {\TitleMark\space\HCode{<title>}\HtmlParOff}
   {\HCode{</title>}\HtmlParOn \ShowPar\par}

\ConfigureMark{subsubsection}
   {\ifnum \c:secnumdepth>\c@secnumdepth \expandafter\:gobble
    \else
       \HCode{<label>}\@seccntformat{section}%
       \HCode{</label>}\fi }


\Configure{paragraph}
{\ifvmode\IgnorePar\fi\EndP\IgnorePar\HCode{<sec>\Hnewline}}
{\ifvmode\IgnorePar\fi\EndP\IgnorePar\HCode{</sec>\Hnewline}}
   {\HCode{<title>}\HtmlParOff}
   {\HCode{</title>}\HtmlParOn \ShowPar\par}

>>>

\<latex unnumbered sections\><<<
\Configure{likesection}
{\ifvmode\IgnorePar\fi\EndP\IgnorePar\HCode{<sec>\Hnewline}}
{\ifvmode\IgnorePar\fi\EndP\IgnorePar\HCode{</sec>\Hnewline}}
   {\HCode{<title>}\HtmlParOff}
   {\HCode{</title>}\HtmlParOn \ShowPar\par}

\Configure{likesubsection}
{\ifvmode\IgnorePar\fi\EndP\IgnorePar\HCode{<sec>\Hnewline}}
{\ifvmode\IgnorePar\fi\EndP\IgnorePar\HCode{</sec>\Hnewline}}
   {\HCode{<title>}\HtmlParOff}
   {\HCode{</title>}\HtmlParOn \ShowPar\par}

\Configure{likesubsubsection}
{\ifvmode\IgnorePar\fi\EndP\IgnorePar\HCode{<sec>\Hnewline}}
{\ifvmode\IgnorePar\fi\EndP\IgnorePar\HCode{</sec>\Hnewline}}
   {\HCode{<title>}\HtmlParOff}
   {\HCode{</title>}\HtmlParOn \ShowPar\par}

\Configure{likeparagraph}
{\ifvmode\IgnorePar\fi\EndP\IgnorePar\HCode{<sec>\Hnewline}}
{\ifvmode\IgnorePar\fi\EndP\IgnorePar\HCode{</sec>\Hnewline}}
   {\HCode{<title>}\HtmlParOff}
   {\HCode{</title>}\HtmlParOn \ShowPar\par}
>>>

%%%%%%%%%%%%%%%%%%%%%%%%%%%%%%%%%%%
\section{Abstracts}
%%%%%%%%%%%%%%%%%%%%%%%%%%%%%%%%%%%

\<abstract\><<<
\ConfigureEnv{abstract}
{\ifvmode\IgnorePar\fi\EndP\HCode{<abstract>\Hnewline}}
{\ifvmode\IgnorePar\fi\EndP\HCode{</abstract>}\par}{}{}

\Configure{abstracttitle}{\ifvmode\IgnorePar\fi\EndP\HCode{<label>}\HtmlParOff\NoFonts}
   {\HCode{</label>}\EndNoFonts\HtmlParOn\par}
>>>

%%%%%%%%%%%%%%%%%%%%%%%%%%%%%%%%%%%
\section{Floats}
%%%%%%%%%%%%%%%%%%%%%%%%%%%%%%%%%%%

\<latex floats\><<<

\Configure{float}
   {\ifOption{refcaption}{}{\csname par\endcsname\ShowPar \leavevmode}}
     {\IgnorePar\EndP\HCode{<fig>}}
   {\ifvmode \IgnorePar \fi\EndP
     \HCode{</fig>}\csname par\endcsname\ShowPar}

\ConfigureEnv{figure}
{\IgnorePar\EndP\HCode{<fig>\Hnewline}%
    \bgroup \Configure{float}{\ShowPar}{}{}%    
   }
   {\egroup
   \IgnorePar\EndP\HCode{</fig>}|<try env inline par|>\par}
   {}{}


\Configure{caption}{\IgnorePar\EndP\HCode{<label>}}
   {: } {\HCode{</label><caption>}\noindent\ShowPar}
   {\HCode{</caption>}\HCode{<!--tex4ht:label?:
   \cur:th\:currentlabel\space-->}%
}
>>>
%%%%%%%%%%%%%%%%%%%
\section{Tables}
%%%%%%%%%%%%%%%%%%%

\<latex tables\><<<
\ConfigureEnv{table}
   {\ifvmode \IgnorePar\fi \EndP \HCode{<table-wrap position="float">}
   \Configure{float}{}{}{}{}
   }
   {\ifvmode \IgnorePar\fi \EndP \HCode{</table-wrap>}\par\ShowPar}
   {}{}

\ConfigureEnv{tabular}
     {\Configure{noalign}%
{\f:tabular\d:tabular \HCode{<tr><td colspan="\ar:cnt">}}
{\HCode{</td></tr>}\pend:def\TableNo{0}\c:tabular\e:tabular}%
%
\IgnorePar\ifvmode\else\HCode{<!--tex4ht:inline-->}\fi
\EndP\PushStack\Col:Marg\AllColMargins
}
{\PopStack\Col:Marg\AllColMargins\ShowPar
\Configure{noalign}{}{}%
%
\ShowPar
%
}{}{}
>>>

%%%%%%%%%%%%%%%%%%%%%
\section{Footnotes}
%%%%%%%%%%%%%%%%%%%%%

\<latex footnotes\><<<
\Configure{footnotemark}{\bgroup\NoFonts\HCode{<fn symbol="}\Configure{textsuperscript}{}{}}{\HCode{">}\EndNoFonts\egroup}
% suppress footnote number in footnotext, it was already used in footnotemark
\Configure{footnotetext}{\NoFonts\ShowPar\setbox0=\vbox\bgroup}
{\egroup\EndNoFonts\SaveEndP\par\ShowPar\normalsize}% force new paragraph and normal font size
{\EndP\RecallEndP\HCode{</fn>}}%
>>>

%%%%%%%%%%%%%%%%%%%%
\section{Quotes}
%%%%%%%%%%%%%%%%%%%%

\<latex quotes\><<<
\ConfigureEnv{quote}
   {}{}
   {\ifvmode\IgnorePar\fi\EndP%
    \HCode{<disp-quote>}}
   {\IgnorePar\EndP\HCode{</disp-quote>}\ShowPar\ShowIndent}
>>>

%%%%%%%%%%%%%%%%%%%%
\section{Lists}
%%%%%%%%%%%%%%%%%%%%

List utilities, copied from the HTML configuration. We need to
keep track of the 

\<save end:itm\><<<
\PushMacro\end:itm
>>>


\<recall end:itm\><<<
\PopMacro\end:itm \global\let\end:itm \end:itm 
>>>

\<list par\><<<
\par\ShowPar
>>>

JATS list have a similar structure, so we can use a custom macro that
takes just the environment name and list-type attribute:

\<latex lists\><<<
\def\ConfigJatsList#1#2{%
\ConfigureList{#1}%
   {\ifvmode\IgnorePar\fi\EndP\EndP\HCode{<list list-type="#2">}%
       |<save end:itm|>\global\let\end:itm=\empty}
   {|<recall end:itm|>\EndP\HCode{</list-item></list>}}
   {\end:itm\global\def\end:itm{\EndP\Tg</list-item>}\DeleteMark}
   {\HCode{<list-item>}|<list par|>}
 }

\ConfigJatsList{itemize}{bulleted}
\ConfigJatsList{enumerate}{order}
>>>

Description lists have a little bit different structure, so we need to configure
them separatelly.

\<latex lists\><<<
\ConfigureList{description}%
   {\ifvmode\IgnorePar\fi\EndP\HCode{<def-list>}%
      |<save end:itm|>\global\let\end:itm=\empty}
   {|<recall end:itm|>\EndP\HCode{</def></def-item></def-list>}\ShowPar}
   {\end:itm \global\def\end:itm{\EndP\HCode{</def></def-item>}}\HCode{<def-item><term>}\NoFonts\HtmlParOff}
   {\EndNoFonts\HCode{</term><def>}\HtmlParOn|<list par|>}
>>>

%%%%%%%%%%%%%%%%%%%%%%%%%%%%%%%%%%%
\chapter{Packages}
%%%%%%%%%%%%%%%%%%%%%%%%%%%%%%%%%%%

\section{Hyperref}

\<configure jats url\><<<
|<url links|>
>>>

\<configure jats hyperref\><<<
|<hyperref links|>
>>>

\section{Bibliographies}

\<configure jats biblatex\><<<
|<biblatex environment|>
|<biblatex fields|>
>>>


\<biblatex environment\><<<
 \def\bibConfigure{%
  \ConfigureList{thebibliography}
  {\ifvmode \IgnorePar \fi \EndP \EndP
    \HCode{<ref-list>}\HtmlParOff%
   \immediate\write\@auxout{%
     \string\providecommand\string\BibFileName[2][]{}
   }%
   \immediate\write\@auxout{%
       \string\BibFileName[\therefsection]{\FileName}}%
    \PushMacro \end:itm \global \let \end:itm =\empty}%
  {\ifvmode \IgnorePar \fi \EndP
    \PopMacro \end:itm \global \let \end:itm \end:itm \EndP
    \HCode {</mixed-citation></ref></ref-list>}\HtmlParOn\ShowPar}%
  {\ifvmode \IgnorePar \fi \EndP \gHAdvance \bibN by 1
    \end:itm \global \def \end:itm {\EndP \HCode{</mixed-citation></ref>}}%
    \Tag{X\therefsection-\abx@field@entrykey}{bibitem-\bibN}
    \HCode {<ref id="bibitem-\bibN"><label>}}%
  {\ifvmode \IgnorePar \fi \EndP
    \HCode {</label><mixed-citation publication-type="\thefield{entrytype}"  id="bib-\bibN">}}%
}
>>>

We use BibLaTeX's formatting commands to add JATS elements around some bibliographic
fields.

\<biblatex fields\><<<
\DeclareFieldFormat{title}{\HCode{<source>}#1\HCode{</source>}}
\DeclareFieldFormat{booktitle}{\HCode{<source>}#1\HCode{</source>}}
\DeclareFieldFormat{maintitle}{\HCode{<source>}#1\HCode{</source>}}
\DeclareFieldFormat{journaltitle}{\HCode{<source>}#1\HCode{</source>}}
\DeclareFieldFormat[article]{title}{\HCode{<article-title>}#1\HCode{</article-title>}}
\DeclareFieldFormat[incollection]{title}{\HCode{<part-title>}#1\HCode{</part-title>}}
\DeclareFieldFormat[inbook]{title}{\HCode{<part-title>}#1\HCode{</part-title>}}
\DeclareFieldFormat{year}{\HCode{<year>}#1\HCode{</year>}}
\DeclareFieldFormat{date}{\HCode{<date-in-citation content-type="published">}#1\HCode{</date-in-citation>}} 
\DeclareFieldFormat{pages}{\HCode{<page-range>}#1\HCode{</page-range>}}
\DeclareListFormat{publisher}{\usebibmacro{list:delim}{#1}\HCode{<publisher-name>}#1\HCode{</publisher-name>}\isdot\usebibmacro{list:andothers}}
\DeclareListFormat{location}{\usebibmacro{list:delim}{#1}\HCode{<publisher-loc>}#1\HCode{</publisher-loc>}\isdot\usebibmacro{list:andothers}}
\DeclareNameFormat{author}{%
	\nameparts{#1}% This command initializes commands used in the following bibmacro
	\HCode{<string-name>}% use 
	\usebibmacro{name:family-given}%
    {\namepartfamily}%
    {\namepartgiveni}%
    {\namepartprefix}%
    {\namepartsuffix}%
  \HCode{</string-name>}%
  \usebibmacro{name:andothers}%
}
>>>

\endinput
