% $Id$
% compile: latex tex4ht-mathjax
%
% Copyright 2018-2024 TeX Users Group
% Released under LPPL 1.3c+.
% See tex4ht-cpright.tex for license text.

\ifx \HTML\UnDef
   \def\HTML{mathjax-latex-4ht} 
   \def\CONFIG{\jobname}
   \def\MAKETITLE{\author{Michal Hoftich}}
   \def\next{\input mktex4ht.4ht  \endinput}
   \expandafter\next
\fi

% $Id$
% A more few common TeX definitions for literate sources.  Not installed
% in runtime.  These are only used in a few files, compared to those in
% common.tex.  Do not know if any harm would come from including them always.
% 
% Copyright 2009, 2010 TeX Users Group
% Copyright 1996--2009 Eitan M. Gurari
%
% This work may be distributed and/or modified under the
% conditions of the LaTeX Project Public License, either
% version 1.3c of this license or (at your option) any
% later version. The latest version of this license is in
%   http://www.latex-project.org/lppl.txt
% and version 1.3c or later is part of all distributions
% of LaTeX version 2005/12/01 or later.
%
% This work has the LPPL maintenance status "maintained".
%
% The Current Maintainer of this work
% is the TeX4ht Project <http://tug.org/tex4ht>.
% 
% If you modify this program, changing the 
% version identification would be appreciated.

\let\AltxModifyShowCode=\ModifyShowCode
\def\ModifyShowCode{%
   \def\by{by}\def\={=}\AltxModifyShowCode}

\let\pReModifyOutputCode=\ModifyOutputCode
\def\ModifyOutputCode{%
   \def\by{}\def\={}%
   \pReModifyOutputCode}

% $Id$
% A few common TeX definitions for literate sources.  Not installed in runtime.
% 
% Copyright 2009-2017 TeX Users Group
% Copyright 1996-2009 Eitan M. Gurari
%
% This work may be distributed and/or modified under the
% conditions of the LaTeX Project Public License, either
% version 1.3c of this license or (at your option) any
% later version. The latest version of this license is in
%   http://www.latex-project.org/lppl.txt
% and version 1.3c or later is part of all distributions
% of LaTeX version 2005/12/01 or later.
%
% This work has the LPPL maintenance status "maintained".
%
% The Current Maintainer of this work
% is the TeX4ht Project <http://tug.org/tex4ht>.
% 
% If you modify this program, changing the 
% version identification would be appreciated.

\newcount\tmpcnt  \tmpcnt\time  \divide\tmpcnt  60
\edef\temp{\the\tmpcnt}
\multiply\tmpcnt  -60 \advance\tmpcnt  \time

\edef\version{\the\year-\ifnum \month<10 0\fi
  \the\month-\ifnum \day<10 0\fi\the\day
   -\ifnum \temp<10 0\fi \temp
   :\ifnum \tmpcnt<10 0\fi\the\tmpcnt}

% a fixed-string version that can be enabled for debugging.
%\edef\versionDebug{000-00-00-00:00}
%\let\version\versionDebug

% #1 is the first year for Eitan.  The last year is always 2009.  RIP.
\def\CopyYear.#1.{#1-2009}

% command for write to terminal and the log file
% this version is used in the .4ht files build
% identical command is defined also in tex4ht-sty.tex, 
% it is used in TeX document compilation
\def\writesixteen#1{\immediate\write1616{#1}}

\<TeX4ht copyright\><<<
%
% This work may be distributed and/or modified under the
% conditions of the LaTeX Project Public License, either
% version 1.3c of this license or (at your option) any
% later version. The latest version of this license is in
%   http://www.latex-project.org/lppl.txt
% and version 1.3c or later is part of all distributions
% of LaTeX version 2005/12/01 or later.
%
% This work has the LPPL maintenance status "maintained".
%
% The Current Maintainer of this work
% is the TeX4ht Project <http://tug.org/tex4ht>.
% 
% If you modify this program, changing the 
% version identification would be appreciated.
>>>


\<mathjax-latex-4ht\><<<
% mathjax-latex-4ht.4ht (|version), generated from |jobname.tex
% Copyright 2018-2024 TeX Users Group
|<TeX4ht copywrite|>
|<required packages|>
|<defined commands|>
\AtBeginDocument{%
|<configurations|>
|<mhchem configurations|>
|<restore environments|>
}
\endinput
>>>

We don't require additional packages anymore, as everything necessary is included
in LaTeX kernel now.
% \RequirePackage{etoolbox,expl3,environ}
\<required packages\><<<
>>>

The \verb|\alteqtoks| command prints the used LaTeX math code to the output document in 
verbatim. 

In the past, we used \verb|\detokenize|. The side effect of this is that detokeize inserts space
after each control sequence. This is completely valid TeX code, but earlier versions of
MathJax didn't like that, rendering resulted in error.

Fortunatelly, MathJax 3 supports these spaces, so we could remove regular expressions for space handling.
The original code was this:

% % convert \ { to \:{ 
% \regex_replace_all:nnN { \x{5C} \x{20} \x{7B} } { \x{5C} \x{3A} \x{7B} } \l_tmpa_tl
% % delete spaces before left brackets
% \regex_replace_all:nnN { \x{20} \x{7B} } { \x{7B} } \l_tmpa_tl
% % replace \\:{ back to \\ { -- this can be introduced by the previous regex
% \regex_replace_all:nnN { \x{5C} \x{5C}  \x{3A} \x{7B} } { \x{5C} \x{5C} \x{20} \x{7B} } \l_tmpa_tl


In July 2024, when I researched another issue, I've found that it is actually possible to avoid these 
extra spaces using LaTeX 3 commands. The inspiration comes from 
\Link[https://tex.stackexchange.com/a/44444/2891]this Bruno Le Foch\EndLink.
One ongoing issue is that newlines are not presented. But they weren't preserved with \verb|\detokenize|
either, so it shouldn't be a problem.

In September 2024 I found that the LaTeX 3 method of preserving spaces didn't work correctly for 
code like this: \verb|$\int   A f(x)\;dx$| -- the space between \verb|\int| and \verb|A| was removed,
so it ended like command \verb|\intA|. So we are moving back to \verb|\detokenize|.

This is the code that was promising, but didn't work in the end:

\begin{verbatim}
  % save tokens, but preserve spaces
  % https://tex.stackexchange.com/a/44444/2891
  \tl_set:Nn \l_tmpa_tl {#1}
  \regex_replace_all:nnN { . } { \c{string} \0 } \l_tmpa_tl
  \tl_set:Nx \l_tmpa_tl { \l_tmpa_tl }
\end{verbatim}


We still use regular expressions to escape invalid XML characters to entities, so it works only with LaTeX.

\<defined commands\><<<
\ExplSyntaxOn
\cs_new_protected:Npn \alteqtoks #1
{
  \tl_set:Ne \l_tmpa_tl {\detokenize{#1}}
  % % replace < > and & with xml entities
  \regex_replace_all:nnN { \x{26} } { &amp; } \l_tmpa_tl
  \regex_replace_all:nnN { \x{3C} } { &lt; } \l_tmpa_tl 
  \regex_replace_all:nnN { \x{3E} } { &gt; } \l_tmpa_tl
  % replace \par command with blank lines
  \regex_replace_all:nnN { \x{5C}par\b } {\x{A}\x{A}} \l_tmpa_tl
  \tl_set:Ne \l_tmpb_tl{ \l_tmpa_tl }
  \HCode{\l_tmpb_tl}
}
\ExplSyntaxOff

>>>

Provide configuration for MathJax

\<defined commands\><<<
\NewConfigure{MathJaxConfig}{1}
>>>

The MathJaxMacros configuration includes a file with TeX macro 
definitions, and include them at the beginning of each HTML page.
MathJax then interprets them, so they are available in math. 
This is an alternative to JavaScript method using MathJaxConfig.

The file cannot contain any other commands than newcommand,
all characters that could cause issues in the HTML parsing must
be escaped, so \verb|<| should became \verb|&lt;|, for example.

\<defined commands\><<<
\NewConfigure{MathJaxMacros}[1]{%
\Configure{@BODY}{\bgroup\NoFonts\ttfamily\detokenize{\(}%
  \special{t4ht*<#1}%
\detokenize{\)}\EndNoFonts\egroup}%
}
>>>

The following commands are used for patching of the standard math commands and
environments. They will then keep the LaTeX code unchanged.

\<defined commands\><<<

\long\def\AltlMath#1\){\expandafter\alteqtoks{\(#1\)}\)}
\long\def\AltlDisplay#1\]{\alteqtoks{\[#1\]}\]}
\long\def\AltMathOne#1${\alteqtoks{\(#1\)}$}
% this seems a bit hacky -- we need to skip some code inserted at the 
% beginning of each display math
\long\def\AltlDisplayDollars#1$${\alteqtoks{\[#1\]}$$}

\newcommand\VerbMathToks[2]{%
  \alteqtoks{\begin{#2}
    #1
  \end{#2}}%
}
>>>

In pictures, we want to use the original version of the math environments. Because we use
an agressive verbatim method for the environments, we cannot use the usual \verb'\HLet' method.
Instead, we save the original meaning, and user needs to restore it manually, using
the \verb'\RestoreMathJaxEnvironments' command.

\<defined commands\><<<
\ExplSyntaxOn
\seq_new:N\:savedmathjaxenvs

\newcommand\:savemathjaxenv[1]{%
  \seq_gput_right:Nn\:savedmathjaxenvs{#1}
  \expandafter\let\csname mathjax-#1\expandafter\endcsname\csname #1\endcsname%
  \expandafter\let\csname mathjax-end#1\expandafter\endcsname\csname end#1\endcsname%
}

% we must not reintroduce the matrix environmeint in TikZ, because it interferes with the \matrix command
\newcommand\:ignoretikzmatrix{\seq_remove_all:Nn\:savedmathjaxenvs{matrix}}

\newcommand\RestoreMathJaxEnvironment[1]{%
  \expandafter\let\csname #1\expandafter\endcsname\csname mathjax-#1\endcsname%
  \expandafter\let\csname end#1\expandafter\endcsname\csname mathjax-end#1\endcsname%
}

\newcommand\RestoreMathJaxEnvironments{%
  \seq_map_function:NN\:savedmathjaxenvs\RestoreMathJaxEnvironment%
}
\ExplSyntaxOff
>>>

The \verb|\VerbMath| command redefines environments to pass their content
verbatim to the HTML output.

The optional command can contain name of the counter that should be updated.
It will also use LaTeX 3 regular expressions to search for label commands.
Thanks to that, cross-referencing to the math environments should work.
It is a bit limited, for example subequations cannot work. There can be also
issues in books, where equation numbers are based on chapters, but MathJax
numbers them consecutively. In these cases, other mechanisms may be necessary.

\<defined commands\><<<
\ExplSyntaxOn
\cs_generate_variant:Nn \regex_extract_once:nnNTF {nV}
\newcommand\VerbMath[2][]{%
  \cs_if_exist:cTF{#2}{
    \:savemathjaxenv{#2}%
    \RenewDocumentEnvironment{#2}{+!b}{%
      \NoFonts\expandafter\VerbMathToks\expandafter{\detokenize{##1}}{#2}\EndNoFonts%
      \ifx\relax#1\relax\else%
      \refstepcounter{#1}%
      \regex_extract_once:nVNTF { label\s* \x{7B}([^\x{7D}]*)\x{7D}} {\l_tmpb_tl} \l_tmp_seq {\label{\seq_item:Nn\l_tmp_seq{2}}} {}%
      \fi
    }{}
  }{}%
}
\ExplSyntaxOff
>>>

The \verb|\fixmathjaxtoc| command is used for patching commands which should
keep their verbatim appearance in the TOC

\<defined commands\><<<
\def\fixmathjaxtoc#1{\Configure{writetoc}{\def#1{\detokenize{#1}}}}
>>>

This was meant to fix some commands in sectioning commands, but it is not
a good solution. Better is to define the problematic commands as robust.

\<not defined commands\><<<
\def\fixmathjaxsec#1{\def#1{\detokenize{#1}}}
\fixmathjaxsec\right
\fixmathjaxsec\left
>>>

Require verbatim math environments.

\<configurations\><<<
\VerbMath{subarray}
\VerbMath{smallmatrix}
\VerbMath{matrix}
\VerbMath{pmatrix}
\VerbMath{bmatrix}
\VerbMath{Bmatrix}
\VerbMath{vmatrix}
\VerbMath{Vmatrix}
\VerbMath{cases}
\VerbMath{subequations}
\VerbMath{aligned}
\VerbMath{alignedat}
\VerbMath{gathered}
\VerbMath{gather}
\VerbMath{gather*}
\VerbMath{alignat}
\VerbMath{alignat*}
\VerbMath{xalignat}
\VerbMath{xalignat*}
\VerbMath{xxalignat}
\VerbMath{align}
\VerbMath{align*}
\VerbMath{flalign}
\VerbMath{flalign*}
\VerbMath{split}
\VerbMath{multline}
\VerbMath{multline*}
\VerbMath[equation]{equation}
\VerbMath{equation*}
\VerbMath{math}
\VerbMath{displaymath}
\VerbMath{eqnarray}
\VerbMath{eqnarray*}
>>>

It is necessary to reset env configurations for multiline 
environments, because default TeX4ht configurations turn
them into pictures.

\<configurations\><<<
\ConfigureEnv{gather}{}{}{}{}
\ConfigureEnv{gather*}{}{}{}{}
\ConfigureEnv{multline}{}{}{}{}
\ConfigureEnv{multline*}{}{}{}{}
>>>


Fixes for tables of contents.

\<configurations\><<<
\fixmathjaxtoc\int
\fixmathjaxtoc\,
\fixmathjaxtoc\sin
\fixmathjaxtoc\cos
\fixmathjaxtoc\tan
\fixmathjaxtoc\arcsin
\fixmathjaxtoc\arccos
\fixmathjaxtoc\arctan
\fixmathjaxtoc\csc
\fixmathjaxtoc\sec
\fixmathjaxtoc\cot
\fixmathjaxtoc\sinh
\fixmathjaxtoc\cosh
\fixmathjaxtoc\tanh
\fixmathjaxtoc\coth
\fixmathjaxtoc\log
\fixmathjaxtoc\ln
\fixmathjaxtoc\sum
\fixmathjaxtoc\(
\fixmathjaxtoc\)
\fixmathjaxtoc\begin
\fixmathjaxtoc\end
\fixmathjaxtoc\\
\fixmathjaxtoc\exp
\fixmathjaxtoc\left
\fixmathjaxtoc\right
>>>

\<mhchem configurations\><<<
\@ifpackageloaded{mhchem}{%
\def\:tempa#1{\texttt{\detokenize{\(\ce{#1}\)}}}
\HLet\ce\:tempa
}{}
>>>

Restore environments redefined to produce verbatim output to their
original meaning in the picture mode. Otherwise math cannot work
inside TikZ etc.

\<restore environments\><<<
\@ifpackageloaded{tikz}{%
\tikzset{every picture/.append code={\:ignoretikzmatrix\RestoreMathJaxEnvironments}} 
}{}
>>>

\endinput
