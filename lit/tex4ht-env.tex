% $Id$
%      latex tex4ht-env
% or htlatex tex4ht-env "xhtml,3"
%
% Copyright 2009-2015 TeX Users Group
% Copyright 1997-2009 Eitan M. Gurari
% Released under LPPL 1.3c+.
% See tex4ht-cpright.tex for license text.

\documentclass{article}

\usepackage{url}
\input tex4ht-dir.tex

\ifx \HCode\Undef
    \usepackage[xhtml,2,sections+]{tex4ht}
    \input ProTex.sty
    \AlProTex{env,<<<>>>,@,title,list}
\else
    \Configure{ProTex}{env,<<<>>>,@,title,list}
\fi

% $Id$
% A few common TeX definitions for literate sources.  Not installed in runtime.
% 
% Copyright 2009-2017 TeX Users Group
% Copyright 1996-2009 Eitan M. Gurari
%
% This work may be distributed and/or modified under the
% conditions of the LaTeX Project Public License, either
% version 1.3c of this license or (at your option) any
% later version. The latest version of this license is in
%   http://www.latex-project.org/lppl.txt
% and version 1.3c or later is part of all distributions
% of LaTeX version 2005/12/01 or later.
%
% This work has the LPPL maintenance status "maintained".
%
% The Current Maintainer of this work
% is the TeX4ht Project <http://tug.org/tex4ht>.
% 
% If you modify this program, changing the 
% version identification would be appreciated.

\newcount\tmpcnt  \tmpcnt\time  \divide\tmpcnt  60
\edef\temp{\the\tmpcnt}
\multiply\tmpcnt  -60 \advance\tmpcnt  \time

\edef\version{\the\year-\ifnum \month<10 0\fi
  \the\month-\ifnum \day<10 0\fi\the\day
   -\ifnum \temp<10 0\fi \temp
   :\ifnum \tmpcnt<10 0\fi\the\tmpcnt}

% a fixed-string version that can be enabled for debugging.
%\edef\versionDebug{000-00-00-00:00}
%\let\version\versionDebug

% #1 is the first year for Eitan.  The last year is always 2009.  RIP.
\def\CopyYear.#1.{#1-2009}

% command for write to terminal and the log file
% this version is used in the .4ht files build
% identical command is defined also in tex4ht-sty.tex, 
% it is used in TeX document compilation
\def\writesixteen#1{\immediate\write1616{#1}}


\begin{document}

%%%%%%%%%%%%%%%%%%%%%%%%%%%%%%%%%%%%%%
\section{Outline of Scripts}
%%%%%%%%%%%%%%%%%%%%%%%%%%%%%%%%%%%%%%

Note: The `-c...' calls are executed in the order of the prsence of
the scripts within tex4ht.env.  Hence, place the xtpipe translations
before validations.


\<tex4ht.env-unix\><<<
@<unix copyright@>%
@<unix opening comment@>%
@<unix note fonts@>%
@<unix texmf dir@>%
@<unix tfm fonts@>%
@<unix htf fonts@>%
@<unix G script@>%
@<unix tex4ht.fls@>%
@<unix t4ht commands@>%
@<xtpipes: unix java help@>
@<xtpipes: unix emacspeak@>
@<xtpipes: unix jsml@>
@<xtpipes: unix docbk@>
@<xtpipes: unix xhtml@> 
@<xtpipes: unix xhtmml@> @%before validate@%
@<xtpipes: unix oo@>%
@<unix validate@> %
@<unix openoffice@>%
% end of file
>>>


\<tex4ht.env-win32\><<<
@<win32 copyright@>%
@<win32 opening comment@>%
@<win32 note fonts@>%
@<win32 texmf dir@>%
@<win32 tfm fonts@>%
@<win32 htf fonts@>%
@<win32 G script@>%
@<dos t4ht commands@>%
@<win32 tex4ht.fls@>%
@<xtpipes: win32 java help@>
@<xtpipes: win32 emacspeak@>
@<xtpipes: win32 jsml@>
@<xtpipes: win32 docbk@>
@<xtpipes: win32 xhtml@> 
@<xtpipes: win32 xhtmml@> @%before validate@%
@<xtpipes: win32 oo@>%
@<win32 validate@> %
@<win32 openoffice@>%
% end of file
>>>



\<tex4ht\><<<
@<private copyright@>%
@<my opening comment@>%
@<my note fonts@>%
@<my texmf dir@>%
@<my tfm fonts@>%
@<my htf fonts@>%
@<my G script@>%
@<my shared@>%
@<my t4ht commands@>%
@<my tex4ht.fls@>%
@<xtpipes: my java help@>
@<xtpipes: my emacspeak@>
@<xtpipes: my jsml@>
@<xtpipes: my docbk@>
@<xtpipes: my xhtml@>
@<xtpipes: my xhtmml@>  @%before validate@%
@<xtpipes: my oo@>
@<my validate@> %
@<unix openoffice@>%
% end of file
>>>





%%%%%%%%%%%%%%%%%%
\section{Opening Comments}
%%%%%%%%%%%%%%%%%%%

\<my opening comment\><<<
%%%%%%%%%%%%%%%%%%%%%%%%%%%%%%%%%%%%%%%%%%%%%%%%%%%%%%%%%%%%%%%%%%%%% 
% Notes:                                                            %
%  1. Empty lines might be harmful                                  %
%  2. Tagged script segments <tag>...</tag> are scanned only if     %
%     their names are specified within -ctag switches of tex4ht.c   %
%     and t4ht.c. When -c switches are not supplied, a -cdefault    %
%     is implicitly assumed.                                        % 
%        The tags <tag> and </tag>, but not their bodies, are       %
%     ignored if they are not located at the start of their lines.  %
%  3. Place this file in your work directory and/or root directory  %
%     and/or in directory `xxx' of your choice. In the latest case, %
%     compile tex4ht.c and t4ht.c with `#define ENVFILE xxx', or    %
%     provide the address of the file to tex4ht and t4ht throught   %
%     the -e switch                                                 %
%%%%%%%%%%%%%%%%%%%%%%%%%%%%%%%%%%%%%%%%%%%%%%%%%%%%%%%%%%%%%%%%%%%%%  
>>>

\<unix opening comment\><<<
%%%%%%%%%%%%%%%%%%%%%%%%%%%%%%%%%%%%%%%%%%%%%%%%%%%%%%%%%%%%%%%%%%%%%  
% Notes:                                                            %
%  1. Empty lines might be harmful                                  %
%  2. Tagged script segments <tag>...</tag> are scanned only if     %
%     their names are specified within -ctag switches of tex4ht.c   %
%     and t4ht.c. When -c switches are not supplied, a -cdefault    %
%     is implicitly assumed.                                        % 
%        The tags <tag> and </tag>, but not their bodies, are       %
%     ignored if they are not located at the start of their lines.  %
%  3. Place this file in your work directory and/or root directory  %
%     and/or in directory `xxx' of your choice. In the latest case, %
%     compile tex4ht.c and t4ht.c with `#define ENVFILE xxx', or    %
%     provide the address of the file to tex4ht and t4ht throught   %
%     the -e switch                                                 %
%  4. Under kpathsea, the substring `%%~' may be employed in        %
%     commands to indirectly obtain the value of ${SELFAUTOPARENT}. %
%%%%%%%%%%%%%%%%%%%%%%%%%%%%%%%%%%%%%%%%%%%%%%%%%%%%%%%%%%%%%%%%%%%%%  
>>>


\<win32 opening comment\><<<
%%%%%%%%%%%%%%%%%%%%%%%%%%%%%%%%%%%%%%%%%%%%%%%%%%%%%%%%%%%%%%%%%%%%%  
% Notes:                                                            %
%  1. Empty lines might be harmful                                  %
%  2. Tagged script segments <tag>...</tag> are scanned only if     %
%     their names are specified within -ctag switches of tex4ht.c   %
%     and t4ht.c. When -c switches are not supplied, a -cdefault    %
%     is implicitly assumed.                                        % 
%        The tags <tag> and </tag>, but not their bodies, are       %
%     ignored if they are not located at the start of their lines.  %
%  3. Place this file in your work directory and/or in the same     %
%     directory as tex4ht.exe and t4ht.exe and/or in directory      %
%     `xxx' of your choice. In the latest case, compile tex4ht.c    %
%     and t4ht.c with `#define ENVFILE xxx', or provide the address %
%     of the file to tex4ht and t4ht throught the -e switch         %
%  4. Under kpathsea, the substring `%%~' may be employed in        %
%     commands to indirectly obtain the value of ${SELFAUTOPARENT}. %
%%%%%%%%%%%%%%%%%%%%%%%%%%%%%%%%%%%%%%%%%%%%%%%%%%%%%%%%%%%%%%%%%%%%%  
>>>


%%%%%%%%%%%%%%%%%%
\section{Fonts}
%%%%%%%%%%%%%%%%%%




%%%%%%%%%%%%%%%%%%
\subsection{TFM Files}
%%%%%%%%%%%%%%%%%%%



\<unix tfm fonts\><<<
tpath/tex/texmf/fonts/tfm/!
>>>


\<win32 tfm fonts\><<<
tc:\path\tfm!
>>>

\<my tfm fonts\><<<
t/usr/local/teTeX/share/texmf/fonts/tfm/!
t/var/tmp/texfonts/tfm/!
 t/usr/local/teTeX/texmf/fonts/vf/! 
>>>

The \verb't/var/tmp/texfonts/tfm/!' are for fonts created on the fly
during the compilation by tex.



%%%%%%%%%%%%%%%%%%
\subsection{HTF Files}
%%%%%%%%%%%%%%%%%%%


\<unix htf fonts\><<<
@<unix iso1 htf@>
@<unix moz htf@>
@<unix iso2 htf@>
@<unix iso5 htf@>
@<unix iso7 htf@>
@<unix iso8 htf@>
@<unix iso15 htf@>
@<unix koi8r htf@>
@<unix sym htf@>
@<unix win1251 htf@>
@<unix uni htf@>
@<unix emspk htf@>
@<unix jsml htf@>
>>>





\<win32 htf fonts\><<<
@<win32 iso1 htf@>
@<win32 moz htf@>
@<win32 iso2 htf@>
@<win32 iso5 htf@>
@<win32 iso7 htf@>
@<win32 iso8 htf@>
@<win32 iso15 htf@>
@<win32 koi8r htf@>
@<win32 sym htf@>
@<win32 win1251 htf@>
@<win32 uni htf@>
@<win32 emspk htf@>
@<win32 jsml htf@>
>>>



\<my htf fonts\><<<
@<unix iso1 htf@>
@<unix moz htf@>
@<unix iso2 htf@>
@<unix iso5 htf@>
@<unix iso7 htf@>
@<unix iso8 htf@>
@<unix iso15 htf@>
@<unix koi8r htf@>
@<unix sym htf@>
@<unix win1251 htf@>
@<unix uni htf@>
@<unix emspk htf@>
@<unix jsml htf@>
>>>



\<unix iso1 htf\><<<
<default>
i~/tex4ht.dir/texmf/tex4ht/ht-fonts/iso8859/1/!
i~/tex4ht.dir/texmf/tex4ht/ht-fonts/ascii/!
i~/tex4ht.dir/texmf/tex4ht/ht-fonts/alias/!
i~/tex4ht.dir/texmf/tex4ht/ht-fonts/mozilla/!
i~/tex4ht.dir/texmf/tex4ht/ht-fonts/unicode/!
</default>             
>>>


\<win32 iso1 htf\><<<
<default>
ic:\tex4ht\texmf\tex4ht\ht-fonts\iso8859\1\!
ic:\tex4ht\texmf\tex4ht\ht-fonts\ascii\!
ic:\tex4ht\texmf\tex4ht\ht-fonts\alias\!
ic:\tex4ht\texmf\tex4ht\ht-fonts\unicode\!
</default>
>>>


\<unix moz htf\><<<
<mozhtf>
i~/tex4ht.dir/texmf/tex4ht/ht-fonts/mozilla/!
i~/tex4ht.dir/texmf/tex4ht/ht-fonts/unicode/!
i~/tex4ht.dir/texmf/tex4ht/ht-fonts/ascii/!
i~/tex4ht.dir/texmf/tex4ht/ht-fonts/alias/!
</mozhtf>
>>>




\<win32 moz htf\><<<
<mozhtf>
ic:\tex4ht\texmf\tex4ht\ht-fonts\mozilla\!
ic:\tex4ht\texmf\tex4ht\ht-fonts\unicode\!
ic:\tex4ht\texmf\tex4ht\ht-fonts\ascii\!
ic:\tex4ht\texmf\tex4ht\ht-fonts\alias\!
</mozhtf>
>>>



\<unix uni htf\><<<
<unihtf>
i~/tex4ht.dir/texmf/tex4ht/ht-fonts/unicode/!
i~/tex4ht.dir/texmf/tex4ht/ht-fonts/ascii/!
i~/tex4ht.dir/texmf/tex4ht/ht-fonts/alias/!
</unihtf>
>>>



\<win32 uni htf\><<<
<unihtf>
ic:\tex4ht\texmf\tex4ht\ht-fonts\unicode\!
ic:\tex4ht\texmf\tex4ht\ht-fonts\ascii\!
ic:\tex4ht\texmf\tex4ht\ht-fonts\alias\!
</unihtf>
>>>





\<unix emspk htf\><<<
<emspkhtf>
i~/tex4ht.dir/texmf/tex4ht/ht-fonts/html-speech/!
i~/tex4ht.dir/texmf/tex4ht/ht-fonts/unicode/!
i~/tex4ht.dir/texmf/tex4ht/ht-fonts/ascii/!
i~/tex4ht.dir/texmf/tex4ht/ht-fonts/alias/!
i~/tex4ht.dir/texmf/tex4ht/ht-fonts/css/emacspeak/!
</emspkhtf>
>>>



\<win32 emspk htf\><<<
<emspkhtf>
ic:\tex4ht\texmf\tex4ht\ht-fonts\html-speech\!
ic:\tex4ht\texmf\tex4ht\ht-fonts\unicode\!
ic:\tex4ht\texmf\tex4ht\ht-fonts\ascii\!
ic:\tex4ht\texmf\tex4ht\ht-fonts\alias\!
ic:\tex4ht\texmf\tex4ht\ht-fonts\css\emacspeak\!
</emspkhtf>
>>>






















\<unix jsml htf\><<<
<jsmlhtf>
i~/tex4ht.dir/texmf/tex4ht/ht-fonts/jsml/!
i~/tex4ht.dir/texmf/tex4ht/ht-fonts/unicode/!
i~/tex4ht.dir/texmf/tex4ht/ht-fonts/ascii/!
i~/tex4ht.dir/texmf/tex4ht/ht-fonts/alias/!
</jsmlhtf>
>>>



\<win32 jsml htf\><<<
<jsmlhtf>
ic:\tex4ht\texmf\tex4ht\ht-fonts\jsml\!
ic:\tex4ht\texmf\tex4ht\ht-fonts\unicode\!
ic:\tex4ht\texmf\tex4ht\ht-fonts\ascii\!
ic:\tex4ht\texmf\tex4ht\ht-fonts\alias\!
</jsmlhtf>
>>>






\<unix iso2 htf\><<<
<iso2htf>
i~/tex4ht.dir/texmf/tex4ht/ht-fonts/iso8859/2/!
i~/tex4ht.dir/texmf/tex4ht/ht-fonts/ascii/!
i~/tex4ht.dir/texmf/tex4ht/ht-fonts/unicode/!
i~/tex4ht.dir/texmf/tex4ht/ht-fonts/alias/!
</iso2htf>
>>>


\<win32 iso2 htf\><<<
<iso2htf>
ic:\tex4ht\texmf\tex4ht\ht-fonts\iso8859\2\!
ic:\tex4ht\texmf\tex4ht\ht-fonts\ascii\!
ic:\tex4ht\texmf\tex4ht\ht-fonts\alias\!
ic:\tex4ht\texmf\tex4ht\ht-fonts\unicode\!
</iso2htf>
>>>









\<unix iso5 htf\><<<
<iso5htf>
i~/tex4ht.dir/texmf/tex4ht/ht-fonts/iso8859/5/!
i~/tex4ht.dir/texmf/tex4ht/ht-fonts/ascii/!
i~/tex4ht.dir/texmf/tex4ht/ht-fonts/unicode/!
i~/tex4ht.dir/texmf/tex4ht/ht-fonts/alias/!
</iso5htf>
>>>


\<win32 iso5 htf\><<<
<iso5htf>
ic:\tex4ht\texmf\tex4ht\ht-fonts\iso8859\5\!
ic:\tex4ht\texmf\tex4ht\ht-fonts\ascii\!
ic:\tex4ht\texmf\tex4ht\ht-fonts\alias\!
ic:\tex4ht\texmf\tex4ht\ht-fonts\unicode\!
</iso5htf>
>>>



\<unix iso7 htf\><<<
<iso7htf>
i~/tex4ht.dir/texmf/tex4ht/ht-fonts/iso8859/7/!
i~/tex4ht.dir/texmf/tex4ht/ht-fonts/ascii/!
i~/tex4ht.dir/texmf/tex4ht/ht-fonts/unicode/!
i~/tex4ht.dir/texmf/tex4ht/ht-fonts/alias/!
</iso7htf>
>>>


\<win32 iso7 htf\><<<
<iso7htf>
ic:\tex4ht\texmf\tex4ht\ht-fonts\iso8859\7\!
ic:\tex4ht\texmf\tex4ht\ht-fonts\ascii\!
ic:\tex4ht\texmf\tex4ht\ht-fonts\alias\!
ic:\tex4ht\texmf\tex4ht\ht-fonts\unicode\!
</iso7htf>
>>>




\<unix iso28htf\><<<
<iso8htf>
i~/tex4ht.dir/texmf/tex4ht/ht-fonts/iso8859/8/!
i~/tex4ht.dir/texmf/tex4ht/ht-fonts/ascii/!
i~/tex4ht.dir/texmf/tex4ht/ht-fonts/unicode/!
i~/tex4ht.dir/texmf/tex4ht/ht-fonts/alias/!
</iso8htf>
>>>


\<win32 iso8 htf\><<<
<iso8htf>
ic:\tex4ht\texmf\tex4ht\ht-fonts\iso8859\8\!
ic:\tex4ht\texmf\tex4ht\ht-fonts\ascii\!
ic:\tex4ht\texmf\tex4ht\ht-fonts\alias\!
ic:\tex4ht\texmf\tex4ht\ht-fonts\unicode\!
</iso8htf>
>>>


\<unix iso15 htf\><<<
<iso15htf>
i~/tex4ht.dir/texmf/tex4ht/ht-fonts/iso8859/15/!
i~/tex4ht.dir/texmf/tex4ht/ht-fonts/ascii/!
i~/tex4ht.dir/texmf/tex4ht/ht-fonts/unicode/!
i~/tex4ht.dir/texmf/tex4ht/ht-fonts/alias/!
</iso15htf>
>>>


\<win32 iso15 htf\><<<
<iso15htf>
ic:\tex4ht\texmf\tex4ht\ht-fonts\iso8859\15\!
ic:\tex4ht\texmf\tex4ht\ht-fonts\ascii\!
ic:\tex4ht\texmf\tex4ht\ht-fonts\alias\!
ic:\tex4ht\texmf\tex4ht\ht-fonts\unicode\!
</iso15htf>
>>>







\<unix koi8r htf\><<<
<koi8rhtf>
i~/tex4ht.dir/texmf/tex4ht/ht-fonts/koi/8r/!
i~/tex4ht.dir/texmf/tex4ht/ht-fonts/ascii/!
i~/tex4ht.dir/texmf/tex4ht/ht-fonts/unicode/!
i~/tex4ht.dir/texmf/tex4ht/ht-fonts/alias/!
</koi8rhtf>
>>>


\<win32 koi8r htf\><<<
<koi8rhtf>
ic:\tex4ht\texmf\tex4ht\ht-fonts\koi\8r\!
ic:\tex4ht\texmf\tex4ht\ht-fonts\ascii\!
ic:\tex4ht\texmf\tex4ht\ht-fonts\alias\!
ic:\tex4ht\texmf\tex4ht\ht-fonts\unicode\!
</koi8rhtf>
>>>




\<unix sym htf\><<<                                                      %
<symhtf>
i~/tex4ht.dir/texmf/tex4ht/ht-fonts/symbol/!
i~/tex4ht.dir/texmf/tex4ht/ht-fonts/unicode/!
i~/tex4ht.dir/texmf/tex4ht/ht-fonts/ascii/!
i~/tex4ht.dir/texmf/tex4ht/ht-fonts/alias/!
</symhtf>
>>>

\<win32 sym htf\><<<                                                     %
<symhtf>
ic:\tex4ht\texmf\tex4ht\ht-fonts\symbol\!
ic:\tex4ht\texmf\tex4ht\ht-fonts\unicode\!
ic:\tex4ht\texmf\tex4ht\ht-fonts\ascii\!
ic:\tex4ht\texmf\tex4ht\ht-fonts\alias\!
</symhtf>
>>>




\<unix win1251 htf\><<<                                                      %
<win1251htf>
i~/tex4ht.dir/texmf/tex4ht/ht-fonts/win/1251/!
i~/tex4ht.dir/texmf/tex4ht/ht-fonts/unicode/!
i~/tex4ht.dir/texmf/tex4ht/ht-fonts/ascii/!
i~/tex4ht.dir/texmf/tex4ht/ht-fonts/alias/!
</win1251htf>
>>>

\<win32 win1251 htf\><<<                                                     %
<win1251htf>
ic:\tex4ht\texmf\tex4ht\ht-fonts\win\1251\!
ic:\tex4ht\texmf\tex4ht\ht-fonts\unicode\!
ic:\tex4ht\texmf\tex4ht\ht-fonts\ascii\!
ic:\tex4ht\texmf\tex4ht\ht-fonts\alias\!
</win1251htf>
>>>





%%%%%%%%%%%%%%%%%%
\subsection{Bookkeeping tex4ht.fls}
%%%%%%%%%%%%%%%%%%%

Currently broken and Disabled.


\<my tex4ht.fls\><<<
% * Address for an automatically created bookkeeping file           %
% * The work directory is assumed for a default                     %
l~/tex4ht.dir/tex4ht.fls
>>>




\<unix tex4ht.fls\><<<
% * Address for an automatically created bookkeeping file           %
% * Ignored under kpathsea                                          %
% * The work directory is assumed for a default                     %
 l~/tex4ht.dir/tex4ht.fls
>>>



\<win32 tex4ht.fls\><<<
% * Address for an automatically created bookkeeping file           %
% * Ignored under kpathsea                                          %
% * The work directory is assumed for a default                     %
lc:\tex4ht\tex4ht.fls
>>>


%%%%%%%%%%%%%%%%%%
\subsection{Notes}
%%%%%%%%%%%%%%%%%%%

\<my note fonts\><<<
% * Replace `path', and possibly what follows, with appropriate     %
%   content                                                         %
% * A path may start with:                                          %
%   ~ for environment variable HOME                                 %
%   ~~ for environment variable TEX4HTTFM                           %
% * A ! requests recursive search into subdirectories               %
% * Multiple entries of each type are allowed                       %
% * When supported by kpathsea                                      %
%   + t entries are ignored                                         %
%   + the path before /ht-fonts/ is ignored in i records that don't %
%     represent real addresses                                      %
%%%%%%%%%%%%%%%%%%%%%%%%%%%%%%%%%%%%%%%%%%%%%%%%%%%%%%%%%%%%%%%%%%%%%  
>>>

\<unix note fonts\><<<
% * Replace `path', and possibly what follows, with appropriate     %
%   content                                                         %
% * A path may start with:                                          %
%   ~ for environment variable HOME                                 %
%   ~~ for environment variable TEX4HTTFM                           %
%   %%~ for ${SELFAUTOPARENT} of KPATHSEA                           %
% * A ! requests recursive search into subdirectories               %
% * Multiple t and i entries are allowed                            %
% * When supported by kpathsea                                      %
%   + t entries are ignored                                         %
%   + the path before /ht-fonts/ is ignored in i records that don't %
%     represent real addresses                                      %
%%%%%%%%%%%%%%%%%%%%%%%%%%%%%%%%%%%%%%%%%%%%%%%%%%%%%%%%%%%%%%%%%%%%%  
>>>


\<win32 note fonts\><<<
% * Fix the paths                                                   %
% * A path may start with:                                          %
%   ~ for environment variable HOME                                 %
%   ~~ for environment variable TEX4HTTFM                           %
%   %%~ for ${SELFAUTOPARENT} of KPATHSEA                           %
% * A ! requests recursive search into  subdirectories              %
% * Multiple t and i entries are allowed                            %
% * When supported by kpathsea                                      %
%   + t entries are ignored                                         %
%   + the path before /ht-fonts/ is ignored in i records that don't %
%     represent real addresses                                      %
%%%%%%%%%%%%%%%%%%%%%%%%%%%%%%%%%%%%%%%%%%%%%%%%%%%%%%%%%%%%%%%%%%%%%  
>>>



%%%%%%%%%%%%%%%%%%
\section{Bitmap Conversion Scripts}
%%%%%%%%%%%%%%%%%%%



%%%%%%%%%%%%%
\subsection{Unix}
%%%%%%%%%%%%%




\<unix G script\><<<
%%%%%%%%%%%%%%%%%%%%%%%%%%%%%%%%%%%%%%%%%%%%%%%%%%%%%%%%%%%%%%%%%%%%%  
% dvi-to-xxx                                                        %
%   choose one G-script                                             %
%%%%%%%%%%%%%%%%%%%%%%%%%%%%%%%%%%%%%%%%%%%%%%%%%%%%%%%%%%%%%%%%%%%%%  
% dvips options                                                     %
%   magnification: -x mag  (e.g., -x 1200)                          %
%   page size:     -T x,y  (e.g., -T 14in,14in)                     %
%%%%%%%%%%%%%%%%%%%%%%%%%%%%%%%%%%%%%%%%%%%%%%%%%%%%%%%%%%%%%%%%%%%%%
 <convert>
@<unix png gs@>
@<unix svg@>
G.
@<unix dvips convert@>
@<variants of convert@>
 </convert>    
<netpbm>
@<unix netpbm@>
</netpbm>
@<dvipng@>
@<request scripts@> %
>>>


%%%%%%%%%%%%%
\subsection{Windows}
%%%%%%%%%%%%%



\<win32 G script\><<<
%%%%%%%%%%%%%%%%%%%%%%%%%%%%%%%%%%%%%%%%%%%%%%%%%%%%%%%%%%%%%%%%%%%%%  
% dvi-to-xxx                                                        %
%   choose one G-script                                             %
%%%%%%%%%%%%%%%%%%%%%%%%%%%%%%%%%%%%%%%%%%%%%%%%%%%%%%%%%%%%%%%%%%%%%  
% dvips options                                                     %
%   magnification: -x mag  (e.g., -x 1200)                          %
%   page size:     -T x,y  (e.g., -T 14in,14in)                     %
%%%%%%%%%%%%%%%%%%%%%%%%%%%%%%%%%%%%%%%%%%%%%%%%%%%%%%%%%%%%%%%%%%%%%
  <convert>
@<dos png gs@>
@<win32 svg@>
G.
@<dos dvips convert@>
  </convert>
@<variants of convert@>
<netpbm>
@<dos netpbm@>
</netpbm>
@<dvipng@>
@<win gs@>
@<request scripts@> %
>>>


%%%%%%%%%%%%%
\subsection{Private (on SUN)}
%%%%%%%%%%%%%



\<my G script\><<<
%%%%%%%%%%%%%%%%%%%%%%%%%%%%%%%%%%%%%%%%%%%%%%%%%%%%%%%%%%%%%%%%%%%%%  
% Select one of the G scripts                                       %
%%%%%%%%%%%%%%%%%%%%%%%%%%%%%%%%%%%%%%%%%%%%%%%%%%%%%%%%%%%%%%%%%%%%%  
% dvips options                                                     %
%   magnification: -x mag  (e.g., -x 1200)                          %
%   page size:     -T x,y  (e.g., -T 14in,14in)                     %
%   gs fonts:      -Pfont  (e.g., -Pcmz -Pamz)                      %
%%%%%%%%%%%%%%%%%%%%%%%%%%%%%%%%%%%%%%%%%%%%%%%%%%%%%%%%%%%%%%%%%%%%%
 <convert>
@<dvips convert (personal)@>
 </convert>
<convert-old>
@<unix png gs@>
@<unix svg@>
G.
@<old dvips + gs + convert (personal)@> 
</convert-old>
%%%%%%%%%%%%%%%%%%%%%%%%%%%%%%%%%%%%%%%%%%%%%%%%%%%%%%%%%%%%%%%%%%%%%  
<netpbm>
@<unix netpbm@>
</netpbm>
@<dvipng@>
@<request scripts@> %
>>>



%%%%%%%%%%%%%
\subsection{Script: dvips + gs}
% duplicated below, as well as twice here.  must fixx!
% we don't want to use convert for the sake of simplifying mactex.
%%%%%%%%%%%%%

\<dvips convert (personal)\><<<
Gdvips -E -q -Ppdf -f %%1 -pp %%2 > zz%%4.ps
Ggs -sDEVICE=pngalpha -sOutputFile=%%3 -r110x110 -dEPSCrop -dBackgroundColor=16#ffffff -dTextAlphaBits=2 -dGraphicsAlphaBits=2 -q -dbatch -dNOPAUSE zz%%4.ps -c quit 
Grm zz%%4.ps >>>

\<unix dvips convert\><<<
%echo "dbg: got to dvips to png via gs (unix dvips convert)"
Gdvips -E -q -Ppdf -f %%1 -pp %%2 > zz%%4.ps
Ggs -sDEVICE=pngalpha -sOutputFile=%%3 -r110x110 -dEPSCrop -dBackgroundColor=16#ffffff -dTextAlphaBits=2 -dGraphicsAlphaBits=2 -q -dbatch -dNOPAUSE zz%%4.ps -c quit 
Grm zz%%4.ps >>>


\<dos dvips convert\><<<
Gif exist zz%%4.ps DEL zz%%4.ps >nul
Gif exist %%3 DEL %%3 >nul
Gdvips -E -Ppdf -mode ibmvga -D 110 -f %%1 -pp %%2 > zz%%4.ps
Gconvert -trim +repage -density 110x110 -transparent "#FFFFFF" zz%%4.ps %%3
Gif exist zz%%4.ps DEL zz%%4.ps >nul   >>>



\<variants of convert\><<<
%--------- alternative instructions for old versions of convert
% Gconvert -trim -density 110x110 -transparent "#FFFFFF" zz%%4.ps %%3 
% Gconvert -crop 0x0 +repage -density 110x110 -transparent "#FFFFFF" zz%%4.ps %%3
% Gconvert -crop 0x0 +page -density 110x110 -transparent "#FFFFFF" zz%%4.ps %%3
%%%%%%%%%%%%%%%%%%%%%%%%%%%%%%%%%%%%%%%%%%%%%%%%%%%%%%%%%%%%%%%%%%%%%  >>>

The `-E' option is required for dvips to produce a tight bounding box.


%%%%%%%%%%%%%
\subsection{Script: dvips + gs + convert}
%%%%%%%%%%%%%

\<old dvips + gs + convert (unix)\><<<
<convert-old>
Gdvips -Ppdf -mode ibmvga -D 110 -f %%1 -pp %%2 > zz%%4.ps
Ggs -sDEVICE=ppm -sOutputFile=zz%%4.ppm -r110x110 -dTextAlphaBits=2 -dGraphicsAlphaBits=2 -q -dbatch -dNOPAUSE zz%%4.ps -c quit
Gconvert zz%%4.ppm -trim +repage -density 110x110 -transparent '#FFFFFF' %%3
Grm zz%%4.ps
Grm zz%%4.ppm
</convert-old>  >>>


The intermediate ppm step is to overcome cropping and other problems
in convert.



\<old dvips + gs + convert (personal)\><<<
Gdvips -Ppdf -mode ibmvga -D 110 -f %%1 -pp %%2 > zz%%4.ps
Ggs -sDEVICE=ppm -sOutputFile=zz%%4.ppm -r110x110 -dTextAlphaBits=2 -dGraphicsAlphaBits=2 -q -dbatch -dNOPAUSE zz%%4.ps -c quit
Gconvert zz%%4.ppm -crop 0x0 +page -density 110x110 -transparent '#FFFFFF' %%3
Grm zz%%4.ps
Grm zz%%4.ppm    >>>

\<old dvips + gs + convert (windows)\><<<
<convert-old>
Gif exist zz%%4.ps DEL zz%%4.ps >nul
Gif exist zz%%4.ppm DEL zz%%4.ppm >nul
Gif exist %%3 DEL %%3 >nul
Gdvips -Ppdf -mode ibmvga -D 110 -f %%1 -pp %%2 > zz%%4.ps
Ggswin32c -dNOPAUSE -sDEVICE=ppmraw -dTextAlphaBits=4 -dGraphicsAlphaBits=4 -r110x110 -sOutputFile=zz%%4.ppm -q zz%%4.ps -c quit
Gconvert zz%%4.ppm -trim +repage -density 110x110 -transparent #FFFFFF %%3
Gif exist zz%%4.ps DEL zz%%4.ps >nul
Gif exist zz%%4.ppm DEL zz%%4.ppm >nul
</convert-old>   >>>

The command line options \verb'-Pcmz -Pamz' of dvips requests type 1
postscript format for the standard Computer Modern fonts of \TeX.


TextAlphaBits, GraphicsAlphaBits: The number of bits of anti-aliasing
information for text or graphics respectively. Legal values are 1 (no
anti-aliasing, the default for most devices), 2, or 4.

-x: magnification * 1000.  A variant of \verb'\magnification'. Should use
values for which the printers can support the font sizes. Typically,
\verb'1.2{\tt\string^}i * 1000'. 

%%%%%%%%%%%%%
\subsection{PNG Script: dvi + gs}
%%%%%%%%%%%%%



\<unix png gs\><<<
 G.png
 Gdvips -E -q -Ppdf -f %%1 -pp %%2 > zz%%4.ps
 Ggs -sDEVICE=pngalpha -sOutputFile=%%3 -r110x110 -dEPSCrop -dBackgroundColor=16#ffffff -dTextAlphaBits=2 -dGraphicsAlphaBits=2 -q -dbatch -dNOPAUSE zz%%4.ps -c quit 
 Grm zz%%4.ps    >>>

A piping of the conversion as in

\begin{verbatim}
dvips -E -q -Ppdf -f %%1 -pp %%2 | Ggs -sDEVICE=pngalpha -sOutputFile=- ...
\end{verbatim}

provides faster compilations, but error messages might get lost if
they happen to be directed to the standard output stream.


\<dos png gs\><<<
 G.png
 Gif exist zz%%4.ps DEL zz%%4.ps >nul
 Gif exist %%3 DEL %%3 >nul
 Gdvips -E -q -Ppdf -f %%1 -pp %%2 > zz%%4.ps
 Ggswin32c -sDEVICE=pngalpha -sOutputFile=%%3 -r110x110 -dEPSCrop -dBackgroundColor=16#ffffff -dTextAlphaBits=2 -dGraphicsAlphaBits=2 -q -dbatch -dNOPAUSE zz%%4.ps -c quit 
 Gif exist zz%%4.ps DEL zz%%4.ps >nul >>>




ImageMagic comes with an `identify' utility for inspecting image
files.  For instance, try `identify -verbatim foo.png'. For additional
information, try `identify' or `man identify'.


%%%%%%%%%%%%%
\subsection{Script: netpbm}
%%%%%%%%%%%%%



\<unix netpbm\><<<
Gdvips -Ppdf -mode ibmvga -D 110 -f %%1 -pp %%2  > zz%%4.ps
Ggs -sDEVICE=ppm -r110 -sOutputFile=zz%%4.ppm -q -dbatch -dTextAlphaBits=4 -dGraphicsAlphaBits=4 -dNOPAUSE zz%%4.ps -c quit
Gmv zz%%4.ppm temp.ppm
Gpnmcrop  temp.ppm > zz%%4.ppm
Gpnmtopng -interlace -transparent 1,1,1 zz%%4.ppm > %%3
 Gppmtopng -interlace -transparent 1,1,1 zz%%4.ppm > %%3
Grm zz%%4.ps
Grm zz%%4.ppm
Grm temp.ppm
% Gppmtogif -interlace -transparent 1,1,1 zz%%4.ppm > %%3 >>>


\<dos netpbm\><<<
Gif exist  zz%%4.ps DEL zz%%4.ps  
Gif exist zz%%4.ppm DEL zz%%4.ppm  
Gif exist temp.ppm DEL temp.ppm  
Gdvips -Ppdf -mode ibmvga -D 110 -f %%1 -pp %%2  > zz%%4.ps 
Ggswin32c -sDEVICE=ppm -r110 -sOutputFile=zz%%4.ppm -q -dbatch -dTextAlphaBits=4 -dGraphicsAlphaBits=4 -dNOPAUSE zz%%4.ps -c quit  
Gmove /y zz%%4.ppm temp.ppm  
Gpnmcrop  temp.ppm > zz%%4.ppm  
Gif exist %%3 DEL %%3  
Gppmtopng -interlace -transparent 1,1,1 zz%%4.ppm > %%3  
Gif exist  zz%%4.ps DEL zz%%4.ps  
Gif exist zz%%4.ppm DEL zz%%4.ppm  
Gif exist temp.ppm DEL temp.ppm  
% Gppmtogif -interlace -transparent 1,1,1 zz%%4.ppm > %%3   >>>

%%%%%%%%%%%%%
\subsection{Script: dvipng}
%%%%%%%%%%%%%

\<dvipng\><<<
<dvipng>
G.png
%echo "dbg: got to dvi to png via (dvipng)"
Gdvipng -T tight -x 1400 -D 72 -bg Transparent -pp %%2:%%2 %%1 -o %%3
G.gif
Gdvipng -T tight -x 1400 -D 72 -bg Transparent -gif -pp %%2:%%2 %%1 -o %%3
G.
Gdvipng -T tight -x 1400 -D 72 -bg Transparent -pp %%2:%%2 %%1 -o %%3
% avoiding convert for mactex.
% Gdvips -Ppdf -mode ibmvga -D 110 -f %%1 -pp %%2 > zz%%4.ps
% Gconvert -crop 0x0 -density 110x110 -transparent '#FFFFFF' zz%%4.ps %%3
%Grm zz%%4.ps
</dvipng>
>>>


%%%%%%%%%%%%%%%%%% 
\subsection{SVG Bitmap Conversion Scripts} 
%%%%%%%%%%%%%%%%%%% 

\<unix svg\><<<
G.svg
Gdvips -Ppdf -mode ibmvga -D 110 -f %%1 -pp %%2 > zz%%4.eps
Gpstoedit -f svg zz%%4.eps %%3      >>>


\<win32 svg\><<<
G.svg
Gif exist zz%%4.ps DEL zz%%4.ps >nul
Gif exist %%3 DEL %%3 >nul
Gdvips -Ppdf -mode ibmvga -D 110 -f %%1 -pp %%2 > zz%%4.eps
Gpstoedit -f svg zz%%4.eps %%3
Gif exist zz%%4.ps DEL zz%%4.ps >nul >>>

\begin{itemize}
\item
\Link[http://www.cityinthesky.co.uk/pdf2svg.html]{}{}pdf2svg\EndLink ?

The LaTeX distribution offers pdfcrop, in case it is needed


\end{itemize}




%%%%%%%%%%%%%
\subsection{Request Scripts}
%%%%%%%%%%%%%

\<request scripts\><<<
%%%%%%%%%%%%%%%%%%%%%%%%%%%%%%%%%%%%%%%%%%%%%%%%%%%%%%%%%%%%%%%%%%%%%  
% Default scripts, shifted rightward 1 pos
%%%%%%%%%%%%%%%%%%%%%%%%%%%%%%%%%%%%%%%%%%%%%%%%%%%%%%%%%%%%%%%%%%%%%  
 s--- needs --- %%1.idv[%%2] ==> %%3 ---
 b--- characters ---
 g.png
% empty bitmaps                                                     %
 Ecopy  empty.png %%1%%2
%%%%%%%%%%%%%%%%%%%%%%%%%%%%%%%%%%%%%%%%%%%%%%%%%%%%%%%%%%%%%%%%%%%%%  
>>>


%%%%%%%%%%%%%
\subsection{Alternative Scripts}
%%%%%%%%%%%%%






\begin{itemize}
\item
\begin{verbatim}
 Gdvips -T 14in,14in -Ppdf  -mode  epstypro -D 360 -f %%1 -pp %%2  > zz%%4.ps
\end{verbatim}

\item
\begin{verbatim}
Gtest ! -f %%3  
Gdvips -pp %%2 -R -E %%1 -o - | convert -transparent '#FFFFFF' -quality 100   
-density 110 ps:- %%3  
\end{verbatim}

\item
{A 
Netpbm-based script
(\Link[ftp://ftp.x.org/R5contrib/netpbm-1mar1994.tar.gz]{}{}src\EndLink{})}

\begin{verbatim}
 Ggs -sDEVICE=ppm -sOutputFile=zz%%4.ppm -q -dbatch -dNOPAUSE zz%%4.ps -c quit
 Gmv zz%%4.ppm yz%%4.ppm
 Gpnmcrop  yz%%4.ppm > zz%%4.ppm
 Gppmtopng -interlace -transparent 1,1,1 zz%%4.ppm > %%3
\end{verbatim}

\item
{A 
PDF-based script
(\Link[http://www.cs.wisc.edu/\string
   ~ghost/doc/AFPL/6.50/Ps2pdf.htm]{}{}ps2pdf\EndLink{})}
\begin{verbatim}
Gdvips -mode ibmvga -D 110 -f %%1 -pp %%2  > zz%%4.ps
Gps2pdf zz%%4.ps zz%%4.pdf
Gconvert zz%%4.pdf -trim -density 110x110 -transparent '#FFFFFF' %%3
\end{verbatim}


\item

\<win gs\><<<
<gs>
Gif exist zz%%4.ps DEL zz%%4.ps >nul
Gif exist %%3 DEL %%3 >nul
Gdvips -E -q -Ppdf -D 110 -mode ibmvga -f %%1 -pp %%2 -o zz%%4.ps
Ggswin32c -dbatch -dNOPAUSE -sDEVICE=pngalpha -dEPSCrop ``-dBackgroundColor=16#ffffff'' -dTextAlphaBits=4 -dGraphicsAlphaBits=4 -r110x110 -sOutputFile=%%3 -q zz%%4.ps -c quit
Gif exist zz%%4.ps DEL zz%%4.ps >nul
</gs>
>>>


\end{itemize}



\begin{itemize}



\item{A variant of the given script}
\begin{verbatim}
GDEL zz%%4.ps
Gdvips -T 14in,14in -Ppdf -mode  epstypro -D 360 -f %%1 -pp %%2  > zz%%4.ps
GDEL %%3
Gc:/imagemagic/convert zz%%4.ps -trim -density 110x110 -transparent #FFFFFF %%3
\end{verbatim}
\item{A 
Netpbm-based script
(\Link[ftp://ftp.x.org/R5contrib/netpbm-1mar1994.tar.gz]{}{}src\EndLink{} /
binaries for Win 95/98/NT:
\Link[http://sourceforge.net/projects/netpbm/]{}{}Netpbm project\EndLink,
\Link[ftp://miktex.sourceforge.net/pub/miktex/util/]{}{}MiK\TeX\EndLink
)}

\begin{verbatim}
 Gif exist  zz%%4.ps DEL zz%%4.ps
 Gif exist zz%%4.ppm DEL zz%%4.ppm
 Gdvips -T 14in,14in -Ppdf -f %%1 -pp %%2  -o zz%%4.ps
 Ggswin32c -sDEVICE=ppm -sOutputFile=zz%%4.ppm -q -dbatch -dNOPAUSE zz%%4.ps -c quit
 Gmove zz%%4.ppm yz%%4.ppm
 Gpnmcrop  yz%%4.ppm > zz%%4.ppm
 Gif exist %%3 DEL %%3
 Gppmtopng -interlace -transparent 1,1,1 zz%%4.ppm > %%3
\end{verbatim}
 
or similarly

% on dvips: -D 600
% on gs :  -r600x600

\begin{verbatim}
 Gif exist  zz%%4.ps DEL zz%%4.eps
 Gif exist zz%%4.ppm DEL zz%%4.ppm
 Gdvips -E -f %%1 -pp %%2  -o zz%%4.eps
 Ggswin32c -dEPSCrop -dBATCH -dNOPAUSE -sDEVICE=pbm -q -sOutputFile=zz%%4.ppm zz%%4.eps -c quit
 Gmove zz%%4.ppm yz%%4.ppm
 Gpnmcrop  yz%%4.ppm > zz%%4.ppm
 Gppmtopng -interlace -transparent 1,1,1 zz%%4.ppm > %%3
\end{verbatim}

\item {IrfanView}
\Link[http://www.irfanview.com]{}{}IrfanView\EndLink{}
offers a possible alternative for the
\`'convert' utility of ImageMagick.

\item{PDF-based script}
\Link{}{pdf-based}\EndLink {A 
(\Link[http://www.cs.wisc.edu/\string
   ~ghost/doc/AFPL/6.50/Ps2pdf.htm]{}{}ps2pdf\EndLink{})}
\begin{verbatim}
Gdvips -mode ibmvga -D 110 -f %%1 -pp %%2  > zz%%4.ps
Gps2pdf zz%%4.ps zz%%4.pdf
Gconvert zz%%4.pdf -trim -density 110x110 -transparent '#FFFFFF' %%3
\end{verbatim}


\end{itemize}



%%%%%%%%%%%%%%%%%%%%
\section{Open Office}
%%%%%%%%%%%%%%%%%%%%



%%%%%%%%%%%%%%%%%%%%
\subsection{Unix}
%%%%%%%%%%%%%%%%%%%%


A `.foo command' record in the script asks to execute the given 
command for each entry 'File: abc.foo' in the .lg file.  A \verb=%%0=
stands for the job name, a \verb=%%1= stands for the file name
(e.g., 'abc') without  the extension. A \verb=%%~= stands for the
value  of \verb=${SELFAUTOPARENT}= under KPATHSEA, and just `\verb=~='
when such a value is unavailable.


\<unix openoffice\><<<
<oo>
%%% openoffice                                                      %
.4oo mkdir sxw-%%0.dir
.4os mkdir sxw-%%0.dir/Pictures
.4os mv    %%1.4os         sxw-%%0.dir/settings.xml
.4oo mv    %%0.4oo         sxw-%%0.dir/content.xml
%                                                                   %
.4om mkdir sxw-%%0.dir/%%1
.4om mv    %%1.4om         sxw-%%0.dir/%%1/content.xml
.4om cp    sxw-%%0.dir/settings.xml  sxw-%%0.dir/%%1/settings.xml
%                                                                   %
.4of mkdir sxw-%%0.dir/META-INF
.4of mv    %%1.4of         sxw-%%0.dir/META-INF/manifest.xml
%                                                                   %
.4ot mv    %%1.4ot         sxw-%%0.dir/meta.xml
.4oy mv    %%1.4oy         sxw-%%0.dir/styles.xml
% .4og cp  --parents %%1   Pictures    
% .4og mkdir -p              sxw-%%0.dir/Pictures/%%1 
% .4og rmdir                 sxw-%%0.dir/Pictures/%%1 
.4og cp    %%1             sxw-%%0.dir/Pictures/
%-- sxw --                                                          %
.4ox cd sxw-%%0.dir; zip -r %%0.sxw * ; cd ..
%                                                                   %
.4ox mv    sxw-%%0.dir/%%0.sxw .
%                                                                   %
.4ox rm    %%1.4ox 
.4ox rm -r -f sxw-%%0.dir
%-- odt --                                                          %
.4od cd sxw-%%0.dir; zip -r %%0.odt * ; cd ..
%                                                                   %
.4od mv    sxw-%%0.dir/%%0.odt .
%                                                                   %
.4od rm    %%1.4od 
.4od rm -r -f sxw-%%0.dir
%%% end openoffice                                                  %
</oo>
>>>

%%%%%%%%%%%%%%%%%%%%%%%%%%%
\subsection{Win32}
%%%%%%%%%%%%%%%%%%%%%%%%%%%




\<win32 openoffice\><<<
<oo> 
%%% openoffice                                                     %
.4oo if exist sxw-%%0.dir rmdir /Q /S sxw-%%0.dir
.4oo mkdir sxw-%%0.dir
.4os mkdir sxw-%%0.dir\Pictures
.4os move  %%1.4os                  sxw-%%0.dir\settings.xml
.4oo move  %%0.4oo                  sxw-%%0.dir\content.xml
%
.4om mkdir sxw-%%0.dir\%%1
.4om move  %%1.4om                  sxw-%%0.dir\%%1\content.xml
.4om copy  sxw-%%0.dir\settings.xml sxw-%%0.dir\%%1\settings.xml
%
.4of mkdir sxw-%%0.dir\META-INF
.4of move  %%1.4of sxw-%%0.dir\META-INF\manifest.xml
%
.4ot move  %%1.4ot    sxw-%%0.dir\meta.xml
.4oy move  %%1.4oy    sxw-%%0.dir\styles.xml
.4og copy  %%1        sxw-%%0.dir\Pictures\
.4od cd    sxw-%%0.dir & zip -r -m  %%0.zip * & cd ..
% .4od cd    sxw-%%0.dir & 7za a -y  %%0.zip * & cd ..
.4od move sxw-%%0.dir\%%0.zip %%0.odt
% 
.4od del %%1.4od
.4od if exist sxw-%%0.dir rmdir /Q /S sxw-%%0.dir
%%% end openoffice      
</oo> 
>>>

7za: \url{http://sourceforge.net/projects/sevenzip/}

Windows uses ampersand character '\verb=&=' as command separator.
There are also conditional separators '\verb=&&=' (execute if previous
command successful) and '\verb=||=' (execute if previous command
failed). It was added in NT so it's not portable to win95(98).



\<win32 openoffice-DELETE\><<<
%%% openoffice                                                     % 
.4oo mkdir sxw-%%0.dir
.4os mkdir sxw-%%0.dir\Pictures
.4os move  %%1.4os                  sxw-%%0.dir\settings.xml
.4oo move  %%0.4oo                  sxw-%%0.dir\content.xml
%
.4om mkdir sxw-%%0.dir\%%1
.4om move  %%1.4om                  sxw-%%0.dir\%%1\content.xml
.4om copy  sxw-%%0.dir\settings.xml sxw-%%0.dir\%%1\settings.xml
%
.4of mkdir sxw-%%0.dir\META-INF
.4of move  %%1.4of sxw-%%0.dir\META-INF\manifest.xml
%
.4ot move  %%1.4ot    sxw-%%0.dir\meta.xml
.4oy move  %%1.4oy    sxw-%%0.dir\styles.xml
.4og copy  %%1        sxw-%%0.dir\Pictures\%%1
%-- sxw --
.4ox cd    sxw-%%0.dir\
.4ox cd    sxw-%%0.dir\META-INF
.4ox zip   sxw-%%0.dir\%%0.sxw        sxw-%%0.dir\META-INF\*.*
.4ox cd    ..
.4ox cd    sxw-%%0.dir\Pictures
.4ox zip   sxw-%%0.dir\%%0.sxw        sxw-%%0.dir\Pictures
.4ox cd ..
.4ox zip -j sxw-%%0.dir\%%0.sxw       sxw-%%0.dir\*
.4ox cd ..
%
.4ox move sxw-%%0.dir\%%0.sxw .
%
.4ox del %%1.4ox
.4ox echo y | del sxw-%%0.dir\Pictures\*.*
.4ox rd sxw-%%0.dir\Pictures
.4ox echo y | del sxw-%%0.dir\META-INF\*.*
.4ox rd sxw-%%0.dir\META-INF
.4ox echo y | del sxw-%%0.dir\*.*
.4ox rd /S sxw-%%0.dir
%-- odt --
.4od cd    sxw-%%0.dir\
.4od cd    sxw-%%0.dir\META-INF
.4od zip   sxw-%%0.dir\%%0.odt        sxw-%%0.dir\META-INF\*.*
.4od cd    ..
.4od cd    sxw-%%0.dir\Pictures
.4od zip   sxw-%%0.dir\%%0.odt        sxw-%%0.dir\Pictures
.4od cd ..
.4od zip -j sxw-%%0.dir\%%0.odt       sxw-%%0.dir\*
.4od cd ..
%
.4od move sxw-%%0.dir\%%0.odt .
%
.4od del %%1.4od
.4od echo y | del sxw-%%0.dir\Pictures\*.*
.4od rd sxw-%%0.dir\Pictures
.4od echo y | del sxw-%%0.dir\META-INF\*.*
.4od rd sxw-%%0.dir\META-INF
.4od echo y | del sxw-%%0.dir\*.*
.4od rd sxw-%%0.dir
% (1) if  'echo y | del ...'  does not work try  'del /Q ...'  instead
% (2) if rd fails to do the job, try replacing 'rd' with 'rd /s /q' 
%%% end openoffice                                                  % 
</oo> 
>>> 


t.m.trzeciak (Tolmek):

\begin{verbatim}
'rd' won't remove non-empty directory. It should be: 
rd /s /q sxw-test-tex4ht.dir 
\end{verbatim}



Contributed by Hans-Peter Kohler:

\<win32 openofficeDELETE\><<<
%
% Activate the following non-concurent sub script, if the previous
% one does not work. That is, interchange <oo> with <oo-alt>
% and </oo> with </oo-alt>.
%
<oo-alt>
%%% openoffice                                                     %
.4os mkdir Pictures
.4os move  %%1.4os settings.xml
.4oo move  %%0.4oo content.xml
%
.4om mkdir %%1
.4om move  %%1.4om      %%1\content.xml
.4om copy  settings.xml %%1\settings.xml
%
.4of mkdir META-INF
.4of move  %%1.4of META-INF\manifest.xml
%
.4ot move  %%1.4ot    meta.xml
.4oy move  %%1.4oy    styles.xml
.4og copy  %%1        Pictures\%%1
%-- sxw ---
.4ox zip -m %%0.sxw Pictures
.4ox zip -m %%0.sxw settings.xml
.4ox zip -m %%0.sxw content.xml
.4ox zip -m -r %%0.sxw %%0-m*
.4ox zip -m -r %%0.sxw META-INF
.4ox zip -m %%0.sxw meta.xml
.4ox zip -m %%0.sxw styles.xml
.4ox zip -m -r %%0.sxw Pictures/*
%-- odt ---
.4od zip -m %%0.odt Pictures
.4od zip -m %%0.odt settings.xml
.4od zip -m %%0.odt content.xml
.4od zip -m -r %%0.odt %%0-m*
.4od zip -m -r %%0.odt META-INF
.4od zip -m %%0.odt meta.xml
.4od zip -m %%0.odt styles.xml
.4od zip -m -r %%0.odt Pictures/*
%%% end openoffice                                                  %
</oo-alt>
>>>







Contributed: Andrzej Or�owski-Skoczyk; Modified: Wilfried Hennings:

\<win32 openofficeDELETE\><<<
<oo-alt-2>
%%% openoffice                                                     %
.4os mkdir sxw-%%0.dir
.4os mkdir Pictures
.4os move  %%1.4os                  sxw-%%0.dir\settings.xml
.4os move  %%0.4oo                  sxw-%%0.dir\content.xml
%
.4om mkdir %%1
.4om move  %%1.4om                  %%1\content.xml
.4om copy  sxw-%%0.dir\settings.xml %%1\settings.xml
.4om zip  %%0.zip       %%1\*
.4om del /Q %%1\*
.4om rmdir %%1
%
.4of mkdir META-INF
.4of move  %%1.4of META-INF\manifest.xml
%
.4ot move  %%1.4ot    sxw-%%0.dir\meta.xml
.4oy move  %%1.4oy    sxw-%%0.dir\styles.xml
.4og copy  %%1        Pictures\%%1
%-- sxw --
.4ox move %%0.zip %%0.sxw
.4ox zip   %%0.sxw        META-INF\*.*
.4ox zip   %%0.sxw        Pictures
.4ox zip -j %%0.sxw       sxw-%%0.dir\*
%
.4ox del %%1.4ox
.4ox del /Q Pictures\*.*
.4ox rd Pictures
.4ox del /Q META-INF\*.*
.4ox rd META-INF
.4ox del /Q sxw-%%0.dir\*.*
.4ox rd /S sxw-%%0.dir
.4ox del /Q *.tmp
%-- odt --
.4od move %%0.zip %%0.odt
.4od zip %%0.odt       META-INF\*.*
.4od zip %%0.odt       Pictures
.4od zip -j %%0.odt    sxw-%%0.dir\*
%
.4od del %%1.4od
.4od del /Q Pictures\*.*
.4od rd Pictures
.4od del /Q META-INF\*.*
.4od rd META-INF
.4od del /Q sxw-%%0.dir\*.*
.4od rd sxw-%%0.dir
.4od del /Q *.tmp
% if rd fails to do the job, try replacing 'rd' with 'rd /s /q' 
%%% end openoffice    
</oo-alt-2>
>>>


\<win32 openoffice-AndrzejDELETE\><<<
<oo-alt-2>
%%% openoffice                                                     %
.4os mkdir sxw-%%0.dir
.4os mkdir Pictures
.4os move  %%1.4os                  sxw-%%0.dir\settings.xml
.4os move  %%0.4oo                  sxw-%%0.dir\content.xml
%
.4om mkdir %%1
.4om move  %%1.4om                  %%1\content.xml
.4om copy  sxw-%%0.dir\settings.xml %%1\settings.xml
%
.4of mkdir META-INF
.4of move  %%1.4of META-INF\manifest.xml
%
.4ot move  %%1.4ot    sxw-%%0.dir\meta.xml
.4oy move  %%1.4oy    sxw-%%0.dir\styles.xml
.4og copy  %%1        Pictures\%%1
%-- sxw --
.4ox zip   %%0.sxw        META-INF\*.*
.4ox zip   %%0.sxw        Pictures
.4ox zip -j %%0.sxw       sxw-%%0.dir\*
%
.4ox del %%1.4ox
.4ox del /Q Pictures\*.*
.4ox rd Pictures
.4ox del /Q META-INF\*.*
.4ox rd META-INF
.4ox del /Q sxw-%%0.dir\*.*
.4ox rd /S sxw-%%0.dir
%-- odt --
.4od zip %%0.odt       META-INF\*.*
.4od zip %%0.odt       Pictures
.4od zip -j %%0.odt    sxw-%%0.dir\*
%
.4od del %%1.4od
.4od del /Q Pictures\*.*
.4od rd Pictures
.4od del /Q META-INF\*.*
.4od rd META-INF
.4od del /Q sxw-%%0.dir\*.*
.4od rd sxw-%%0.dir
% if rd fails to do the job, try replacing 'rd' with 'rd /s /q' 
%%% end openoffice            
</oo-alt-2>
>>>





Contributed by Andreas Gobel:


\<win32 openoffice\><<<
%
% Platforms (e.g.,  cygwin) with cp and mv utilities of
% http://www.mingw.org/msys.shtml may use the following script.
% It is capable of moving pictures from subdirectories required,
% for instance, by \includegraphics{bitmaps/test.eps}
%
<oo-msys> 
%%% openoffice                                                     %  
.4os mkdir Pictures  
.4os mv  %%1.4os settings.xml             
.4oo mv  %%0.4oo content.xml  
%  
.4om mkdir %%1  
.4om mv  %%1.4om      %%1\content.xml  
.4om cp  settings.xml %%1\settings.xml    
%  
.4of mkdir META-INF  
.4of mv  %%1.4of META-INF\manifest.xml  
%  
.4ot mv  %%1.4ot    meta.xml  
.4oy mv  %%1.4oy    styles.xml  
.4og cp   %%1        Pictures\
%-- sxw --  
.4ox zip  %%0.sxw Pictures  
.4ox zip  %%0.sxw settings.xml            
.4ox zip  %%0.sxw content.xml  
.4ox zip  -r %%0.sxw %%0-m*  
.4ox zip  -r %%0.sxw META-INF  
.4ox zip  %%0.sxw meta.xml  
.4ox zip  %%0.sxw styles.xml  
.4ox zip  -r %%0.sxw Pictures/*           
%-- odt --  
.4od zip  %%0.odt Pictures  
.4od zip  %%0.odt settings.xml            
.4od zip  %%0.odt content.xml  
.4od zip  -r %%0.odt %%0-m*  
.4od zip  -r %%0.odt META-INF  
.4od zip  %%0.odt meta.xml  
.4od zip  %%0.odt styles.xml  
.4od zip  -r %%0.odt Pictures/*           
%  
%%% end openoffice                                          % 
</oo-msys> 
>>> 
 




%%%%%%%%%%%%%
\subsection{xtpipes}
%%%%%%%%%%%%%







\<xtpipes: my oo\><<<
<ooxtpipes>
.4oo mv %%0.4oo %%0.tmp 
.4oo java -classpath %%~/texmf-dist/tex4ht/bin/tex4ht.jar xtpipes -i %%~/texmf-dist/tex4ht/xtpipes/ -o %%0.4oo %%0.tmp 
.4om mv %%1.4om %%1.tmp
.4om java -classpath %%~/texmf-dist/tex4ht/bin/tex4ht.jar xtpipes -i %%~/texmf-dist/tex4ht/xtpipes/ -o %%1.4om %%1.tmp
</ooxtpipes>
>>>


\<xtpipes: unix oo\><<<
<ooxtpipes>
.4oo mv %%0.4oo %%0.tmp 
.4oo java -classpath %%~/texmf-dist/tex4ht/bin/tex4ht.jar xtpipes -i %%~/texmf-dist/tex4ht/xtpipes/ -o %%0.4oo %%0.tmp 
.4om mv %%1.4om %%1.tmp
.4om java -classpath %%~/texmf-dist/tex4ht/bin/tex4ht.jar xtpipes -i %%~/texmf-dist/tex4ht/xtpipes/ -o %%1.4om %%1.tmp
</ooxtpipes>
>>>


\<xtpipes: win32 oo\><<<
<ooxtpipes>
.4oo move %%0.4oo %%0.tmp
.4oo java -classpath  c:/tex4ht/texmf/tex4ht/bin/tex4ht.jar xtpipes -i c:\tex4ht\texmf\tex4ht\xtpipes\ -o %%0.4oo %%0.tmp
.4om move %%1.4om %%1.tmp
.4om java -classpath  c:/tex4ht/texmf/tex4ht/bin/tex4ht.jar xtpipes -i c:\tex4ht\texmf\tex4ht\xtpipes\ -o %%1.4om %%1.tmp
</ooxtpipes>
>>>




%%%%%%%%%%%%%%%%%%%%%%%%%%%
\section{JavaHelp}
%%%%%%%%%%%%%%%%%%%%%%%%%%%

\<xtpipes: my java help\><<<
<javahelp>
.html mv %%1.html %%1.tmp
.html java -classpath %%~/texmf-dist/tex4ht/bin/tex4ht.jar xtpipes -i %%~/texmf-dist/tex4ht/xtpipes/ -s javahelp.4xt -o %%1.html %%1.tmp 
.jhm echo "java -jar ${HOME}/javahelp.dir/jh2.0/javahelp/bin/jhindexer.jar -db %%0-doc/%%1-jhs %%1*.html"
.jhm echo "--- Warning --- Beware of leftover %%1*.html files from other compilations"
.jhm java -jar ${HOME}/javahelp.dir/jh2.0/javahelp/bin/jhindexer.jar -db %%0-doc/%%1-jhs %%1*.html
</javahelp>
<javahelp-prog>
.java rm -r -f %%0-java.dir
.java mkdir %%0-java.dir
.java cp *.java %%0-java.dir/.
.java javac -classpath ${HOME}/javahelp.dir/jh2.0/javahelp/lib/jh.jar %%0-java.dir/%%1.java
.java chdir %%0-java.dir; jar -cf %%1.jar *.class ;  chdir  ..
.java mv %%0-java.dir/%%1.jar .
.java rm -r -f %%0-java.dir
.java zip -r %%1 %%0-doc %%1.jar
</javahelp-prog>
>>>


\<xtpipes: my java help\><<<
<javahelp1>
.jhm java -jar ${HOME}/javahelp.dir/jh1.1.3/javahelp/bin/jhindexer.jar -db %%0-doc/%%1-jhs %%1*.html
</javahelp1>
<javahelp1-prog>
.java rm -r -f %%0-java.dir
.java mkdir %%0-java.dir
.java cp *.java %%0-java.dir/.
.java javac -classpath ${HOME}/javahelp.dir/jh1.1.3/javahelp/lib/jh.jar %%0-java.dir/%%1.java
.java chdir %%0-java.dir; jar -cf %%1.jar *.class ;  chdir  ..
.java mv %%0-java.dir/%%1.jar .
.java rm -r -f %%0-java.dir
.java zip -r %%1 %%0-doc %%1.jar
</javahelp1-prog>
>>>


\begin{verbatim}
jhlatex jhprog "html,3,xml,xmldtd"
 javac -classpath ${HOME}/javahelp.dir/jh2.0/javahelp/lib/jh.jar jhprog.java 

jh1latex jhprog "html,3,xml,xmldtd"
javac -classpath ${HOME}/javahelp.dir/jh1.1.3/javahelp/lib/jh.jar jhprog.java 
\end{verbatim}


\<xtpipes: unix java help\><<<
<javahelp>
.html mv %%1.html %%1.tmp
.html java -classpath  %%~/texmf-dist/tex4ht/bin/tex4ht.jar xtpipes -i %%~/texmf-dist/tex4ht/xtpipes/ -s javahelp.4xt -o %%1.html %%1.tmp
.jhm echo "java -jar ${HOME}/jh2.0/javahelp/bin/jhindexer.jar -db %%0-doc/%%1-jhs %%1*.html"
.jhm echo "--- Warning ---  Beware of leftover %%1*.html files from other compilations"
.jhm java -jar ${HOME}/jh2.0/javahelp/bin/jhindexer.jar -db %%0-doc/%%1-jhs %%1*.html
</javahelp>
<javahelp1>
.jhm java -jar ${HOME}/jh1.1.3/javahelp/bin/jhindexer.jar -db %%0-doc/%%1-jhs %%1*.html
</javahelp1>
>>>


\<xtpipes: win32 java help\><<<
<javahelp>
.html move %%1.html %%1.tmp
.html java -classpath  c:\tex4ht\texmf\tex4ht\bin\tex4ht.jar xtpipes -i c:\tex4ht\texmf\tex4ht\xtpipes\ -s javahelp.4xt -o %%1.html %%1.tmp
.jhm java -jar c:\jh2.0\javahelp\bin\jhindexer.jar -db %%0-doc\%%1-jhs %%1*.html
</javahelp>
<javahelp1>
.jhm java -jar c:\jh1.1.3\javahelp\bin\jhindexer.jar -db %%0-doc\%%1-jhs %%1*.html
</javahelp1>
>>>

%%%%%%%%%%%%%%%%%%
\section{Access to OS}
%%%%%%%%%%%%%%%%%%

\<unix t4ht commands\><<<
% t4ht -d%%2                                                        %
Mmv %%1 %%2%%3
Ccp %%1 %%2%%3
% t4ht -d%%2 -m%%1                                                  %
Achmod %%1 %%2%%3
% removed protection                                                %
S*
>>>


\<my t4ht commands\><<<
% t4ht -d%%2                                                        %
Mmv %%1 %%2%%3
Ccp %%1 %%2%%3
% t4ht -e%%1 -d%%2                                                  %
Ecp empty.png %%1%%2
% t4ht -d%%2 -m%%1                                                  %
Achmod %%1 %%2%%3
% removed protection                                                %
S*
>>>

\<dos t4ht commands\><<<
% t4ht -d%%2                                                        %
Mmove %%1 %%2%%3
Ccopy %%1 %%2%%3
% removed protection                                                %
S*
>>>

Note: the permission option `S' takes into account also trailing spaces.


%%%%%%%%%%%%%%%%%%
\section{Validation}
%%%%%%%%%%%%%%%%%%



\<unix validate\><<<
<validatehtml>
 .html xmllint --noout --valid --html %%1.html
</validatehtml>
<validate>
 .xml xmllint --noout --valid %%1.xml
 .html xmllint --noout --valid %%1.html
 .css mycssparser %%1.css
</validate>
>>>

%  Xmake -f NSGMLS name=%%1  ext=%%2


\<win32 validate\><<<
<validatehtml>
</validatehtml>
<validate>
 .xml xmllint --noout --valid %%1.xml
 .html xmllint --noout --valid %%1.html
 .css mycssparser %%1.css
</validate>
>>>


% Xif "%%2"=="html" nsgmls -wxml -wall -c catalog -s xml.dcl %%1.%%2
% Xecho "*****************************************"




\def\.#1.{\csname #1\endcsname}


\<my validate\><<<
<validatehtml>
.html xmllint --noout --valid --html %%1.html
.xml xmllint --noout --valid %%1.xml
 .hs java -cp @.HOME.texmf/tex4ht/bin/xv4ht.jar xv4ht %%1.hs @.HOME.texmf/tex4ht/base/solaris/xv4ht.cat
 .jhm java -cp @.HOME.texmf/tex4ht/bin/xv4ht.jar xv4ht %%1.jhm @.HOME.texmf/tex4ht/base/solaris/xv4ht.cat
.css vcss %%1.css
</validatehtml>
<validate>
.xml xmllint --noout --valid %%1.xml
.html xmllint --noout --valid %%1.html
.css vcss %%1.css
 .4om echo "rxp -V -s -e %%1.4om"   
 .4om rxp -V -s -e %%1.4om
 .4oo xmllint --relaxng ~/dtd.dir/openoffice/relaxng.dir/OpenDocument-schema-v1.0-os.rng -noout %%0.4oo
.4oo xmllint --relaxng  ~/dtd.dir/openoffice/relaxng.dir/OpenDocument-schema-v1.1.rng -noout %%0.4oo
 .4oy xmllint --relaxng ~/dtd.dir/openoffice/relaxng.dir/OpenDocument-schema-v1.0-os.rng -noout %%0-styles.4oy
.4oy xmllint --relaxng  ~/dtd.dir/openoffice/relaxng.dir/OpenDocument-schema-v1.1.rng -noout %%0-styles.4oy
.4om perl -pe 's/math://g' %%1.4om | perl -pe 's/xmlns:[^=]*="[^"]*"//g' | perl -pe 's/xlink:[^=]*="[^"]*"//g' > tmp.tmp ; xmllint --valid --noout --catalogs tmp.tmp
</validate>
 Xmake -f \HOME texmf/base/NSGMLS name=%%1  ext=%%2 %%3
>>>





%%%%%%%%%%%%%%%%%%%%%%%%%%%
\section{Emacspeak}
%%%%%%%%%%%%%%%%%%%%%%%%%%%

\<xtpipes: my emacspeak\><<<
<emspk>
.html mv %%1.html %%1.tmp
.html java -classpath %%~/texmf-dist/tex4ht/bin/tex4ht.jar xtpipes -i %%~/texmf-dist/tex4ht/xtpipes/ -o %%1.html %%1.tmp
.html xmllint --noout %%1.html 
</emspk>
>>>


\<xtpipes: unix emacspeak\><<<
<emspk>
.html mv %%1.html %%1.tmp
.html java -classpath  %%~/texmf-dist/tex4ht/bin/tex4ht.jar xtpipes -i %%~/texmf-dist/tex4ht/xtpipes/ -o %%1.html %%1.tmp
</emspk>
>>>


\<xtpipes: win32 emacspeak\><<<
<emspk>
.html move %%1.html %%1.tmp
.html java -classpath  c:\tex4ht\texmf\tex4ht\bin\tex4ht.jar xtpipes -i c:\tex4ht\texmf\tex4ht\xtpipes\ -o %%1.html %%1.tmp
</emspk>
>>>


%%%%%%%%%%%%%%%%%%%%%%%%%%%
\section{JSML}
%%%%%%%%%%%%%%%%%%%%%%%%%%%

\<xtpipes: my jsml\><<<
<jsml>
.jsml mv %%1.jsml %%1.tmp
.jsml java -classpath %%~/texmf-dist/tex4ht/bin/tex4ht.jar xtpipes -i %%~/texmf-dist/tex4ht/xtpipes/ -o %%1.jsml %%1.tmp
.jsml xmllint --noout %%1.jsml 
</jsml>
>>>


\<xtpipes: unix jsml\><<<
<jsml>
.jsml mv %%1.jsml %%1.tmp
.jsml java -classpath %%~/texmf-dist/tex4ht/bin/tex4ht.jar xtpipes -i %%~/texmf-dist/tex4ht/xtpipes/ -o %%1.jsml %%1.tmp
</jsml>
>>>


\<xtpipes: win32 jsml\><<<
<jsml>
.jsml move %%1.jsml %%1.tmp
.jsml java -classpath c:\tex4ht\texmf\tex4ht\bin\tex4ht.jar xtpipes -i c:\tex4ht\texmf\tex4ht\xtpipes\ -o %%1.jsml %%1.tmp
</jsml>
>>>







%%%%%%%%%%%%%%%%%%%%%%%%%%%
\section{DocBook}
%%%%%%%%%%%%%%%%%%%%%%%%%%%

\<xtpipes: my docbk\><<<
<docbk>
.xml cp %%1.xml %%1.tmp
.xml java -classpath %%~/texmf-dist/tex4ht/bin/tex4ht.jar xtpipes -i %%~/texmf-dist/tex4ht/xtpipes/ -o %%1.xml %%1.tmp
.xml xmllint --relaxng /home/4/gurari/dtd.dir/docbook.dir/docbook-5.0/rng/docbook.rng -noout -dropdtd -nonet %%0.xml
</docbk>
>>>


\<xtpipes: unix docbk\><<<
<docbk>
.xml cp %%1.xml %%1.tmp
.xml java -classpath %%~/texmf-dist/tex4ht/bin/tex4ht.jar xtpipes -i %%~/texmf-dist/tex4ht/xtpipes/ -o %%1.xml %%1.tmp
</docbk>
>>>


\<xtpipes: win32 docbk\><<<
<docbk>
.xml copy %%1.xml %%1.tmp
.xml java -classpath  c:\tex4ht\texmf\tex4ht\bin\tex4ht.jar xtpipes -i c:\tex4ht\texmf\tex4ht\xtpipes\ -o %%1.xml %%1.tmp
</docbk>
>>>








%%%%%%%%%%%%%%%%%%%%%%%%%%%
\section{XHTML}
%%%%%%%%%%%%%%%%%%%%%%%%%%%

\<xtpipes: my xhtml\><<<
<xhtml>
.html mv %%1.html %%1.tmp
.html java -classpath %%~/texmf-dist/tex4ht/bin/tex4ht.jar xtpipes -i %%~/texmf-dist/tex4ht/xtpipes/ -s xhtml.4xt -o %%1.html %%1.tmp 
.html xmllint --noout %%1.html 
.xht mv %%1.xht %%1.tmp
.xht java -classpath %%~/texmf-dist/tex4ht/bin/tex4ht.jar xtpipes -i %%~/texmf-dist/tex4ht/xtpipes/ -s xhtml.4xt -o %%1.xht %%1.tmp 
.xht xmllint --noout %%1.xht 
.xml mv %%1.xml %%1.tmp
.xml java -classpath %%~/texmf-dist/tex4ht/bin/tex4ht.jar xtpipes -i %%~/texmf-dist/tex4ht/xtpipes/ -s xhtml.4xt -o %%1.xml %%1.tmp 
.xml xmllint --noout %%1.xml 
</xhtml>
>>>


\<xtpipes: unix xhtml\><<<
<xhtml>
.html mv %%1.html %%1.tmp
.html java -classpath  %%~/texmf-dist/tex4ht/bin/tex4ht.jar xtpipes -i %%~/texmf-dist/tex4ht/xtpipes/ -s xhtml.4xt -o %%1.html %%1.tmp
.xht mv %%1.xht %%1.tmp
.xht java -classpath  %%~/texmf-dist/tex4ht/bin/tex4ht.jar xtpipes -i %%~/texmf-dist/tex4ht/xtpipes/ -s xhtml.4xt -o %%1.xht %%1.tmp
.xml mv %%1.xml %%1.tmp
.xml java -classpath  %%~/texmf-dist/tex4ht/bin/tex4ht.jar xtpipes -i %%~/texmf-dist/tex4ht/xtpipes/ -s xhtml.4xt -o %%1.xml %%1.tmp
</xhtml>
>>>


\<xtpipes: win32 xhtml\><<<
<xhtml>
.html move %%1.html %%1.tmp
.html java -classpath  c:\tex4ht\texmf\tex4ht\bin\tex4ht.jar xtpipes -i c:\tex4ht\texmf\tex4ht\xtpipes\ -s xhtml.4xt -o %%1.html %%1.tmp
.xht move %%1.xht %%1.tmp
.xht java -classpath  c:\tex4ht\texmf\tex4ht\bin\tex4ht.jar xtpipes -i c:\tex4ht\texmf\tex4ht\xtpipes\ -s xhtml.4xt -o %%1.xht %%1.tmp
.xml move %%1.xml %%1.tmp
.xml java -classpath  c:\tex4ht\texmf\tex4ht\bin\tex4ht.jar xtpipes -i c:\tex4ht\texmf\tex4ht\xtpipes\ -s xhtml.4xt -o %%1.xml %%1.tmp
</xhtml>
>>>



%%%%%%%%%%%%%%%%%%%%%%%%%%%
\section{XHTML + MathML}
%%%%%%%%%%%%%%%%%%%%%%%%%%%

\<xtpipes: my xhtmml\><<<
<xhtmml>
.html mv %%1.html %%1.tmp
.html java -classpath %%~/texmf-dist/tex4ht/bin/tex4ht.jar xtpipes -i %%~/texmf-dist/tex4ht/xtpipes/ -s xhtmml.4xt -o %%1.html %%1.tmp 
.html xmllint --noout %%1.html 
.xht mv %%1.xht %%1.tmp
.xht java -classpath %%~/texmf-dist/tex4ht/bin/tex4ht.jar xtpipes -i %%~/texmf-dist/tex4ht/xtpipes/ -s xhtmml.4xt -o %%1.xht %%1.tmp 
.xht xmllint --noout %%1.xht 
.xml mv %%1.xml %%1.tmp
.xml java -classpath %%~/texmf-dist/tex4ht/bin/tex4ht.jar xtpipes -i %%~/texmf-dist/tex4ht/xtpipes/ -s xhtmml.4xt -o %%1.xml %%1.tmp 
.xml xmllint --noout %%1.xml 
</xhtmml>
>>>


\<xtpipes: unix xhtmml\><<<
<xhtmml>
.html mv %%1.html %%1.tmp
.html java -classpath  %%~/texmf-dist/tex4ht/bin/tex4ht.jar xtpipes -i %%~/texmf-dist/tex4ht/xtpipes/ -s xhtmml.4xt -o %%1.html %%1.tmp
.xht mv %%1.xht %%1.tmp
.xht java -classpath  %%~/texmf-dist/tex4ht/bin/tex4ht.jar xtpipes -i %%~/texmf-dist/tex4ht/xtpipes/ -s xhtmml.4xt -o %%1.xht %%1.tmp
.xml mv %%1.xml %%1.tmp
.xml java -classpath  %%~/texmf-dist/tex4ht/bin/tex4ht.jar xtpipes -i %%~/texmf-dist/tex4ht/xtpipes/ -s xhtmml.4xt -o %%1.xml %%1.tmp
</xhtmml>
>>>


\<xtpipes: win32 xhtmml\><<<
<xhtmml>
.html move %%1.html %%1.tmp
.html java -classpath  c:\tex4ht\texmf\tex4ht\bin\tex4ht.jar xtpipes -i c:\tex4ht\texmf\tex4ht\xtpipes\ -s xhtmml.4xt -o %%1.html %%1.tmp
.xht move %%1.xht %%1.tmp
.xht java -classpath  c:\tex4ht\texmf\tex4ht\bin\tex4ht.jar xtpipes -i c:\tex4ht\texmf\tex4ht\xtpipes\ -s xhtmml.4xt -o %%1.xht %%1.tmp
.xml move %%1.xml %%1.tmp
.xml java -classpath  c:\tex4ht\texmf\tex4ht\bin\tex4ht.jar xtpipes -i c:\tex4ht\texmf\tex4ht\xtpipes\ -s xhtmml.4xt -o %%1.xml %%1.tmp
</xhtmml>
>>>





%%%%%%%%%%%%%%%%%%
\section{Copyright}
%%%%%%%%%%%%%%%%%%

\<TeX4ht copyright\><<<
%
% This work may be distributed and/or modified under the
% conditions of the LaTeX Project Public License, either
% version 1.3c of this license or (at your option) any
% later version. The latest version of this license is in
%   http://www.latex-project.org/lppl.txt
% and version 1.3c or later is part of all distributions
% of LaTeX version 2005/12/01 or later.
%
% This work has the LPPL maintenance status "maintained".
%
% The Current Maintainer of this work
% is the TeX4ht Project <http://tug.org/tex4ht>.
% 
% If you modify this program, changing the 
% version identification would be appreciated.
>>>



\<unix copyright\><<<
% tex4ht.env-unix (@version), generated from @jobname.tex
% (can be used as ~/.tex4ht)
% Copyright 2009-2015 TeX Users Group
% Copyright @CopyYear.1997. Eitan M. Gurari
@<TeX4ht copyright@>>>>

\<private copyright\><<<
% tex4ht.env (private) (@version), generated from @jobname.tex
% (can be used as ~/.tex4ht)
% Copyright 2009-2015 TeX Users Group
% Copyright @CopyYear.1997. Eitan M. Gurari
@<TeX4ht copyright@>>>>

\<win32 copyright\><<<
% tex4ht.env-win32 (@version), generated from @jobname.tex
% Copyright 2009-2015 TeX Users Group
% Copyright @CopyYear.1997. Eitan M. Gurari
@<TeX4ht copyright@>>>>

%%%%%%%%%%%%%%%%%%%%%%%%%%%%%%%%%%%%%%%%%%%%%%%%%%

\OutputCode\<tex4ht\>
\OutputCodE\<tex4ht.env-unix\>
\OutputCodE\<tex4ht.env-win32\>

\end{document}

Date: Thu, 14 Nov 2013 13:08:12 -0800
From: Richard Koch <koch@math.uoregon.edu>
To: Karl Berry <karl@freefriends.org>

[...]

The syntax of tex4ht.env is certainly unclear. Eitan says at the top
that <tag> is ignored if it doesn't start a line, and then has a vague
comment about defaults.

Looking at the <convert> section, everything here seems to call dvips to get a
ps, and then output a graphic, either png or svg. The section has

	G.png prefaced by a space
	G.svg prefaced by no space
	G. prefaced by no space

I guessed this meant that output of either a png or svg was acceptable, and
the preferred method is G.

So as a test, I left everything alone except the existing "G." section, which I
commented out. Then I added my own "G." section, containing the existing
G.png code, because it calls ghostscript rather than convert.

Next there is a netpbm section. The Mac doesn't have netpbm, so this section
seems to do no harm and I left it alone.

The final graphic section has tag <dvipng> and the programs inside seem to
all start with a dvi, skip the ps step, and get either a png or a gif. The
section has

	G.png with no initial space
	G.gif with no initial space
	G. with no initial space

With little evidence, I took this to mean that conversion starting with a dvi
file can end with either a png or a gif, but that the preferred method is in G.
As before, I commented out the old "G." because it used convert, and replaced
it with a "G." section containing the contents of G.png because it uses
Gdvipng instead of convert.

Then I tried a sample file. It worked fine, even when I disabled
convert.  Earlier attempts to change tex4ht.env resulted in no png files
and lots "missing picture" icons in the html, but this time all those
png files were created and the html output looked fine on a browser.

This evidence suggests, but does not prove, that Ghostscript itself
would be a satisfactory conversion method. The change seems to take more
time, but perhaps this is startup time and large projects wouldn't see
much difference.

END OF TODAY'S HOBBY SESSION

Dick
