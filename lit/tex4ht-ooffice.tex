% $Id$
% compile 3 times: latex tex4ht-ooffice
%      htlatex tex4ht-ooffice "xhtml,4,sections+"
% Copyright 2009-2019 TeX Users Group
% Copyright 2001-2009 Eitan M. Gurari
% Released under LPPL 1.3c+.
% See tex4ht-cpright.tex for license text.


%%%%%%%%%%%%%%%%%% load style files %%%%%%%%%%%%%%%%%%%%%%%%%%

\ifx \HTML\UnDef
   \def\HTML{ooffice,ooffice-mml}
   \def\PREAMBLE{%
      \usepackage{url}%
      \usepackage{hyperref}%
   }
   \def\OPTIONS{html,sections+,3}%
   \def\CONFIG{\jobname}
   \def\MAKETITLE{\author{Eitan M. Gurari, 
                          James Naughton, and Maarten Wisse}}         
   \def\next{\input mktex4ht.4ht  \endinput}
   \expandafter\next
\fi


% $Id$
% A few common TeX definitions for literate sources.  Not installed in runtime.
% 
% Copyright 2009-2017 TeX Users Group
% Copyright 1996-2009 Eitan M. Gurari
%
% This work may be distributed and/or modified under the
% conditions of the LaTeX Project Public License, either
% version 1.3c of this license or (at your option) any
% later version. The latest version of this license is in
%   http://www.latex-project.org/lppl.txt
% and version 1.3c or later is part of all distributions
% of LaTeX version 2005/12/01 or later.
%
% This work has the LPPL maintenance status "maintained".
%
% The Current Maintainer of this work
% is the TeX4ht Project <http://tug.org/tex4ht>.
% 
% If you modify this program, changing the 
% version identification would be appreciated.

\newcount\tmpcnt  \tmpcnt\time  \divide\tmpcnt  60
\edef\temp{\the\tmpcnt}
\multiply\tmpcnt  -60 \advance\tmpcnt  \time

\edef\version{\the\year-\ifnum \month<10 0\fi
  \the\month-\ifnum \day<10 0\fi\the\day
   -\ifnum \temp<10 0\fi \temp
   :\ifnum \tmpcnt<10 0\fi\the\tmpcnt}

% a fixed-string version that can be enabled for debugging.
%\edef\versionDebug{000-00-00-00:00}
%\let\version\versionDebug

% #1 is the first year for Eitan.  The last year is always 2009.  RIP.
\def\CopyYear.#1.{#1-2009}

% command for write to terminal and the log file
% this version is used in the .4ht files build
% identical command is defined also in tex4ht-sty.tex, 
% it is used in TeX document compilation
\def\writesixteen#1{\immediate\write1616{#1}}

\<TeX4ht copyright\><<<
%
% This work may be distributed and/or modified under the
% conditions of the LaTeX Project Public License, either
% version 1.3c of this license or (at your option) any
% later version. The latest version of this license is in
%   http://www.latex-project.org/lppl.txt
% and version 1.3c or later is part of all distributions
% of LaTeX version 2005/12/01 or later.
%
% This work has the LPPL maintenance status "maintained".
%
% The Current Maintainer of this work
% is the TeX4ht Project <http://tug.org/tex4ht>.
% 
% If you modify this program, changing the 
% version identification would be appreciated.
>>>


%%%%%%%%%%%%%%%%%%%%%%%%%%%%%%%%%%%%%%%%%%%%%%%%%%%%%%%%%%%%%%%%%%%%%%%%
\chapter{The Code}
%%%%%%%%%%%%%%%%%%%%%%%%%%%%%%%%%%%%%%%%%%%%%%%%%%%%%%%%%%%%%%%%%%%%%%%%

%%%%%%%%%%%%%%%%%%%%%%%%%%%%%%%%%%%%%%%%%%%%%%%%%%%%%%%%%%%%%%%%%%%%%%%%
\section{The Root}
%%%%%%%%%%%%%%%%%%%%%%%%%%%%%%%%%%%%%%%%%%%%%%%%%%%%%%%%%%%%%%%%%%%%%%%%

%%%%%%%%%%%%%%%%%%%%%%%%%%%%%%%%%%%%%%%%%%%%%%%%%%%%%%%%%%%%%%%%%%%%%%%%
\subsection{Imported Configurations}
%%%%%%%%%%%%%%%%%%%%%%%%%%%%%%%%%%%%%%%%%%%%%%%%%%%%%%%%%%%%%%%%%%%%%%%%


\<configure ooffice tex4ht\><<<
\if:latex  |<Hinclude latex|>
\else      |<Hinclude plain|>  \fi
>>>


\<configure ooffice plain\><<<
|<Hinclude plain lib|>
|<Hinclude plain + latex lib|>
>>>

\<configure ooffice latex\><<<
|<Hinclude latex lib|>
|<Hinclude plain + latex lib|>
>>>

%%%%%%%%%%%%%%%%%%%%%%%%%%%%%%%%%%%%%%%%%%%%%%%%%%%%%%%%%%%%%%%%%%%%%%%%
\subsection{Common Code Fragments}
%%%%%%%%%%%%%%%%%%%%%%%%%%%%%%%%%%%%%%%%%%%%%%%%%%%%%%%%%%%%%%%%%%%%%%%%

\<par del\><<<
!*?: >>>

\<tag of Tag\><<<
 cw:>>>

\<tail\><<<
tail>>>

\<addr for Tag and Ref of Sec\><<<
\xdef\:cursec{|<section html addr|>}%
>>>

\<utilities\><<<
\newif\ifHCond
\def\:gobbleM#1->{}
>>>


\<try env inline par\><<<
\ShowPar
>>>

\<try inline par\><<<
\ShowPar\par{\HCondtrue\noindent}%
>>>



%%%%%%%%%%%%%%%%%%%%%%
\subsection{Copyright}
%%%%%%%%%%%%%%%%%%%%%%

\<ooffice\><<<
% ooffice.4ht (|version), generated from |jobname.tex
|<copy right notice|>  >>>

\<ooffice-math\><<<
% ooffice-math.4ht (|version), generated from |jobname.tex
|<copy right notice|>  >>>


\<ooffice-mml\><<<
% ooffice-mml.4ht (|version), generated from |jobname.tex
|<copy right notice|>  >>>


\<copy right notice\><<<
% Copyright 2009-2019 TeX Users Group
% Copyright |CopyYear.2001. Maarten Wisse, James Naughton, Eitan M. Gurari
|<TeX4ht copywrite|>
>>>

\<configure ooffice tex4ht\><<<
|<utilities|>
>>>




%%%%%%%%%%%%%%%%%%%%%%
\section{The XML Files}
%%%%%%%%%%%%%%%%%%%%%%

%%%%%%%%%%%%%%%%%%%%%%
\subsection{content.xml (= jobname.xml)}
%%%%%%%%%%%%%%%%%%%%%%

\<configure ooffice-mml tex4ht\><<<
\Configure{math-xmlns}
  { 
     xmlns="http://www.w3.org/1998/Math/MathML"\Hnewline
     xmlns:xlink="http://www.w3.org/1999/xlink"
 }
>>>

\<configure ooffice-mml Preamble\><<<
\Configure{mathml}{}
>>>

\<configure ooffice Preamble\><<<
\Configure{ext}{4oo}
\Configure{PROLOG}{VERSION,DOCTYPE}
|<od definitions|>
\def\oo:dtd:ver{2.0}
>>>

The DOCTYPE can actually make the ODT file invalid. Just insert xtpipes instruction.

\<configure ooffice tex4ht\><<<
\Configure{VERSION}{\HCode{|<VERSION|>}}
\:CheckOption{-xtpipes} \if:Option 
   \Configure{DOCTYPE}{}
\else
   \Configure{DOCTYPE}{\HCode{<?xtpipes file="oo-text.4xt" ?>\Hnewline}}
\fi
>>>

\<configure ooffice tex4ht\><<<
   |<oo2 tex4ht|>
>>>

\<oo2 tex4ht\><<< 
   {\Configure{Needs}{File: \jobname.odt}\Needs{}}
   \Configure{HTML}
      {\HCode{<office:document-content \Hnewline
          |<xml name spaces 2.0|>%
          office:version="1.0"
       >\Hnewline
       <office:automatic-styles>          
          <?tex4ht-lg file="\jobname.lg" ?>
          \expandafter\ifx\csname a:automatic-styles\endcsname\relax \else
             \csname a:automatic-styles\endcsname
          \fi  \Hnewline
       </office:automatic-styles>
      }}
      {\EndP\HCode{</office:document-content>}\Configure{newpage}{}}
\NewConfigure{automatic-styles}[1]{%
   \expandafter\concat:config\csname a:automatic-styles\endcsname{#1}} 
\expandafter\let\csname a:automatic-styles\endcsname\relax 
|<oo2 automatic styles|>
>>>








\<configure ooffice tex4ht\><<<
\Configure{HEAD}{}{}
>>>



\<oo2 tex4ht\><<<
\Configure{BODY}
  {\HCode{<office:body><office:text>}\par\ShowPar}
  {\IgnorePar\EndP   
   \HCode{</office:text></office:body>\Hnewline}}
>>>



\<xml name spaces 1.0\><<<
xmlns:chart="http://openoffice.org/2000/chart"      \Hnewline
xmlns:dr3d="http://openoffice.org/2000/dr3d"        \Hnewline
xmlns:draw="http://openoffice.org/2000/drawing"     \Hnewline
xmlns:form="http://openoffice.org/2000/form"        \Hnewline
xmlns:number="http://openoffice.org/2000/datastyle" \Hnewline
xmlns:office="http://openoffice.org/2000/office"    \Hnewline
xmlns:script="http://openoffice.org/2000/script"    \Hnewline
xmlns:style="http://openoffice.org/2000/style"      \Hnewline
xmlns:table="http://openoffice.org/2000/table"      \Hnewline
xmlns:text="http://openoffice.org/2000/text"        \Hnewline
xmlns:fo="http://www.w3.org/1999/XSL/Format"        \Hnewline
xmlns:math="http://www.w3.org/1998/Math/MathML"     \Hnewline
xmlns:svg="http://www.w3.org/2000/svg"              \Hnewline
xmlns:xlink="http://www.w3.org/1999/xlink"          \Hnewline
>>>


\<xml name spaces 2.0\><<<
xmlns:office="urn:oasis:names:tc:opendocument:xmlns:office:1.0"   \Hnewline
xmlns:style="urn:oasis:names:tc:opendocument:xmlns:style:1.0"     \Hnewline
xmlns:table="urn:oasis:names:tc:opendocument:xmlns:table:1.0"     \Hnewline
xmlns:draw="urn:oasis:names:tc:opendocument:xmlns:drawing:1.0"    \Hnewline
xmlns:fo="http://www.w3.org/1999/XSL/Format"                      \Hnewline
xmlns:xlink="http://www.w3.org/1999/xlink"                        \Hnewline
xmlns:text="urn:oasis:names:tc:opendocument:xmlns:text:1.0"       \Hnewline
xmlns:dc="http://purl.org/dc/elements/1.1/"                       \Hnewline
xmlns:meta="urn:oasis:names:tc:opendocument:xmlns:meta:1.0"       \Hnewline
xmlns:number="urn:oasis:names:tc:opendocument:xmlns:datastyle:1.0"\Hnewline
xmlns:svg="http://www.w3.org/2000/svg"                            \Hnewline
xmlns:chart="urn:oasis:names:tc:opendocument:xmlns:chart:1.0"     \Hnewline
xmlns:dr3d="urn:oasis:names:tc:opendocument:xmlns:dr3d:1.0"       \Hnewline
xmlns:math="http://www.w3.org/1998/Math/MathML"                   \Hnewline
xmlns:form="urn:oasis:names:tc:opendocument:xmlns:form:1.0"       \Hnewline
xmlns:script="urn:oasis:names:tc:opendocument:xmlns:script:1.0"   \Hnewline
xmlns:ooo="http://openoffice.org/2004/office"                     \Hnewline
xmlns:ooow="http://openoffice.org/2004/writer"                    \Hnewline
xmlns:oooc="http://openoffice.org/2004/calc"                      \Hnewline
xmlns:dom="http://www.w3.org/2001/xml-events"                     \Hnewline
xmlns:xforms="http://www.w3.org/2002/xforms"                      \Hnewline
xmlns:xsd="http://www.w3.org/2001/XMLSchema"                      \Hnewline
xmlns:xsi="http://www.w3.org/2001/XMLSchema-instance"             \Hnewline
xmlns:anim="urn:oasis:names:tc:opendocument:xmlns:animation:1.0"  \Hnewline
xmlns:smil="urn:oasis:names:tc:opendocument:xmlns:smil-compatible:1.0"\Hnewline
xmlns:presentation="urn:oasis:names:tc:opendocument:xmlns:presentation:1.0"\Hnewline
>>>


\<xml styles name spaces 2.0\><<<    
xmlns:anim="urn:oasis:names:tc:opendocument:xmlns:animation:1.0"\Hnewline
xmlns:chart="urn:oasis:names:tc:opendocument:xmlns:chart:1.0"  \Hnewline
xmlns:dc="http://purl.org/dc/elements/1.1/"  \Hnewline
xmlns:dom="http://www.w3.org/2001/xml-events"  \Hnewline
xmlns:dr3d="urn:oasis:names:tc:opendocument:xmlns:dr3d:1.0"  \Hnewline
xmlns:draw="urn:oasis:names:tc:opendocument:xmlns:drawing:1.0"  \Hnewline
xmlns:fo="urn:oasis:names:tc:opendocument:xmlns:xsl-fo-compatible:1.0"\Hnewline
xmlns:form="urn:oasis:names:tc:opendocument:xmlns:form:1.0"  \Hnewline
xmlns:math="http://www.w3.org/1998/Math/MathML"  \Hnewline
xmlns:meta="urn:oasis:names:tc:opendocument:xmlns:meta:1.0"  \Hnewline
xmlns:number="urn:oasis:names:tc:opendocument:xmlns:datastyle:1.0"  \Hnewline
xmlns:office="urn:oasis:names:tc:opendocument:xmlns:office:1.0"\Hnewline
xmlns:ooo="http://openoffice.org/2004/office"  \Hnewline
xmlns:oooc="http://openoffice.org/2004/calc"  \Hnewline
xmlns:ooow="http://openoffice.org/2004/writer"  \Hnewline
xmlns:presentation="urn:oasis:names:tc:opendocument:xmlns:presentation:1.0"\Hnewline
xmlns:script="urn:oasis:names:tc:opendocument:xmlns:script:1.0"  \Hnewline
xmlns:smil="urn:oasis:names:tc:opendocument:xmlns:smil-compatible:1.0"\Hnewline
xmlns:style="urn:oasis:names:tc:opendocument:xmlns:style:1.0"  \Hnewline
xmlns:svg="urn:oasis:names:tc:opendocument:xmlns:svg-compatible:1.0"\Hnewline
xmlns:table="urn:oasis:names:tc:opendocument:xmlns:table:1.0"  \Hnewline
xmlns:text="urn:oasis:names:tc:opendocument:xmlns:text:1.0"\Hnewline
xmlns:xlink="http://www.w3.org/1999/xlink"\Hnewline
>>>

\<VERSION\><<<
<?xml version="1.0" encoding="UTF-8"?>\Hnewline
>>>

\<DOCTYPE\><<<
>>>


\<dtd lang\><<<
\expandafter
\ifx \csname a:dtd-lang\endcsname\relax EN\else
  \csname a:dtd-lang\endcsname
\fi
>>>


\<configure ooffice Preamble\><<<
\let\Css=\:gobble
>>>

%%%%%%%%%%%%%%%%%%%%%%%%%%%%%%%%%%%%%%%%%%%%
\subsection{styles.xml (jobname-styles.xml)}
%%%%%%%%%%%%%%%%%%%%%%%%%%%%%%%%%%%%%%%%%%%%



The contents.xml file can contain only styles that are known ahead of
time, before starting the processing of the content.

The styles.xml file allows us

\begin{enumerate}
\item To collect styles as we go along the body of the document, and
\item Introduce default built in styles that user may override
\end{enumerate}



The content.xml must get the style information at the start of the
compilation, before the body is processed.



\verb+\OOstyles+ immediatey sends it content to the styles.xml
file. In the case of \verb+\ConfigureOO+, the last assignment to the
hook is sent to the file upon reaching the end of the source.


\<utilities\><<<
\let\:OOstyles=\empty
\def\OOstyles#1{\edef\:OOstyles{\:OOstyles #1}}
\OOstyles{|<oo2 styles.xml|>}
>>>

\<switch to main body OOstlyles\><<<
\def\OOstyles##1{%  
\ht:special{t4ht>|<styles file name|>}%
  \ht:special{t4ht=##1}%
  \ht:special{t4ht*>}%
}
\OOstyles{\:OOstyles}
\ht:special{t4ht*>|<styles file name|>}%
\let\:OOstyles=\:UnDef
>>>

\<styles file name\><<<
\jobname-styles.4oy>>>






\<oo2 styles at /@BODY\><<<
\OOstyles{%
  \oo:style
  </office:styles>\Hnewline
     <office:automatic-styles>
        |<styles.xml: automatic-styles|>\Hnewline         
     </office:automatic-styles>
     <office:master-styles>\Hnewline         
        |<styles.xml: master-styles|>\Hnewline         
     </office:master-styles>\Hnewline         
  </office:document-styles>\Hnewline
}%
>>>





\<oo2 styles.xml\><<<
|<VERSION|>%
<office:document-styles \Hnewline
       |<xml styles name spaces 2.0|>\Hnewline
       office:version="1.0">\Hnewline
   <office:font-face-decls>
      |<oo2 fonts|>%
   </office:font-face-decls>
   <office:styles>\Hnewline
>>>





[\HPage{dirty trick for fonts}
\begin{verbatim}
\OOstyles{% 
   </office:styles> 
   <office:font-face-decls> 
     <style:font-face  
         style:name="Palatino Linotype" 
         svg:font-family="&apos;Palatino Linotype&apos;" 
         style:font-family-generic="roman"  
         style:font-pitch="variable"/> 
   </office:font-face-decls> 
   <office:styles> 
       <style:style style:name="grmn-1000" style:family="text"> 
          <style:text-properties style:font-name="Palatino Linotype"/> 
       </style:style> 
} 
\end{verbatim}

\EndHPage{}]

\<utilities\><<<
\let\oo:style=\empty
\def\NewConfigureOO#1{%
  \expandafter\ifx \csname c:#1:oo:style:\endcsname\relax
     \append:def\oo:style{\csname a:#1:oo:style\endcsname}%
     \:CheckOption{info}\if:Option
         \Log:Note{\string\NewConfigureOO{#1}}%
     \fi
  \fi
  \NewConfigure{#1:oo:style}{1}}
\def\ConfigureOO#1#2{%
  \edef\:temp{\noexpand\Configure{#1:oo:style}{#2}}\:temp
  \:CheckOption{info-oo}\if:Option
    \Log:Note{\string\ConfigureOO{#1}{%
       \expandafter\expandafter\expandafter\:gobbleM
       \expandafter\meaning\csname a:#1:oo:style\endcsname
     }}
  \fi
}
>>>

\<utilities\><<<
\:CheckOption{info}\if:Option
    \Log:Note{For tracing of oo configurations, 
       use the `info-oo' option}%
\fi
>>>



%%%%%%%%%%%%%%%%%%%%%%%%%%%%%%%
\subsection{settings.xml for Math Formulas}
%%%%%%%%%%%%%%%%%%%%%%%%%%%%%%%









\<oo2 settings.xml\><<<
\ht:special{t4ht>|<settings file name|>}%
  \ht:special{t4ht=|<VERSION|>}%
  \ht:special{t4ht=|<settings DOCTYPE|>}%
  \ht:special{t4ht=|<oo2 start mathml settings|>}%
  \ht:special{t4ht=|<close mathml settings|>}%
\ht:special{t4ht*>}%
\ht:special{t4ht<|<settings file name|>}%
>>>






\<settings file name\><<<
\jobname-settings.4os>>>

\<settings DOCTYPE\><<<
>>>




\<oo2 start mathml settings\><<<
<office:document-settings \Hnewline
   xmlns:office="urn:oasis:names:tc:opendocument:xmlns:office:1.0" \Hnewline
   xmlns:xlink="http://www.w3.org/1999/xlink"       \Hnewline
   xmlns:config="urn:oasis:names:tc:opendocument:xmlns:config:1.0" \Hnewline
   office:version="1.0">                            \Hnewline
>>>

% <office:settings>                                   \Hnewline


%</office:settings>

\<close mathml settings\><<<
</office:document-settings>\Hnewline
>>>





%%%%%%%%%%%%%%%%%%%%%%%%%%%%%%%
\subsection{manifest.xml}
%%%%%%%%%%%%%%%%%%%%%%%%%%%%%%%


\<utilities\><<<
\let\media:type\relax
\let\:OOmanifest=\empty
\def\OOmanifest#1{\edef\:OOmanifest{\:OOmanifest #1}}
\OOmanifest{|<oo2 manifest.xml|>}
\def\media:type{application/vnd.oasis.opendocument.text}
>>>

\<switch to main body OOmanifest\><<<
\def\OOmanifest##1{%
  \ht:special{t4ht>|<manifest file name|>}%
  \ht:special{t4ht=##1}%
  \ht:special{t4ht*>}%
}
\OOmanifest{\:OOmanifest}
\ht:special{t4ht*>|<manifest file name|>}%
\let\:OOmanifest=\:UnDef
>>>


\<manifest file name\><<<
\jobname-manifest.4of>>>





\<oo2 manifest.xml\><<<
|<VERSION|>%
<manifest:manifest       
       xmlns:manifest =
       "urn:oasis:names:tc:opendocument:xmlns:manifest:1.0">\Hnewline
   <manifest:file-entry
       manifest:media-type="\media:type"
       manifest:full-path="/"/>
   |<shared manifest|>%
>>>


\<manifest at /@BODY\><<<
\OOmanifest{</manifest:manifest>\Hnewline}%
>>>


\<math obj manifest\><<<
<manifest:file-entry 
   manifest:media-type="text/xml" 
   manifest:full-path="|<math obj name|>/content.xml"/>\Hnewline
<manifest:file-entry 
   manifest:media-type="text/xml" 
   manifest:full-path="|<math obj name|>/settings.xml"/>\Hnewline
<manifest:file-entry
    manifest:media-type="application/vnd.oasis.opendocument.formula"
    manifest:full-path="|<math obj name|>/"/>
>>>

\<graphics obj manifest\><<<
<manifest:file-entry
    manifest:full-path="Pictures/\PictureFile"/>\Hnewline
>>>

\<shared manifest\><<<
<manifest:file-entry 
   manifest:media-type=""
   manifest:full-path="Pictures/"/>\Hnewline
<manifest:file-entry 
   manifest:media-type="text/xml" 
   manifest:full-path="content.xml"/>\Hnewline
<manifest:file-entry 
   manifest:media-type="text/xml" 
   manifest:full-path="styles.xml"/>\Hnewline
<manifest:file-entry 
   manifest:media-type="text/xml" 
   manifest:full-path="meta.xml"/>\Hnewline
<manifest:file-entry 
   manifest:media-type="text/xml" 
   manifest:full-path="settings.xml"/>\Hnewline
>>>





%%%%%%%%%%%%%%%%%%%%%%%%%%%%%%%
\subsection{meta.xml}
%%%%%%%%%%%%%%%%%%%%%%%%%%%%%%%




\<oo2 meta.xml\><<<
\ht:special{t4ht>|<meta file name|>}%
  \ht:special{t4ht=|<VERSION|>}%
  \ht:special{t4ht=|<oo2 start meta|>}%
  \ht:special{t4ht=\a:META}%
  \ht:special{t4ht=|<close meta|>}%
\ht:special{t4ht*>}%
\ht:special{t4ht<|<meta file name|>}%
>>>















\<meta file name\><<<
\jobname-meta.4ot>>>





\<oo2 start meta\><<<
<office:document-meta \Hnewline
  xmlns:office="urn:oasis:names:tc:opendocument:xmlns:office:1.0"\Hnewline
  xmlns:xlink="http://www.w3.org/1999/xlink"\Hnewline
  xmlns:dc="http://purl.org/dc/elements/1.1/"\Hnewline
  xmlns:meta="urn:oasis:names:tc:opendocument:xmlns:meta:1.0"\Hnewline
  xmlns:ooo="http://openoffice.org/2004/office"\Hnewline
  office:version="1.0">\Hnewline
<office:meta>\Hnewline
>>>

\<close meta\><<<
</office:meta></office:document-meta>\Hnewline
>>>






\<oo2 tex4ht\><<<
\Configure{META}
   {<meta:initial-creator>|<creator/generator|>\Hnewline
    </meta:initial-creator>\Hnewline}
\Configure{META}
   {<meta:generator>|<creator/generator|>\Hnewline
    </meta:generator>\Hnewline}
\:CheckOption{timestamp}\if:Option
\Configure{META}
   {<meta:creation-date>\:today</meta:creation-date>\Hnewline}
\Configure{META}
   {<dc:date>\:today</dc:date>\Hnewline}
\else
  \Log:Note{To print a timestamp, use the command line option timestamp}
\fi
>>>



   

\<creator/generator\><<<
TeX4ht from \jobname.tex, 
       options:  \expandafter\:gobbleM\meaning\Preamble\Hnewline
      (http://www.cse.ohio-state.edu/\string
       ~gurari/TeX4ht/)
>>>


\<utilities\><<<
\tmp:cnt=\time  \divide\tmp:cnt by 60
\edef\:temp{\the\tmp:cnt}
\multiply\tmp:cnt by -60 \advance\tmp:cnt by \time
\edef\:today{\the\year -\ifnum \month<10 0\fi
  \the\month -\ifnum \day<10 0\fi\the\day
   T\ifnum \:temp<10 0\fi \:temp
   :\ifnum \tmp:cnt<10 0\fi\the\tmp:cnt :00}
>>>

\<utilities\><<<
\NewConfigure{META}[1]{\concat:config\a:META{#1}}
\let\a:META\relax
>>>






%%%%%%%%%%%%%%%%%%%%%%%%%%%%%%%
\subsection{MathML}
%%%%%%%%%%%%%%%%%%%%%%%%%%%%%%%








\<oo2 inline mathml object\><<<
<draw:frame
         draw:style-name="mml-inline" 
         draw:name="mobj-\math:obj"\Hnewline 
         text:anchor-type="as-char"
         draw:z-index="0"
     >%
   <draw:object\Hnewline
      xlink:href="./|<math obj name|>"\Hnewline
      xlink:type="simple" 
      xlink:show="embed" 
      xlink:actuate="onLoad" 
    />%
</draw:frame>%
>>>


\<oo2 display mathml object\><<<
<draw:frame
         draw:style-name="mml-display" 
         draw:name="mobj-\math:obj"\Hnewline 
\ifTab\else         text:anchor-type="paragraph"       \fi
         draw:z-index="0"
     >%
   <draw:object\Hnewline
      xlink:href="./|<math obj name|>"\Hnewline
      xlink:type="simple" 
      xlink:show="embed" 
      xlink:actuate="onLoad" 
    />%
</draw:frame>%
>>>






\<configure ooffice-mml latex\><<<
   |<oo2 mml latex|>
>>>



\<oo2 mml latex\><<<
\OOstyles{\Hnewline
  <style:style style:name="mml-inline"
               style:family="graphic"
               style:parent-style-name="Formula"
  >\Hnewline
     <style:graphic-properties
               style:vertical-pos="middle"
               style:vertical-rel="text"/>\Hnewline
  </style:style>\Hnewline}
>>>



\<oo2 mml latex\><<<
\OOstyles{\Hnewline
    <style:style style:name="mml-display"
                style:family="graphic"
                style:parent-style-name="Formula">
    <style:graphic-properties
                style:wrap="none"
%                style:vertical-pos="center"
%                style:vertical-rel="paragraph-content"
                style:horizontal-pos="center"
                style:horizontal-rel="paragraph-content"
                fo:background-color="transparent"
                style:background-transparency="100\%"
                fo:margin-left="5\%"
      >
      </style:graphic-properties>
    </style:style>\Hnewline
}
>>>



\<configure ooffice-mml latex\><<<
\OOstyles{\Hnewline
   <style:style style:name="Formula"
                style:family="graphic">
      <style:graphic-properties
                text:anchor-type="as-char"
                svg:y="0in"
                fo:margin-left="0.1mm"
                fo:margin-right="0.1mm"
                style:vertical-pos="middle"
                style:vertical-rel="text"
                 />
    </style:style>\Hnewline
}
>>>






\<utilities\><<<
\newif\ifCenteredFrame
>>>


\<oo2 start math obj\><<<
\gHAdvance\math:obj by 1
\NextFile{|<math obj name|>.4om}%
\bgroup
    \HCode{\ifCenteredFrame
       |<oo2 display mathml object|>\else 
       |<oo2 inline mathml object|>\fi
    }%
    \OOmanifest{|<math obj manifest|>}%
    |<config mathml object file|>%
    |<sv dvimath par|>\IgnorePar
    \HPage{}%
>>>




\<end math obj\><<<
\EndHPage{}|<recall dvimath par|>\egroup 
>>>



\<configure ooffice-mml tex4ht\><<<
|<oo2 mml tex4ht|>
>>>





\<oo2 mml tex4ht\><<<
\HAssign\math:obj=0
\Configure{DviMath}
    {\a:DviMathML \mathmltrue
       \ifOption{mml-fonts}{}{|%\mml:htfsym|%\NoFonts}%
       \HCode{|<input line num|>}\IgnoreRule 
       |<oo2 start math obj|>\Tg<\a:mathml math\Hnewline
           \csname a:math-xmlns\endcsname \Hnewline
       >\a:math
    }
    {\Tg</\a:mathml math>%
     |<end math obj|>\EndIgnoreRule
     \ifOption{mml-fonts}{}{|%\ext:htfsym|%\EndNoFonts}%
     \mathmlfalse \b:DviMathML }
>>>



\<config mathml object file\><<<
\Configure{DOCTYPE}{\HCode{<!DOCTYPE math PUBLIC "-//W3C//DTD MathML 2.0//EN" "http://www.w3.org/Math/DTD/mathml2/mathml2.dtd">}}%
|<oo math xtpipes|>%
\Configure{HTML}{}{\noindent}%
\Configure{BODY}{}{}%
>>>


\<oo math xtpipes\><<<
\:CheckOption{-xtpipes} \if:Option 
   \Log:Note{Use the `-xtpipes' command line option to 
             avoid xtpipes post processing}
\else
      \Configure{DOCTYPE}{%%
          \HCode{|<tex4ht math note|>}%
      }%
\fi
>>>

\<tex4ht math note\><<<
<!DOCTYPE math PUBLIC "-//W3C//DTD MathML 2.0//EN" "http://www.w3.org/Math/DTD/mathml2/mathml2.dtd">\Hnewline
<?xtpipes file="oo-math.4xt" ?>\Hnewline
<!-- |<math obj name|>\space
     by TeX4ht from \jobname.tex line \the\inputlineno\space
     |version\Hnewline
     (http://www.cse.ohio-state.edu/\string ~gurari/TeX4ht/)
      -->\Hnewline
>>>





\<recall dvimath par\><<<
\sv:ignore
>>>

\<sv dvimath par\><<<
\edef\sv:ignore{\if:nopar
    \noexpand\IgnorePar\else \noexpand\ShowPar\fi}%
>>>


\<math obj name\><<<
\jobname-m\math:obj
>>>


%%%%%%%%%%%%%%%%%%%%%%%%%%%%%%%
\subsection{Graphics}
%%%%%%%%%%%%%%%%%%%%%%%%%%%%%%%





\<configure ooffice graphics\><<<
\Configure{graphics}
   {\bgroup
      \let\E:IMG=\e:IMG \let\e:IMG=\empty
    \Picture+{}}
   {\EndPicture 
      \E:IMG\HCode{<!--tex4ht:graphics 
      \Hnewline  name="\PictureFile"  src="\Gin@base\Gin@ext"
       svg:width="\the\Gin@req@width"
       svg:height="\the\Gin@req@height"
      \Hnewline -->}%
    \egroup}
>>>

\<\><<<
\:CheckOption{graphics-dim} \if:Option
   |<graphics quality|>
   \Configure{graphics}
      {\bgroup
         \let\E:IMG=\e:IMG \let\e:IMG=\empty
       \Picture+{}}
      {\EndPicture        \ifx\noBoundingBox\UnDefined
           |<pixels graphics width/height|>\fi
          \E:IMG\HCode{<!--tex4ht:graphics 
          \Hnewline  name="\PictureFile"  src="\Gin@base\Gin@ext"
          \Hnewline -->}%
       \egroup}
\else

\fi
>>>




\<pixels graphics width/height\><<<
\ifx \gr:density\empty
   \HCode{
      width="\the\Gin@req@width"
      height="\the\Gin@req@height"
   }%
\else
   \bgroup
     \divide\Gin@req@width by 72 
     \multiply\Gin@req@width by \gr:density \relax   
     \divide\Gin@req@height by 72 
     \multiply\Gin@req@height by \gr:density \relax   
     \HCode{
        width="\expandafter\int:pt\the\Gin@req@width|<par del|>"
        height="\expandafter\int:pt\the\Gin@req@height|<par del|>"
     }%   
   \egroup
\fi
>>>


\<graphics quality\><<<
\def\:temp#1graphics-#2,#3|<par del|>{\def\gr:density{#2}}
\expandafter\:temp\Preamble,graphics-,|<par del|>
\ifx \gr:density\empty 
   \Log:Note{if included graphics are
      of degraded quality, try the command line option
      `graphics-num'. The `num' should provide
      the density of pixels in the bitmaps (e.g., 110).
   }
\else
   \def\int:pt#1.#2|<par del|>{#1}
\fi
>>>


All images should be saved in one directory (`Pictures' by default),
regardless of the original path structure. We must save image names and
original paths to be able to copy pictures to correct location.

\<configure ooffice graphics\><<<
\def\getImgFileParser#1/{%
  \ifx#1\relax
    \let\next\relax
  \else
    \do{#1}\let\next\getImgFileParser
  \fi\next
}
\newcommand\getImgFile[2]{%
  \def\do##1{\def\imgBaseName{##1#2}\def\imgBase{##1}}
  \def\imgFullName{#1#2}
  \getImgFileParser#1/\relax/
}

\Configure{graphics*}
   {gif}
   {|<save image name|>\Picture[IMG]{\imgBase.gif |<graphics dimensions|> }}
\Configure{graphics*}
   {png}
   {|<save image name|>\Picture[IMG]{\imgBase.png |<graphics dimensions|> }}
\Configure{graphics*}
   {jpg}
   {|<save image name|>\Picture[IMG]{\imgBase.jpg  |<graphics dimensions|> }}
\Configure{graphics*}
   {jpeg}
   {|<save image name|>\Picture[IMG]{\imgBase.jpeg |<graphics dimensions|> }}

\NewConfigure{Ghostscript}{1}
\NewConfigure{PdfConvert}{2}
\Configure{Ghostscript}{rungs}
\Configure{PdfConvert}{png}{"\a:Ghostscript\space -q -dNOPAUSE -dBATCH -sDEVICE=pngalpha -r300 \ifx\Gin@page\@empty\else -dFirstPage='\Gin@page'  -dLastPage='\Gin@page'\fi\space -sOutputFile='\Gin@base-\Gin@page.\a:PdfConvert' \Gin@base.pdf"}
\Configure{graphics*}
   {pdf}
   {\ifdefined\Gin@page\else\def\Gin@page{1}\fi%
     \edef\svg:filename{\Gin@base-\Gin@page.\a:PdfConvert}%
     % test for existence of the converted file
     \openin15=\svg:filename\relax
      \ifeof15 
      \Needs{\b:PdfConvert}%
      \fi 
      \closein15 
      \def\Gin@ext{.\a:PdfConvert}
    \expandafter\getImgFile\expandafter{\Gin@base}{\a:PdfConvert}%
   \Picture[IMG]{\svg:filename |<graphics dimensions|>}}
>>>

\<\><<<
\Configure{graphics*}
   {eps}
   {|<save image name|>\Picture[IMG]{\imgBase.eps  |<graphics dimensions|> }}
>>>



\<save image name\><<<
\expandafter\getImgFile\expandafter{\Gin@base}{\Gin@ext}
>>>



\<graphics dimensions\><<<
\ifx\noBoundingBox\UnDefined
   svg:width="\the\Gin@req@width"
   svg:height="\the\Gin@req@height"
\fi
>>>  


\<configure ooffice tex4ht\><<<
\def\svg:width{0.0pt}\def\svg:height{0.0pt}
\Configure{IMG}
  {|<insert shadowing fig file|>%
   \ht:special{t4ht=<draw:frame  
      draw:name="\PictureFile"  
      text:anchor-type="as-char"  % "paragraph"  
%      svg:width="...pt"  
%      svg:height="...pt"  
				% \ifx\noBoundingBox\UnDefined
				 \string svg:width="\the\Gin@req@width"
				  svg:height="\the\Gin@req@height"
				% \fi
      draw:z-index="0"
    >%
      <draw:image\Hnewline
         xlink:href="Pictures/}}            
  {\ht:special{t4ht=" \Hnewline
         xlink:type="simple"  
        xlink:show="embed"  
        xlink:actuate="onLoad" 
        /><!--draw:name="}}
  {" }
  {\ht:special{t4ht=" }}
  {\ht:special{t4ht=--></draw:frame>}}
>>>

\<configure ooffice pst-jtree\><<<
\Configure{jtree}
  {\bgroup
      \Configure{IMG}
        {|<insert shadowing fig file|>%
         \ht:special{t4ht=<draw:frame  
            draw:name="|<graphics base name|>"  
            text:anchor-type="as-char"  
            draw:style-name="jtree"
            draw:z-index="0"
          >%
            <draw:image\Hnewline
               xlink:href="Pictures/}}            
        {\ht:special{t4ht=" \Hnewline
               xlink:type="simple"  
              xlink:show="embed"  
              xlink:actuate="onLoad" 
              /><!--draw:name="}}
        {" }
        {\ht:special{t4ht=" }}
        {\ht:special{t4ht=--></draw:frame>}}%
  }
  {\egroup
  }
>>>


\<configure ooffice pst-jtree\><<<
\OOstyles{\Hnewline
<style:style style:name="jtree" 
             style:family="graphic" 
             style:parent-style-name="Graphics">
     <style:graphic-properties 
             style:vertical-pos="bottom" 
             style:vertical-rel="baseline" 
             style:mirror="none" 
             fo:margin-left="0.01em" 
             fo:margin-right="0.01em" 
             fo:margin-top="0.01em" 
             fo:margin-bottom="0.01em" 
     />
</style:style>
}
>>>



\<insert shadowing fig file\><<<
% \ht:special{t4ht>\PictureFile.4og}%
\ht:special{t4ht>\Gin@base\Gin@ext.4og}%
\ht:special{t4ht*>}%
% \ht:special{t4ht<\PictureFile.4og}%
\ht:special{t4ht<\Gin@base\Gin@ext.4og}%
\OOmanifest{|<graphics obj manifest|>}%
>>>


\<graphics file name\><<<
\PictureFile>>>

\<graphics base name\><<<
\imgBaseName>>>

\<graphics full name\><<<
\imgFullName>>>

\<configure ooffice tex4ht\><<<
\Configure{PictureAlt*+[]}
   {\ht:special{t4ht>\jobname .tmp}}
   {\ht:special{t4ht<\jobname .tmp}}
\Configure{PictureAlt*+}
   {\let\sv:HtmlPar=\HtmlPar   \let\HtmlPar=\empty
     |<postscript for /Picture|>%
     |<tex halign and cr/crcr|>%
     \NoFonts\csname PauseMathClass\endcsname \SUBOff \SUPOff
     \let\HCode=\:gobble     |%\offinterlineskip|%
     \let\EndPicture=\empty
     \ht:special{t4ht>\jobname 1.tmp}%
   }
   {\ht:special{t4ht<\jobname 1.tmp}%
     \let\HCode=\:HCode
     \let\EndPicture=\:UnDef \let\HtmlPar=\sv:HtmlPar \SUBOn \SUPOn
     \csname EndPauseMathClass\endcsname \EndNoFonts
     |<tex4ht halign and cr/crcr|>%
     |<delay postscript|>}
>>>



Was \verb+\SUBOff \SUPOff+. Any problems?

\<postscript for /Picture\><<<
\def\PsCode##1{{\ht:special{\PsCodeSpecial##1}}}%
>>>

\<tex halign and cr/crcr\><<<
\iffalse{\fi   
\let\sv:halign=\halign
\let\sv:cr=\cr
\let\sv:crcr=\crcr
\iffalse}\fi 
\RecallTeXcr \let\halign =\TeXhalign
>>>

\<tex4ht halign and cr/crcr\><<<
\iffalse{\fi   
\let\halign=\sv:halign
\let\cr=\sv:cr
\let\crcr=\sv:crcr
\iffalse}\fi 
>>>

\<delay postscript\><<<
\let\PsCode=\relax
>>>


%%%%%%%%%%%%%%%%%%%%%%%%%%%%%%%
\subsection{Last File}
%%%%%%%%%%%%%%%%%%%%%%%%%%%%%%%

A utility file for letting the openoffice script at tex4ht.env know
when the files are ready for zipping.

\<insert last file\><<<
   \ht:special{t4ht>\jobname .4od}%
   \ht:special{t4ht*>}%
   \ht:special{t4ht<\jobname .4od}%
>>>


\<configure ooffice tex4ht\><<<
\:CheckOption{.gif} \if:Option
    \Configure{Picture}{.gif}
\fi
\:CheckOption{.wmf} \if:Option
    \Configure{Picture}{.wmf}
\fi 
\:CheckOption{.svg} \if:Option
    \Configure{Picture}{.svg}
\fi 
\:CheckOption{info}\if:Option 
    \Log:Note{to request a format for dvi pictures,
         use the desired command line option from:
        `.gif', `.svg', `.wmf'}
\fi
>>>





%%%%%%%%%%%%%%%%%%%%%%
\subsection{Shared}
%%%%%%%%%%%%%%%%%%%%%%




\<oo2 tex4ht\><<<
\Configure{Preamble}      
  {}
  {|<oo2 settings.xml|>%             |%before mathml|%
   |<switch to main body OOstlyles|>%
   |<switch to main body OOmanifest|>%
   |<oo2 meta.xml|>%
  } 
>>>




\<oo2 tex4ht\><<<
\Configure{@/BODY}
  {\relax
   \ifnum \FileNumber=1
     |<oo2 styles at /@BODY|>%
     |<manifest at /@BODY|>%
     |<insert last file|>%
   \fi
  }
>>>







\<configure ooffice tex4ht\><<<
\Configure{CutAtTITLE+}{}
\Configure{HPageTITLE+}{}
\Configure{TITLE}{}{}
\Configure{TITLE+}{}{}
>>>



%%%%%%%%%%%%%%%%%%%%%%%%%%%%%
\section{Title Page}
%%%%%%%%%%%%%%%%%%%%%%%%%%%%

%%%%%%%%%%%%%%%%%%%%%%%%%%%%%
\subsection{LaTeX}
%%%%%%%%%%%%%%%%%%%%%%%%%%%%


\<ooffice title page\><<<
\Configure{maketitle}
   {\HCode{<!--begin maketitle-->}}
   {\HCode{<!--end maketitle-->}\par}
   {\ifvmode \IgnorePar\fi \EndP
    \HCode{<text:p text:style-name="Title|<cond rtl class|>"
              >}}
   {\HCode{</text:p>}}
>>>


\<ooffice title page\><<<
\Configure{thanks author date and} 
   {\ifvmode \IgnorePar\fi \EndP
    \HCode{<text:p text:style-name="Text-body|<cond rtl class|>"
             >}}
   {\HCode{</text:p>}}
   {\ifvmode \IgnorePar\fi \EndP
    \HCode{<text:p text:style-name="author|<cond rtl class|>"
              >}}
   {\HCode{</text:p>}}
   {\ifvmode \IgnorePar\fi \EndP
    \HCode{<text:p text:style-name="date|<cond rtl class|>"
              >}}
   {\HCode{</text:p>}}
   {}
   {}%   {\HCode{<text:line-break/>}}
>>>


\<footnote thank\><<<
\Configure{thank}
   {\setbox\tmp:bx=\hbox\bgroup}
   {\egroup}
   {\NoFonts   \leavevmode   \bgroup \protect\Configure{$}{}{}{}%
    \hbox{\HCode{<\text:footnote\Hnewline\text:note:class
                  text:id="}|<thank id|>%
               \HCode{"><\text:footnote-citation>}}%
   }
   {\egroup \EndNoFonts
    \HCode{</\text:footnote-citation><\text:footnote-body>
    <text:p\Hnewline text:style-name="Footnote" ><text:s/>}}
   {\HCode{</text:p></\text:footnote-body></\text:footnote>}}
>>>






\<oo2 tex4ht\><<<
\def\text:footnote{text:note}
\:CheckOption{endnotes}\if:Option 
   \def\text:note:class{text:note-class="endnote" }
\else
   \def\text:note:class{text:note-class="footnote" }
   \Log:Note{for end notes instead of footnotes,
         use the command line option `endnotes'}
\fi
>>>




\<ooffice title page\><<<
\NewConfigureOO{title}
\ConfigureOO{title}{%
<style:style style:name="Title"
             style:family="paragraph"
             style:parent-style-name="Heading"
             style:next-style-name="author"
             style:class="chapter">\Hnewline
<style:paragraph-properties  fo:margin-top="0.42cm"
                  fo:margin-bottom="0.51cm"
                  fo:font-size="20pt"
                  fo:font-weight="normal"
                  style:font-size-asian="18pt"
                  style:font-weight-asian="bold"
                  style:font-size-complex="18pt"
                  style:font-weight-complex="bold"
                  fo:text-align="center"
                  style:justify-single-word="false"/>
</style:style> \Hnewline |<Title rtl|> }
>>>


\<ooffice title page\><<<
\NewConfigureOO{author}
\ConfigureOO{author}{%
<style:style      style:name="author"
                  style:family="paragraph"
                  style:parent-style-name="Text-body"
                  style:next-style-name="date">\Hnewline
<style:paragraph-properties  fo:margin-top="0cm"
                  fo:margin-bottom="0.199cm"
                  fo:font-size="14pt"
                  fo:text-align="center"
                  style:justify-single-word="false"/>
</style:style>\Hnewline |<author rtl|>}
>>>

\<ooffice title page\><<<
\NewConfigureOO{date}
\ConfigureOO{date}{%
<style:style style:name="date"
                  style:family="paragraph"
                  style:parent-style-name="Text-body"
                  style:next-style-name="Text-body">\Hnewline
<style:paragraph-properties  fo:margin-top="0cm"
                  fo:margin-bottom="0.199cm"
                  fo:font-size="14pt"
                  fo:text-align="center"
                  style:justify-single-word="false"/>
</style:style>\Hnewline |<date rtl|>}
>>>

%%%%%%%%%%%%%%%%%%%%%%%%%%%%%
\subsection{Scr}
%%%%%%%%%%%%%%%%%%%%%%%%%%%%



\<configure ooffice scrbook\><<<
|<scr title page|>
>>>


\<configure ooffice scrreprt\><<<
|<scr title page|>
>>>


\<scr title page\><<<
\ConfigureEnv{titlepage}
   {\bgroup 
      \ifvmode \IgnorePar\fi \EndP
%     \HCode{<div class="titlepage">}%
      \Configure{tabular}{}{}{}{}{}{}% 
      \ConfigureEnv{tabular}{}{}{}{}{}{}% 
      \ConfigureEnv{center}{}{}{}{}}
   {\ifvmode \IgnorePar\fi \EndP
%     \HCode{</div>}%
    \egroup}
   {} {}
\Configure{dedication}
   {\ifvmode \IgnorePar\fi\EndP
    \HCode{<text:p text:style-name="Text-body" >}}
   {\ifvmode \IgnorePar\fi\EndP \HCode{</text:p>}}
>>>

%%%%%%%%%%%%%%%%%%%%%%%%%%%%%
\section{Abstract}
%%%%%%%%%%%%%%%%%%%%%%%%%%%%

%%%%%%%%%%%%%%%%%%%%%%%%%%%%%
\subsection{LaTeX}
%%%%%%%%%%%%%%%%%%%%%%%%%%%%

\<ooffice title page\><<<
\ConfigureEnv{abstract}
   {\IgnorePar\EndP \bgroup
      \Configure{HtmlPar}
         {\EndP \HCode{<text:p
                    text:style-name="abstract|<cond rtl class|>">}}
         {\EndP \HCode{<text:p  
                    text:style-name="abstract|<cond rtl class|>">}}
         {\HCode{</text:p>}}
         {\HCode{</text:p>}}%
      \par \ShowPar
   }
   {\ifvmode \IgnorePar\fi \EndP \egroup \ShowPar\par \IgnoreIndent }
   {\ifvmode \IgnorePar\fi\EndP
    \HCode{<text:p text:style-name="abstract-title"  >}%
   }
   {\HCode{</text:p>}}
>>>


\<ooffice title page\><<<
\NewConfigureOO{abstract}
\ConfigureOO{abstract}{\Hnewline
<style:style style:name="abstract"
                  style:family="paragraph"
                  style:parent-style-name="Text-body"
                  style:next-style-name="Text-body">\Hnewline
<style:paragraph-properties  fo:margin-left="3cm"
                  fo:margin-right="3cm"
                  fo:margin-top="0cm"
                  fo:margin-bottom="0.199cm"
                  fo:font-size="10pt"
                  fo:text-indent="0cm"
                  style:auto-text-indent="false"/>\Hnewline
</style:style>\Hnewline |<abstract rtl|> }
>>>

\<ooffice title page\><<<
\NewConfigureOO{abstract-title}
\ConfigureOO{abstract-title}{\Hnewline
<style:style style:name="abstract-title"
                  style:family="paragraph"
                  style:parent-style-name="abstract"
                  style:next-style-name="abstract">\Hnewline
<style:paragraph-properties  fo:margin-top="0.499cm"
                  fo:margin-bottom="0cm"
                  fo:font-weight="bold"
                  fo:text-align="center"
                  style:justify-single-word="false"/>\Hnewline
</style:style>\Hnewline
}
>>>




%%%%%%%%%%%%%%%%%%%%%%%%%%%%%
\section{List Environments}
%%%%%%%%%%%%%%%%%%%%%%%%%%%%


%%%%%%%%%%%%%
\subsection{LaTeX}
%%%%%%%%%%%%%

\begin{description}
\item[Itemized Lists]



\<configure ooffice latex\><<<     
\ConfigureList{itemize}%
   {\EndP\HCode{|<input line num|><\text:unorderedlist
      text:style-name="Itemize|<cond rtl class|>"
        >\Hnewline}\def\end@Item{}\bgroup
     \Configure{HtmlPar}
        {\EndP\HCode{<text:p
            text:style-name="Inside-itemize|<cond rtl class|>"
                     >}}
        {\EndP\HCode{<text:p
            text:style-name="Inside-itemize|<cond rtl class|>"
                      >}}
        {\HCode{</text:p>\Hnewline}}
        {\HCode{</text:p>\Hnewline}}%
     \gHAdvance\BegEnd:D by 1
   }
   {\gHAdvance\BegEnd:D by -1
    \EndP\IgnorePar \egroup \HCode{</text:list-item>\Hnewline
      </\text:unorderedlist>\Hnewline}\ShowPar}
   {\EndP\end@Item\DeleteMark}
   {\HCode{<text:list-item>}\par\ShowPar
      \def\end@Item{\HCode{</text:list-item>\Hnewline}}}
>>>







\<configure ooffice latex\><<<
\NewConfigureOO{itemize}
\ConfigureOO{itemize}
    {|<shared oo+ itemize style|>%
     |<oo2+ itemize style|>%
     |<oo2+ itemize style rtl|>}
>>>














\<oo2+ itemize style\><<<
<text:list-style style:name="Itemize" >
  <text:list-level-style-bullet
       text:level="1"
       text:style-name="Bullet-Symbols"
       text:bullet-char="\string&\:sharp x2022;" >\Hnewline
     <style:list-level-properties
       text:space-before="0.15in" 
       text:min-label-width="0.2in"
       text:min-label-distance="0.05in" />\Hnewline
  </text:list-level-style-bullet>
  <text:list-level-style-bullet
       text:level="2"
       text:style-name="Bullet-Symbols"
       text:bullet-char="-" >\Hnewline
    <style:list-level-properties
       text:space-before="0.35in" 
       text:min-label-width="0.2in"
       text:min-label-distance="0.05in"
     />\Hnewline
  </text:list-level-style-bullet>
  <text:list-level-style-bullet
       text:level="3"
       text:style-name="Bullet-Symbols"
  text:bullet-char="\string&\:sharp x204E;" >\Hnewline
    <style:list-level-properties
       text:space-before="0.55in"
       text:min-label-width="0.2in"
       text:min-label-distance="0.05in"/>\Hnewline
  </text:list-level-style-bullet>
  <text:list-level-style-bullet
       text:level="4"
       text:style-name="Bullet-Symbols"
       text:bullet-char="\string&\:sharp x00B7;" >\Hnewline
    <style:list-level-properties
       text:space-before="0.75in"
       text:min-label-width="0.2in"
       text:min-label-distance="0.05in"/>\Hnewline 
  </text:list-level-style-bullet>
</text:list-style>
>>>




\<shared oo+ itemize style\><<<
<style:style style:name="Inside-itemize"
             style:family="paragraph"
             style:parent-style-name="Standard"
             style:list-style-name="Itemize">\Hnewline
   <style:paragraph-properties 
       fo:margin-left="0in"
       fo:margin-right="0in" 
       fo:margin-top="0in" 
       fo:margin-bottom="0.15in"
       fo:text-indent="0in"
       style:auto-text-indent="false"/>
</style:style>
>>>


\item[Enumerated Lists]


\<configure ooffice latex\><<<   
\NewConfigureOO{enumerate}  
\:CheckOption{enumerate+} \if:Option
   |<oo-description for enumerate|>
\else
   |<oo-enumerate for enumerate|>
   |<description-based enumerate list note|>
\fi
>>>

\<description-based enumerate list note\><<<
\Log:Note{for enumerated list elements with valued data,
                      use the command line option `enumerate+'}
>>>


\<oo-enumerate for enumerate\><<<
\ConfigureList{enumerate}%
   {\EndP\HCode{<\text:orderedlist
      text:style-name="Enumerate|<cond rtl class|>"
       >\Hnewline}\def\end@Item{}\bgroup
     \Configure{HtmlPar}
      {\EndP\HCode{<text:p
           text:style-name="Inside-enumerate|<cond rtl class|>"
            >}}
      {\EndP\HCode{<text:p
           text:style-name="Inside-enumerate|<cond rtl class|>"
             >}}
      {\HCode{</text:p>\Hnewline}}     {\HCode{</text:p>\Hnewline}}}
   {\EndP\IgnorePar \egroup \HCode{</text:list-item>\Hnewline
     </\text:orderedlist>\Hnewline}\ShowPar}
   {\EndP\end@Item\DeleteMark}
   {\HCode{<text:list-item>}\par\ShowPar
     \def\end@Item{\HCode{</text:list-item>\Hnewline}}}
>>>




\<shared oo+ enumerate style\><<<
<style:style style:name="Inside-enumerate" 
             style:family="paragraph"
             style:parent-style-name="Text-body"
             style:list-style-name="Enumerate">\Hnewline 
   <style:paragraph-properties 
       fo:margin-left="0in"
       fo:margin-right="0in" 
       fo:margin-top="0in" 
       fo:margin-bottom="0.15in"
       fo:text-indent="0in"
       style:auto-text-indent="false"/>
</style:style>
>>>





\<oo-enumerate for enumerate\><<<
\ConfigureOO{enumerate}%
    {|<shared oo+ enumerate style|>%
     |<oo2+ enumerate style|>%
     |<shared oo+ enumerate style rtl|>%
     |<oo2+ enumerate style rtl|>}
>>>


\<configure ooffice latex\><<<
\NewConfigure{enumerate:1}{3}
\NewConfigure{enumerate:2}{3}
\NewConfigure{enumerate:3}{3}
\NewConfigure{enumerate:4}{3}
\Configure{enumerate:1}{1}{}{.}
\Configure{enumerate:2}{a}{(}{)}
\Configure{enumerate:3}{i}{}{.}
\Configure{enumerate:4}{A}{}{.}
>>>





\<oo2+ enumerate style\><<<
<text:list-style style:name="Enumerate">\Hnewline
  <text:list-level-style-number
       text:level="1"
       text:style-name="Numbering-Symbols"
       style:num-prefix="\csname b:enumerate:1\endcsname" 
       style:num-suffix="\csname c:enumerate:1\endcsname"
       style:num-format="\csname a:enumerate:1\endcsname"
      >\Hnewline
    <style:list-level-properties
       text:min-label-width="0.2in"
       text:space-before="0.15in"       
       text:min-label-distance="0.05in"/>\Hnewline
  </text:list-level-style-number>\Hnewline
  <text:list-level-style-number
       text:level="2"
       text:style-name="Numbering-Symbols"
       style:num-prefix="\csname b:enumerate:2\endcsname" 
       style:num-suffix="\csname c:enumerate:2\endcsname"
       style:num-format="\csname a:enumerate:2\endcsname"
      >\Hnewline
    <style:list-level-properties
       text:min-label-width="0.2in"
       text:space-before="0.35in"
       text:min-label-distance="0.05in"/>\Hnewline
  </text:list-level-style-number>\Hnewline
  <text:list-level-style-number
       text:level="3"
       text:style-name="Numbering-Symbols"
       style:num-prefix="\csname b:enumerate:3\endcsname" 
       style:num-suffix="\csname c:enumerate:3\endcsname"
       style:num-format="\csname a:enumerate:3\endcsname"
     >\Hnewline
    <style:list-level-properties
       text:min-label-width="0.2in"
       text:space-before="0.6in" 
       text:min-label-distance="0.05in"/>\Hnewline
  </text:list-level-style-number>\Hnewline
  <text:list-level-style-number
       text:level="4"
       text:style-name="Numbering-Symbols"
       style:num-prefix="\csname b:enumerate:4\endcsname" 
       style:num-suffix="\csname c:enumerate:4\endcsname"
       style:num-format="\csname a:enumerate:4\endcsname"
     >\Hnewline
    <style:list-level-properties
       text:min-label-width="0.2in"
       text:space-before="0.8in"  
       text:min-label-distance="0.05in"/>\Hnewline
  </text:list-level-style-number>\Hnewline
</text:list-style>\Hnewline
>>>


%%%%%%%%%%%%%%%%%%%%%%%%%%%%%%%%
\item[Description Lists]
%%%%%%%%%%%%%%%%%%%%%%%%%%%%%%%%



\<configure ooffice latex\><<<     
\let\end:itm=\empty
\ConfigureList{description}%
   {\EndP
    \bgroup
    \HCode{<text:list
         text:style-name="description\if@rl-rtl\fi"
         text:name="description"\Hnewline>}%
      \PushMacro\end:itm
      \global\let\end:itm=\empty      
      \HTML:PAR{dd|<cond rtl class|>}{dd|<cond rtl class|>}%
      \gHAdvance\BegEnd:D by 1
   }
   {\gHAdvance\BegEnd:D by -1
      \PopMacro\end:itm \global\let\end:itm \end:itm
      \EndP
      \HCode{<!--/dd-->}%
      \HCode{</text:list-item>}
      \HCode{</text:list>}%
      \egroup
      \ShowPar}
   {\end:itm 
    \global\def\end:itm{\EndP
                        \HCode{<!--/dd-->}%
                        \HCode{</text:list-item>}%
   }%
    \HCode{<text:list-item>%
           <text:h text:style-name="dt|<cond rtl class|>"\Hnewline >%
          }%
   }
   {\EndP
    \HCode{</text:h>}%
    \HCode{<!--dd\Hnewline class="description"-->}%
    \par
   }
>>>






\<configure ooffice enumerate\><<<
|<oo-description for enumerate|>
>>>

\<oo-description for enumerate\><<<
\ConfigureList{enumerate}%
   {\EndP\HCode{<\text:orderedlist
      text:style-name="Enumerate|<cond rtl class|>"
       >\Hnewline}\def\end@Item{}\bgroup
     \Configure{HtmlPar}
      {\EndP\HCode{<text:p
           text:style-name="Inside-enumerate|<cond rtl class|>"
            >}}
      {\EndP\HCode{<text:p
           text:style-name="Inside-enumerate|<cond rtl class|>"
             >}}
      {\HCode{</text:p>\Hnewline}}     {\HCode{</text:p>\Hnewline}}}
   {\EndP\IgnorePar \egroup \HCode{</text:list-item>\Hnewline
     </\text:orderedlist>\Hnewline}\ShowPar}
   {\EndP\end@Item  \let\sv:DeleteMark\DeleteMark
     \let\DeleteMark\DeleteMark: \DeleteMark}
   {\HCode{<text:list-item>}%
       \par\ShowPar
       \HCode{<text:span
       text:style-name="enum-label">}\AnchorLabel
          \box\tmp:bx
          \let\DeleteMark\sv:DeleteMark
     \HCode{</text:span><text:s/>}%
     \def\end@Item{\HCode{</text:list-item>\Hnewline}}}
\def\DeleteMark:#1\@labels{\hfill\break 
   \setbox\tmp:bx\vbox{\box\@labels}}
>>>




\<oo-description for enumerate\><<<
\ConfigureOO{enumerate}%
    {|<shared oo+ enumerate style|>%
     |<enumerate.sty style|>}
>>>



\<enumerate.sty style\><<<
<text:list-style style:name="Enumerate">\Hnewline
  <text:list-level-style-number
       text:level="1"
       text:style-name="Numbering-Symbols"
       style:num-prefix=""
       style:num-suffix=""
       style:num-format=""
      >\Hnewline
    <style:list-level-properties
       text:min-label-width="0.2in"
       text:space-before="0.15in"       
       text:min-label-distance="0.05in"/>\Hnewline
  </text:list-level-style-number>\Hnewline
  <text:list-level-style-number
       text:level="2"
       text:style-name="Numbering-Symbols"
       style:num-prefix=""
       style:num-suffix=""
       style:num-format=""
      >\Hnewline
    <style:list-level-properties
       text:min-label-width="0.2in"
       text:space-before="0.35in"
       text:min-label-distance="0.05in"/>\Hnewline
  </text:list-level-style-number>\Hnewline
  <text:list-level-style-number
       text:level="3"
       text:style-name="Numbering-Symbols"
       style:num-prefix=""
       style:num-suffix=""
       style:num-format=""
     >\Hnewline
    <style:list-level-properties
       text:min-label-width="0.2in"
       text:space-before="0.6in" 
       text:min-label-distance="0.05in"/>\Hnewline
  </text:list-level-style-number>\Hnewline
  <text:list-level-style-number
       text:level="4"
       text:style-name="Numbering-Symbols"
       style:num-prefix=""
       style:num-suffix=""
       style:num-format=""
     >\Hnewline
    <style:list-level-properties
       text:min-label-width="0.2in"
       text:space-before="0.8in"  
       text:min-label-distance="0.05in"/>\Hnewline
  </text:list-level-style-number>\Hnewline
</text:list-style>\Hnewline
>>>






\<configure ooffice latex\><<<
\NewConfigureOO{description}
\ConfigureOO{description}{|<oo2+ description style|>}
>>>








\<oo2+ description style\><<<
<text:list-style style:name="description" >
  <text:list-level-style-bullet
       text:level="1"
       text:style-name="Source-Text"
       text:bullet-char=" " >\Hnewline
     <style:list-level-properties
       text:space-before="0in" 
       text:min-label-width="0in"
       text:min-label-distance="0in" />\Hnewline
  </text:list-level-style-bullet>
  <text:list-level-style-bullet
       text:level="2"
       text:style-name="Source-Text"
       text:bullet-char=" " >\Hnewline
    <style:list-level-properties
       text:space-before="0.2in" 
       text:min-label-width="0.2in"
       text:min-label-distance="0.05in"
     />\Hnewline
  </text:list-level-style-bullet>
  <text:list-level-style-bullet
       text:level="3"
       text:style-name="Source-Text"
       text:bullet-char=" " >\Hnewline
    <style:list-level-properties
       text:space-before="0.4in"
       text:min-label-width="0.2in"
       text:min-label-distance="0.05in"/>\Hnewline
  </text:list-level-style-bullet>
  <text:list-level-style-bullet
       text:level="4"
       text:style-name="Source-Text"
       text:bullet-char=" " >\Hnewline
    <style:list-level-properties
       text:space-before="0.6in"
       text:min-label-width="0.2in"
       text:min-label-distance="0.05in"/>\Hnewline 
  </text:list-level-style-bullet>
</text:list-style>
>>>






\<oo2+ description style\><<<
<style:style 
    style:name="dd" 
    style:family="paragraph" 
    style:parent-style-name="Standard"> 
 <style:paragraph-properties  
    fo:margin-left="0.35in" 
    fo:margin-right="0in"  
    fo:margin-top="0cm" 
    fo:margin-bottom="0.2cm" 
    fo:text-indent="0in"  
    style:auto-text-indent="false"/> 
</style:style> 
<style:style
       style:name="dt"
       style:family="paragraph"
       style:parent-style-name="">
  <style:paragraph-properties
       fo:margin-left="0.1in"
       fo:margin-right="0in"
       fo:text-indent="0in"
       style:auto-text-indent="false"/>
  <style:text-properties
       fo:font-weight="bold"
  />
</style:style>
>>>





\end{description}


%%%%%%%%%%%%%
\subsection{List}
%%%%%%%%%%%%%


\<configure ooffice latex\><<<     
\let\end:itm=\empty
\ConfigureList{list}%
   {\EndP\HCode{<text:list
         text:style-name="list|<cond rtl class|>"
         text:name="list"\Hnewline>}%
    \PushMacro\end:itm
    \global\let\end:itm=\empty
    \SaveHtmlPar \HTML:PAR{dd}{dd}%
   }
   {\PopMacro\end:itm \global\let\end:itm \end:itm
    \EndP
    \RecallHtmlPar
    \HCode{</text:list-item></text:list>}\ShowPar}
   {\end:itm \global\def\end:itm{\EndP\HCode{</text:list-item>}}%
    \HCode{<text:list-item><text:h
      text:style-name="dt|<cond rtl class|>"\Hnewline >}\bgroup \bf}
   {\egroup\HCode{</text:h>}\par\ShowPar}
>>>



\<configure ooffice latex\><<<
\NewConfigureOO{list}
\ConfigureOO{list}{|<oo2+ list style|>}
>>>




\<oo2+ list style\><<<
<text:list-style style:name="list" >
  <text:list-level-style-bullet
       text:level="1"
       text:style-name="Source-Text"
       text:bullet-char=" " >\Hnewline
     <style:list-level-properties
       text:space-before="0in" 
       text:min-label-width="0in"
       text:min-label-distance="0in" />\Hnewline
  </text:list-level-style-bullet>
  <text:list-level-style-bullet
       text:level="2"
       text:style-name="Source-Text"
       text:bullet-char=" " >\Hnewline
    <style:list-level-properties
       text:space-before="0.2in" 
       text:min-label-width="0.2in"
       text:min-label-distance="0.05in"
     />\Hnewline
  </text:list-level-style-bullet>
  <text:list-level-style-bullet
       text:level="3"
       text:style-name="Source-Text"
       text:bullet-char=" " >\Hnewline
    <style:list-level-properties
       text:space-before="0.4in"
       text:min-label-width="0.2in"
       text:min-label-distance="0.05in"/>\Hnewline
  </text:list-level-style-bullet>
  <text:list-level-style-bullet
       text:level="4"
       text:style-name="Source-Text"
       text:bullet-char=" " >\Hnewline
    <style:list-level-properties
       text:space-before="0.6in"
       text:min-label-width="0.2in"
       text:min-label-distance="0.05in"/>\Hnewline 
  </text:list-level-style-bullet>
</text:list-style>
>>>


\<oo2+ list style\><<<
<style:style 
    style:name="dd" 
    style:family="paragraph" 
    style:parent-style-name="Standard"> 
 <style:paragraph-properties  
    fo:margin-left="0.35in" 
    fo:margin-right="0in"  
    fo:margin-top="0cm" 
    fo:margin-bottom="0.2cm" 
    fo:text-indent="0in"  
    style:auto-text-indent="false"/> 
</style:style> 
<style:style
       style:name="dt"
       style:family="paragraph"
       style:parent-style-name="">
  <style:paragraph-properties
       fo:margin-left="0.1in"
       fo:margin-right="0in"
       fo:text-indent="0in"
       style:auto-text-indent="false"/>
  <style:text-properties
       fo:font-weight="bold"
  />
</style:style>
>>>



%%%%%%%%%%%%%
\subsection{Apa}
%%%%%%%%%%%%%

\<configure ooffice apacite\><<<
\Configure{bibcite}{\def\ { }}
>>>


\<configure ooffice apa\><<<
\Configure{maketitle}
   {\HCode{<!--begin maketitle-->}|<apa maketitle|>}
   {\HCode{<!--end maketitle-->}\par}
   {\ifvmode \IgnorePar\fi \EndP
    \HCode{<text:p text:style-name="Title|<cond rtl class|>" 
               >}}
   {\HCode{</text:p>}}
\Configure{affil}
   {\ifvmode \IgnorePar\fi \EndP
    \HCode{<text:p text:style-name="affiliation\if@rl
                                        -rtl\fi"
                     >}}
   {\HCode{</text:p>}}
\Configure{abstract}
   {\Hclearpage \IgnorePar\EndP \bgroup 
      \Configure{HtmlPar}
         {\EndP \HCode{<text:p 
                  text:style-name="abstract\if@rl-rtl\fi"
                    >}}
         {\EndP \HCode{<text:p 
                  text:style-name="abstract\if@rl-rtl\fi"
                    >}}
         {\HCode{</text:p>}}
         {\HCode{</text:p>}}%
      \ht:everypar{\HtmlPar}\par \ShowPar
   }
   {\ifvmode \IgnorePar\fi \EndP 
    \egroup \Hclearpage \ShowPar\par \IgnoreIndent }
>>>


\<apa maketitle\><<<
\Configure{centercr} 
   {\ht:everypar{\ht:everypar{\HtmlPar}}}%
   {\ht:everypar{\ht:everypar{\HtmlPar}}}%
>>>







\<configure ooffice apa\><<<
\ConfigureOO{itemize}
    {|<apa itemize style|>|<apa itemize style rtl|>}
\ConfigureList{APAenumerate}%
   {\EndP\HCode{<\text:orderedlist
      text:style-name="Enumerate|<cond rtl class|>"
        >\Hnewline}\def\end@Item{}\bgroup
     \Configure{HtmlPar}
      {\EndP\HCode{<text:p text:style-name="Inside-enumerate|<cond rtl class|>"
                             >}}
      {\EndP\HCode{<text:p text:style-name="Inside-enumerate|<cond rtl class|>"
                             >}}
      {\HCode{</text:p>\Hnewline}}     {\HCode{</text:p>\Hnewline}}}
   {\EndP\IgnorePar \egroup \HCode{</text:list-item>\Hnewline
     </\text:orderedlist>\Hnewline}\ShowPar}
   {\EndP\end@Item\DeleteMark}
   {\HCode{<text:list-item>}\par\ShowPar
     \def\end@Item{\HCode{</text:list-item>\Hnewline}}}
\ConfigureList{APAitemize}%
   {\EndP\HCode{<\text:unorderedlist
      text:style-name="Itemize|<cond rtl class|>"
        >\Hnewline}\def\end@Item{}\bgroup
     \Configure{HtmlPar}
      {\EndP\HCode{<text:p text:style-name="Inside-itemize|<cond rtl class|>"
                             >}}
      {\EndP\HCode{<text:p text:style-name="Inside-itemize|<cond rtl class|>"
                              >}}
      {\HCode{</text:p>\Hnewline}}     {\HCode{</text:p>\Hnewline}}}
   {\EndP\IgnorePar \egroup \HCode{</text:list-item>\Hnewline
      </\text:unorderedlist>\Hnewline}\ShowPar}
   {\EndP\end@Item\DeleteMark}
   {\HCode{<text:list-item>}\par\ShowPar
      \def\end@Item{\HCode{</text:list-item>\Hnewline}}}
>>>




\<apa itemize style\><<<
<text:list-style style:name="Itemize">\Hnewline
  <text:list-level-style-bullet
      \list:text:level="1" 
      text:style-name="Bullet-Symbols"
      text:bullet-char="\string&\:sharp x2022;" >\Hnewline
     <style:text-properties 
        text:space-before="0.15in" 
        text:min-label-distance="0.05in"/>\Hnewline
  </text:list-level-style-bullet>\Hnewline
  <text:list-level-style-bullet
      \list:text:level="2"
      text:bullet-char="\string&\:sharp x2022;" >\Hnewline
     <style:text-properties
       text:space-before="0.35in"
       text:min-label-distance="0.05in" />\Hnewline
  </text:list-level-style-bullet>\Hnewline
  <text:list-level-style-bullet
      \list:text:level="3"
      text:bullet-char="\string&\:sharp x2022;" >\Hnewline
     <style:text-properties
       text:space-before="0.55in"
       text:min-label-distance="0.05in" />\Hnewline
  </text:list-level-style-bullet>\Hnewline
  <text:list-level-style-bullet
      \list:text:level="4" 
      text:bullet-char="\string&\:sharp x2022;" >\Hnewline
     <style:text-properties 
       text:space-before="0.75in" 
       text:min-label-distance="0.05in" />\Hnewline
  </text:list-level-style-bullet>\Hnewline
</text:list-style>\Hnewline
>>>


\<apa itemize style\><<<
<style:style style:name="Inside-itemize"
             style:family="paragraph"
             style:parent-style-name="Text-body"
             style:list-style-name="Itemize"/>\Hnewline
>>>



\<configure ooffice apa\><<<
   |<oo2 apa|>
>>>




\<oo2 apa\><<<
\ConfigureOO{enumerate}%
    {|<shared apa enumerate style|>%
     |<oo2 apa enumerate style|>%
     |<oo2 apa enumerate style rtl|>}
>>>





\<oo2 apa enumerate style\><<<
<text:list-style style:name="Enumerate">\Hnewline
  <text:list-level-style-number
       text:level="1"
       style:num-prefix="" 
       style:num-suffix="." >\Hnewline
    <style:list-level-properties
        text:space-before="0.15in"  
       style:num-format="1"
        text:min-label-distance="0.05in"/>\Hnewline
  </text:list-level-style-number>\Hnewline
  <text:list-level-style-number
       text:level="2"
       style:num-prefix="" 
       style:num-suffix="." >\Hnewline
    <style:list-level-properties
       text:space-before="0.35in" 
       style:num-format="1"
       text:min-label-distance="0.05in"/>\Hnewline
  </text:list-level-style-number>\Hnewline
  <text:list-level-style-number
       text:level="3"
       style:num-prefix="" 
       style:num-suffix="." >\Hnewline
    <style:list-level-properties
       text:space-before="0.6in" 
       style:num-format="1"
       text:min-label-distance="0.05in"/>\Hnewline
  </text:list-level-style-number>\Hnewline
  <text:list-level-style-number
       text:level="4"
       style:num-prefix="" 
       style:num-suffix="." >\Hnewline
    <style:list-level-properties
       text:space-before="0.8in"  
       style:num-format="1"
       text:min-label-distance="0.05in"/>\Hnewline
  </text:list-level-style-number>\Hnewline
</text:list-style>\Hnewline
>>>

\<shared apa enumerate style\><<<
<style:style style:name="Inside-enumerate" 
             style:family="paragraph"
             style:parent-style-name="Text-body"
             style:list-style-name="Enumerate"/>\Hnewline 
>>>


%%%%%%%%%%%%%
\subsection{Paralist}
%%%%%%%%%%%%%

\<configure ooffice paralist\><<<
\ConfigureList{compactenum}%
   {\EndP \end:itm  
    \SaveHtmlPar \HCode{<text:list  
      text:style-name="compactenum|<cond rtl class|>"  
         text:name="compactenum"\Hnewline>}%  
      \PushMacro\end:Itm
      \global\let\end:itm=\empty  
      \global\let\end:Itm=\empty  
     \Configure{HtmlPar}  
        {\end:itm\gdef\end:itm{\HCode{</text:p>\Hnewline}}%  
          \HCode{<text:p  
          text:style-name="dd|<cond rtl class|>" >}}  
        {\end:itm\gdef\end:itm{\HCode{</text:p>\Hnewline}}%  
          \HCode{<text:p  
          text:style-name="dd|<cond rtl class|>" >}}  
        {}{}%  
   }  
   {\end:itm \end:Itm \global\let\end:itm=\empty  
    \PopMacro\end:Itm \global\let\end:Itm\end:Itm
    \HCode{</text:list>}\RecallHtmlPar  \ShowPar}  
   {\end:itm \end:Itm \gdef\end:Itm{\HCode{</text:list-item>}}% 
      \gdef\end:itm{\HCode{</text:h>}}\HCode{<text:list-item><text:h  
      text:style-name="dt|<cond rtl class|>"\Hnewline >}\bgroup \bf}  
   {\egroup \end:itm \gdef\end:itm{\HCode{</text:p>}}%  
      \HCode{<text:p\Hnewline  
      text:style-name="dd|<cond rtl class|>"\Hnewline  
          >}} 
\NewConfigureOO{compactenum}
\ConfigureOO{compactenum}{|<compactenum style|>}
>>>


\<compactenum style\><<<
<text:list-style style:name="compactenum" >
  <text:list-level-style-bullet
       text:level="1"
       text:style-name="Source-Text"
       text:bullet-char=" " >\Hnewline
     <style:list-level-properties
       text:space-before="0in" 
       text:min-label-width="0in"
       text:min-label-distance="0in" />\Hnewline
  </text:list-level-style-bullet>
  <text:list-level-style-bullet
       text:level="2"
       text:style-name="Source-Text"
       text:bullet-char=" " >\Hnewline
    <style:list-level-properties
       text:space-before="0.2in" 
       text:min-label-width="0.2in"
       text:min-label-distance="0.05in"
     />\Hnewline
  </text:list-level-style-bullet>
  <text:list-level-style-bullet
       text:level="3"
       text:style-name="Source-Text"
       text:bullet-char=" " >\Hnewline
    <style:list-level-properties
       text:space-before="0.4in"
       text:min-label-width="0.2in"
       text:min-label-distance="0.05in"/>\Hnewline
  </text:list-level-style-bullet>
  <text:list-level-style-bullet
       text:level="4"
       text:style-name="Source-Text"
       text:bullet-char=" " >\Hnewline
    <style:list-level-properties
       text:space-before="0.6in"
       text:min-label-width="0.2in"
       text:min-label-distance="0.05in"/>\Hnewline 
  </text:list-level-style-bullet>
</text:list-style>
>>>

\<configure ooffice paralist\><<<
\ConfigureList{compactitem}%
   {\EndP\HCode{|<input line num|><\text:unorderedlist
      text:style-name="compactitem|<cond rtl class|>"
        >\Hnewline}\def\end@Item{}\bgroup
     \Configure{HtmlPar}
        {\EndP\HCode{<text:p
            text:style-name="Inside-compactitem|<cond rtl class|>"
                     >}}
        {\EndP\HCode{<text:p
            text:style-name="Inside-compactitem|<cond rtl class|>"
                      >}}
        {\HCode{</text:p>\Hnewline}}
        {\HCode{</text:p>\Hnewline}}%
     \gHAdvance\BegEnd:D by 1
   }
   {\gHAdvance\BegEnd:D by -1
    \EndP\IgnorePar \egroup \HCode{</text:list-item>\Hnewline
      </\text:unorderedlist>\Hnewline}\ShowPar}
   {\EndP\end@Item\DeleteMark}
   {\HCode{<text:list-item>}\par\ShowPar
      \def\end@Item{\HCode{</text:list-item>\Hnewline}}}
\NewConfigureOO{compactitem}
\ConfigureOO{compactitem}
    {|<shared oo+ compactitem style|>%
     |<oo2+ compactitem style|>%
     |<oo2+ compactitem style rtl|>}
>>>



\<oo2+ compactitem style\><<<
<text:list-style style:name="compactitem" >
  <text:list-level-style-bullet
       text:level="1"
       text:style-name="Bullet-Symbols"
       text:bullet-char="\string&\:sharp x2022;" >\Hnewline
     <style:list-level-properties
       text:space-before="0.15in" 
       text:min-label-width="0.2in"
       text:min-label-distance="0.05in" />\Hnewline
  </text:list-level-style-bullet>
  <text:list-level-style-bullet
       text:level="2"
       text:style-name="Bullet-Symbols"
       text:bullet-char="-" >\Hnewline
    <style:list-level-properties
       text:space-before="0.35in" 
       text:min-label-width="0.2in"
       text:min-label-distance="0.05in"
     />\Hnewline
  </text:list-level-style-bullet>
  <text:list-level-style-bullet
       text:level="3"
       text:style-name="Bullet-Symbols"
  text:bullet-char="\string&\:sharp x204E;" >\Hnewline
    <style:list-level-properties
       text:space-before="0.55in"
       text:min-label-width="0.2in"
       text:min-label-distance="0.05in"/>\Hnewline
  </text:list-level-style-bullet>
  <text:list-level-style-bullet
       text:level="4"
       text:style-name="Bullet-Symbols"
       text:bullet-char="\string&\:sharp x00B7;" >\Hnewline
    <style:list-level-properties
       text:space-before="0.75in"
       text:min-label-width="0.2in"
       text:min-label-distance="0.05in"/>\Hnewline 
  </text:list-level-style-bullet>
</text:list-style>
>>>




\<shared oo+ compactitem style\><<<
<style:style style:name="Inside-compactitem"
             style:family="paragraph"
             style:parent-style-name="Standard"
             style:list-style-name="compactitem">\Hnewline
   <style:paragraph-properties 
       fo:margin-left="0in"
       fo:margin-right="0in" 
       fo:margin-top="0in" 
       fo:margin-bottom="0.15in"
       fo:text-indent="0in"
       style:auto-text-indent="false"/>
</style:style>
>>>

\<configure ooffice paralist\><<<
\ConfigureList{compactdesc}%
   {\EndP \end:itm  
    \SaveHtmlPar \HCode{<text:list  
      text:style-name="compactdesc|<cond rtl class|>"  
         text:name="compactdesc"\Hnewline>}%  
      \PushMacro\end:Itm
      \global\let\end:itm=\empty  
      \global\let\end:Itm=\empty  
     \Configure{HtmlPar}  
        {\end:itm\gdef\end:itm{\HCode{</text:p>\Hnewline}}%  
          \HCode{<text:p  
          text:style-name="dd|<cond rtl class|>" >}}  
        {\end:itm\gdef\end:itm{\HCode{</text:p>\Hnewline}}%  
          \HCode{<text:p  
          text:style-name="dd|<cond rtl class|>" >}}  
        {}{}%  
   }  
   {\end:itm \end:Itm \global\let\end:itm=\empty  
    \PopMacro\end:Itm \global\let\end:Itm\end:Itm
    \HCode{</text:list>}\RecallHtmlPar  \ShowPar}  
   {\end:itm \end:Itm \gdef\end:Itm{\HCode{</text:list-item>}}% 
      \gdef\end:itm{\HCode{</text:h>}}\HCode{<text:list-item><text:h  
      text:style-name="dt|<cond rtl class|>"\Hnewline >}\bgroup \bf}  
   {\egroup \end:itm \gdef\end:itm{\HCode{</text:p>}}%  
      \HCode{<text:p\Hnewline  
      text:style-name="dd|<cond rtl class|>"\Hnewline  
          >}} 
\NewConfigureOO{compactdesc}
\ConfigureOO{compactdesc}{|<compactdesc style|>}
>>>


\<compactdesc style\><<<
<text:list-style style:name="compactdesc" >
  <text:list-level-style-bullet
       text:level="1"
       text:style-name="Source-Text"
       text:bullet-char=" " >\Hnewline
     <style:list-level-properties
       text:space-before="0in" 
       text:min-label-width="0in"
       text:min-label-distance="0in" />\Hnewline
  </text:list-level-style-bullet>
  <text:list-level-style-bullet
       text:level="2"
       text:style-name="Source-Text"
       text:bullet-char=" " >\Hnewline
    <style:list-level-properties
       text:space-before="0.2in" 
       text:min-label-width="0.2in"
       text:min-label-distance="0.05in"
     />\Hnewline
  </text:list-level-style-bullet>
  <text:list-level-style-bullet
       text:level="3"
       text:style-name="Source-Text"
       text:bullet-char=" " >\Hnewline
    <style:list-level-properties
       text:space-before="0.4in"
       text:min-label-width="0.2in"
       text:min-label-distance="0.05in"/>\Hnewline
  </text:list-level-style-bullet>
  <text:list-level-style-bullet
       text:level="4"
       text:style-name="Source-Text"
       text:bullet-char=" " >\Hnewline
    <style:list-level-properties
       text:space-before="0.6in"
       text:min-label-width="0.2in"
       text:min-label-distance="0.05in"/>\Hnewline 
  </text:list-level-style-bullet>
</text:list-style>
>>>

\<configure ooffice paralist\><<<
\ConfigureList{asparaitem}%
   {\EndP|<boxed env|>\HCode{<!--asparaitem-->}\ShowPar\ShowIndent}
   {\EndP|<end boxed env|>\ShowPar}
   {\par\ShowPar{\parindent=10pt\leavevmode}}
   {}
\ConfigureList{asparaenum}%
   {\EndP|<boxed env|>\HCode{<!--asparaenum-->}\ShowPar\ShowIndent}
   {\EndP|<end boxed env|>\ShowPar}
   {\par\ShowPar{\parindent=10pt\leavevmode}}
   {}
\ConfigureList{asparadesc}%
   {\EndP|<boxed env|>\HCode{<!--asparadesc-->}\ShowPar\ShowIndent}
   {\EndP|<end boxed env|>\ShowPar}
   {\par\ShowPar{\parindent=10pt\leavevmode}}
   {}
>>>


%%%%%%%%%%%%%%%%%%%%%%%%%%%%%
\section{Bibliography}
%%%%%%%%%%%%%%%%%%%%%%%%%%%%

%%%%%%%%%%%%%
\subsection{Background}
%%%%%%%%%%%%%

\begin{verbatim}

<text:bibliography 
     text:style-name="Sect1" 
     text:name="Bibliography1">
   <text:bibliography-source>

      <text:index-title-template 
        text:style-name="Bibliography_20_Heading">Bibliography
      </text:index-title-template>

      <text:bibliography-entry-template 
              text:bibliography-type="article" 
              text:style-name="Bibliography_20_1">
         <text:index-entry-bibliography 
              text:bibliography-data-field="identifier"/>
         <text:index-entry-span>:  </text:index-entry-span>
         <text:index-entry-bibliography 
              text:bibliography-data-field="author"/>
         <text:index-entry-span>, </text:index-entry-span>
         <text:index-entry-bibliography 
              text:bibliography-data-field="title"/>
         <text:index-entry-span>, </text:index-entry-span>
         <text:index-entry-bibliography 
              text:bibliography-data-field="year"/>
      </text:bibliography-entry-template>

      <text:bibliography-entry-template 
              text:bibliography-type="book"
              text:style-name="Bibliography_20_1">
         <text:index-entry-bibliography 
              text:bibliography-data-field="identifier"/>
         <text:index-entry-span>: </text:index-entry-span>
         <text:index-entry-bibliography 
              text:bibliography-data-field="author"/>
         <text:index-entry-span>, </text:index-entry-span>
         <text:index-entry-bibliography 
              text:bibliography-data-field="title"/>
         <text:index-entry-span>, </text:index-entry-span>
         <text:index-entry-bibliography text:bibliography-data-field="year"/>
      </text:bibliography-entry-template>

   </text:bibliography-source>
   <text:index-body>
      <text:index-title 
        text:style-name="Sect2" 
        text:name="Bibliography1_Head">
           <text:p 
              text:style-name="Bibliography_20_Heading">Bibliography</text:p>
      </text:index-title>
      <text:p text:style-name="P1">1: author, title, year</text:p>
      <text:p text:style-name="P1">2: auth, my title, </text:p>
   </text:index-body>
</text:bibliography>
\end{verbatim}


%%%%%%%%%%%%%%%%%%%
\subsection{cite}
%%%%%%%%%%%%%%%%%%%

\<configure ooffice latex\><<<
\Configure{cite} 
   {\HCode{<text:bibliography-mark 
                 text:bibliography-type="custom1">}}  
   {\HCode{</text:bibliography-mark>}} 
   {\Link}         {\EndLink} 
\let\lat:cite\@cite
\def\@cite#1#2{{#1\if@tempswa , #2\fi}} 
>>>

\<\><<<
\Configure{cite}
   {\HCode{<text:bibliography-mark text:identifier="}\bgroup
      \def\:nbsp{ }\NoFonts
      \setbox0=\hbox\bgroup
   }  
   {\egroup
    \copy0
    \EndNoFonts  
    \HCode{"\Hnewline text:bibliography-type="custom1"
      >}[\box0]\HCode{</text:bibliography-mark>}\egroup
   }
%   {}{}
   {\Link}{\EndLink}
\let\lat:cite\@cite
\def\@cite#1#2{{#1\if@tempswa , #2\fi}} 
>>>

%   {\HCode{<!--cite-->}}  {\HCode{<!--/cite-->}}
%   {\Link}         {\EndLink}



\<configure ooffice apacite\><<<
\Configure{cite}
   {\HCode{<!--cite-->}\bgroup
      \def\:nbsp{ }\NoFonts
   }
   {\EndNoFonts\egroup \HCode{<!--/cite-->}}
   {}{}
\let\@cite\lat:cite
>>>






\begin{verbatim}
<text:bibliography-mark 
     text:identifier="shortname" 
     text:bibliography-type="article" 
     text:address="address" 
     text:annote="annotation" 
     text:author="author" 
     text:booktitle="book title" 
     text:chapter="chapter" 
     text:edition="edition" 
     text:editor="editor" 
     text:howpublished="publication type" 
     text:institution="institution" 
     text:journal="journal" 
     text:month="month" 
     text:note="note" 
     text:number="number" 
     text:organizations="organization" 
     text:pages="pages" 
     text:publisher="publisher" 
     text:school="university" 
     text:series="series" 
     text:title="title" 
     text:report-type="report type" 
     text:volume="volume" 
     text:year="year" 
     text:url="url" 
     text:isbn="isbn 1234567890">
[1]
</text:bibliography-mark>
\end{verbatim}



%%%%%%%%%%%%%%%%%%%
\subsection{Items}
%%%%%%%%%%%%%%%%%%%
  
How the problem in the following post can be resolved?

\url{http://www.mail-archive.com/dev@bibliographic.openoffice.org/msg00840.html}


\<ooffice bib with bibname\><<<
\def\ref:name{\bibname}
|<ooffice bib|>
>>>


\<ooffice bib with refname\><<<
\def\ref:name{\refname}
|<ooffice bib|>
>>>



\<ooffice bib\><<<
\HAssign\bib:N=0
\:CheckOption{bib-}\if:Option 
   \ConfigureList{thebibliography}%
      {\IgnorePar\EndP \gHAdvance\bib:N by 1
       \let\en:bib=\empty
      }
      {\en:bib}
      {\en:bib\def\en:bib{\HCode{</text:p>}}\HCode
                     {<text:p text:style-name="p-bibitem"  >}}
      {}
\else
   \Log:Note{for a degraded bibliography friendlier for conversion to .doc,
         use the command line option `bib-'}
   \ConfigureList{thebibliography}%
      {\IgnorePar\EndP \gHAdvance\bib:N by 1
       \HCode{<text:bibliography text:name="bib-\bib:N" >
          <text:bibliography-source>\Hnewline
             |<bibliography source|>\Hnewline
          </text:bibliography-source>\Hnewline
          <text:index-body>\Hnewline
       }\let\en:bib=\empty
      }
      {\en:bib\HCode{</text:index-body></text:bibliography>}}
      {\en:bib\def\en:bib{\HCode{</text:p>}}\HCode
                     {<text:p text:style-name="p-bibitem"  >}}
      {}
\fi
\Configure{bibitem}{\Link}{\EndLink\space }
>>>


\<ooffice bib\><<<
\NewConfigureOO{p-bibitem}
\ConfigureOO{p-bibitem}{%
   <style:style style:name="p-bibitem"\Hnewline
             style:family="paragraph" \Hnewline
             style:parent-style-name="Standard">\Hnewline
      <style:paragraph-properties  fo:margin-left="1cm"\Hnewline
                     fo:text-indent="-1cm"\Hnewline
                     fo:margin-right="0cm" \Hnewline
                     style:auto-text-indent="false">\Hnewline
              <style:tab-stops/>\Hnewline
      </style:paragraph-properties >\Hnewline
    </style:style>\Hnewline
}
>>>


\<bibliography source\><<<
<text:index-title-template>}\NoFonts\ref:name\EndNoFonts\HCode{%
</text:index-title-template>\Hnewline
%
  <text:bibliography-entry-template\Hnewline
      text:bibliography-type="custom1"
      text:style-name="Bibliography11"  >\Hnewline
%  
    <text:index-entry-bibliography
      text:bibliography-data-field="identifier"  />\Hnewline
%
    <text:index-entry-span>: </text:index-entry-span>\Hnewline
    <text:index-entry-bibliography
                  text:bibliography-data-field="author" />\Hnewline
    <text:index-entry-span>, </text:index-entry-span>\Hnewline
%    
    <text:index-entry-bibliography
       text:bibliography-data-field="title" />\Hnewline
    <text:index-entry-span>, </text:index-entry-span>\Hnewline
%    
    <text:index-entry-bibliography
       text:bibliography-data-field="year" />\Hnewline
    </text:bibliography-entry-template>\Hnewline
>>>



%%%%%%%%%%%%%%%%%%%
\subsection{jurabib}
%%%%%%%%%%%%%%%%%%%




\<configure ooffice jurabib\><<<
\HAssign\jbancorDepth=0
\Configure{jblink}
   {\HAdvance\jbancorDepth by 1 \Link}
   {\EndLink \HAdvance\jbancorDepth by -1 }
\Configure{jbanchor}
  {\bgroup
     \ifnum \jbancorDepth=0
       \Configure{Link}{text:a}{xlink:href=}%
                               {office:name=}{}%
     \else
       \Configure{Link}{text:span}{}{text:style-name=}{}% 
       \def\NOHREF:##1{}%
     \fi
   \Link}
  {\EndLink  \egroup}
>>>

Nested anchors are broken
See html4 comment regarding \verb+\Configure{jbanchor}{\Link}{\EndLink}+

%%%%%%%%%%%%%
\subsection{biblatex}
%%%%%%%%%%%%%

\<configure ooffice biblatex\><<<
\Configure{thebibliography}%
   {\ifvmode \IgnorePar\fi \EndP
    \EndP\HCode{<text:list text:style-name="printthebibliography"
                                 text:name="printthebibliography"\Hnewline>}%
    \PushMacro\end:itm
    \global\let\end:itm=\empty}
   {\ifvmode \IgnorePar\fi \EndP
    \PopMacro\end:itm \global\let\end:itm \end:itm
    \EndP\HCode{</text:p></text:list-item></text:list>}\ShowPar}
   {\ifvmode \IgnorePar\fi \EndP
    \end:itm \global\def\end:itm{\EndP\HCode{</text:p></text:list-item>}}%
    \HCode{<text:list-item><text:h  
           text:style-name="printthebibliography-dt"\Hnewline>}\bgroup \bf}
   {\ifvmode \IgnorePar\fi \EndP
    \egroup\HCode{</text:h><text:p 
           text:style-name="printthebibliography-dd"\Hnewline>}\par\ShowPar}
>>>

\<configure ooffice biblatex\><<<
\Configure{bibinit}{\IgnorePar}
>>>



\<configure ooffice biblatex\><<<
\Configure{printshorthands}
   {\ifvmode \IgnorePar\fi \EndP   
    \HCode{<text:list  
               text:style-name="printshorthands"
                     text:name="printshorthands"\Hnewline
           ><text:list-item><text:h  
           text:style-name="printshorthands-dt"\Hnewline>}%
   }
   {\ifvmode \IgnorePar\fi \EndP  
    \HCode{</text:list-item></text:list>}}
   {\ifvmode \IgnorePar\fi \EndP
    \HCode{</text:list-item><text:list-item><text:h  
           text:style-name="printshorthands-dt"\Hnewline>}}
   {\HCode{</text:h><text:p text:style-name="printshorthands-dd">}%
    \Next:EndP{\HCode{</text:p>}}%     
   }
\Configure{@shorthands}
   {\Configure{printfield-annotation}
      {\HTML:PAR{printshorthands-dd}{printshorthands-dd}}
      {}%
   }
>>>

\<configure ooffice biblatex\><<<
\Configure{biblatex-style}{authortitle}{}
\Configure{biblatex-style}{authortitle-comp}{}
\Configure{biblatex-style}{authortitle-tcomp}{}
\Configure{biblatex-style}{authortitle-icomp}{}
\Configure{biblatex-style}{authortitle-ibid}{}
\Configure{biblatex-style}{authoryear}{}
\Configure{biblatex-style}{authoryear-ibid}{}
\Configure{biblatex-style}{authoryear-comp}{}
\Configure{biblatex-style}{verbose}{}
\Configure{biblatex-style}{verbose-ibid}{}
\Configure{biblatex-style}{verbose-note}{}
\Configure{biblatex-style}{alphabetic-verb}{}
\Configure{biblatex-style}{alphabetic}{}
\Configure{biblatex-style}{numeric}{}
\Configure{biblatex-style}{numeric-verb}{}
\Configure{biblatex-style}{numeric-comp}{}
\Configure{biblatex-style}{draft}{}
\Configure{biblatex-style}{reading}{}
\Configure{biblatex-style}{verbose-trad1}{}
\Configure{biblatex-style}{verbose-trad2}{}
>>>



\<configure ooffice latex\><<<
\NewConfigureOO{printshorthands}
\ConfigureOO{printshorthands}{|<printshorthands style|>}
>>>



\<printshorthands style\><<<
<text:list-style style:name="printshorthands" >
  <text:list-level-style-bullet
       text:level="1"
       text:style-name="Source-Text"
       text:bullet-char=" " >\Hnewline
     <style:list-level-properties
       text:space-before="0in" 
       text:min-label-width="0in"
       text:min-label-distance="0in" />\Hnewline
  </text:list-level-style-bullet>
</text:list-style>
<style:style 
    style:name="printshorthands-dd" 
    style:family="paragraph" 
    style:parent-style-name="Standard"> 
 <style:paragraph-properties  
    fo:margin-left="0.35in" 
    fo:margin-right="0in"  
    fo:margin-top="0cm" 
    fo:margin-bottom="0.2cm" 
    fo:text-indent="0in"  
    style:auto-text-indent="false"/> 
</style:style> 
<style:style
       style:name="printshorthands-dt"
       style:family="paragraph"
       style:parent-style-name="">
  <style:paragraph-properties
       fo:margin-left="0.1in"
       fo:margin-right="0in"
       fo:text-indent="0in"
       style:auto-text-indent="false"/>
</style:style>
>>>








\<configure ooffice latex\><<<
\NewConfigureOO{printthebibliography}
\ConfigureOO{printthebibliography}{|<printthebibliography style|>}
>>>



\<printthebibliography style\><<<
<text:list-style style:name="printthebibliography" >
  <text:list-level-style-bullet
       text:level="1"
       text:style-name="Source-Text"
       text:bullet-char=" " >\Hnewline
     <style:list-level-properties
       text:space-before="0in" 
       text:min-label-width="0in"
       text:min-label-distance="0in" />\Hnewline
  </text:list-level-style-bullet>
</text:list-style>
<style:style 
    style:name="printthebibliography-dd" 
    style:family="paragraph" 
    style:parent-style-name="Standard"> 
 <style:paragraph-properties  
    fo:margin-left="0.35in" 
    fo:margin-right="0in"  
    fo:margin-top="0cm" 
    fo:margin-bottom="0.2cm" 
    fo:text-indent="-0.35in"
    style:auto-text-indent="false"/> 
</style:style> 
<style:style
       style:name="printthebibliography-dt"
       style:family="paragraph"
       style:parent-style-name="">
  <style:paragraph-properties
       fo:margin-top="0cm" 
       fo:margin-bottom="0cm" 
       style:auto-text-indent="false"/>
</style:style>
>>>






%%%%%%%%%%%%%
\subsection{footnote-dw}
%%%%%%%%%%%%%

\<configure ooffice footnote-dw\><<<
\protected\long\def\mkfootnotecite#1{#1}
>>>




%%%%%%%%%%%%%%%%%%%%%%%%%%%%%
\section{List-Based Environments}
%%%%%%%%%%%%%%%%%%%%%%%%%%%%



%%%%%%%%%%%%%%%%%%%%%%%%%%%%%
\subsection{Quotes}
%%%%%%%%%%%%%%%%%%%%%%%%%%%%


\<ooffice begin-end env\><<<
\ConfigureEnv{quote}
   {|<begin env section|>}
   {|<end env section|>\par\ShowPar}
   {\EndP \ifvmode \IgnorePar\fi
    \bgroup |<HtmlPar at quote|>%
   }
   {\IgnorePar\EndP \egroup \ShowPar \ShowIndent}
>>>


\<HtmlPar at quote\><<<
\Configure{HtmlPar}
   {\EndP \HCode{|<input line num|>%
                 <text:p text:style-name="quote|<cond rtl class|>" 
                         >}}
   {\EndP \HCode{|<input line num|>%
                 <text:p text:style-name="quote|<cond rtl class|>"
                          >}}
   {\HCode{</text:p>}}
   {\HCode{</text:p>}}%
>>>

\<configure ooffice latex\><<<
\NewConfigureOO{quote}
\ConfigureOO{quote}{\Hnewline
<style:style style:name="quote"
             style:family="paragraph"
             style:parent-style-name="Text-body"
             style:next-style-name="Text-body">\Hnewline
<style:paragraph-properties  fo:margin-left="1cm"
             fo:margin-right="1cm"
             fo:margin-top="0.199cm"
             fo:margin-bottom="0.199cm"
             fo:text-indent="0cm"
             style:auto-text-indent="false"/>\Hnewline
</style:style>
\Hnewline |<quote rtl|>}
>>>



%%%%%%%%%%%%%
\subsection{Environment Envelop}
%%%%%%%%%%%%%




\<configure ooffice latex\><<<
\HAssign\BegEnd:N=0
\HAssign\BegEnd:D=0
>>>



\<begin env section\><<<
\ifvmode \IgnorePar\fi \EndP
\ifnum \BegEnd:D=0   
   |<sectioned env|>%
\else
   |<boxed env|>%
\fi
\gHAdvance\BegEnd:D by 1
>>>

\<end env section\><<<
\gHAdvance\BegEnd:D by -1
\ifvmode \IgnorePar\fi\EndP 
\ifnum \BegEnd:D=0
   |<end sectioned env|>%
\else
   |<end boxed env|>%
\fi
>>>




\<sectioned env\><<<
\gHAdvance\BegEnd:N by 1
\HCode{<text:section
        text:style-name="begin-end-env" 
        text:name="begin-end-env-\BegEnd:N"
       >%
       <text:p text:style-name="begin-env-p" ></text:p>}%
>>>

\<end sectioned env\><<<
\HCode{<text:p text:style-name="end-env-p" ></text:p></text:section>}%
>>>


\<boxed env\><<<
\gHAdvance\BegEnd:N by 1
\hbox{\HCode{<text:p><draw:frame 
                draw:name="begin-end-env-\BegEnd:N"  
                 draw:style-name="env-frame"  
                 text:anchor-type="as-char"  
%                 fo:min-width="0.14in"  
                 svg:width="90\%"  
                 draw:z-index="0"
       >  
       <draw:text-box 
%           fo:min-height="0.14in" 
       >}}% 
>>>


\<end boxed env\><<<
\hbox{\HCode{</draw:text-box>  
</draw:frame></text:p> }}%
>>>





\<configure ooffice latex\><<<
\NewConfigureOO{begin-end-env}
\ConfigureOO{begin-end-env}{\Hnewline
<style:style style:name="begin-env-p" style:family="paragraph">\Hnewline
   <style:paragraph-properties  fo:margin-top="2mm"
                     fo:margin-bottom="0mm" 
                     fo:line-height="0.1mm" />\Hnewline
</style:style>\Hnewline
<style:style style:name="end-env-p" style:family="paragraph">\Hnewline
   <style:paragraph-properties  fo:margin-bottom="2mm"
                     fo:margin-top="0mm" 
                     fo:line-height="0.1mm" />\Hnewline
</style:style>\Hnewline
}
>>>

The indirect use of paragraphs is due to  OpenOffice not recognizing
vertical spaces around sections.

\begin{verbatim}
<style:style style:name="begin-end-env"
             style:family="section">\Hnewline
  <style:text-properties  fo:margin-top="0.4cm"
                     fo:margin-bottom="0.4cm" />\Hnewline
</style:style>\Hnewline
\end{verbatim}


\<configure ooffice latex\><<<
\NewConfigureOO{env-frame}
\ConfigureOO{env-frame}{\Hnewline
<style:style  
    style:name="env-frame"  
    style:family="graphic"  
>\Hnewline
  <style:graphic-properties 
     style:vertical-pos="top"  
     style:vertical-rel="baseline"  
     style:horizontal-pos="from-left" 
     style:horizontal-rel="paragraph-content"
     fo:padding="0in"  
     fo:border="none"  
     style:shadow="none" 
  />\Hnewline
</style:style>\Hnewline 
}
>>>



\<\><<<
    style:parent-style-name="EnvFrame" 


<style:style 
    style:name="EnvFrame" 
    style:family="graphic" 
>\Hnewline
   <style:graphic-properties  
        text:anchor-type="paragraph"  
%        style:wrap="parallel"  
%        style:number-wrapped-paragraphs="no-limit"  
        style:wrap-contour="false"  
        style:vertical-pos="top"  
        style:vertical-rel="paragraph-content"  
        style:horizontal-pos="center"  
        style:horizontal-rel="paragraph-content"  
        fo:padding="0in"  
        fo:border="0.0in"/> \Hnewline
</style:style> \Hnewline

>>>



%%%%%%%%%%%%%%%%%%%%%%%%%%%%%
\subsection{Quotations}
%%%%%%%%%%%%%%%%%%%%%%%%%%%%

\<ooffice begin-end env\><<<
\ConfigureEnv{quotation}
   {|<begin env section|>}
   {|<end env section|>\par\ShowPar \noindent}
   {\IgnorePar\EndP \bgroup  |<HtmlPar for quatation|>}       
   {\EndP\IgnorePar \egroup \ShowPar }  
>>>

\<HtmlPar for quatation\><<<
\Configure{HtmlPar}
   {\EndP \HCode{<text:p text:style-name="quotation|<cond rtl class|>"
                           >}}
   {\EndP \HCode{<text:p text:style-name="quotation|<cond rtl class|>"
                           >}}
   {\HCode{</text:p>}}
   {\HCode{</text:p>}}%
>>>

\<ooffice begin-end env\><<<
\NewConfigureOO{quotation}
\ConfigureOO{quotation}
   {\Hnewline
      <style:style style:name="quotation"
                   style:family="paragraph" 
                   style:parent-style-name="Standard">\Hnewline
        <style:paragraph-properties 
                   fo:margin-left="1cm"
                   fo:margin-right="1cm"
                   fo:margin-top="0cm"
                   fo:margin-bottom="0cm"
                   fo:text-indent="0.499cm"
                   style:auto-text-indent="true"/>\Hnewline
      </style:style>\Hnewline |<quotation rtl|>%
   }
>>>




%%%%%%%%%%%%%%%%%%%%%%%%%%%%%
\subsection{flush right and left}
%%%%%%%%%%%%%%%%%%%%%%%%%%%%




\<ooffice begin-end env\><<<
\ConfigureEnv{flushright}
  {|<begin env section|>\HCode{<!--begin flushright-->}}
  {|<end env section|>%
   \HCode{<!--end flushright-->}|<try env inline par|>}
  {} {}
\ConfigureList{flushright}
 {}{}{|<HtmlPar at flushright|>\par}{}
>>>


\<HtmlPar at flushright\><<<
\Configure{HtmlPar}
   {\EndP \HCode{<text:p text:style-name="flushright"  >}}
   {\EndP \HCode{<text:p text:style-name="flushright"  >}}
   {\HCode{</text:p>}}
   {\HCode{</text:p>}}%
>>>



\<ooffice begin-end env\><<<
\NewConfigureOO{flushright}
\ConfigureOO{flushright}
  {\Hnewline
<style:style style:name="flushright"
             style:family="paragraph"
             style:parent-style-name="Text-body"
             style:next-style-name="Text-body"
             style:master-page-name="">
<style:paragraph-properties  fo:margin-top="0.199cm"
             fo:margin-bottom="0.199cm"
             fo:text-align="end"
             style:justify-single-word="false"
             style:page-number="0"/>
</style:style>\Hnewline
  }
>>>




\<ooffice begin-end env\><<<
\ConfigureEnv{flushleft}
  {|<begin env section|>\HCode{<!--begin flushleft-->}}
  {|<end env section|>%
   \HCode{<!--end flushleft-->}\par\ShowPar}
  {} {}
\ConfigureList{flushleft}
 {}{}{|<HtmlPar at flushleft|>\par}{}
>>>


\<HtmlPar at flushleft\><<<
\Configure{HtmlPar}
   {\EndP \HCode{<text:p text:style-name="flushleft"  >}}
   {\EndP \HCode{<text:p text:style-name="flushleft"  >}}
   {\HCode{</text:p>}}
   {\HCode{</text:p>}}%
>>>



\<ooffice begin-end env\><<<
\NewConfigureOO{flushleft}
\ConfigureOO{flushleft}
  {\Hnewline
<style:style style:name="flushleft"
             style:family="paragraph"
             style:parent-style-name="Text-body"
             style:next-style-name="Text-body">
<style:paragraph-properties  fo:margin-top="0.199cm"
             fo:margin-bottom="0.199cm"
             fo:text-align="start"
             style:justify-single-word="false"/>
</style:style>
  \Hnewline}
>>>




%%%%%%%%%%%%%%%%%%%%%%%%%%%%%
\subsection{center}
%%%%%%%%%%%%%%%%%%%%%%%%%%%%


\<ooffice begin-end env\><<<
\ConfigureEnv{center}
  {|<begin env section|>\HCode{<!--begin center-->}}
  {|<end env section|>%
   \HCode{<!--end center-->}\par|<try env inline par|>}
  {} {}
\ConfigureList{center}
 {}{}{|<HtmlPar at center|>\par}{}
>>>

The \verb=\par= at the end of \verb=\ConfigureEnv{center}=
is required for cases like the following one.

\begin{verbatim}
\documentclass[titlepage]{article}  
\begin{document}  
    \begin{abstract}  
    file. (see below)  
     
    file. (see below)  
     
    file. (see below)  
    \end{abstract}  
\end{document}  
\end{verbatim}

\<HtmlPar at center\><<<
\Configure{HtmlPar}
   {\EndP \HCode{<text:p text:style-name="center"  >}}
   {\EndP \HCode{<text:p text:style-name="center"  >}}
   {\HCode{</text:p>}}
   {\HCode{</text:p>}}%
>>>

\<ooffice begin-end env\><<<
\NewConfigureOO{center}
\ConfigureOO{center}
  {\Hnewline
   <style:style style:name="center"
             style:family="paragraph"
             style:parent-style-name="Text-body"
             style:next-style-name="Text-body">
<style:paragraph-properties  fo:margin-top="0.199cm"
             fo:margin-bottom="0.199cm"
             fo:text-align="center"
             style:justify-single-word="false">
   <style:tab-stops/>
</style:paragraph-properties >
</style:style>\Hnewline
   }
>>>


%%%%%%%%%%%%%%%%%%%%%%%%%%%%%
\subsection{centerline}
%%%%%%%%%%%%%%%%%%%%%%%%%%%%


\<ooffice begin-end env\><<<
\Configure{centerline}
   {\EndP \ifvmode \IgnorePar \fi
    \HCode{<text:p text:style-name="FirstCenterPar"  >}}
   {\HCode{</text:p>}|<try inline par|>}
>>>


%%%%%%%%%%%%%%%%%%%%%%%%%%%%%%%
\section{Sections}
%%%%%%%%%%%%%%%%%%%%%%%%%%%%%%%




%%%%%%%%%%%%%%%%%%%%%%%%%%%%%%%
\subsection{Shared}
%%%%%%%%%%%%%%%%%%%%%%%%%%%%%%%


THe following protects section titles from side effects, and also is 
introduced for protecting writeups to tables of contents.



\<configure ooffice latex\><<<
\Configure{NoSection}
  {\let\sv:index\index \let\sv:label\label
   \let\sv:newline\newline \def\newline{ }%
   \let\sv:setfontsize\@setfontsize  \let\@setfontsize\:gobbleIII
   \let\index\@gobble  \let\label\@gobble
   \let\sva:rEfLiNK\a:rEfLiNK
   \let\svb:rEfLiNK\b:rEfLiNK
   \let\svc:rEfLiNK\c:rEfLiNK
   \expandafter\let\expandafter\svC:rEfLiNK\c:rEfLiNK
   \Configure{ref}{}{}{}%  \let\sv:ref\ref  \let\ref\@gobble
  }
  {\let\index\sv:index \let\label\sv:label
   \let\a:rEfLiNK\sva:rEfLiNK
   \let\b:rEfLiNK\svb:rEfLiNK
   \let\c:rEfLiNK\svc:rEfLiNK
   \expandafter\let\c:rEfLiNK\svC:rEfLiNK  %  \let\ref\sv:ref
   \let\newline\sv:newline    \let\@setfontsize\sv:setfontsize
  }

>>>




\<configure ooffice latex\><<<
\Configure{InsertTitle}{\let\label\lb:l
   \let\ref\o:ref \Configure{ref}{}{}{}}
\Configure{AfterTitle}{\let\index\:index
       \let\ref\:ref  \let\label\lb:l }
>>>


The labels are needed for labels \HPage{nested}
\begin{verbatim}
\documentclass{article} 
\begin{document} 
\section{Introduction} 
 
AAA \ref{1111} 
 
\section{BBB\label{1111}} 
 
\end{document} 
\end{verbatim}
\EndHPage{}
within titles.

\<basic ooffice sections\><<<
\ifx \part\:UnDef\else
  \Configure{part}
     {\IgnorePar\EndP \HCode{<!--start part-->}}   
     {\IgnorePar\EndP \HCode{<!--end part-->}}
     {\HCode{<text:p text:style-name="part"  >
             <text:span text:style-name="part">}\TitleMark\space
      \HCode{</text:span>}%
      \HCode{<text:line-break/>}}
     {\HCode{</text:p>\Hnewline}\par\ShowPar\IgnoreIndent}
  \ConfigureMark{part}{\partname\space\thepart}
  \NewConfigureOO{part}
  \ConfigureOO{part}{\Hnewline
  <style:style style:name="part"
                    style:family="paragraph"
                    style:parent-style-name="Heading|<cond rtl class|>"
                    style:next-style-name="Text-body"
                    style:list-style-name="part">\Hnewline
  <style:paragraph-properties  fo:margin-top="0.52cm"
                    fo:margin-bottom="0.21cm"
                    fo:font-size="18pt"/>\Hnewline
  </style:style>\Hnewline
  }
\fi
>>>




\<od definitions\><<<
\def\list:text:level{text:level}
\def\text:outlinelevel{text:outline-level}
\def\style:tableproperties{style:table-properties }
\def\style:sectionproperties{style:section-properties }
\def\text:orderedlist{text:list }
\def\text:unorderedlist{text:list }
>>>


\<ooffice chapters\><<<
\Configure{chapter}
   {\IgnorePar\EndP \HCode{<!--start of chapter-->|<new page|>}}   
   {\IgnorePar\EndP \HCode{<!--end of chapter-->}}
   {\HCode{<text:h 
           text:style-name="\csname a:Heading-1|<cond rtl class|>\endcsname" 
        \text:outlinelevel="1"  ><text:span 
        text:style-name="chapter"  >}\TitleMark\space
    \HCode{</text:span>}%
    \HCode{<text:line-break/>}\IgnoreIndent}
   {\HCode{</text:h>\Hnewline}\par\ShowPar}
\Configure{likechapter}
   {\IgnorePar\EndP \HCode{<!--start of chapter*-->|<new page|>}}   
   {\IgnorePar\EndP \HCode{<!--end of chapter*-->}}
   {\HCode{<text:h 
           text:style-name="\csname a:Heading-1|<cond rtl class|>\endcsname" 
        \text:outlinelevel="1"  >}\IgnoreIndent}
   {\HCode{</text:h>\Hnewline}\par\ShowPar}
>>>

\<report chapters\><<<
\ConfigureMark{chapter}
   {\chaptername \ \thechapter}
>>>

\<book chapters\><<<
\ConfigureMark{chapter}
   {\if@mainmatter \chaptername \ \thechapter\fi}
>>>



\<book chapters\><<<
\ConfigureMark{appendix}
   {\if@mainmatter \appendixname \ \thechapter\fi}
\Configure{appendix}
   {\IgnorePar\EndP \HCode{<!--start of appendix-->|<new page|>}}   
   {\IgnorePar\EndP \HCode{<!--end of appendix-->}}
   {\HCode{<text:h text:style-name="\csname a:Heading-1|<cond rtl class|>\endcsname" 
        \text:outlinelevel="1"  ><text:span 
        text:style-name="chapter"  >}\TitleMark\space
    \HCode{</text:span>}%
    \HCode{<text:line-break/>}\IgnoreIndent}
   {\HCode{</text:h>\Hnewline}\par\ShowPar}
>>>






\<basic ooffice sections\><<<
\ConfigureMark{section}
   {\ifnum \c:secnumdepth>\c@secnumdepth \expandafter\:gobble
    \else  \@seccntformat{section}\fi }
\Configure{section}
   {\IgnorePar\EndP \HCode{<!--start of section-->}}   
   {\IgnorePar\EndP \HCode{<!--end of section-->}}
   {\HCode{<text:h 
            text:style-name="\csname a:Heading-2|<cond rtl class|>\endcsname" 
                   \text:outlinelevel="2" ><text:span
                   text:style-name="section" >}\TitleMark\space
    \HCode{</text:span>}\IgnoreIndent}
   {\HCode{</text:h>\Hnewline}\par\ShowPar}
>>>



\<basic ooffice sections\><<<
\ConfigureMark{subsection}
   {\ifnum \c:secnumdepth>\c@secnumdepth  \expandafter\:gobble
    \else \@seccntformat{subsection}\fi }
\Configure{subsection}
   {\IgnorePar\EndP \HCode{<!--start of subsection-->}}   
   {\IgnorePar\EndP \HCode{<!--end of subsection-->}}
   {\HCode{<text:h 
           text:style-name="\csname a:Heading-3|<cond rtl class|>\endcsname"
                   \text:outlinelevel="3"   ><text:span
                   text:style-name="subsection"  >}\TitleMark\space
    \HCode{</text:span>}\IgnoreIndent}
   {\HCode{</text:h>\Hnewline}\par\ShowPar}
>>>  


\<basic ooffice sections\><<<
\ConfigureMark{subsubsection}
   {\ifnum \c:secnumdepth>\c@secnumdepth  \expandafter\:gobble
    \else \@seccntformat{subsubsection}\fi }
\Configure{subsubsection}
   {\IgnorePar\EndP \HCode{<!--start of subsubsection-->}}   
   {\IgnorePar\EndP \HCode{<!--end of subsubsection-->}}
   {\HCode{<text:h 
           text:style-name="\csname a:Heading-4|<cond rtl class|>\endcsname"
                   \text:outlinelevel="4"   ><text:span
                   text:style-name="subsubsection"  >}\TitleMark\space
    \HCode{</text:span>}\IgnoreIndent}
   {\HCode{</text:h>\Hnewline}\par\ShowPar}
>>>  


\<basic ooffice sections\><<<
\Configure{likesection}
   {\IgnorePar\EndP \HCode{<!--start of section*-->}}   
   {\IgnorePar\EndP \HCode{<!--end of section*-->}}
   {\HCode{<text:h 
           text:style-name="\csname a:Heading-2|<cond rtl class|>\endcsname" 
                   \text:outlinelevel="2"  >}\IgnoreIndent}
   {\HCode{</text:h>\Hnewline}\par\ShowPar}
\Configure{likesubsection}
   {\IgnorePar\EndP \HCode{<!--start of subsection*-->}}   
   {\IgnorePar\EndP \HCode{<!--end of subsection*-->}}
   {\HCode{<text:h 
           text:style-name="\csname a:Heading-3|<cond rtl class|>\endcsname" 
                   \text:outlinelevel="3"  >}\IgnoreIndent}
   {\HCode{</text:h>\Hnewline}\par\ShowPar}
>>>



\<basic ooffice sections\><<<
\Configure{paragraph}
   {\par\ShowPar \noindent\HCode{<!--start of paragraph-->}}   
   {\HCode{<!--end of paragraph-->}\par\ShowPar}
   {\HCode{<text:span text:style-name="paragraph-h" >}}
   {\HCode{</text:span><text:s/><text:s/>\Hnewline}\IgnorePar}
\Configure{subparagraph}
   {\par\ShowPar \noindent\HCode{<!--start of subparagraph-->}}   
   {\HCode{<!--end of subparagraph-->}\par\ShowPar}
   {\HCode{<text:span text:style-name="subparagraph-h" >}}
   {\HCode{</text:span><text:s/><text:s/>\Hnewline}\IgnorePar}
>>>  

\<basic ooffice sections\><<<
\NewConfigureOO{paragraph-h}
\ConfigureOO{paragraph-h}{
   <style:style style:name="paragraph-h" style:family="text">\Hnewline
      <style:text-properties fo:font-style="normal" 
                        fo:font-weight="bold"/>\Hnewline
   </style:style>\Hnewline
}
>>>



\<basic ooffice sections\><<<
\NewConfigureOO{subparagraph-h}
\ConfigureOO{subparagraph-h}{
   <style:style style:name="subparagraph-h" style:family="text">\Hnewline
      <style:text-properties fo:font-style="normal" 
                        fo:font-weight="bold"/>\Hnewline
   </style:style>\Hnewline
}
>>>


\<basic ooffice sectionsNO\><<<
\Configure{paragraph}
   {\IgnorePar\EndP \HCode{<!--start of paragraph-->}}   
   {\IgnorePar\EndP \HCode{<!--end of paragraph-->}}
   {\HCode{<text:h 
           text:style-name="\csname a:Heading-5|<cond rtl class|>\endcsname"
                   \text:outlinelevel="5"  >}}
   {\HCode{</text:h>\Hnewline}\par\ShowPar}
\Configure{subparagraph}
   {\IgnorePar\EndP \HCode{<!--start of subparagraph-->}}   
   {\IgnorePar\EndP \HCode{<!--end of subparagraph-->}}
   {\HCode{<text:h 
           text:style-name="\csname a:Heading-6|<cond rtl class|>\endcsname"
                   \text:outlinelevel="6"  >}}
   {\HCode{</text:h>\Hnewline}\par\ShowPar}
>>>  




\<configure ooffice tex4ht\><<<
\Configure{TocAt}{}{}
\Configure{TocAt*}{}{}
>>>


A configuration `\verb!\Configure{Heading-1}{Heading 1}!'
will provide the default setting of OpenOffice.  It might be 
a more desirable for translations into MS Word.

\<basic ooffice section styles\><<<
\NewConfigure{Heading-1}{1}
\Configure{Heading-1}{Heading-1}
\NewConfigureOO{Heading-1}
\ConfigureOO{Heading-1}{\Hnewline
<style:style style:name="Heading-1"
                  style:family="paragraph"
                  style:parent-style-name="Heading"
                  style:next-style-name="Text-body"
                  style:class="text">\Hnewline
<style:paragraph-properties  fo:font-size="17.2999992370605pt"
                  fo:font-weight="bold"
                  style:font-size-asian="115\%"
                  style:font-weight-asian="bold"
                  style:font-size-complex="115\%"
                  style:font-weight-complex="bold"
                  fo:margin-top="40pt"
                  fo:margin-bottom="25pt"
                />
</style:style>
\Hnewline |<Heading 1 rtl|> }
>>>




\<basic ooffice section styles\><<<
\NewConfigure{Heading-2}{1}
\Configure{Heading-2}{Heading-2}
\NewConfigureOO{Heading-2}
\ConfigureOO{Heading-2}{\Hnewline
<style:style style:name="Heading-2"
                  style:family="paragraph"
                  style:parent-style-name="Heading"
                  style:next-style-name="Text-body"
                  style:class="text">\Hnewline
<style:paragraph-properties  fo:font-size="14.3999996185303pt"
                  style:font-size-complex="14pt"
                  style:font-style-complex="italic"
                  style:font-weight-complex="bold"
                  fo:margin-top="12pt"
                  fo:margin-bottom="9pt"
               />
</style:style>
\Hnewline |<Heading 2 rtl|> }
>>>




\<basic ooffice section styles\><<<
\NewConfigure{Heading-3}{1}
\Configure{Heading-3}{Heading-3}
\NewConfigureOO{Heading-3}
\ConfigureOO{Heading-3}{\Hnewline
<style:style style:name="Heading-3"
                  style:family="paragraph"
                  style:parent-style-name="Heading"
                  style:next-style-name="Text-body"
                  style:class="text">\Hnewline
<style:paragraph-properties  fo:font-size="12pt"
                  fo:font-weight="bold"
                  style:font-size-asian="14pt"
                  style:font-weight-asian="bold"
                  style:font-size-complex="14pt"
                  style:font-weight-complex="bold"/>
</style:style>
\Hnewline |<Heading 3 rtl|>}
>>>



\<basic ooffice section styles\><<<
\NewConfigure{Heading-4}{1}
\Configure{Heading-4}{Heading-4}
\NewConfigureOO{Heading-4}
\ConfigureOO{Heading-4}{\Hnewline
<style:style style:name="Heading-4"
                  style:family="paragraph"
                  style:parent-style-name="Heading"
                  style:next-style-name="Text-body"
                  style:class="text">\Hnewline
<style:paragraph-properties  fo:font-size="12pt"
                  fo:font-weight="bold"
                  fo:margin-top="0.42cm"
                  fo:margin-bottom="0cm"
                  style:font-size-complex="85\%"
                  style:font-style-complex="italic"
                  style:font-weight-complex="bold"
/>
</style:style>
\Hnewline  |<Heading 4 rtl|>}
>>>



\<basic ooffice section styles\><<<
\NewConfigure{Heading-5}{1}
\Configure{Heading-5}{Heading-5}
\NewConfigureOO{Heading-5}
\ConfigureOO{Heading-5}{\Hnewline
<style:style style:name="Heading-5"
                  style:family="paragraph"
                  style:parent-style-name="Heading"
                  style:next-style-name="Text-body"
                  style:class="text">\Hnewline
<style:paragraph-properties  fo:margin-left="0cm"
                  fo:margin-right="0cm"
                  fo:margin-top="0.42cm"
                  fo:margin-bottom="0cm"
                  fo:font-size="12pt"
                  fo:font-style="italic"
                  fo:font-weight="normal"
                  style:font-size-complex="85\%"
                  style:font-weight-complex="bold"
                  fo:text-indent="0cm"
                  style:auto-text-indent="false"/>
</style:style>
\Hnewline |<Heading 5 rtl|>}
>>>


\<basic ooffice section styles\><<<
\NewConfigure{Heading-6}{1}
\Configure{Heading-6}{Heading-6}
\NewConfigureOO{Heading-6}
\ConfigureOO{Heading-6}{\Hnewline
<style:style style:name="Heading-6"
                  style:family="paragraph"
                  style:parent-style-name="Heading"
                  style:next-style-name="Text-body"
                  style:class="text">\Hnewline
<style:paragraph-properties  fo:font-size="75\%"
                  fo:font-weight="bold"
                  style:font-size-asian="75\%"
                  style:font-weight-asian="bold"
                  style:font-size-complex="75\%"
                  style:font-weight-complex="bold"/>
</style:style>
\Hnewline |<Heading 6 rtl|>}
>>>

\<basic ooffice section styles\><<<
\NewConfigure{Heading-7}{1}
\Configure{Heading-7}{Heading-7}
\NewConfigureOO{Heading-7}
\ConfigureOO{Heading-7}{\Hnewline
<style:style style:name="Heading-7"
                  style:family="paragraph"
                  style:parent-style-name="Heading"
                  style:next-style-name="Text-body"
                  style:class="text">\Hnewline
<style:paragraph-properties  fo:font-size="75\%"
                  fo:font-weight="bold"
                  style:font-size-asian="75\%"
                  style:font-weight-asian="bold"
                  style:font-size-complex="75\%"
                  style:font-weight-complex="bold"/>
</style:style>
\Hnewline |<Heading 7 rtl|>}
>>>

\<basic ooffice section styles\><<<
\NewConfigure{Heading-8}{1}
\Configure{Heading-8}{Heading-8}
\NewConfigureOO{Heading-8}
\ConfigureOO{Heading-8}{\Hnewline
<style:style style:name="Heading-8"
                  style:family="paragraph"
                  style:parent-style-name="Heading"
                  style:next-style-name="Text-body"
                  style:class="text">\Hnewline
<style:paragraph-properties  fo:font-size="75\%"
                  fo:font-weight="bold"
                  style:font-size-asian="75\%"
                  style:font-weight-asian="bold"
                  style:font-size-complex="75\%"
                  style:font-weight-complex="bold"/>
</style:style>
\Hnewline |<Heading 8 rtl|>}
>>>


\<basic ooffice section styles\><<<
\NewConfigure{Heading-9}{1}
\Configure{Heading-9}{Heading-9}
\NewConfigureOO{Heading-9}
\ConfigureOO{Heading-9}{\Hnewline
<style:style style:name="Heading-9"
                  style:family="paragraph"
                  style:parent-style-name="Heading"
                  style:next-style-name="Text-body"
                  style:class="text">\Hnewline
<style:paragraph-properties  fo:font-size="75\%"
                  fo:font-weight="bold"
                  style:font-size-asian="75\%"
                  style:font-weight-asian="bold"
                  style:font-size-complex="75\%"
                  style:font-weight-complex="bold"/>
</style:style>
\Hnewline |<Heading 9 rtl|>}
>>>


\<basic ooffice section styles\><<<
\NewConfigure{Heading-10}{1}
\Configure{Heading-10}{Heading-10}
\NewConfigureOO{Heading-10}
\ConfigureOO{Heading-10}{\Hnewline
<style:style style:name="Heading-10"
                  style:family="paragraph"
                  style:parent-style-name="Heading"
                  style:next-style-name="Text-body"
                  style:class="text">\Hnewline
<style:paragraph-properties  fo:font-size="75\%"
                  fo:font-weight="bold"
                  style:font-size-asian="75\%"
                  style:font-weight-asian="bold"
                  style:font-size-complex="75\%"
                  style:font-weight-complex="bold"/>
</style:style>
\Hnewline |<Heading 10 rtl|>}
>>>


%%%%%%%%%%%%%%%%%%%%%%%%%%%%%%%
\subsection{Scr}
%%%%%%%%%%%%%%%%%%%%%%%%%%%%%%%

\<configure ooffice scrbook\><<<
\ConfigureMark{chapter}
   {\if@mainmatter \thechapter.~~\fi}
>>>


\<configure ooffice scrreprt\><<<
\ConfigureMark{chapter}{\thechapter~~}
>>>

\<scr ooffice chapters\><<<
\Configure{chapter}
   {\IgnorePar\EndP \HCode{<!--start of chapter-->|<new page|>}}   
   {\IgnorePar\EndP \HCode{<!--end of chapter-->}}
   {\HCode{<text:h text:style-name="\csname a:Heading-1\endcsname" 
                   \text:outlinelevel="1"  ><text:span
                   text:style-name="chapter">}\TitleMark\space
    \HCode{</text:span>}\IgnoreIndent}
   {\HCode{</text:h>\Hnewline}\par\ShowPar}
\Configure{likechapter}
   {\IgnorePar\EndP \HCode{<!--start of chapter*-->|<new page|>}}   
   {\IgnorePar\EndP \HCode{<!--end of chapter*-->}}
   {\HCode{<text:h text:style-name="\csname a:Heading-1\endcsname" 
                   \text:outlinelevel="1"  >}\IgnoreIndent}
   {\HCode{</text:h>\Hnewline}\par\ShowPar}
\Configure{addchap}
   {\IgnorePar\EndP \HCode{<!--start of addchap-->|<new page|>}}   
   {\IgnorePar\EndP \HCode{<!--end of addchap-->}}
   {\HCode{<text:h text:style-name="\csname a:Heading-1\endcsname" 
                   \text:outlinelevel="1"  >}\IgnoreIndent}
   {\HCode{</text:h>\Hnewline}\par\ShowPar}
>>>


\<basic scr ooffice sections\><<<
\ConfigureMark{section}
   {\ifnum \c:secnumdepth>\c@secnumdepth \expandafter\:gobble
    \else  \thesection\space\fi }
\Configure{section}
   {\IgnorePar\EndP \HCode{<!--start of section-->}}   
   {\IgnorePar\EndP \HCode{<!--end of section-->}}
   {\HCode{<text:h text:style-name="\csname a:Heading-2\endcsname" 
                   \text:outlinelevel="2"  ><text:span
                   text:style-name="section">}\TitleMark\space
    \HCode{</text:span>}\IgnoreIndent}
   {\HCode{</text:h>\Hnewline}\par\ShowPar}
>>>



\<basic scr ooffice sections\><<<
\ConfigureMark{subsection}
   {\ifnum \c:secnumdepth>\c@secnumdepth  \expandafter\:gobble
    \else \thesubsection\space\fi }
\Configure{subsection}
   {\IgnorePar\EndP \HCode{<!--start of subsection-->}}   
   {\IgnorePar\EndP \HCode{<!--end of subsection-->}}
   {\HCode{<text:h text:style-name="\csname a:Heading-3\endcsname"
                   \text:outlinelevel="3"  ><text:span 
                   text:style-name="subsection"   >}\TitleMark\space
    \HCode{</text:span>}\IgnoreIndent}
   {\HCode{</text:h>\Hnewline}\par\ShowPar}
>>>  


\<basic scr ooffice sections\><<<
\Configure{likesection}
   {\IgnorePar\EndP \HCode{<!--start of section*-->}}   
   {\IgnorePar\EndP \HCode{<!--end of section*-->}}
   {\HCode{<text:h text:style-name="\csname a:Heading-2\endcsname" 
                   \text:outlinelevel="2"  >}\IgnoreIndent}
   {\HCode{</text:h>\Hnewline}\par\ShowPar}
\Configure{likesubsection}
   {\IgnorePar\EndP \HCode{<!--start of subsection*-->}}   
   {\IgnorePar\EndP \HCode{<!--end of subsection*-->}}
   {\HCode{<text:h text:style-name="\csname a:Heading-3\endcsname" 
                   \text:outlinelevel="3"  >}\IgnoreIndent}
   {\HCode{</text:h>\Hnewline}\par\ShowPar}
>>>






\<scrbook end units\><<<
\Configure{endlikeparagraph}
   {paragraph,subsubsection,likesubsubsection,subsection,%
    likesubsection,section,%
    likesection,chapter,likechapter,addchap,appendix,part,likepart}
\Configure{endsubparagraph}
   {likesubparagraph,likeparagraph,subsubsection,likesubsubsection,%
    subsection,likesubsection,section,%
    likesection,chapter,likechapter,addchap,appendix,part,likepart}
\Configure{endlikesubparagraph}
   {subparagraph,likeparagraph,subsubsection,likesubsubsection,%
    subsection,likesubsection,section,%
    likesection,chapter,likechapter,addchap,appendix,part,likepart}
\Configure{endsection}
     {likesection,chapter,likechapter,addchap,appendix,part,likepart}
\Configure{endlikesection}
     {section,chapter,likechapter,addchap,appendix,part,likepart}
\Configure{endsubsection}
   {likesubsection,section,likesection,chapter,%
      likechapter,addchap,appendix,part,likepart}
\Configure{endlikesubsection}
   {subsection,section,likesection,chapter,%
      likechapter,addchap,appendix,part,likepart}
\Configure{endsubsubsection}
   {likesubsubsection,subsection,likesubsection,section,%
      likesection,chapter,likechapter,addchap,appendix,part,likepart}
\Configure{endlikesubsubsection}
   {subsubsection,subsection,likesubsection,section,%
      likesection,chapter,likechapter,addchap,appendix,part,likepart}
\Configure{endparagraph}
   {likeparagraph,subsubsection,likesubsubsection,subsection,%
    likesubsection,section,%
    likesection,chapter,likechapter,addchap,appendix,part,likepart}
>>>


%%%%%%%%%%%%%
\subsection{alphanum}
%%%%%%%%%%%%%

\<configure ooffice alphanum\><<<
\bgroup
   \let\ifx=\relax
   \let\:temp=\relax
   \let\:tempa=\expandafter
   \let\expandafter=\relax
   \let\:TableOfContents=\relax
   \let\else=\relax
   \let\if=\relax
   \let\fi=\relax
   \def\Auto:ent#1{\noexpand\Auto:ent{#1,toclvla,liketoclvla,toclvlb,%
      liketoclvlb,toclvlc,liketoclvlc,toclvld,liketoclvld,toclvle,%
      liketoclvle,toclvlf,liketoclvlf,toclvlg,liketoclvlg,toclvlh,%
      liketoclvlh,toclvli,liketoclvli,toclvlj,%
      liketoclvlj,toclvlk,liketoclvlk,toclvll,liketoclvll}}
   \edef\:TOC{\def\noexpand\:TOC{\:TOC}}
   \:tempa 
\egroup \:TOC
>>>


\<configure ooffice alphanum\><<<
\ConfigureToc{chapter}
  {\HCode{<text:p text:style-name="P1"  ><text:span 
                              text:style-name="toc-mark"  >}}
  { \HCode{</text:span>}}
  {}
  {\HCode{</text:p>}}
>>>

\<configure ooffice alphanum\><<<
\def\:tempc#1#2#3#4{%
   \ConfigureMark{toclvl#1}
      {\HCode{<text:span 
          text:style-name="toclvl#1"   >}\csname thelvl#1\endcsname
       \HCode{</text:span><text:s/>}}%
   \Configure{toclvl#1}
      {}{}
      {\ifvmode \IgnorePar\fi \EndP\IgnorePar
       \HCode{<text:h
              text:style-name="\csname a:Heading-#4\endcsname" 
              \text:outlinelevel="#4"
                     >}\TitleMark\space\HtmlParOff}
      {\HCode{</text:h>}\HtmlParOn \ShowPar\IgnoreIndent \par}%
   \ConfigurToc{toclvl#1}
      {\HCode{<text:p text:style-name="P#4"><text:span 
                                          text:style-name="toc-mark">}%
       |<alphanum toc spaces|>%
       \def\:temp{\J@INumberRoot{#2}{\tmp:cnt}}%
       \afterassignment\:temp \tmp:cnt}
      {#3\HCode{</text:span><text:s/>}}
      {}
      {\HCode{</text:p>}}%
   \ConfigureToc{liketoclvl#1}
      {\HCode{<text:p text:style-name="P#4"  ><text:span 
                                 text:style-name="toc-mark"  >}%
       |<alphanum toc spaces|>%
       \def\:temp{\J@INumberRoot{#2}{\tmp:cnt}}%
       \afterassignment\:temp \tmp:cnt}
      {#3\HCode{</text:span><text:s/>}}
      {}
      {\HCode{</text:p>}}%
}
\:tempc a1{.}{3}
\:tempc b2{.}{4}
\:tempc c3{.}{5}
\:tempc d4{)}{5}
\:tempc e5{)}{5}
\:tempc f6{)}{5}
\:tempc g7{)}{5}
\:tempc h8{)}{5}
\:tempc i9{)}{5}
\:tempc j{10}{)}{5}
\:tempc k{11}{)}{5}
\:tempc l{12}{)}{5}
>>>

\<alphanum toc spaces\><<<
\tmp:cnt=0 \loop \advance\tmp:cnt by 1
                  \ifnum \tmp:cnt<#2 ~~\repeat
>>>


%%%%%%%%%%%%%
\subsection{amsart}
%%%%%%%%%%%%%






\<basic ams ooffice sections\><<<
\ConfigureMark{section}
   {\ifnum \c:secnumdepth>\c@secnumdepth \expandafter\:gobble
    \else  \@seccntformat{section}\fi }
\Configure{section}
   {\IgnorePar\EndP \HCode{<!--start of section-->}}   
   {\IgnorePar\EndP \HCode{<!--end of section-->}}
   {\HCode{<text:h
           text:style-name="\csname a:Heading-2|<cond rtl class|>\endcsname" 
                   \text:outlinelevel="2" ><text:span
                   text:style-name="section" >}\TitleMark\space
    \HCode{</text:span>}\IgnoreIndent}
   {\HCode{</text:h>\Hnewline}\par\ShowPar}
>>>

\<basic ams ooffice sections\><<<
\ConfigureMark{subsection}
   {\ifnum \c:secnumdepth>\c@secnumdepth  \expandafter\:gobble
    \else \@seccntformat{subsection}\fi }
\Configure{subsection}
   {\IgnorePar\EndP \HCode{<!--start of subsection-->}}   
   {\IgnorePar\EndP \HCode{<!--end of subsection-->}}
   {\HCode{<text:h 
           text:style-name="\csname a:Heading-3|<cond rtl class|>\endcsname"
                   \text:outlinelevel="3"   ><text:span
                   text:style-name="subsection"  >}\TitleMark\space
    \HCode{</text:span>}\IgnoreIndent}
   {\HCode{</text:h>\Hnewline}\par\ShowPar}
>>>  


\<basic ams ooffice sections\><<<
\ConfigureMark{subsubsection}
   {\ifnum \c:secnumdepth>\c@secnumdepth  \expandafter\:gobble
    \else \@seccntformat{subsubsection}\fi }
\Configure{subsubsection}
   {\IgnorePar\EndP \HCode{<!--start of subsubsection-->}}   
   {\IgnorePar\EndP \HCode{<!--end of subsubsection-->}}
   {\HCode{<text:h 
           text:style-name="\csname a:Heading-4|<cond rtl class|>\endcsname"
                   \text:outlinelevel="4"   ><text:span
                   text:style-name="subsubsection"  >}\TitleMark\space
    \HCode{</text:span>}\IgnoreIndent}
   {\HCode{</text:h>\Hnewline}\par\ShowPar}
>>>  

\<basic ams ooffice sections\><<<
\Configure{paragraph}
   {\IgnorePar\EndP \HCode{<!--start of paragraph-->}}   
   {\IgnorePar\EndP \HCode{<!--end of paragraph-->}}
   {\HCode{<text:h
           text:style-name="\csname a:Heading-5|<cond rtl class|>\endcsname"
                   \text:outlinelevel="5"  >}}
   {\HCode{</text:h>\Hnewline}\par\ShowPar}
\Configure{subparagraph}
   {\IgnorePar\EndP \HCode{<!--start of subparagraph-->}}   
   {\IgnorePar\EndP \HCode{<!--end of subparagraph-->}}
   {\HCode{<text:h
           text:style-name="\csname a:Heading-6|<cond rtl class|>\endcsname"
                   \text:outlinelevel="6"  >}}
   {\HCode{</text:h>\Hnewline}\par\ShowPar}
>>>  




%%%%%%%%%%%%%%%%%%%%%%%%%%%%%%%
\section{Tables Of Contents}
%%%%%%%%%%%%%%%%%%%%%%%%%%%%%%%




%%%%%%%%%%%%%%%%%%%%%%%%%%%%%%%
\subsection{Entries}
%%%%%%%%%%%%%%%%%%%%%%%%%%%%%%%


 \<non chap tocs\><<<
\ConfigureToc{part}
  {\HCode{<text:p text:style-name="P1"  ><text:span 
                        text:style-name="toc-mark"  >}}
  { \HCode{</text:span>}}
  {}
  {\HCode{</text:p>}}
\ConfigureToc{likepart}
  {\HCode{<text:p text:style-name="P1"  ><text:span 
                        text:style-name="toc-mark"  >}}
  { \HCode{</text:span>}}
  {}
  {\HCode{</text:p>}}
\ConfigureToc{section}
  {\HCode{<text:p text:style-name="P1"  ><text:span 
                        text:style-name="toc-mark"  >}}
  { \HCode{</text:span>}}
  {}
  {\HCode{</text:p>}}
\ConfigureToc{likesection}
  {\HCode{<text:p text:style-name="P1"  ><text:span 
                        text:style-name="toc-mark"  >}}
  { \HCode{</text:span>}}
  {}
  {\HCode{</text:p>}}
\ConfigureToc{subsection}
  {\HCode{<text:p text:style-name="P2"  ><text:span 
                        text:style-name="toc-mark"  >}}
  { \HCode{</text:span>}}
  {}
  {\HCode{</text:p>}}
\ConfigureToc{likesubsection}
  {\HCode{<text:p text:style-name="P2"  ><text:span 
                        text:style-name="toc-mark"  >}}
  { \HCode{</text:span>}}
  {}
  {\HCode{</text:p>}}
\ConfigureToc{subsubsection}
  {\HCode{<text:p text:style-name="P3"  ><text:span 
                        text:style-name="toc-mark"  >}}
  { \HCode{</text:span>}}
  {}
  {\HCode{</text:p>}}
\ConfigureToc{likesubsubsection}
  {\HCode{<text:p text:style-name="P3"  ><text:span 
                        text:style-name="toc-mark"  >}}
  { \HCode{</text:span>}}
  {}
  {\HCode{</text:p>}}
>>>




\<chap tocs\><<<
\ConfigureToc{chapter}
  {\HCode{<text:p text:style-name="P1"  ><text:span 
                        text:style-name="toc-mark"  >}%
   \ifx \chaptername\:UnDef\else \chaptername~\fi}
  { \HCode{</text:span>}}
  {}
  {\HCode{</text:p>}}
\ConfigureToc{likechapter}
  {\HCode{<text:p text:style-name="P1"  ><text:span 
                        text:style-name="toc-mark"  >}}
  { \HCode{</text:span>}}
  {}
  {\HCode{</text:p>}}
\ConfigureToc{appendix}
  {\HCode{<text:p text:style-name="P1"  ><text:span 
                        text:style-name="toc-mark"  >}}
  { \HCode{</text:span>}}
  {}
  {\HCode{</text:p>}}
>>>



 \<chap tocs\><<<
\ConfigureToc{part}
  {\HCode{<text:p text:style-name="P1"  ><text:span 
                        text:style-name="toc-mark"  >}}
  { \HCode{</text:span>}}
  {}
  {\HCode{</text:p>}}
\ConfigureToc{likepart}
  {\HCode{<text:p text:style-name="P1"  ><text:span 
                        text:style-name="toc-mark"  >}}
  { \HCode{</text:span>}}
  {}
  {\HCode{</text:p>}}
\ConfigureToc{section}
  {\HCode{<text:p text:style-name="P2"  ><text:span 
                        text:style-name="toc-mark"  >}}
  { \HCode{</text:span>}}
  {}
  {\HCode{</text:p>}}
\ConfigureToc{likesection}
  {\HCode{<text:p text:style-name="P2"><text:span 
                                       text:style-name="toc-mark" >}}
  { \HCode{</text:span>}}
  {}
  {\HCode{</text:p>}}
\ConfigureToc{subsection}
  {\HCode{<text:p text:style-name="P3"  ><text:span 
                        text:style-name="toc-mark"  >}}
  { \HCode{</text:span>}}
  {}
  {\HCode{</text:p>}}
\ConfigureToc{likesubsection}
  {\HCode{<text:p text:style-name="P3"  ><text:span 
                        text:style-name="toc-mark"  >}}
  { \HCode{</text:span>}}
  {}
  {\HCode{</text:p>}}
\ConfigureToc{subsubsection}
  {\HCode{<text:p text:style-name="P4"  ><text:span 
                        text:style-name="toc-mark"  >}}
  { \HCode{</text:span>}}
  {}
  {\HCode{</text:p>}}
\ConfigureToc{likesubsubsection}
  {\HCode{<text:p text:style-name="P4"  ><text:span 
                        text:style-name="toc-mark"  >}}
  { \HCode{</text:span>}}
  {}
  {\HCode{</text:p>}}
>>>

\verb+<text:tab-stop/>+
 

%%%%%%%%%%%%%%%%%%%%%%%%%%%%%%%
\subsection{Content.xml Styles}
%%%%%%%%%%%%%%%%%%%%%%%%%%%%%%%

Why we need the following in content.xml instead of styles.xml

\<automatic style 1\><<<
\Hnewline
   <style:style style:name="P1"
                style:family="paragraph"
                style:parent-style-name="Contents-1">\Hnewline
      <style:paragraph-properties >\Hnewline
         <style:tab-stops>\Hnewline
            <style:tab-stop style:position="16.999cm"
                   style:type="right"
                   style:leader-char="."/>\Hnewline
         </style:tab-stops>\Hnewline
      </style:paragraph-properties >\Hnewline
   </style:style>
   \Hnewline 
>>>


\<automatic style 2\><<<
\Hnewline
   <style:style style:name="P2"
                style:family="paragraph"
                style:parent-style-name="Contents-2">\Hnewline
   <style:paragraph-properties >\Hnewline
      <style:tab-stops>\Hnewline
      <style:tab-stop style:position="16.999cm"
                   style:type="right"
                   style:leader-char="."/>\Hnewline
      </style:tab-stops>\Hnewline
   </style:paragraph-properties >\Hnewline
   </style:style>
   \Hnewline 
>>>


\<automatic style 3\><<<
\Hnewline
   <style:style style:name="P3"
                style:family="paragraph"
                style:parent-style-name="Contents-3">\Hnewline
   <style:paragraph-properties >\Hnewline
      <style:tab-stops>\Hnewline
         <style:tab-stop style:position="16.999cm"
                      style:type="right"
                      style:leader-char="."/>\Hnewline
      </style:tab-stops>\Hnewline
   </style:paragraph-properties >\Hnewline
   </style:style>\Hnewline
   \Hnewline 
>>>


\<automatic style 4\><<<
\Hnewline
   <style:style style:name="P4"
                style:family="paragraph"
                style:parent-style-name="Contents-4">\Hnewline
   <style:paragraph-properties >\Hnewline
      <style:tab-stops>\Hnewline
         <style:tab-stop style:position="16.999cm"
                      style:type="right"
                      style:leader-char="."/>\Hnewline
      </style:tab-stops>\Hnewline
   </style:paragraph-properties >\Hnewline
   </style:style>
      \Hnewline 
>>>


\<automatic style 5\><<<
\Hnewline
   <style:style style:name="P5"
                style:family="paragraph"
                style:parent-style-name="Contents-5">\Hnewline
   <style:paragraph-properties >\Hnewline
      <style:tab-stops>\Hnewline
         <style:tab-stop style:position="16.999cm"
                      style:type="right"
                      style:leader-char="."/>\Hnewline
      </style:tab-stops>\Hnewline
   </style:paragraph-properties >\Hnewline
   </style:style>   \Hnewline 
>>>
 

%%%%%%%%%%%%%%%%%%%%%%%%%%%%%%%
\subsection{Style.xml Styles}
%%%%%%%%%%%%%%%%%%%%%%%%%%%%%%%



\<content style 1\><<<
\NewConfigureOO{Contents 1}
\ConfigureOO{Contents 1}{\Hnewline
   <style:style style:name="Contents-1"
                style:family="paragraph"
                style:parent-style-name="Index"
                style:class="index">\Hnewline
   <style:paragraph-properties  fo:margin-left="0cm"
                fo:margin-right="0cm"
                fo:margin-top="0cm"
                fo:margin-bottom="0.101cm"
                fo:font-weight="bold"
                fo:text-indent="0cm"
                style:auto-text-indent="false">\Hnewline
      <style:tab-stops>\Hnewline
         <style:tab-stop style:position="16.999cm"
                      style:type="right"
                      style:leader-char="."/>\Hnewline
      </style:tab-stops>\Hnewline
   </style:paragraph-properties >\Hnewline
   </style:style>\Hnewline
}
>>>




 


\<content style 2\><<<
\NewConfigureOO{Contents 2}
   \ConfigureOO{Contents 2}{\Hnewline
   <style:style style:name="Contents-2"
                style:family="paragraph"
                style:parent-style-name="Index"
                style:class="index">
   <style:paragraph-properties  fo:margin-left="0.499cm"
                fo:margin-right="0cm"
                fo:text-indent="0cm"
                style:auto-text-indent="false">
      <style:tab-stops>
      <style:tab-stop style:position="16.499cm"
                   style:type="right"
                   style:leader-char="."/>
      </style:tab-stops>
   </style:paragraph-properties >
   </style:style>
   \Hnewline}
>>>
 


\<content style 3\><<<
\NewConfigureOO{Contents 3}
\ConfigureOO{Contents 3}{\Hnewline
   <style:style style:name="Contents-3"
                style:family="paragraph"
                style:parent-style-name="Index"
                style:class="index">
   <style:paragraph-properties  fo:margin-left="0.998cm"
                fo:margin-right="0cm"
                fo:text-indent="0cm"
                style:auto-text-indent="false">
      <style:tab-stops>
         <style:tab-stop style:position="16cm"
                      style:type="right"
                      style:leader-char="."/>
      </style:tab-stops>
   </style:paragraph-properties >
   </style:style>
   \Hnewline}
>>>
 


\<content style 4\><<<
\NewConfigureOO{Contents 4}
\ConfigureOO{Contents 4}{\Hnewline
   <style:style style:name="Contents-4"
                style:family="paragraph"
                style:parent-style-name="Index"
                style:class="index">\Hnewline
   <style:paragraph-properties  fo:margin-left="1.498cm"
                fo:margin-right="0cm"
                fo:text-indent="0cm"
                style:auto-text-indent="false">\Hnewline
      <style:tab-stops>
         <style:tab-stop style:position="15.501cm"
                      style:type="right"
                      style:leader-char="."/>\Hnewline
      </style:tab-stops>\Hnewline
   </style:paragraph-properties >\Hnewline
   </style:style>
   \Hnewline}
>>>
 


\<content style 5\><<<
\NewConfigureOO{Contents 5}
\ConfigureOO{Contents 5}{\Hnewline
   <style:style style:name="Contents-5"
                style:family="paragraph"
                style:parent-style-name="Index"
                style:class="index">\Hnewline
   <style:paragraph-properties  fo:margin-left="1.997cm"
                fo:margin-right="0cm"
                fo:text-indent="0cm"
                style:auto-text-indent="false">\Hnewline
      <style:tab-stops>\Hnewline
         <style:tab-stop style:position="15.002cm"
                      style:type="right"
                      style:leader-char="."/>\Hnewline
      </style:tab-stops>\Hnewline
   </style:paragraph-properties >\Hnewline
   </style:style>
\Hnewline}
>>>
   
%%%%%%%%%%%%%
\subsection{Toc Style Outline}
%%%%%%%%%%%%%






\<oo2 tex4ht\><<<
\def\index:entry:chapter{text:index-entry-chapter}
>>>



\<toc template 1\><<<
<text:table-of-content-entry-template 
         text:outline-level="1" 
         text:style-name="Contents-1"   >\Hnewline
<\index:entry:chapter />\Hnewline
<text:index-entry-text/>\Hnewline
<text:index-entry-tab-stop 
         style:type="right"
         style:leader-char="."  />\Hnewline
<text:index-entry-page-number/>\Hnewline
</text:table-of-content-entry-template>\Hnewline
>>>

\<toc template 2\><<<
<text:table-of-content-entry-template 
        text:outline-level="2" 
        text:style-name="Contents-2"  >\Hnewline
<\index:entry:chapter />\Hnewline
<text:index-entry-text/>\Hnewline
<text:index-entry-tab-stop 
        style:type="right" 
        style:leader-char="."/>\Hnewline
<text:index-entry-page-number/>\Hnewline
</text:table-of-content-entry-template>\Hnewline
>>>


\<toc template 3\><<<
<text:table-of-content-entry-template 
        text:outline-level="3" 
        text:style-name="Contents-3"  >\Hnewline
<\index:entry:chapter />\Hnewline
<text:index-entry-text/>\Hnewline
<text:index-entry-tab-stop 
        style:type="right" 
        style:leader-char="."/>\Hnewline
<text:index-entry-page-number/>\Hnewline
</text:table-of-content-entry-template>\Hnewline
>>>


\<toc template 4\><<<
<text:table-of-content-entry-template 
        text:outline-level="4" 
        text:style-name="Contents-4"  >\Hnewline
<\index:entry:chapter />\Hnewline
<text:index-entry-text/>\Hnewline
<text:index-entry-tab-stop 
        style:type="right" 
        style:leader-char="."/>\Hnewline
<text:index-entry-page-number/>\Hnewline
</text:table-of-content-entry-template>\Hnewline
>>>

\<toc template 5\><<<
<text:table-of-content-entry-template 
        text:outline-level="5" 
        text:style-name="Contents-5"  >\Hnewline
<\index:entry:chapter />\Hnewline
<text:index-entry-text/>\Hnewline
<text:index-entry-tab-stop 
        style:type="right" 
        style:leader-char="."/>\Hnewline
<text:index-entry-page-number/>\Hnewline
</text:table-of-content-entry-template>\Hnewline
>>>

\<toc template 6\><<<
<text:table-of-content-entry-template 
        text:outline-level="6" 
        text:style-name="Contents-6"  >\Hnewline
<\index:entry:chapter />\Hnewline
<text:index-entry-text/>\Hnewline
<text:index-entry-tab-stop 
        style:type="right" 
        style:leader-char="."/>\Hnewline
<text:index-entry-page-number/>\Hnewline
</text:table-of-content-entry-template>\Hnewline
>>>

\<toc template 7\><<<
<text:table-of-content-entry-template 
        text:outline-level="7" 
        text:style-name="Contents-7"  >\Hnewline
<\index:entry:chapter />\Hnewline
<text:index-entry-text/>\Hnewline
<text:index-entry-tab-stop 
        style:type="right" 
        style:leader-char="."/>\Hnewline
<text:index-entry-page-number/>\Hnewline
</text:table-of-content-entry-template>\Hnewline
>>>

\<toc template 8\><<<
<text:table-of-content-entry-template 
        text:outline-level="8" 
        text:style-name="Contents-8"  >\Hnewline
<\index:entry:chapter />\Hnewline
<text:index-entry-text/>\Hnewline
<text:index-entry-tab-stop 
        style:type="right" 
        style:leader-char="."/>\Hnewline
<text:index-entry-page-number/>\Hnewline
</text:table-of-content-entry-template>\Hnewline
>>>

\<toc template 9\><<<
<text:table-of-content-entry-template 
        text:outline-level="9" 
        text:style-name="Contents-9"  >\Hnewline
<\index:entry:chapter />\Hnewline
<text:index-entry-text/>\Hnewline
<text:index-entry-tab-stop 
        style:type="right" 
        style:leader-char="."/>\Hnewline
<text:index-entry-page-number/>\Hnewline
</text:table-of-content-entry-template>\Hnewline
>>>

\<toc template 10\><<<
<text:table-of-content-entry-template 
        text:outline-level="10" 
        text:style-name="Contents-10"  >\Hnewline
<\index:entry:chapter />\Hnewline
<text:index-entry-text/>\Hnewline
<text:index-entry-tab-stop 
        style:type="right" 
        style:leader-char="."/>\Hnewline
<text:index-entry-page-number/>\Hnewline
</text:table-of-content-entry-template>\Hnewline
>>>
  


%%%%%%%%%%%%%%%%%%%%%%%%%%%%%%%
\subsection{SCR Entries}
%%%%%%%%%%%%%%%%%%%%%%%%%%%%%%%


\<scr chap tocs\><<<
\ConfigureToc{chapter}
  {\HCode{<text:p text:style-name="chapterToc"  ><text:span 
                              text:style-name="toc-mark"  >}}
  { \HCode{</text:span>}}
  {}
  {\HCode{</text:p>}}
\ConfigureToc{addchap}
  {}
  {\HCode{<text:p text:style-name="addchapToc"  >}}
  {}
  {\HCode{</text:p>}}
>>>




%%%%%%%%%%%%%%%%%%%%%%%%%%%%%%%
\subsection{Links}
%%%%%%%%%%%%%%%%%%%%%%%%%%%%%%%

\<configure ooffice tex4ht\><<<
\:CheckOption{TocLink}\if:Option 
   \Configure{TocLink}{\Link{#2}{#3}#4\ifvmode\IgnorePar\fi\EndLink} 
\else
    \Log:Note{to request links from the tables of contents,
         use the command line option from `TocLink'}
   \Configure{TocLink}{\Link{#2}{#3}\EndLink#4\ifvmode\IgnorePar\fi} 
\fi
\Configure{toTocLink}{\Link}{\EndLink}     
>>>

%%%%%%%%%%%%%%%%%%%%%%%%%%%%%%%
\subsection{Envelop}
%%%%%%%%%%%%%%%%%%%%%%%%%%%%%%%

\<configure ooffice latex\><<<
\Configure{tableofcontents}
   {\IgnorePar\EndP\HCode{|<start toc|>}\IgnorePar}
   {}
   {\IgnorePar\EndP\HCode{|<end toc|>}\ShowPar}
   {\HCode{\Hnewline}}   {}
>>>

\<start toc\><<<
<text:table-of-content 
        text:name="toc" 
        text:style-name="Sect1"
        text:protected="true"  >\Hnewline
   <text:table-of-content-source 
               text:outline-level="|<outline level|>"  >\Hnewline
      |<toc outline|>%
   </text:table-of-content-source>\Hnewline
   <text:index-body>\Hnewline
   \ifx\contentsname\empty \else
     \ifx\contentsname\:UnDef \else
      <text:index-title text:style-name="Sect2" text:name="contents"
                         >
         <text:p text:style-name="Contents-Heading"
                         >\contentsname</text:p>
      </text:index-title>
   \fi \fi
>>>

\<configure ooffice latex\><<<
\NewConfigureOO{Sect1}
\ConfigureOO{Sect1}{\Hnewline
   <style:style style:name="Sect1"
                style:family="section">
   <\style:sectionproperties>
      <style:columns fo:column-count="1"
                   fo:column-gap="0cm">
         <style:column fo:margin-top="0.5cm"
                      fo:margin-left="0cm"
                      fo:margin-right="0cm"
         />
      </style:columns>
   </\style:sectionproperties>
   </style:style>
\Hnewline }
>>>

\<configure ooffice latex\><<<
\NewConfigureOO{Sect2}
\ConfigureOO{Sect2}{\Hnewline
<style:style    style:name="Sect2"
   style:family="section">
   <\style:sectionproperties>
      <style:columns fo:column-count="0" fo:column-gap="0cm"/>
   </\style:sectionproperties>
</style:style>
\Hnewline }
>>>

\<configure ooffice latex\><<<
\NewConfigureOO{Contents Heading}
\ConfigureOO{Contents Heading}{\Hnewline
   <style:style style:name="Contents-Heading"
                style:family="paragraph"
                style:parent-style-name="Heading"
                style:class="index">
   <style:paragraph-properties 
                fo:margin-top="0.5cm"
                fo:margin-bottom="0.2cm"
                fo:margin-left="0cm"
                fo:margin-right="0cm"
                fo:font-size="16pt"
                fo:font-weight="bold"
                style:font-size-asian="16pt"
                style:font-weight-asian="bold"
                style:font-size-complex="16pt"
                style:font-weight-complex="bold"
                fo:text-indent="0cm"
                style:auto-text-indent="false"
                text:number-lines="false"
                text:line-number="0"
   />
   </style:style>
\Hnewline }
>>>

\<configure ooffice article\><<<
\def\tableofcontents{\:tableofcontents}
>>>

\<configure ooffice book\><<<
\def\tableofcontents{\:tableofcontents}
>>>


\<configure ooffice report\><<<
\def\tableofcontents{\:tableofcontents}
>>>


\<configure ooffice amsart\><<<
\def\tableofcontents{\:tableofcontents}
>>>

\<configure ooffice amsproc\><<<
\def\tableofcontents{\:tableofcontents}
>>>

\<configure ooffice nwart\><<<
\def\tableofcontents{\:tableofcontents}
>>>

\<configure ooffice nwrep\><<<
\def\tableofcontents{\:tableofcontents}
>>>

\<configure ooffice nwbk\><<<
\def\tableofcontents{\:tableofcontents}
>>>




\<outline level\><<<
6>>>


\<end toc\><<<
   </text:index-body>\Hnewline
</text:table-of-content>%
\Hnewline
>>>



\<toc outline\><<<
<text:index-title-template
   text:style-name="Contents-Heading"  >\contentsname
</text:index-title-template>\Hnewline
|<toc template 1|>%
|<toc template 2|>%
|<toc template 3|>%
|<toc template 4|>%
|<toc template 5|>%
|<toc template 6|>%
|<toc template 7|>%
|<toc template 8|>%
|<toc template 9|>%
|<toc template 10|>%
>>>


\<automatic styles\><<<
|<automatic style 1|>%
|<automatic style 2|>%
|<automatic style 3|>%
|<automatic style 4|>%
|<automatic style 5|>%
|<automatic style 6|>%
|<automatic style 7|>%
|<automatic style 8|>%
>>>

\<content styles\><<<
|<content style 1|>%
|<content style 2|>%
|<content style 3|>%
|<content style 4|>%
|<content style 5|>%
|<content style 6|>%
|<content style 7|>%
|<content style 8|>%
>>>


%%%%%%%%%%%%%
\subsection{Tables of Figures}
%%%%%%%%%%%%%


\<config lists of figures and tables\><<<
\Configure{listoffigures}
   {\let\contentsname\listfigurename \let\listfigurename\empty}
   {}
\Configure{listoftables}
   {\let\contentsname\listtablename \let\listtablename\empty}
   {}
\ConfigureToc{lof}
  {\HCode{<text:p><text:span text:style-name="toc-mark"
                              >}}{\ }{}%
                     {\HCode{</text:span></text:p>}}
\ConfigureToc{lot}
  {\HCode{<text:p><text:span text:style-name="toc-mark" 
                              >}}{\ }{}%
                     {\HCode{</text:span></text:p>}}
>>>



%%%%%%%%%%%%%%%%%%%%%%%%%%%%%%%
\section{Tabular}
%%%%%%%%%%%%%%%%%%%%%%%%%%%%%%%


%%%%%%%%%%%%%%%%%%%%%%%%%%%%%%%
\subsection{Examples}
%%%%%%%%%%%%%%%%%%%%%%%%%%%%%%%


General Format:

\begin{verbatim}
<table:table table:name="..." table:style-name="tbl">
  <table:table-column table:style-name="..."/>... <table:table-column/>
  <table:table-row table:name="...">
    <table:table-cell> <text:p>...</text:p> </table:table-cell>
    ....
  </table:table-row>
  .....
</table:table>
\end{verbatim}

May include subtables

\begin{verbatim}
<table:sub-table>
  .......
</table:sub-table>
\end{verbatim}



%%%%%%%%%%%%%%%%%%%%%%%%%%%%%%%
\subsection{The External Envelope}
%%%%%%%%%%%%%%%%%%%%%%%%%%%%%%%

\<configure ooffice latex\><<<
|<latex tabular|>
>>>

\<latex tabular\><<<
\ConfigureEnv{tabular}
  {|<configure tabular noalign|>%
   \IgnorePar|<inline comment|>\EndP
   |<save cols info|>%
   \ifnum \BegEnd:D>0
     |<boxed env|>%
   \fi
   \gHAdvance\BegEnd:D by 1
   \bgroup |<Text-Body HtmlPar|>%
  }
  {\egroup
   \gHAdvance\BegEnd:D by -1
   \ifnum \BegEnd:D>0
      |<end boxed env|>%
   \fi
   |<recall cols info|>\ShowPar
   |<configure clear noalign|>%
   |<try env inline par|>%
  }{}{}
>>>


\<configure ooffice tabularx\><<<
\ConfigureEnv{tabularx}
  {|<configure tabular noalign|>%
   \IgnorePar|<inline comment|>\EndP
   |<save cols info|>%
  }
  {|<recall cols info|>\ShowPar
   |<configure clear noalign|>%
   |<try env inline par|>%
  }{}{}
>>>





\<latex tabular\><<<
\Configure{tabular}
   {\ifvmode \IgnorePar\fi \EndP \halignTB{tabular}%
    |<list the grouping of columns|>%
   }
   {|<insert Clr|>\HCode{</table:table>}}
   {|<cond trail row|>%
    \HCode{<table:table-row \Hnewline 
      |<trail row|>%
      table:style-name="TBL-\TableNo-\HRow-}%
      |<row attributes|>\HCode{">}}
   {\HCode{</table:table-row>}|<end cond trail row|>%
   }
   {\HCode{<table:table-cell \Hnewline}%
    |<multispan cell|>%
    \HCode{ table:style-name="TBL-\TableNo-\HRow-\HCol">}%
    |<start par at cell|>|<save table num|>}
   {|<end array par box|>|<recall table num|>% 
      |<end par at cell|>\HCode{</table:table-cell>}}
>>>


\<cond trail row\><<<
\ifTag{vis-\TableNo-\HRow}%
      {\special{t4ht@[}\gdef\end:box{\special{t4ht@]}}}%
      {\global\let\end:box\empty}%
>>>

\<end cond trail row\><<<
\end:box
>>>

\<trail row\><<<
\ifTag{vis-\TableNo-\HRow}{table:visibility="collapse" }{}
>>>

\<latex tabular\><<<
\Configure{endtabular} {\Tag{vis-\TableNo-\HRow}{}}
>>>




\<latex tabular\><<<
\def\:MT{\ifmathml \a:mathml m\else table:\fi}
\Configure{halignTB}
   {\HCode{<\:MT table }}
   {\HCode{>}}
>>>




\<configure tabular noalign\><<<
\Configure{noalign}%
  {\f:tabular\d:tabular \HCode{<tr><td colspan="\ar:cnt">}}
  {\HCode{</td></tr>}\pend:def\TableNo{0}\c:tabular\e:tabular}%
>>>


\<configure clear noalign\><<<
\Configure{noalign}{}{}%
>>>


\<latex tabular\><<<
\Configure{halignTB<>}{tabular}{\HCode{\Hnewline
     table:style-name="tabular"
     }<>\HAlign}
>>>


For some reason, the following doesn't have effect at the office tyles
section


\<automatic styles\><<<
<style:style style:name="tabular" 
             style:family="table">\Hnewline
<style:table-properties style:rel-width="76\%"
                 fo:margin-left="1cm"
                 fo:margin-right="1cm"
                 table:align="margins"/>\Hnewline
</style:style>\Hnewline
>>>


\<inline comment\><<<
\ifvmode\else\HCode{<!--tex4ht:inline-->}\fi
>>>



\<latex tabular\><<<
\HAssign\Next:TableNo=0 
\global\let\TableNo=\Next:TableNo
>>>

\<increase table num\><<<
\gHAdvance\Next:TableNo by 1
\global\let\TableNo=\Next:TableNo
>>>


\<save table num\><<<
\PushStack\Table:Stck\TableNo
>>>

\<recall table num\><<<
\PopStack\Table:Stck\TableNo
>>>



\<Text-Body HtmlPar\><<<
\Configure{HtmlPar}
   {\EndP \HCode{<text:p text:style-name="Text-body|<cond rtl class|>"
                          >}}
   {\EndP \HCode{<text:p text:style-name="Text-body|<cond rtl class|>"
                          >}}
   {\HCode{</text:p>}}
   {\HCode{</text:p>}}
>>>





%%%%%%%%%%%%%%%%%%%%%%%%%%%%%%%%
\subsection{Description of Columns}
%%%%%%%%%%%%%%%%%%%%%%%%%%%%%%%%



The variable
\verb+\NewGroup+ is introduced to record the index of entry in current
group.


\<configure ooffice-mml latex\><<<
|<visit start of pattern|>
\Configure{VBorder}
   {\HAssign\NewGroup = 0 
    |<increase table num|>%
    |<visit start of pattern|>%
   }
   {|<visit a v-bar entry|>\HAssign\NewGroup = 0 }
   {\Advance:\NewGroup by 1 
    |<visit a non-@ entry|>%
   }
   {|<visit a @ entry|>}
>>>



\<configure ooffice-mml array\><<<
|<visit start of pattern|>
\Configure{VBorder}
   {\HAssign\NewGroup = 0 
    |<increase table num|>%
    |<visit start of pattern|>%
   }
   {|<visit a v-bar entry|>\HAssign\NewGroup = 0 }
   {\Advance:\NewGroup by 1 
    |<visit a non-@ entry|>%
   }
   {|<visit a @ entry|>}
>>>




\<configure ooffice-mml latexNO\><<<
\let\VBorder=\empty   \let\AllColMargins=\empty
\Configure{VBorder}
   {\let\VBorder=\empty \let\AllColMargins=\empty
    \global\let\GROUPS=\empty \HAssign\NewGroup = 0 
    \gHAdvance\Next:TableNo by 1 \global\let\TableNo=\Next:TableNo
    |<init vborder??|>%
   }
   {|<vborder solid??|>%
    \xdef\VBorder{\VBorder\ifnum \NewGroup> 0 </colgroup>\fi}
    \HAssign\NewGroup = 0  \gdef\GROUPS{rules="groups"}}
   {\Advance:\NewGroup by 1
    |<record cur vborder??|>% 
    \ifnum \NewGroup=1 \xdef\VBorder{\VBorder<colgroup 
        id="TBL-\TableNo-\ar:cnt g">}\fi      
    \xdef\VBorder{\VBorder<col\Hnewline id="TBL-\TableNo
       -\ar:cnt"\xml:empty>}\xdef\AllColMargins{\AllColMargins1}}
   {\xdef\AllColMargins{\AllColMargins 0}}
>>>


\<insert ClrNO\><<<
\expandafter\tmp:toks\expandafter{\Clr}%
\HCode{<!--\the\tmp:toks-->}%
>>>

The above doesn't work, for instance on `\verb+|r||r@{--}l|p{1.25in}|+'
due to thge extra \verb+--+..





\<configuring @()\><<<
\Configure{@{}}{}
>>>


\<row attributes\><<<
\halignTR
>>>

\<latex tabular\><<<
\Configure{halignTR} 
   {}
\Configure{halignTBL} 
%   {t}{\HCode{valign-top}}
%   {b}{\HCode{valign-baseline}}
%   {c}{\HCode{valign-middle}}
    {}
>>>




\<start par at cell\><<<
|<extract cell attributes|>\if !\Aligh:Wrap!\let\not:Pcell=\:gobble
\else \let\not:Pcell\empty \fi
\not:Pcell{\HCode{<text:p 
    \Hnewline  text:style-name="}|<style at cell|>\HCode{"
     >}}%
|<start array par box|>%
>>>



\<style at cell\><<<
\cell:sty
>>>


\<latex tabular\><<<
\def\cell:sty{\NoFonts 
    |<extract cell margins info|>%
    |<extract cell h-border info|>%
    |<extract cell attributes|>%    
    |<cell style name|>%
    |<cell style properties|>%
  \EndNoFonts}
>>>



\<cell style name\><<<
sty-\LR:VBars\Left:Marg\Right:Marg\Aligh:Wrap
>>>



\<end par at cell\><<<
\csname not:Pcell\endcsname{\HCode{</text:p>}}%
>>>    



\<multispan cell\><<<
\ifnum \HMultispan>1 
   \HCode{table:number-columns-spanned="\HMultispan"\Hnewline}%
   \bgroup
      \def\:temp{\ifnum \HMultispan>1
          \HAdvance\HMultispan by -1 
          \HCode{<table:covered-table-cell/>}%
          \expandafter\:temp
        \fi}%
   \egroup
\fi
>>>




%%%%%%%%%%%%%%%%%%%%%%%%%%%%%%%%
\subsection{Cell Style Properties}
%%%%%%%%%%%%%%%%%%%%%%%%%%%%%%%%

\<extract cell attributes\><<<
\edef\Aligh:Wrap{\halignTD}%
>>>
    

\<configure ooffice tex4ht\><<<
\Configure{halignTD} {}{}
   {<}{L}     %{\HCode{align-left nowrap}}
   {-}{C}     %{\HCode{align-center nowrap}}
   {>}{R}     %{\HCode{align-right nowrap}}
   {^}{T}     %{\HCode{valign-top nowrap}}
   {=}{N}     %{\HCode{valign-baseline nowrap}}
   {||}{M}     %{\HCode{valign-middle nowrap}}
   {_}{B}     %{\HCode{valign-bottom nowrap}}
   {p}{}
   {m}{m}     %{\HCode{valign-middle   align-left}}
   {b}{n}     %{\HCode{valign-baseline align-left}}
   {}
>>>





\<configure ooffice-mml tex4ht\><<<
\Configure{halignTD} {}{}
   {<}{\ifmathml \HCode{ columnalign="left"}\else
        L\fi}
   {-}{\ifmathml \HCode{ columnalign="center"}\else
        C\fi}
   {>}{\ifmathml \HCode{ columnalign="right"}\else
        R\fi}
   {^}{\ifmathml \HCode{ rowalign="top"}\else
        T\fi}
   {=}{\ifmathml \HCode{ rowalign="baseline"}\else
        N\fi}
   {||}{\ifmathml \HCode{ rowalign="center"}\else
        M\fi}
   {_}{\ifmathml \HCode{ rowalign="bottom"}\else
        B\fi}
   {p}{\ifmathml \HCode{ columnalign="left"}\else
        \fi}
   {m}{\ifmathml\else m\fi}     %{\HCode{valign-middle   align-left}}
   {b}{\ifmathml\else n\fi}     %{\HCode{valign-baseline align-left}}
   {}
>>>



\<cell style properties\><<<
\xdef\cell:style:name{|<cell style name|>}%
\expandafter\ifx \csname\cell:style:name\endcsname \relax
   \global\expandafter\let\csname\cell:style:name\endcsname=\def
   \OOstyles{%
      <style:style
        style:name="\cell:style:name"
        style:family="paragraph"
        style:parent-style-name="Standard">\Hnewline
        <style:paragraph-properties  
   }%
   |<cell h-margins property|>%
   |<cell alignment properties|>%
   |<cell LR borders property|>%
   \OOstyles{%
         style:justify-single-word="false"/>\Hnewline
    </style:style>\Hnewline
   }%
\fi
>>>










%%%%%%%%%%%%%%%%%%%%%%%%%%%%%%%%
\subsection{Cell Alignments}
%%%%%%%%%%%%%%%%%%%%%%%%%%%%%%%%


\<cell alignment properties\><<<
\def\:temp{L}\ifx \Aligh:Wrap\:temp
   \OOstyles{%
     fo:wrap-option="no-wrap"
     fo:text-align="start"
   }%
\fi
\def\:temp{C}\ifx \Aligh:Wrap\:temp
   \OOstyles{%
     fo:wrap-option="no-wrap"
     fo:text-align="center"
   }%
\fi
\def\:temp{R}\ifx \Aligh:Wrap\:temp
   \OOstyles{%
     fo:wrap-option="no-wrap"
     fo:text-align="end"
   }%
\fi
\ifx \Aligh:Wrap\empty
   \OOstyles{%
     fo:text-align="start"
   }%
\fi
>>>


%%%%%%%%%%%%%%%%%%%%%%%%%%%%%%%%
\subsection{Groups of Columns}
%%%%%%%%%%%%%%%%%%%%%%%%%%%%%%%%


A group consists of a non-empty pattern \verb+|...|+.  

\<list the grouping of columns\><<<
\putVBorder
>>>


When \verb'\putVBorder' is called,  in \verb'\:VBorder' 
we have a sequence of the form
\verb'<COLGROUP><COL ...">...</COLGROUP>...' with the last 
tag possibly missing.

\<latex tabular\><<<
\let\remove:eog=\relax
\def\putVBorder{\expandafter\put:VBorder\:VBorder<</>>|<par del|>}
\def\put:VBorder#1<</#2>>#3|<par del|>{%
  \def\:temp{#2}\ifx\:temp\empty
     \def\remove:eog<</##1>>{}%
       \HCode{\:VBorder}%
     \let\remove:eog=\relax
     |<insert last end of group, if missing|>%
  \else  \def\:temp{\put:VBorder#3|<par del|>}\expandafter\:temp\fi
}
>>>


\<visit start of pattern\><<<
\let\:VBorder=\empty 
>>>

\<visit a non-@ entry\><<<
\ifnum \NewGroup=1  |<start a group of cols|>\fi     
|<add a group of columns|>% 
>>>


\<start a group of cols\><<<
\xdef\:VBorder{\:VBorder
   <!--table:table-column-group ???="TBL-\TableNo-\ar:cnt g"-->}%
>>>

\<add a group of columns\><<<
\xdef\:VBorder{\:VBorder<table:table-column\Hnewline 
    \csname a:table-column-style\endcsname\Hnewline
 />}%
>>>

\<latex tabular\><<<
\NewConfigure{table-column-style}{1}
\Configure{table-column-style}
   {table:style-name="TBL-\TableNo -\ar:cnt"}
>>>

\<visit a v-bar entry\><<<
\xdef\:VBorder{\:VBorder
   \ifnum \NewGroup> 0
       <!--/table:table-column-group-->\remove:eog <</ >>\fi}%
>>>    

\<insert last end of group, if missing\><<<
\def\:temp{#1}\ifx\:temp\empty 
\else\HCode{<!--/table:table-column-group-->}\fi
>>>



%%%%%%%%%%%%%%%%%%%%%%%%%%%%%%%%
\subsection{Vertical Lines}
%%%%%%%%%%%%%%%%%%%%%%%%%%%%%%%%



\<cell LR borders property\><<<
\cell:vbars
>>>


\<latex tabular\><<<
\def\cell:vbars{%
  \def\:tempa##1##2{%
    \tmp:cnt=0
    \def\:temp####1{%       
       \if ####1##1\advance\tmp:cnt by 1 \fi
       \if ####1.\else \expandafter\:temp \fi
    }%
    \expandafter\:temp\LR:VBars.%
    \ifnum \tmp:cnt>0
       \OOstyles{ fo:border-##2="|<hline resolution|>\the\tmp:cnt
                                 cm solid \#000000" }%
    \fi
  }%
  \:tempa L{left}%
  \:tempa R{right}%
}
>>>


\<hline resolution\><<<
0.0>>>

\<extract cell h-border info\><<<
\ifx \:AllColBars\empty
   \let\LR:VBars=\empty
\else
   \edef\LR:VBars{\Col:VBars}%
\fi
>>>



\verb'\AllColMargins' Return a binary string in which 1 represents
a column, and 0 represents a `@'. \verb'\ColMargins' retrieves the
zeros before the 1's that represent the current and following 1's.

\<latex tabular\><<<
\def\Col:VBars{\expandafter\get:vbar\:AllColBars,,}
\def\get:vbar#1,{\if :#1:\else
       \ifnum #1=0 \ifnum \HCol=1 L\fi
       \else       \ifnum \HCol=#1R\fi \fi
   \expandafter\get:vbar \fi}
>>>




\<configure ooffice-mml latex\><<<
\let\AllColMargins\empty
\ConfigureEnv{array}
   {\PushStack\Col:Marg\AllColMargins
    \bgroup\Configure{$}{}{}{}} 
   {\egroup \PopStack\Col:Marg\AllColMargins}
   {}{}
>>>

\<configure ooffice-mml array\><<<
\ConfigureEnv{array}
   {\PushStack\Col:Marg\AllColMargins
    \bgroup\Configure{$}{}{}{}} 
   {\egroup \PopStack\Col:Marg\AllColMargins}
   {}{}
>>>


\<configure ooffice latex\><<<
\ConfigureEnv{tabular}
   {\PushStack\Col:Marg\AllColMargins} 
   {\PopStack\Col:Marg\AllColMargins}
   {}{}
>>>










\<visit start of pattern\><<<
\let\:AllColBars=\empty
>>>

\<visit a v-bar entry\><<<
\xdef\:AllColBars{\:AllColBars 
   \ifx\:AllColBars\empty \else ,\fi \ar:cnt}%
>>>


\<save cols info\><<<
\PushStack\Col:Bars\:AllColBars
>>>


\<recall cols info\><<<
\PopStack\Col:Bars\:AllColBars
>>>







%%%%%%%%%%%%%%%%%%%%%%%%%%%%%%%%
\subsection{Horizontal Marging}
%%%%%%%%%%%%%%%%%%%%%%%%%%%%%%%%

\verb'\AllColMargins' Return a binary string in which 1 represents
a column, and 0 represents a `@'. \verb'\ColMargins' retrieves the
zeros before the 1's that represent the current and following 1's.

\<cell h-margins property\><<<
\OOstyles{%
   fo:padding-left="0\ifnum \Left:Marg >0 .17\fi cm" |% 5pt |%
  fo:padding-right="0\ifnum \Right:Marg>0 .17\fi cm"
}%
>>>

\<extract cell margins info\><<<
\ColMargins
>>>


\<latex tabular\><<<
\def\ColMargins{\tmp:cnt=\HCol\relax 
   \expandafter\nosp:hbr\AllColMargins 1//}
\def\nosp:hbr#11#2//{\ifnum \tmp:cnt>1 \advance\tmp:cnt by -1
   \nosp:hbr#2//\else\nosp:gt#11#2//\fi}
\def\nosp:gt#11#21#3//{%
   \def\Left:Marg{#1}\def\Right:Marg{#2}%
   \edef\Left:Marg{\ifx \Left:Marg\empty 1\else 0\fi}%
   \edef\Right:Marg{\ifx \Right:Marg\empty 1\else 0\fi}}


|<configuring @()|>
>>>


\<visit start of pattern\><<<
\let\AllColMargins=\empty
>>>

\<visit a non-@ entry\><<<
\xdef\AllColMargins{\AllColMargins 1}%
>>>

\<visit a @ entry\><<<
\xdef\AllColMargins{\AllColMargins 0}%
>>>


\<save cols info\><<<
\PushStack\Col:Marg\AllColMargins
>>>


\<recall cols info\><<<
\PopStack\Col:Marg\AllColMargins
>>>









%%%%%%%%%%%%%%%%%%%%%%%%%%%%%%%%
\subsection{Horizontal Borders}
%%%%%%%%%%%%%%%%%%%%%%%%%%%%%%%%


\<latex tabular\><<<
\Configure{hline}{\ifx \ar:cnt\:UnDef
   \else\o:noalign:{|<hline body for array/tabular|>}%
        \expandafter\ignoreEndTr
   \fi}
\Configure{//[]}{\ifx \ar:cnt\:UnDef
   \else\o:noalign:{|<vspace body for array/tabular|>}\fi}
>>>


\<vspace body for array/tabular\><<<
\append:def\vspc:s{\h:HBorder}%
\def\:tempb{\ifnum \tmp:cnt<\ar:cnt
    \advance\tmp:cnt by 1 \append:def\vspc:s{\i:HBorder}%
    \expandafter\:tempb
  \fi }
\tmp:cnt=0 \:tempb
\append:def\vspc:s{\j:HBorder}\global\let\vspc:s=\vspc:s
>>>

\<hline body for array/tabular\><<<
\append:def\hline:s{\a:HBorder}%
\def\:tempb{\ifnum \tmp:cnt<\ar:cnt
    \advance\tmp:cnt by 1 \append:def\hline:s{\b:HBorder}%
    \expandafter\:tempb
  \fi }%
\tmp:cnt=0 \:tempb
\append:def\hline:s{\c:HBorder}\global\let\hline:s=\hline:s
>>>

\<latex tabular\><<<
\def\ignore:EndTr{\o:noalign:{\global\futurelet\:AfterHline}}
\long\def\ifProperTr#1{%
   \ifx \:AfterHline\end
       #1\global\let\:AfterHline\endinput
   \else \ifx \:AfterHline\endinput \else #1\fi\fi}
\def\ProperTrTrue{\global\let\:AfterHline=\relax}
>>>


\<latex tabular\><<<
\Configure{putHBorder}{\HCode{\HBorder}}
\Configure{HBorder}
   {|<HBorder hline|>}
      {|<HBorder hline cell|>}  
   {|<end HBorder hline|>}
   {|<HBorder cline|>}
      {|<HBorder cline p cell|>}  
      {|<HBorder cline p-hr cell|>}  
   {|<end HBorder cline|>}
   {|<HBorder vspace|>} 
       {|<HBorder vspace td|>}
   {|<end HBorder vspace|>}
\OOstyles {|<hline-p-hr style|>}
>>>




\<configure ooffice-mml latex\><<<
\Configure{HBorder}
   |<HBorder hline row|>%
   |<HBorder cline row|>%
   |<HBorder vspace row|>%
>>>


\<HBorder hline row\><<<
{\ifmathml <\a:mathml mtr\Hnewline class="hline">
 \else     |<HBorder hline|>\fi} 
{\ifmathml |<mml mtd hline|>
   \else   |<HBorder hline cell|>\fi} 
{\ifmathml </\a:mathml mtr>
 \else     |<end HBorder hline|>\fi}
>>>


\<HBorder cline row\><<<
{\ifmathml <\a:mathml mtr\Hnewline class="cline">
 \else |<HBorder cline|>\fi}
{\ifmathml  <\a:mathml mtd></\a:mathml mtd>
 \else |<HBorder cline p cell|>\fi }
{\ifmathml |<mml mtd hline|>
 \else |<HBorder cline p-hr cell|>\fi}
{\ifmathml </\a:mathml mtr>
 \else |<end HBorder cline|>\fi}
>>>

\<HBorder vspace row\><<<
{\ifmathml <\a:mathml mtr\Hnewline class="vspace"  style="font-size:\HBorderspace">
 \else     |<HBorder vspace|>\fi}
{\ifmathml <\a:mathml mtd\Hnewline></\a:mathml mtd>
 \else     |<HBorder vspace td|>\fi}
{\ifmathml </\a:mathml mtr>
 \else     |<end HBorder vspace|>\fi}
>>>






\<HBorder hline\><<<
<table:table-row  \Hnewline table:style-name="hline-row">%
>>>


\<mml mtd hline\><<<
<\a:mathml mtd><\a:mathml mo> \string&\#x0332; </\a:mathml mo></\a:mathml mtd>
>>>


\<HBorder hline cell\><<<
<table:table-cell table:style-name="hline-cell"
                         table:value-type="string">%
          <text:p text:style-name="hline-p-hr"  />%
       </table:table-cell>\Hnewline
>>>

\<end HBorder hline\><<<
</table:table-row>\Hnewline
>>>

\<HBorder cline\><<<
<table:table-row  \Hnewline table:style-name="cline-row">\Hnewline
>>>

\<HBorder cline p cell\><<<
<table:table-cell table:style-name="cline-cell">%
          <text:p text:style-name="cline-p"  />%
       </table:table-cell>\Hnewline
>>>

\<HBorder cline p-hr cell\><<<
<table:table-cell table:style-name="cline-cell">%
          <text:p text:style-name="cline-p-hr"  />%
       </table:table-cell>\Hnewline
>>>

\<end HBorder cline\><<<
</table:table-row>\Hnewline
>>>



\<HBorder vspace\><<<
<tr\Hnewline class="vspace" style="font-size:\HBorderspace">%
>>>

\<HBorder vspace td\><<<
<td\Hnewline>\string&\csname :sharp\endcsname xA0;</td>%
>>>

\<end HBorder vspace\><<<
</tr>%
>>>




\<hline-p-hr style\><<<
<style:style style:name="hline-p-hr"
             style:family="paragraph">\Hnewline
 <style:paragraph-properties 
      fo:padding="0.00001cm"
      fo:margin-top="0.00001cm"
      fo:margin-bottom="0.00001cm"
      fo:border-left="none"
      fo:border-right="none"
      fo:border-top="none"
      fo:line-height="|<hline resolution|>11cm"
      fo:border-bottom="|<hline resolution|>1cm solid \#000000"
   />\Hnewline
</style:style>
>>>


\<hline-p-hr style\><<<
<style:style style:name="cline-p-hr"
             style:family="paragraph">\Hnewline
 <style:paragraph-properties 
      fo:padding="0.00001cm"
      fo:margin-top="0.00001cm"
      fo:margin-bottom="0.00001cm"
      fo:border-left="none"
      fo:border-right="none"
      fo:border-top="none"
      fo:line-height="0.01cm"
      fo:border-bottom="|<hline resolution|>1cm solid \#000000"
   />\Hnewline
</style:style>
<style:style style:name="cline-p"
             style:family="paragraph">\Hnewline
 <style:paragraph-properties 
      fo:margin="0.00001cm"
      fo:border="none"
      fo:line-height="0.001cm"
   />\Hnewline
</style:style>
>>>



\begin{verbatim}
 > In addition, of course the formatting of borders for OOo tables doesn't
 > come out exactly the same as when entered in OOo itself, so that
 > round-tripping won't work,

TeX4ht introduces the borders in segments: top, down, left,
right. Moreover, the horizontal segments appear in separate cells. The
segmentation is there to provide for special effects in LaTeX
tables. I don't know how to improve the outcome, short of trying to
post process the output.
\end{verbatim}



%%%%%%%%%%%%%
\subsubsection{Multicolumn}
%%%%%%%%%%%%%


\<latex tabular\><<<
\Configure{multicolumn} 
   {\let\col:Css\empty 
    \expandafter\let\csname col:Css0\endcsname\empty 
    \expandafter\let\csname col:Css1\endcsname\empty 
    \Configure{VBorder}  
       {} 
       {\edef\col:Css{\col:Css 
                      \ifnum \ar:cnt=0 L\else R\fi }%
        \expandafter\edef\csname col:Css\ar:cnt\endcsname{%
                      fo:border-\ifnum \ar:cnt=0 left\else right\fi
                      ="0.01cm solid \#000000"  
                     fo:padding-\ifnum \ar:cnt=0 left\else right\fi
                     ="0.17cm" |% 5pt |%
            }%
       }{}{}% 
   } 
   {|<multicolumn sty|>}
   {|<multicolumn sty|>%
    \Configure{HtmlPar} 
       {|<multicolumn par|>}
       {|<multicolumn par|>}
       {}{}%
   } 
   {}
>>>





\<multicolumn par\><<<
\Configure{HtmlPar} 
   {\HCode{<text:line-break/><text:line-break/>}} 
   {\HCode{<text:line-break/><text:line-break/>}}
   {} 
   {}% 
>>>

The above text:p doesn't get into the code.  Where the contributions comes from???


\<multicolumn sty\><<<
\ifx \col:Css\empty \else
  \edef\col:Css{multicolumn-\col:Css 11\halignTD-\cell:style:name}%
  \expandafter\ifx \csname \col:Css\endcsname\relax
     \expandafter\xdef\csname \col:Css\endcsname{}%
     |<multicolumn oOOstyles|>%
  \fi
  \HCode{<text:span
     text:style-name="\col:Css"></text:span>}%
\fi
>>>

\<multicolumn oOOstyles\><<<
\OOstyles{%
   <style:style 
       style:name="\col:Css" 
       style:family="paragraph" 
       style:parent-style-name="\cell:style:name">\Hnewline
     <style:paragraph-properties
       \csname col:Css0\endcsname
       \csname col:Css1\endcsname
      />\Hnewline 
   </style:style>\Hnewline 
}%
>>>



\<\><<<
\OOstyles{%
  <style:style style:name="TBL-\TableNo-\HRow-\HCol"
               style:family="table-cell">\Hnewline 
      <style:table-cell-properties
%           style:vertical-align="bottom"
          \col:Css
      />\Hnewline 
  </style:style>\Hnewline 
}
>>>

%%%%%%%%%%%%%%%%%%%%%%%%%%%%%%%
\subsection{Other}
%%%%%%%%%%%%%%%%%%%%%%%%%%%%%%%




\verb|\par| is empty if not in par box.

\<start array par box\><<<
\par  \ShowPar
>>>

\<end array par box\><<<
\IgnorePar \EndP
>>>


\<latex tabular\><<<
\Configure{halign}
   {\EndP\halignTB{halign}}   {\HCode{</table>}}
   {\HCode{<tr \Hnewline}\halignTR\HCode{>}}
   {\HCode{</tr>}}
   {\HCode{<td \ifnum \HMultispan>1 colspan="\HMultispan"\fi}%
       \halignTD \HCode{\Hnewline>}}
   {\HCode{</td>}}
\Configure{pic-halign}{}
>>>



\<latex tabular\><<<
\Configure{HVerbatim+}{\z@}{\:nbsp}
\Configure{Needs}{l. 
   \the\inputlineno\space--- needs --- #1 ---}
\Configure{Needs-}{l.
   \the\inputlineno\space--- needs --- #1 ---}
>>>



%%%%%%%%%%%%%%%%%%%%%%%%%%%%%%%
\subsection{Color}
%%%%%%%%%%%%%%%%%%%%%%%%%%%%%%%

\<configure ooffice color\><<<
\Configure{color}{\:gobble}
>>>

\<configure ooffice color\><<<
\Configure{HColor:gray}{%
   \int:of\:tempa{255}{#1}%
   \hex:of\:tempa
   \Configure{HColor}{}{\#\:tempa\:tempa\:tempa}%
}
\Configure{HColor:rgb}{%
   \int:of\:tempa{255}{#1}%
   \int:of\:tempb{255}{#2}%
   \int:of\:tempc{255}{#3}%
   \hex:of\:tempa
   \hex:of\:tempb
   \hex:of\:tempc
   \Configure{HColor}{}{\#\:tempa\:tempb\:tempc}%
}
\Configure{HColor:cmyk}{%
   \int:of\:Cyan{255}{#1}%
   \int:of\:Magenta{255}{#2}%
   \int:of\:Yellow{255}{#3}%
   \int:of\:Black{255}{#4}%
   \:cmyk\:tempa\:Cyan
   \:cmyk\:tempb\:Magenta
   \:cmyk\:tempc\:Yellow
   \hex:of\:tempa
   \hex:of\:tempb
   \hex:of\:tempc
   \Configure{HColor}{}{\#\:tempa\:tempb\:tempc}%
}
|<HColor util|>
>>>





\<HColor util\><<<
\def\int:of#1#2#3{%
   \tmp:dim=#3pt \tmp:dim=#2\tmp:dim
   \edef\:temp{\tmp:cnt\the\tmp:dim//}%
   \def#1##1//{}\afterassignment#1\:temp
   \edef#1{\the\tmp:cnt}%
}
\def\hex:of#1{%
   \tmp:cnt=#1
   \divide\tmp:cnt by 16
   \edef\:temp{\ifcase \tmp:cnt 0\or 1\or 2\or 3\or 4\or 5\or 6\or 7\or 
                                8\or 9\or A\or B\or C\or D\or E\or F\fi}%
   \multiply\tmp:cnt by -16 \advance\tmp:cnt by #1
   \edef#1{\:temp 
           \ifcase \tmp:cnt 0\or 1\or 2\or 3\or 4\or 5\or 6\or 7\or 
                                8\or 9\or A\or B\or C\or D\or E\or F\fi}%
}
>>>


\begin{verbatim}
cmyk -> gray
(1 - 0.299c - 0.587m - 0.114y)(1 - k) 

cmyk -> rgb
Red   = 1-minimum(1,Cyan   *(1-Black)+Black)
Green = 1-minimum(1,Magenta*(1-Black)+Black)
Blue  = 1-minimum(1,Yellow *(1-Black)+Black)
\end{verbatim}



\<HColor util\><<<
\def\:cmyk#1#2{%
   \tmp:cnt=255 \advance\tmp:cnt by -\:Black
   \multiply\tmp:cnt by#2 \advance\tmp:cnt by \:Black 
   \advance\tmp:cnt by -255 \tmp:cnt=-\tmp:cnt
   \ifnum \tmp:cnt<0 \tmp:cnt=0 \fi
   \edef#1{\the\tmp:cnt}%
}
>>>


The following doesn't work yet.


\begin{verbatim}
\documentclass{article} 
\usepackage[dvips]{xcolor} 
 
\begin{document} 
 
\begin{center} 
\fcolorbox{black}{green}{\parbox{.775\columnwidth}{\textbf{Hint:}\vspace{.2cm}\\ Abc.}} 
\end{center} 
 
 
\begin{center} 
 
\fcolorbox{black}{green}{\parbox{.775\columnwidth}{\textbf{Hint:}\vspace{.2cm}\\ Abc.}}  
 
\fcolorbox{black}{green}{\parbox{.775\columnwidth}{\textbf{Hint:}\vspace{.2cm}\\ Abc.}}   
 
\end{center} 
\end{document} 
 

\documentclass {article}  
  
  \usepackage[table,svgnames]{xcolor}   
  
\begin {document}  

    \rowcolors[]{1}{gray!10}{gray!5}  

    \begin{tabular}{|*{2}{r|c|}}  
      Blue & Red & Yellow & Green\\\hline  
      Blue & Red & Yellow & Green\\  
      Blue & Red & Yellow & Green\\\hline  
      Blue & Red & Yellow & Green\\  
      Blue & Red & Yellow & Green\\  
      Blue & Red & Yellow & Green\\  
      Blue & Red & Yellow & Green\\  
    \end{tabular}  

\end{verbatim}

\<configure ooffice color\><<<
\Configure{fcolorbox}  
   {\bgroup\gHAdvance\textcolorN by 1  
     \ifvmode \let\v:colorbox\def \IgnorePar \EndP  
     \else \let\v:colorbox=\relax \fi  
     \ifx \v:colorbox\def 
        \HCode{<text:p  \Hnewline  
                    text:style-name="colorbox\textcolorN">}%  
%%%%%%%%%%%%%%%%%%%%%%%%%%%%%%%%%% 
\let\A:color=\a:color  
\Configure{color}%  
    {\Configure{color}%  
       {\let\a:color=\A:color  
         \Configure{SetHColor}{%  
            \OOstyles{<style:style style:name="colorbox\textcolorN"
                                   style:family="paragraph"\>  
%              <style:text-properties style:text-background-color="\HColor"/>  
              <style:text-properties fo:color="\HColor"
                                     fo:background-color="transparent" />
           </style:style>}% 
       }}%  
       \Configure{SetHColor}{%  
%         \hshow{\#colorbox\textcolorN{border: solid 1px  \HColor;}}%
       }%  
     }%   
     \Configure{fbox}{\leavevmode }{}%  
%%%%%%%%%%%%%%%%%%%%%%%%%%%%%%%%%% 
     \else 
        \HCode{<text:span \Hnewline  
                    text:style-name="colorbox\textcolorN">}%  
%%%%%%%%%%%%%%%%%%%%%%%%%%%%%%%%%% 
\let\A:color=\a:color  
\Configure{color}%  
    {\Configure{color}%  
       {\let\a:color=\A:color  
         \Configure{SetHColor}{%  
            \OOstyles{<style:style style:name="colorbox\textcolorN" style:family="text">  
%              <style:text-properties style:text-background-color="\HColor" />  
              <style:text-properties fo:color="\HColor"
                                     fo:background-color="transparent" />
           </style:style>}% 
       }}%  
       \Configure{SetHColor}{%  
%         \hshow{\#colorbox\textcolorN{border: solid 1px  \HColor;}}%
       }%  
     }%   
     \Configure{fbox}{\leavevmode }{}%  
%%%%%%%%%%%%%%%%%%%%%%%%%%%%%%%%%% 
     \fi 
} 
{\ifx \v:colorbox\def  
    \EndP\HCode{</text:p>}\par  
  \else \HCode{</text:span>}\fi  
  \egroup 
} 
>>>


%%%%%%%%%%%%%%%%%%%%%%%%%%%%%%%
\subsection{XColor}
%%%%%%%%%%%%%%%%%%%%%%%%%%%%%%%


\<configure ooffice xcolor\><<<
\Configure{rowcolors}
   {\Configure{SetHColor}{%
      \expandafter\ifx \csname HColor\endcsname\relax \else
%          \hshow{tr\#TBL-\TableNo-\the\rownum-{background-color:\HColor}}%
      \fi
   }}
>>>

\<configure ooffice xcolor\><<<
\Configure{HColor!gray}{%
   \XC@c@lcD\rangeGray{}{#1}\@@tmp 
   \int:of\:tempa{255}{\@@tmp}% 
   \hex:of\:tempa
   \Configure{HColor}{}{\#\:tempa\:tempa\:tempa}%
}
>>>



%%%%%%%%%%%%%%%%%%%%%%%%%%%%%%%
\section{tabbing}
%%%%%%%%%%%%%%%%%%%%%%%%%%%%%%%

\<ooffice begin-end env\><<<
\HAssign\tabbing:N=0
\Configure{tabbing}[1.5]{\IgnorePar\leavevmode
     \gHAdvance\tabbing:N by 1
     \gHAssign\tabbing:Nrow=0
     \ht:special{t4ht=<text:p 
         text:style-name="tabbing" \Hnewline   >}}
   {\ht:special{t4ht=</text:p>\Hnewline}\ShowPar}
   {\gHAdvance\tabbing:Nrow by 1
    \ht:special{t4ht=<text:span\Hnewline 
       text:style-name="tabbing-\tabbing:N-\tabbing:Nrow"
         >}%
       |<style of current tabbing cell|>%
   }
   {\ht:special{t4ht=</text:span>}}
\ConfigureEnv{tabbing}{\IgnorePar\EndP}{}{}{}
>>>

\<ooffice begin-end env\><<<
\NewConfigureOO{tabbing}
\ConfigureOO{tabbing}{|<oo tabbing style|>}
>>>


\<oo tabbing style\><<<
<style:style style:name="tabbing" 
             style:family="paragraph"
             style:parent-style-name="Standard" >\Hnewline
  <style:paragraph-properties  fo:margin-left="0cm"
             fo:margin-right="0cm"
             fo:text-indent="0.5cm"
             style:auto-text-indent="false"/>\Hnewline
</style:style>\Hnewline
>>>

The following style sdon't work.


\<style of current tabbing cell\><<<
\OOstyles{%
    <style:style style:name="tabbing-\tabbing:N-\tabbing:Nrow"
                 >\Hnewline
         <style:text-properties
               fo:wrap-option="no-wrap"
               \ifx \TabWidth\empty \else
                   fo:width="\TabWidth pt"
               \fi
               \ifx \`\TabType
                   style:text-align="right"
               \fi
              />\Hnewline
    </style:style>\Hnewline
  }    
>>>


\begin{verbatim}
Is there a way to request a no-wrap element <text:span> ...
</text:span> of a specified width?

      LaTeX source:
      \begin{tabbing}
      Armadillo \= Armament \=       \kill
      Gnat      \> Gnu      \> Gnome \\
      Armadillo \> Armament \> Armorer
      \end{tabbing}

Unfortunately not, you have to use either tab-stops or a table for this.
\end{verbatim}

%%%%%%%%%%%%%%%%%%%%%%%%%%%%%%%
\section{Index}
%%%%%%%%%%%%%%%%%%%%%%%%%%%%%%%


%%%%%%%%%%%%%%%%%%%%%%%%%%%%%%%
\subsection{Envelop}
%%%%%%%%%%%%%%%%%%%%%%%%%%%%%%%


\<ooffice index\><<<
\Configure{theindex}
   {\ifvmode\IgnorePar\fi 
    \EndP\HCode{<text:alphabetical-index text:name="theindex"  >
      |<index format|>%
      <text:index-body>\Hnewline}%
    \csname put:IndexTitle\endcsname
   }
   {\HCode{</text:index-body></text:alphabetical-index>}}
   {\HCode{<text:p text:style-name="P6"  >}}
      {\HCode{<text:tab-stop/></text:p>}\hfill\break}
   {\HCode{<text:p text:style-name="P7"  >}}
      {\HCode{<text:tab-stop/></text:p>}\hfill\break}
   {\HCode{<text:p text:style-name="P8"  >}}
      {\HCode{<text:tab-stop/></text:p>}\hfill\break}
   {}
>>>





\<index format\><<<
<text:alphabetical-index-source 
       text:sort-algorithm="alphanumeric"  >\Hnewline
   <text:index-title-template 
                text:style-name="Index-Heading"  >\indexname
   </text:index-title-template>\Hnewline
   |<index entry format|>\Hnewline
</text:alphabetical-index-source>\Hnewline
>>>

\<index entry format\><<<
<text:alphabetical-index-entry-template 
       text:outline-level="1" 
       text:style-name="Index-1"  >\Hnewline
  <text:index-entry-text/>\Hnewline
  <text:index-entry-tab-stop 
     style:type="left" 
     style:position="0in" 
     style:leader-char=" "
      />\Hnewline
  <text:index-entry-page-number/>\Hnewline
</text:alphabetical-index-entry-template>\Hnewline
>>>

\<index entry format\><<<
<text:alphabetical-index-entry-template 
       text:outline-level="2" 
       text:style-name="Index-2"  >\Hnewline
  <text:index-entry-text/>\Hnewline
  <text:index-entry-tab-stop 
     style:type="left" 
     style:position="0in" 
     style:leader-char=" "
       />\Hnewline
  <text:index-entry-page-number/>\Hnewline
</text:alphabetical-index-entry-template>\Hnewline
>>>

\<index entry format\><<<
<text:alphabetical-index-entry-template 
       text:outline-level="3" 
       text:style-name="Index-3"  >\Hnewline
  <text:index-entry-text/>\Hnewline
  <text:index-entry-tab-stop 
     style:type="left" 
     style:position="0in" 
     style:leader-char=" "
       />\Hnewline
  <text:index-entry-page-number/>\Hnewline
</text:alphabetical-index-entry-template>\Hnewline
>>>








\<content style 6\><<<
\NewConfigureOO{Index 1}
\ConfigureOO{Index 1}{\Hnewline
   <style:style style:name="Index-1"
                style:family="paragraph"
                style:parent-style-name="Index"
                style:class="index">
   <style:paragraph-properties  fo:margin-left="0cm"
                fo:margin-right="0cm"
                fo:text-indent="0cm"
                style:auto-text-indent="false"/>
   </style:style>
   \Hnewline }
>>>
 


\<content style 7\><<<
\NewConfigureOO{Index 2}
\ConfigureOO{Index 2}{\Hnewline
   <style:style style:name="Index-2"
                style:family="paragraph"
                style:parent-style-name="Index"
                style:class="index">\Hnewline
   <style:paragraph-properties  fo:margin-left="0.499cm"
                fo:margin-right="0cm"
                fo:text-indent="0cm"
                style:auto-text-indent="false"/>\Hnewline
   </style:style>
\Hnewline}
>>>
 


\<content style 8\><<<
\NewConfigureOO{Index 3}
\ConfigureOO{Index 3}{\Hnewline
   <style:style style:name="Index-3"
                style:family="paragraph"
                style:parent-style-name="Index"
                style:class="index">\Hnewline
   <style:paragraph-properties  fo:margin-left="0.998cm"
                fo:margin-right="0cm"
                fo:text-indent="0cm"
                style:auto-text-indent="false"/>\Hnewline
   </style:style>
   \Hnewline}
>>>




\<automatic style 6\><<<
\Hnewline
   <style:style style:name="P6"
                style:family="paragraph"
                style:parent-style-name="Index-1">\Hnewline
   <style:paragraph-properties >\Hnewline
      <style:tab-stops>\Hnewline
         <style:tab-stop style:position="0cm"/>\Hnewline
      </style:tab-stops>\Hnewline
   </style:text-properties>\Hnewline
   </style:style>
   \Hnewline 
>>>


\<automatic style 7\><<<
\Hnewline
   <style:style style:name="P7"
                style:family="paragraph"
                style:parent-style-name="Index-2">\Hnewline
   <style:paragraph-properties >\Hnewline
      <style:tab-stops>\Hnewline
         <style:tab-stop style:position="0cm"/>\Hnewline
      </style:tab-stops>\Hnewline
   </style:paragraph-properties >\Hnewline
   </style:style>
   \Hnewline 
>>>


\<automatic style 8\><<<
\Hnewline
   <style:style style:name="P8"
                style:family="paragraph"
                style:parent-style-name="Index-3">\Hnewline
   <style:paragraph-properties >\Hnewline
      <style:tab-stops>\Hnewline
         <style:tab-stop style:position="0cm"/>\Hnewline
      </style:tab-stops>\Hnewline
   </style:paragraph-properties >\Hnewline
   </style:style>
   \Hnewline 
>>>


%%%%%%%%%%%%%
\subsection{Making Index Entries}
%%%%%%%%%%%%%



\<configure ooffice latex\><<<
\catcode`\!=13
\HAssign\indexmark:N = 0
\immediate\openin15=\jobname.4ix
\ifeof15 \else
   \Configure{wrindex}
     {{\special{t4ht@(}\catcode`\\=0 \catcode`\{=1 
       \catcode`\}=2 \catcode`\ =10 \catcode`\!=13
       \def!{\c:indexmark\def!{\d:indexmark\def!{\string!}}}%
       \gHAdvance\indexmark:N by 1
       \def\indexmark##1##2{%
          \ifnum \indexmark:N=##2 \special{t4ht@)}\a:indexmark
              ##1\b:indexmark\special{t4ht@(}\fi}%
       \hbox{\input \jobname.4ix \special{t4ht@)}}%
     }}
\fi
\catcode`\!=12
\NewConfigure{indexmark}{4}
\Configure{indexmark}
   {\HCode{<text:alphabetical-index-mark\Hnewline
                                       text:id="dx\last:haddr"
                                       text:string-value="}%
    \NoFonts}
   {\EndNoFonts \HCode{"  />}}
   {\HCode{" text:key1="}}
   {\HCode{" text:key2="}}
>>>


% %%%%%%%%%%%%%%%%%%%%%%%%%%%%%%%%%%%%%%%%%%%%%%%%%%%%%%%%
% \pend:def\a:wrindex{%
%    \let\title::chs=\title:chs
%   \def\title:chs{\let\title:chs\title::chs
%                  \title:chs{\html:addr}{}\:gobbleII}%
% }
% %%%%%%%%%%%%%%%%%%%%%%%%%%%%%%%%%%%%%%%%%%%%%%%%%%%%%%%%%%
% \Configure{indexmark}
%    {\HCode{<!--}\Link-{}{dx\last:haddr}\EndLink\HCode{-->}%
%     \HCode{<text:alphabetical-index-mark\Hnewline


\<configure ooffice latex\><<<
\pend:def\index{\leavevmode}
>>>

The following two sample codes provide different behavior.

\begin{verbatim}
\documentclass[12pt]{article}  
\usepackage{makeidx}     
\makeindex  
\begin{document}  
\index{Hola: Hallo} Hola  
\printindex                       
\end{document}  
\end{verbatim}

\begin{verbatim}
\documentclass{article}  
\usepackage[frenchb]{babel}  
\usepackage{makeidx}  
\makeindex  
  
\begin{document}  
\begin{itemize}  
\item \index{blabla} blabla  
\item blabla  
\end{itemize}  
\end{document}  
\end{verbatim}






%%%%%%%%%%%%%
\subsection{Insed Title Head}
%%%%%%%%%%%%%

\<ooffice index\><<<
\Configure{LNK}{\oo:LNK}{\EndLink}{}
\Odef\oo:LNK[#1]#2#3{\Link{#2}{#3}}
>>>


\<chapter index\><<<
\Configure{@begin}{theindex}{\ifx \indexname\empty \else
    \def\put:IndexTitle{|<index title|>}\fi}
>>>

\<non-chapter index\><<<
\Configure{@begin}{theindex}{\ifx \indexname\empty \else
    \def\put:IndexTitle{|<index title|>}\fi}
>>>

\<index title\><<<
\HCode{<text:index-title
   text:style-name="Sect1"
   text:name="Alphabetical_Index1_Head"
     ><text:p
   text:style-name="Index-Heading"  >}\indexname
\HCode{</text:p></text:index-title>\Hnewline}
>>>


\<configure ooffice latex\><<<
\NewConfigureOO{Index Heading}
\ConfigureOO{Index Heading}{\Hnewline
  <style:style
           style:name="Index-Heading"
           style:family="paragraph"
           style:parent-style-name="Heading"
           style:class="index">\Hnewline
     <style:paragraph-properties 
        fo:margin-left="0in"
        fo:margin-right="0in"
        fo:font-size="16pt"
        fo:font-weight="bold"
        style:font-size-asian="16pt"
        style:font-weight-asian="bold"
        style:font-size-complex="16pt"
        style:font-weight-complex="bold"
        fo:text-indent="0in"
        style:auto-text-indent="false"
        text:number-lines="false"
        text:line-number="0"/>\Hnewline
  </style:style>\Hnewline
}
>>>

\<configure html4 multind\><<<
\Configure{@begin}{theindex}
   {\ifx \indexname\empty \else|<toc for multind|>\fi}
>>>


% \ifx \indexname\empty \else
%       \ifx\chapter\:Undef \section*{\indexname}\else
%                           \chapter*{\indexname}\fi
%       \let\indexname=\empty 
%     \fi
%     



\<configure html4 index\><<<
\Configure{NoSection}
  {\let\sv:index=\p@index \let\sv:label=\label 
   \let\sv:newline=\newline \def\newline{ }%
   \def\p@index[##1]{\@gobble}\let\label=\@gobble 
   |<nolink ref|>%
  }
  {\let\p@index=\sv:index \let\label=\sv:label 
   |<restore ref|>%
   \let\newline=\sv:newline
  }
>>>




\<configure html4 texinfo\><<<
\Configure{index}
  {\IgnorePar\leavevmode\EndP\Tg<div class="index">}
  {\IgnorePar\EndP\Tg</div>}
  {\leavevmode\Tg<span class="entry">}
  {\Tg</span>\Tg<br />\IgnorePar}
>>>







%%%%%%%%%%%%%%%%%%%%%%%%%%%%%%%
\subsection{mwcls}
%%%%%%%%%%%%%%%%%%%%%%%%%%%%%%%

\<mwcls index\><<<
\Configure{@begin}{theindex}{\section*{\indexname}}
>>>


%%%%%%%%%%%%%%%%%%
\section{Begin/End Environments}
%%%%%%%%%%%%%%%%%%


%%%%%%%%%%%%%
\subsection{Verse}
%%%%%%%%%%%%%



\<configure ooffice latex\><<<
\ConfigureEnv{verse}
   {|<begin env section|>%
    \HCode{<!--start verse-->}%
    \bgroup  |<HtmlPar at verse|>}
   {\egroup \ifvmode \IgnorePar\fi \EndP
    \HCode{<!--end verse-->}|<end env section|>\par\ShowPar}
   {}{}
\ConfigureList{verse}
   {}{}
   {\parindent=\z@}  {\par}
\NewConfigureOO{verse}
\ConfigureOO{verse}{\Hnewline
<style:style style:name="verse"
             style:family="paragraph"
             style:parent-style-name="Text-body"
             style:next-style-name="Text-body">
<style:paragraph-properties  fo:margin-left="1.499cm"
             fo:margin-right="1cm"
             fo:margin-top="0.199cm"
             fo:margin-bottom="0.199cm"
             fo:text-indent="-0.499cm"
             style:auto-text-indent="false">
   <style:tab-stops/>
</style:paragraph-properties >
</style:style>
\Hnewline}
>>>

\<configure ooffice verse\><<<
\ConfigureEnv{verse}
   {|<begin env section|>%
    \HCode{<!--start verse-->}%
    \bgroup
\Configure{HtmlPar}
   {\ifvmode \IgnorePar\fi \EndP\end:verseln
    \HCode{<text:p text:style-name="verse-stanza"  >}}
   {\ifvmode \IgnorePar\fi \EndP\end:verseln
     \HCode{<text:p text:style-name="verse-stanza"  >}}
   {\HCode{</text:p>}\let\end:verseln\empty}
   {\HCode{</text:p>}\let\end:verseln\empty}%
       \let\end:verseln\empty
   }
   {\egroup 
   \ifvmode \IgnorePar\fi \EndP\HCode{<!--end verse-->}|<end env section|>
   \par\ShowPar}
   {}{}

\ConfigureList{verse}
   {} {\ifvmode \IgnorePar\fi \EndP \end:verseln}
   {\parindent=\z@}  {}

\Configure{verseline}
   {\ifvmode \IgnorePar\fi \EndP \end:verseln  
    \HCode{<text:p text:style-name="verse">}\def\end:verseln{\HCode{</text:p>}}}
   {\ifvmode \IgnorePar\fi \EndP \end:verseln
    \HCode{<text:p text:style-name="verse">}%
    \def\end:verseln{\HCode{</text:p>}}}  

\ConfigureOO{verse}{\Hnewline
<style:style style:name="verse"
             style:family="paragraph"
             style:parent-style-name="Text-body"
             style:next-style-name="Text-body">
<style:paragraph-properties  fo:margin-left="1.499cm"
             fo:margin-right="1cm"
             fo:margin-top="0cm"
             fo:margin-bottom="0cm"
             fo:text-indent="-0.499cm"
             style:auto-text-indent="false">
   <style:tab-stops/>
</style:paragraph-properties >
</style:style>
\Hnewline}

\NewConfigureOO{verse-stanza}
\ConfigureOO{verse-stanza}{\Hnewline
<style:style style:name="verse-stanza"
             style:family="paragraph"
             style:parent-style-name="Text-body"
             style:next-style-name="Text-body">
<style:paragraph-properties  fo:margin-left="1.499cm"
             fo:margin-right="1cm"
             fo:margin-top="0.5cm"
             fo:margin-bottom="0cm"
             fo:text-indent="-0.499cm"
             style:auto-text-indent="false">
   <style:tab-stops/>
</style:paragraph-properties >
</style:style>
\Hnewline}
>>>

%%%%%%%%%%%%%
\subsection{Paragraph}
%%%%%%%%%%%%%




\<HtmlPar at verse\><<<
\Configure{HtmlPar}
   {\ifvmode \IgnorePar\fi \EndP
    \HCode{<text:p text:style-name="verse"  >}}
   {\ifvmode \IgnorePar\fi \EndP
     \HCode{<text:p text:style-name="verse"  >}}
   {\HCode{</text:p>}}
   {\HCode{</text:p>}}%
>>>






%%%%%%%%%%%%%
\subsection{Table}
%%%%%%%%%%%%%



\<ooffice begin-end env\><<<
\ConfigureEnv{table}
   {\ifvmode \IgnorePar\fi \EndP
    \HCode{<text:section text:name="table"  >}%
    \bgroup  \Configure{float}{\ShowPar}{}{\ShowPar\IgnoreIndent\csname par\endcsname}%
   }
   {\egroup
    \ifvmode \IgnorePar\fi\EndP \HCode{</text:section>}%
    \csname par\endcsname\ShowPar}
   {}{}
>>>






%%%%%%%%%%%%%
\subsection{Float}
%%%%%%%%%%%%%



\<configure ooffice latex\><<<
\Configure{float}
   {\csname par\endcsname\ShowPar \leavevmode}
   {\ifvmode \IgnorePar\fi \EndP
    \HCode{<text:section text:name="float"  >}}
   {\ifvmode \IgnorePar\fi\EndP 
    \HCode{</text:section>}%
    \csname par\endcsname\ShowPar
   }
\ConfigureEnv{figure}
   {\ifvmode \IgnorePar\fi \EndP
    \HCode{<text:section text:name="fig" 
                         text:style-name="Figure"
                          >|<big skip|>}%
    \bgroup \Configure{float}{\ShowPar}{}{}%    
   }
   {\egroup
    \ifvmode \IgnorePar\fi\EndP \HCode{|<big skip|></text:section>}%
    \csname par\endcsname\ShowPar}
   {}{}
>>>

\<configure ooffice Preamble\><<<
\edef\Preamble{\Preamble,refcaption}
>>>

The section style  can't be in style.xml?


\<oo2 automatic styles\><<<
\Configure{automatic-styles}
   {\csname a:figure-style\endcsname}
\NewConfigure{figure-style}{1}
\Configure{figure-style}{%
   <style:style style:name="Figure" style:family="section">\Hnewline
     <style:section-properties 
                               fo:margin-left="0.25in" 
                               fo:margin-right="0.25in" 
                               style:editable="true">\Hnewline
         <style:columns fo:column-count="0" 
                        fo:column-gap="0in"/> 
     </style:section-properties> 
   </style:style>\Hnewline 
}
>>>



%%%%%%%%%%%%%%%%%%%%%%%%%%%
\section{lineno.sty}
%%%%%%%%%%%%%%%%%%%%%%%%%%%

\<configure ooffice lineno\><<<
\:warning{The line numbers are commented out. To reintroduce them, use
    the \string\Configure{lineno}{...}{...} command}
\Configure{lineno}
    {\bgroup \let\LineNumber=\empty}
    {\egroup}
>>>



%%%%%%%%%%%%%%%%%%
\section{Text Decorations}
%%%%%%%%%%%%%%%%%%


%%%%%%%%%%%%%%%%%%
\subsection{Fonts}
%%%%%%%%%%%%%%%%%%



\<configure ooffice tex4ht\><<<
\Configure{htf}{0}{+}{<text:span\Hnewline 
   text:style-name="}{\%s}{-\%s}{x-x-\%d}{}{">}{</text:span>}
\Configure{htf}{4}{+}{<text:span\Hnewline
   text:style-name="}{}{}{}{}{small-caps">}{</text:span>}
>>>

\<configure ooffice tex4ht\><<<
\NewConfigureOO{small-caps}
\ConfigureOO{small-caps}{%
<style:style style:name="small-caps" style:family="text">\Hnewline
   <style:text-properties
             fo:font-variant="small-caps"
   />\Hnewline
</style:style>\Hnewline    
}
>>>


\<configure ooffice fontmath\><<<
\ifx \find:set:mgroup\:UnDef 
   \def\find:set:mgroup#1#2#3{\bgroup
      \let\HPage\:gobble
      \let\EndHPage\:gobble
      \def\use@mathgroup##1##2##3{\relax
        \tmp:cnt=##2%
        \xdef\:temp####1{\noexpand\Configure{MathClass}{#2}%
            {}{}{}{\mathchar"0\the\tmp:cnt ####1}}%
      }%
      \setbox\tmp:bx=\hbox{$#1$}\setbox\tmp:bx=\hbox{}% 
      \set:mgroup#3{}{}%
    \egroup
    }
\fi
>>>


\<configure ooffice fontmath\><<<
\Configure{texttt}
     {\ifmathml \Configure{@mtext}{\mml:class="texttt" mathvariant="monospace" }%
      \else     \HCode{<text:span text:style-name="texttt"  >}\fi}
     {\ifmathml\else \HCode{</text:span>}\fi}
  \Configure{textit}
     {\ifmathml \Configure{@mtext}{\mml:class="textit" mathvariant="italic" }%
      \else     \HCode{<text:span text:style-name="textit"  >}\fi}
     {\ifmathml\else \HCode{</text:span>}\fi}
  \Configure{textrm}
     {\ifmathml \Configure{@mtext}{ \mml:class="textrm" mathvariant="normal" }%
      \else     \HCode{<text:span text:style-name="textrm"  >}\fi}
     {\ifmathml\else \HCode{</text:span>}\fi}
  \Configure{textsl}
     {\ifmathml \Configure{@mtext}{ \mml:class="textst" mathvariant="italic" }%
      \else     \HCode{<text:span text:style-name="textsl"  >}\fi}
     {\ifmathml\else \HCode{</text:span>}\fi}
  \Configure{textsf}
     {\ifmathml \Configure{@mtext}{ \mml:class="textsf" mathvariant="sans-serif" }%
      \else     \HCode{<text:span text:style-name="textsf"  >}\fi}
     {\ifmathml\else \HCode{</text:span>}\fi}
  \Configure{textbf}
     {\ifmathml \Configure{@mtext}{ \mml:class="textbf" mathvariant="bold" }%
      \else     \HCode{<text:span text:style-name="textbf"  >}\fi}
     {\ifmathml\else \HCode{</text:span>}\fi}
  \Configure{textsc}
     {\ifmathml \Configure{@mtext}{ \mml:class="textsc" mathvariant="normal" }%
      \else     \HCode{<text:span text:style-name="textsc"  >}\fi}
     {\ifmathml\else \HCode{</text:span>}\fi}
  \Configure{emph}
     {\ifmathml \Configure{@mtext}{ \mml:class="emph" mathvariant="italic" }%
      \else     \HCode{<text:span text:style-name="Emphasis"  >}\fi}
     {\ifmathml\else \HCode{</text:span>}\fi}
  \Configure{textup}
     {\ifmathml \Configure{@mtext}{ \mml:class="textup" mathvariant="normal" }%
      \else     \HCode{<text:span text:style-name="textup"  >}\fi}
     {\ifmathml\else \HCode{</text:span>}\fi}
>>>



\<configure ooffice fontmath\><<<
\NewConfigureOO{emph}
\ConfigureOO{emph}{|<oo emph style|>}
\NewConfigureOO{textit}
\ConfigureOO{textit}{|<oo textit style|>}
\NewConfigureOO{textbf}
\ConfigureOO{textbf}{|<oo textbf style|>}
\NewConfigureOO{textsc}
\ConfigureOO{textsc}{|<oo textsc style|>}
\NewConfigureOO{texttt}
\ConfigureOO{texttt}{|<oo texttt style|>}
\NewConfigureOO{textrm}
\ConfigureOO{textrm}{|<oo textrm style|>}
\NewConfigureOO{textsf}
\ConfigureOO{textsf}{|<oo textsf style|>}
\NewConfigureOO{textsl}
\ConfigureOO{textsl}{|<oo textsl style|>}
\NewConfigureOO{textsl}
\ConfigureOO{textsl}{|<oo textup style|>}
>>>


\<oo textsf style\><<<
<style:style style:name="textsf" style:family="text">\Hnewline
   <style:text-properties fo:font-style="normal" 
                     fo:font-weight="bold"
                     style:font-family-generic="swiss" 
                     style:font-pitch="variable" />\Hnewline
</style:style>\Hnewline
>>>


\<oo emph style\><<<
<style:style style:name="Emphasis" style:family="text">\Hnewline
   <style:text-properties fo:font-style="italic"/>\Hnewline
</style:style>\Hnewline
>>>

\<oo textit style\><<<
<style:style style:name="textit" style:family="text">\Hnewline
   <style:text-properties fo:font-style="italic"/>\Hnewline
</style:style>\Hnewline
>>>




\<oo textrm style\><<<
<style:style style:name="textrm" style:family="text">\Hnewline
<style:text-properties fo:font-weight="normal" 
             style:font-weight-asian="normal" 
             style:font-weight-complex="normal"/>\Hnewline
</style:style>\Hnewline
>>>





\<oo textsl style\><<<
<style:style style:name="textsl" style:family="text">\Hnewline
   <style:text-properties fo:font-style="italic"/>\Hnewline
</style:style>\Hnewline
>>>

\<oo textbf style\><<<
<style:style style:name="textbf" style:family="text">\Hnewline
   <style:text-properties fo:font-style="normal" 
                     fo:font-weight="bold"/>\Hnewline
</style:style>\Hnewline
>>>

\<oo textsc style\><<<
<style:style style:name="textsc" style:family="text">\Hnewline
   <style:text-properties
             fo:font-variant="small-caps"
             fo:font-style="normal"
             fo:font-weight="normal"
%              fo:font-size="75\%"
             style:text-position="0\% 100\%"
   />\Hnewline
</style:style>\Hnewline    
>>>


\<oo texttt style\><<<
<style:style style:name="texttt" style:family="text">\Hnewline
   <style:text-properties
             fo:font-family="monospace"
   />\Hnewline
</style:style>\Hnewline    
>>>

\<oo textup style\><<<
<style:style style:name="textup" style:family="text">\Hnewline
<style:text-properties fo:font-weight="normal" 
             style:font-weight-asian="normal" 
             style:font-weight-complex="normal"/>\Hnewline
</style:style>\Hnewline
>>>



%%%%%%%%%%%%%
\subsection{Soul}
%%%%%%%%%%%%%

\<configure ooffice soul\><<<
\Configure{textcaps}
     {\HCode{<text:span text:style-name="textcaps">}}
     {\HCode{</text:span>}}
\Configure{texthl}
     {\HCode{<text:span text:style-name="texthl">}\bgroup 
     \ifx\setstcolor\:gobble\else |<soul color|>\fi}
     {\egroup\HCode{</text:span>}}
\Configure{textso}
     {\HCode{<text:span text:style-name="textso">}}
     {\HCode{</text:span>}}
\Configure{textst}
     {\HCode{<text:span text:style-name="textst">}\bgroup 
     \ifx\setstcolor\:gobble\else |<soul color|>\fi}
     {\egroup\HCode{</text:span>}}
\Configure{textul}
     {\HCode{<text:span text:style-name="textul">}\bgroup 
     \ifx\setstcolor\:gobble\else |<soul color|>\fi}
     {\egroup\HCode{</text:span>}}
>>>





\<configure ooffice soul\><<<
\NewConfigureOO{textcaps}
\ConfigureOO{textcaps}{|<oo textcaps style|>}
\NewConfigureOO{textso}
\ConfigureOO{textso}{|<oo textso style|>}
\NewConfigureOO{textst}
\ConfigureOO{textst}{|<oo textst style|>}
\NewConfigureOO{textul}
\ConfigureOO{textul}{|<oo textul style|>}
>>>


\<oo textcaps style\><<<
<style:style style:name="textcaps" style:family="text">\Hnewline
  <style:text-properties fo:font-variant="small-caps" />\Hnewline
</style:style>\Hnewline
>>>



\<oo textso style\><<<
<style:style style:name="textso" style:family="text">
   <style:text-properties fo:letter-spacing="0.035in"/>
</style:style>
>>>

\<oo textst style\><<<
<style:style style:name="textst" style:family="text">
   <style:text-properties style:text-line-through-style="solid"
                          style:text-underline-style="none"     />
</style:style>
>>>

\<oo textul style\><<<
<style:style style:name="textul" style:family="text">
    <style:text-properties style:text-underline-style="solid"
                           style:text-underline-width="auto"
                           style:text-underline-color="font-color"
    />
</style:style>
>>>

\<soul color\><<<
\Configure{textcolor} 
   {\gHAdvance\textcolorN by 1 
    \Configure{color}%
      {\Configure{SetHColor}{%
         \OOstyles{%
           <style:style style:name="textcolor\textcolorN" style:family="text">
              <style:text-properties fo:color="\HColor"
                                     fo:background-color="transparent" />
           </style:style>}%
    }}%
    \HCode{<text:span text:style-name="textcolor\textcolorN">}}%
   {\HCode{</text:span>}}%
>>>


% <style:text-properties style:text-background-color="\HColor"/>

%    <style:style style:name="T7" style:family="text">
%      <style:text-properties fo:color="#008000" fo:background-color="#ffffff"/>
%    </style:style>


%%%%%%%%%%%%%
\subsection{Color}
%%%%%%%%%%%%%


\<configure ooffice color\><<<
\HAssign\textcolorN=0
>>>


\<configure ooffice color\><<<
\Configure{textcolor} 
   {\gHAdvance\textcolorN by 1 
    \Configure{color}%
      {\Configure{SetHColor}{%
         \OOstyles{%
           <style:style style:name="textcolor\textcolorN" style:family="text">
              <style:text-properties fo:color="\HColor"
                                     fo:background-color="transparent" />
           </style:style>}%
    }}%
    \HCode{<text:span text:style-name="textcolor\textcolorN">}}%
   {\HCode{</text:span>}}%
>>>


%%%%%%%%%%%%%%%%%%
\section{Pictures}
%%%%%%%%%%%%%%%%%%

 
\<configure ooffice latex\><<< 
\Configure{picture} 
    {\Picture+[PICT]{}} 
    {\EndPicture} 
>>>



\<configure ooffice pst-jtree\><<< 
\Configure{picture} 
    {\Picture+[PICT]{}} 
    {\EndPicture} 
>>>





%%%%%%%%%%%%%%%%%%%%%%%%%%%%%%%
\section{Equation Environment}
%%%%%%%%%%%%%%%%%%%%%%%%%%%%%%%


%%%%%%%%%%%%%
\subsection{LaTeX}
%%%%%%%%%%%%%


\<configure ooffice-mml latex\><<<
\:CheckOption{tab-eq}\if:Option 
    |<tab-based display equation|>
\else
    \Log:Note{for tab-based layoutof equation environment
          use the command line option `tab-eq'}
    |<tabular-based display equation|>
\fi
>>>


\<tab-based display equation\><<<
\Configure{equation}
  {\IgnorePar\ifvmode\else \HCode{<!--tex4ht:inline-->}\fi
   \EndP \HCode{<text:p text:style-name="equ-p-tab"><text:tab/>}%
   \Tabtrue
   |<disp math within equation|>%
   \IgnorePar}
  {\Tabfalse \ifvmode \IgnorePar\fi\EndP \HCode{<text:tab/>}}
  {\HCode{</text:p>}|<try inline par|>%
}
>>>

\<configure ooffice Preamble\><<<
\newif\ifTab \Tabfalse
>>>



\<tab-based display equation\><<<
\OOstyles{%
<style:style
     style:name="equ-p-tab"
     style:family="paragraph"
     style:parent-style-name="Standard">\Hnewline
  <style:paragraph-properties
     fo:text-align="center"
     style:justify-single-word="false">\Hnewline
     <style:tab-stops>    
        <style:tab-stop   
            style:position="7.5cm"   
            style:type="center"/>  
        <style:tab-stop   
            style:position="17cm"   
            style:type="right"/>  
     </style:tab-stops>\Hnewline   
  </style:paragraph-properties> 
</style:style>\Hnewline
}
>>>











\<tabular-based display equation\><<<
\Configure{equation}
  {\IgnorePar\ifvmode\else \HCode{<!--tex4ht:inline-->}\fi
   \EndP |<equation start table|>%
   |<disp math within equation|>%
   \IgnorePar}
  {\ifvmode \IgnorePar\fi\EndP
   |<equation table cells separator|>}
  {|<equation end table|>|<try inline par|>%
}
>>>


\<equation start table\><<<
\HCode{<table:table table:style-name="equation">%
       <table:table-column table:style-name="equ-col" />\Hnewline
       <table:table-column table:style-name="equ-num-col" />\Hnewline
       <table:table-row>%
       <table:table-cell table:style-name="equ-cell">%
       <text:p text:style-name="equ-p">}%
>>>




\<equation end table\><<<
\HCode{</text:p></table:table-cell>\Hnewline
          </table:table-row></table:table>}%
>>>





\<equation table cells separator\><<<
\HCode{</text:p></table:table-cell>\Hnewline
       <table:table-cell table:style-name="equ-num-cell">%
       <text:p text:style-name="equ-num-p">}%
>>>





\<configure ooffice-mml latex\><<<
\OOstyles{%
<style:style
     style:name="equ-p"
     style:family="paragraph"
     style:parent-style-name="Standard">\Hnewline
  <style:paragraph-properties 
     fo:text-align="center"
     style:justify-single-word="false"/>\Hnewline
</style:style>\Hnewline
}
\OOstyles{%
<style:style
     style:name="equ-num-p"
     style:family="paragraph"
     style:parent-style-name="Standard">\Hnewline
  <style:paragraph-properties 
     fo:text-align="end"
     style:justify-single-word="false"/>\Hnewline
</style:style>\Hnewline
}
\OOstyles{%
<style:style style:name="equation" 
             style:family="table">\Hnewline
     <\style:tableproperties
                 style:rel-width="88\%"
                 fo:margin-left="1cm"
                 fo:margin-right="0cm"
                 fo:margin-top="0.3cm"
                 fo:margin-bottom="0.3cm"
                 table:align="margins"/>\Hnewline
</style:style>\Hnewline
}
>>>

%%%%%%%%%%%%%
\subsection{AMS Non-Starred}
%%%%%%%%%%%%%


\<configure ooffice-mml amsmath\><<<
\:CheckOption{tab-eq}\if:Option 
    |<tab-based ams display equation|>
\else
    \Log:Note{for tab-based layout of equation environment
          use the command line option `tab-eq'}
    |<tabular-based ams display equation|>
\fi
>>>






\<tabular-based ams display equation\><<<
\Configure{equation}
  {\ifvmode \IgnorePar\fi\EndP
   |<equation start table|>% 
   |<amsmath equation july 2000|>%
   |<disp math within equation|>%
  }
  {\ifvmode \IgnorePar\fi\EndP |<equation table cells separator|>}
  {\ifvmode \IgnorePar\fi\EndP |<equation end table|>%
   |<try inline par|>}
>>>





\<tab-based ams display equation\><<<
\Configure{equation}
  {\IgnorePar\ifvmode\else \HCode{<!--tex4ht:inline-->}\fi
   \EndP \HCode{<text:p text:style-name="equ-p-tab"><text:tab/>}%
   \Tabtrue
   |<amsmath equation july 2000|>%
   \IgnorePar}
  {\Tabfalse \ifvmode \IgnorePar\fi\EndP \HCode{<text:tab/>}}
  {\HCode{</text:p>}|<try inline par|>%
}
>>>












\<disp math within equation\><<<
\Configure{$$}
   {\Configure{@math}{}\IgnorePar\EndP 
    \CenteredFrametrue \DviMath \CenteredFramefalse }
   {\EndDviMath\EndP}{}%
>>>




\<amsmath equation july 2000\><<<
\Configure{gather}
     {\HCode{<\a:mathml mtable\Hnewline
         class="equation"><\a:mathml mtr><\a:mathml mtd>}\IgnorePar 
         |<math display in equation|>}
     {\HCode{</\a:mathml mtd></\a:mathml mtr></\a:mathml mtable>}}
     {}
     {}
     {\ifnum\HCol=2 \IgnorePar
         \HCode{</\a:mathml mtd><\a:mathml mtd>}\fi }
     {}%
>>>



%%%%%%%%%%%%%
\subsection{AMS Starred}
%%%%%%%%%%%%%



\<configure ooffice-mml amsmath\><<<
\Configure{equation*}
  {\ifvmode \IgnorePar\fi \EndP
   |<equation-star start table|>%
   |<disp math within equation|>%
   \IgnorePar
  }
  {\ifvmode \IgnorePar\fi\EndP
   |<equation-star end table|>%
   \IgnoreIndent\par}
>>>


\<equation-star start table\><<<
\HCode{<text:p text:style-name="display-math">}%
>>>


\<equation-star end table\><<<
\HCode{</text:p>}%
>>>



\<configure ooffice-mml amsmath\><<<
\ConfigureEnv{align}
  {\ifmathml\else
     \ifvmode \IgnorePar\fi \EndP
     \IgnorePar
     \HCode{<text:p text:style-name="display-math">}%
   \fi
  }
  {\ifmathml\else
     \ifvmode \IgnorePar\fi\EndP
     \HCode{</text:p>}%
     \IgnoreIndent\par
   \fi
  }
  {} {}
>>>


\<configure ooffice-mml amsmath\><<<
\ConfigureEnv{align*}
  {\ifmathml\else
     \ifvmode \IgnorePar\fi \EndP
     \IgnorePar
     \HCode{<text:p text:style-name="display-math">}%
   \fi
  }
  {\ifmathml\else
     \ifvmode \IgnorePar\fi\EndP
     \HCode{</text:p>}%
     \IgnoreIndent\par
   \fi
  }
  {} {}
>>>



%%%%%%%%%%%%%%%%%%%
\section{algorithmicx}
%%%%%%%%%%%%%%%%%%%

\<configure ooffice algorithmicx\><<<
\ConfigureEnv{algorithmic}
   {\ifvmode \IgnorePar\fi \EndP
%    \HCode{<div \a:LRdir class="algorithmic">}%
    \bgroup
       \Configure{HtmlPar}
         {\EndP}
         {\EndP}
         {}
         {}%
   }
   {\egroup
%    \HCode{</div>}%
    \par\ShowPar 
   }
   {}{}
>>>

\<configure ooffice algorithmicx\><<<
\let\end:itm=\empty
\ConfigureList{algorithmic}%  
   {\EndP \end:itm  
    \SaveHtmlPar \HCode{<text:list  
      text:style-name="algorithmic|<cond rtl class|>"  
         text:name="algorithmic"\Hnewline>}%  
      \PushMacro\end:Itm
      \global\let\end:itm=\empty  
      \global\let\end:Itm=\empty  
     \Configure{HtmlPar}  
        {\end:itm\gdef\end:itm{\HCode{</text:p>}}%  
          \HCode{<text:p  
          text:style-name="dd|<cond rtl class|>" >}%
        }  
        {\end:itm\gdef\end:itm{\HCode{</text:p>}}%  
          \HCode{<text:p  
          text:style-name="dd|<cond rtl class|>" >}%
        }  
        {}  
        {}%  
   }  
   {\end:itm \end:Itm \global\let\end:itm=\empty  
    \PopMacro\end:Itm \global\let\end:Itm\end:Itm
    \HCode{</text:list>}\RecallHtmlPar  \ShowPar
    \special{t4ht@.}%
   }  
   {\end:itm \end:Itm \gdef\end:Itm{\HCode{</text:list-item>}}% 
      \gdef\end:itm{\HCode{</text:h>}\special{t4ht@.}}\HCode{<text:list-item><text:h  
      text:style-name="algorithmic-dt|<cond rtl class|>"\Hnewline >}\bgroup \bf
     \special{t4ht@. }%
   }  
   {\special{t4ht@.}%
    \egroup 
    \special{t4ht@. }%
   } 
>>>







\<\><<<
\ConfigureList{algorithmic}%
   {\EndP
    \PushMacro\end:itm
    \global\let\end:itm=\empty}
   {\PopMacro\end:itm \global\let\end:itm \end:itm
    \EndP
    \special{t4ht@.}%
    \HCode{</span>}%
   }
   {\end:itm
    \HCode{\ifx \end:itm\empty\else <br class="algorithmic"/>\fi
            <span class="label-\the\labelwidth">}%
    \global\def\end:itm{\EndP\special{t4ht@,}\Tg</span>}%
    \ifnum \c@ALG@line=10
        \Css{.algorithmic span.label{width:3em; text-align:right;}}\fi
   }
   {\HCode{</span>&\#xA0;<span\Hnewline class="algorithmic">}%
    \special{t4ht@. }}
\Css{div.algorithmic span.algorithmic{white-space:pre}}
>>>





\<configure ooffice latex\><<<
\NewConfigureOO{algorithmic}
\ConfigureOO{algorithmic}{|<algorithmic style|>}
>>>




\<algorithmic style\><<<
<text:list-style style:name="algorithmic" >
  <text:list-level-style-bullet
       text:level="1"
       text:style-name="Source-Text"
       text:bullet-char=" " >\Hnewline
     <style:list-level-properties
       text:space-before="0in" 
       text:min-label-width="0in"
       text:min-label-distance="0in" />\Hnewline
  </text:list-level-style-bullet>
  <text:list-level-style-bullet
       text:level="2"
       text:style-name="Source-Text"
       text:bullet-char=" " >\Hnewline
    <style:list-level-properties
       text:space-before="0.2in" 
       text:min-label-width="0.2in"
       text:min-label-distance="0.05in"
     />\Hnewline
  </text:list-level-style-bullet>
  <text:list-level-style-bullet
       text:level="3"
       text:style-name="Source-Text"
       text:bullet-char=" " >\Hnewline
    <style:list-level-properties
       text:space-before="0.4in"
       text:min-label-width="0.2in"
       text:min-label-distance="0.05in"/>\Hnewline
  </text:list-level-style-bullet>
  <text:list-level-style-bullet
       text:level="4"
       text:style-name="Source-Text"
       text:bullet-char=" " >\Hnewline
    <style:list-level-properties
       text:space-before="0.6in"
       text:min-label-width="0.2in"
       text:min-label-distance="0.05in"/>\Hnewline 
  </text:list-level-style-bullet>
</text:list-style>
>>>


\<algorithmic style\><<<
<style:style 
    style:name="algorithmic-dd" 
    style:family="paragraph" 
    style:parent-style-name="Standard"> 
 <style:paragraph-properties  
    fo:margin-left="0.35in" 
    fo:margin-right="0in"  
    fo:margin-top="0cm" 
    fo:margin-bottom="0.2cm" 
    fo:text-indent="0in"  
    style:auto-text-indent="false"/> 
  <style:text-properties
       fo:wrap-option="no-wrap"
  />
</style:style> 
<style:style
       style:name="algorithmic-dt"
       style:family="paragraph"
       style:parent-style-name="">
  <style:paragraph-properties
       fo:margin-left="0.1in"
       fo:margin-right="0in"
       fo:text-indent="0in"
       style:auto-text-indent="false"/>
  <style:text-properties
       fo:wrap-option="no-wrap"
  />
</style:style>
>>>




%%%%%%%%%%%%%%%%%%%%%%%%%%%%%%%
\section{Links}
%%%%%%%%%%%%%%%%%%%%%%%%%%%%%%%

%%%%%%%%%%%%%
\subsection{Cross References}
%%%%%%%%%%%%%


\url{http://www.linuxjournal.com/article/8112}

\begin{verbatim}
<text:reference-mark-start text:name="xx"/>
is 
<text:reference-mark-end text:name="xx"/>

<text:reference-ref text:reference-format="direction" text:ref-name="xx">
above
</text:reference-ref> 
\end{verbatim}



An approximation is provided here, to be cleaned by xtpipes.

\<configure ooffice latex\><<< 
\Configure{Link}{t4htlink}{href=}{name=}{}%
\Configure{ref}  
   {\Link}  
   {\ifx \RefArg\:UnDef\else  
      \Tg<!--tex4ht:ref: \RefArg\space-->\fi\EndLink}  
   {}  
\def\NOHREF:#1{\space  href="\:sharp#1"}%
>>>



\<configure ooffice latex\><<< 
\:CheckOption{hidden-ref}\if:Option 
   \OOstyles{%
   <style:style style:name="reference-ref" style:family="text">
       <style:text-properties 
           style:text-scale = "0\%"
       />
   </style:style>
   }
\else
   \Log:Note{to hide clickable index and bibliography references 
          use the command line option `hidden-ref'}
   \OOstyles{%
   <style:style style:name="reference-ref" style:family="text">
       <style:text-properties 
           style:text-underline-style="solid" 
           style:text-underline-width="thick" 
           style:text-underline-height="0.2em" 
           style:text-underline-color="\#0000FF"
       />
   </style:style>
   }
\fi
>>>

\<configure ooffice latex \><<< 
\Configure{Link}{text:a}{ xlink:href=}
                        { office:name=}{}%
\def\NOHREF:#1{\space  xlink:href="\:sharp#1"}%
>>>


% \Configure{Link}{text:span}{xlink:href=}{xlink:label=}{}




%%%%%%%%%%%%%%%%%%%
\subsection{url.sty}
%%%%%%%%%%%%%%%%%%%

\<configure ooffice url\><<<
\:CheckOption{url-il2-pl} \if:Option
\else \:CheckOption{url-enc} \fi
\if:Option
   \Configure{string-encoder}{url-encoder}
%   \:CheckOption{url-il2-pl} \if:Option
      |<il2-pl.tcx url encoder|>
%   \else      
      |<default url encoder WAIT|>
%      \Log:Note{for il2-pl url encoding,
%                        use the command line option 'url-il2-pl'}
%   \fi
   \Configure{url}%
     {|<:temp = url encoded string|>%
      \Link[\noexpand\the\tmp:toks \space
               class="url"]{}{}{\let\UrlBigBreaks\empty
                         \let\UrlBreaks\empty #1}\EndLink}
   |<url encoded string utilities|>
\else
   \:CheckOption{url-} \if:Option
      \Configure{url}%
        {{\let\UrlBigBreaks\empty \let\UrlBreaks\empty #1}}%
   \else
      \Log:Note{for URL encoding within href
             use the command line option `url-enc'; for hyperlink-less
             entry use the command line option `url-'}
      \Configure{url}%
        {\tmp:toks{#1}\Link[\noexpand\the\tmp:toks\space
                  class="url"]{}{}{\let\UrlBigBreaks\empty
                            \let\UrlBreaks\empty #1}\EndLink}
   \fi
\fi
|<path config|>
>>>

\<:temp = url encoded string\><<<
\tmp:toks={}\set:spaces#1|<par del|>%
\edef\:temp{\tmp:toks{\csname url-encoder\expandafter\endcsname
                      \the\tmp:toks |<par del|>}}\:temp
>>>

\<url encoded string utilities\><<<
\def\set:spaces{\futurelet\:temp\set:spacesoo}
\def\set:spacesoo{%
   \expandafter\ifx \space\:temp 
      \tmp:toks\expandafter{\the\tmp:toks{ }}%
   \fi    
   \set:spaceso
} 
\def\set:spaceso#1#2|<par del|>{%
   \tmp:toks\expandafter{\the\tmp:toks #1}%
   \if :#2:\else \set:spaces#2|<par del|>\fi
}
>>>

\<default url encoder\><<<
\Configure{url-encoder}
     {%}{%25}
     {&}{%26}
     {'}{%27}
     {<}{%3C}
     {>}{%3E}
     {_}{_}
     {^}{^}
     {}{}
\Log:Note{current url encoding setting: \string\Configure{url-encoder}%
     {\%}{\%25}%
     {&}{\%26}%
     {'}{\%27}%
     {<}{\%3C}%
     {>}{\%3E}%
     {\string _}{\string _}%
     {\string ^}{\string ^}%
     {}{}}
>>>




\<il2-pl.tcx url encoder\><<<
\input{html4-l2-url.4ht}
>>>


\Link[http://www.faqs.org/rfcs/rfc2396.html]{}{}http://www.faqs.org/rfcs/rfc2396.html\EndLink

See issues in:

\begin{verbatim}
\documentclass{article} 
  \usepackage{url} 
\begin{document} 
 
\url{a%b c&d} 
 
\url{a%25b+c&amp;d} 
 
\end{document} 
\end{verbatim}

\<\><<<
\expandafter\ifx\csname Url@OT1encSpecials\endcsname\relax
   |<url.sty|>  
\else
   |<url-2004.sty|>  
\fi
>>>

\<url-2004.styNO\><<<

>>>

\<url.styNO\><<<
\Configure{url}{\Link[#1 class="url"]{}{}{\let\UrlBigBreaks\empty
                   \let\UrlBreaks\empty #1}\EndLink}
>>>

%%%%%%%%%%%%%%%%%%
\section{Other}
%%%%%%%%%%%%%%%%%%





%%%%%%%%%%%%%%%%%%%%%%%%%%%%%%%
\subsection{??? Move to Unicode}
%%%%%%%%%%%%%%%%%%%%%%%%%%%%%%%


\<configure ooffice fontmath\><<<
\Configure{mathellipsis}
   {\HCode{\string&\#x2026;}}
>>>







%%%%%%%%%%%%%%%%%%%%%%%%%%%%%%%
\subsection{Math}
%%%%%%%%%%%%%%%%%%%%%%%%%%%%%%%



\<configure ooffice-mml latex\><<<
\Configure{[]}
   {\Configure{@math}{|<block display|>}\IgnorePar
    \ifvmode\EndP
    \else\EndP \HCode{<!--tex4ht:inline-->}\fi
    \HCode{<text:p text:style-name="display-math">}%
    \CenteredFrametrue \DviMath \CenteredFramefalse
    $$\DisplayMathtrue}
   {$$\EndDviMath\HCode{</text:p>}\ShowPar\par
    {\csname HCondtrue\endcsname\noindent}}
   {}
>>>




\<configure ooffice tex4ht\><<<
\def\mml:class#1"#2"{ }
>>>




\<oo2 mml tex4ht\><<<
\Configure{$$}
   {\Configure{@math}{|<block display|>}\IgnorePar
    \ifvmode\EndP
    \else\EndP \HCode{<!--tex4ht:inline-->}\fi
    \HCode{<text:p text:style-name="display-math">}%
     \CenteredFrametrue \DviMath \CenteredFramefalse }
   {\EndDviMath
     \HCode{</text:p>}
     \ShowPar\par
    {\csname HCondtrue\endcsname\noindent}}
   {\DisplayMathtrue}
>>>


\<configure ooffice-mml rlbabel\><<<
\Configure{[]}
   {\Configure{@math}{|<block display|>}\IgnorePar
    \ifvmode\EndP
    \else\EndP \HCode{<!--tex4ht:inline-->}\fi
    \HCode{<text:p text:style-name="display-math">}%
     \CenteredFrametrue \DviMath \CenteredFramefalse
    $$\DisplayMathtrue}
   {$$\EndDviMath\HCode{</text:p>}\ShowPar\par
    {\csname HCondtrue\endcsname\noindent}}
   {}
>>>



\<configure ooffice-mml tex4ht\><<<
\NewConfigureOO{display-math}
\ConfigureOO{display-math}
   {<style:style style:name="display-math" 
         style:family="paragraph" 
         style:parent-style-name="Text-body">\Hnewline
      <style:paragraph-properties  
         fo:text-align="center" 
         style:auto-text-indent="false"
         style:justify-single-word="false"/>\Hnewline
    </style:style>\Hnewline}
>>>






%%%%%%%%%%%%%%%%%%%%%%%%%%%%%%%
\subsection{Obeying lines and Verbatims}
%%%%%%%%%%%%%%%%%%%%%%%%%%%%%%%


\<configure ooffice latex\><<<
\Configure{verb}
   {\HCode{<text:span\Hnewline text:style-name="verb">}}
   {\HCode{</text:span>}}
\NewConfigureOO{verb}
\ConfigureOO{verb}{|<verb style|>}
>>>

\<configure ooffice latex\><<<
\Configure{obeylines}
   {\ifvmode \IgnorePar\EndP \IgnoreIndent
       \HCode{<text:p\Hnewline text:style-name="obeylines-v">}%
       \gdef\ob:ln{\ht:special{t4ht=</text:p>}}%
    \else
       \SaveEndP \HCode{<text:span\Hnewline
                               text:style-name="obeylines-h">}%
       \gdef\ob:ln{\ht:special{t4ht=</text:span>}\RecallEndP}%
    \fi
   }
   {\EndP\ob:ln\ShowPar}
   {\hbox{\HCode{<text:line-break/>}}}
\NewConfigureOO{obeylines}
\ConfigureOO{obeylines}{|<obeylines style|>}
>>>



\<configure ooffice latex\><<<
\ConfigureEnv{verbatim}
  {\ifinner\else \vfill\break\fi
   \IgnorePar\EndP  \HAssign\Verb:N=0
   \HCode{|<input line num|><text:p 
     text:style-name="Preformatted-Text|<cond rtl class|>">}\EndNoFonts
  }
  {\NoFonts \EndP \ht:special{t4ht=</text:p>}\ShowPar\par}
  {}{}
\Configure{verbatim}
  {\HAdvance\Verb:N by 1 \ifnum \Verb:N>1
     \HCode{<text:line-break/>}%
   \fi
  }
  {\HCode{<text:s/>}}
\NewConfigureOO{verbatim}
\ConfigureOO{verbatim}{|<verbatim style|>}
>>>




\<obeylines style\><<<
<style:style style:name="obeylines-h" style:family="text">\Hnewline
   <style:text-properties
           fo:wrap-option="no-wrap"
   />\Hnewline
</style:style>\Hnewline    
<style:style style:name="obeylines-v" style:family="text">\Hnewline
   <style:text-properties
           fo:wrap-option="no-wrap"
   />\Hnewline
</style:style>\Hnewline    
>>>



\<verb style\><<<
<style:style style:name="verb" style:family="text">\Hnewline
   <style:text-properties
       fo:font-family="monospace"
   />\Hnewline
</style:style>\Hnewline    
>>>



\<verbatim style\><<<
<style:style style:name="Preformatted-Text" 
             style:family="paragraph"
             style:parent-style-name="Standard">\Hnewline
   <style:paragraph-properties 
%      fo:font-family="monospace"
%      fo:wrap-option="no-wrap"
%      style:font-name="Courier"
      fo:margin-top="0.2cm" 
      fo:margin-bottom="0.2cm" 
      fo:text-indent="0cm"
   />\Hnewline
   <style:text-properties fo:font-family="monospace"/>\Hnewline    
</style:style>\Hnewline    
|<Preformatted Text rtl|>  %
>>>







\<W2L verbatim style\><<<
<style:style style:name="Preformatted-Text"  
             style:family="paragraph"  
             style:parent-style-name="Standard"  
             style:class="html"> 
<style:paragraph-properties  fo:margin-top="0cm"  
             fo:margin-bottom="0cm"  
             style:font-name="Cumberland2"  
             fo:font-size="12pt"  
             style:font-name-asian="Cumberland2"  
             style:font-size-asian="10pt"  
             style:font-name-complex="Cumberland2"  
             style:font-size-complex="10pt"/> 
</style:style>
>>>


















\<oo2 fonts \><<<
  <style:font-face
      style:font-name="Courier" 
      svg:font-family="Courier" 
      style:font-family-generic="modern"
      style:font-pitch="fixed"/>\Hnewline
>>>

 


% \begin{verbatim}
% fo:wrap-option is not supported by OOo
% 
% The option fo:font-family is not used for generic font names in OOo, you
% should use the combination
% fo:font-family="Courier"  (or some other suitable font name, not important)
% style:font-family-generic="modern"
% style:font-pitch="fixed"
% 
%    The `momospace' is recognized, and needed for the spaces.
%    Do we need the Courier?
% 
%   See:
% 
%        \ begin{verbatim}
%        1234
%        x  x
%        \ end{verbatim}
% 
% 
% I would recommend to declare the styles verb and verbatim as follows:
% 
% <style:font-decl style:name="Courier" fo:font-family="Courier"
% style:font-family-generic="modern" style:font-pitch="fixed"/>
% 
% <style:style style:name="verbatim" style:family="paragraph"
% style:parent-style-name="Standard">
% <style:paragraph-properties  style:font-name="Courier" fo:margin-top="0.2cm"
% fo:margin-bottom="0.2cm" fo:text-indent="0cm" />
% </style:style>
% 
% <style:style style:name="verb" style:family="text">
% <style:text-properties style:font-name="Courier" />
% </style:style>
% 
% OOo comes with built-in styles for this, which you could choose to use
% instead:
% "Preformatted-Text" and "Teletype"
% \end{verbatim}

%%%%%%%%%%%%%%%%%%%%%%%%%%%%%%%
\subsection{Standard Paragraphs}
%%%%%%%%%%%%%%%%%%%%%%%%%%%%%%%



\<configure ooffice tex4ht\><<<
\expandafter\ifx \csname if@rl\endcsname\relax  
   \expandafter \csname newif\expandafter\endcsname \csname if@rl\endcsname
   \@rlfalse
\fi
\HTML:PAR{\ifHCond Like-\fi Text-body|<cond rtl class|>}%
         {\ifdim \parindent=\z@ Text-body\else 
                                First-line-indent\fi |<cond rtl class|>}
>>>


% \HTML:PAR{p-\ifHCond nopar\else noindent\fi}%
%         {p-\ifdim \parindent=\z@ no\fi indent}


\<utilities\><<<
\def\HTML:PAR#1#2{%
  \Configure{HtmlPar}%
     {\EndP\HCode{|<input line num|><text:p text:style-name="#1">}}%
     {\EndP\HCode{|<input line num|><text:p text:style-name="#2">}}%
     {\HCode{</text:p>\Hnewline}}%
     {\HCode{</text:p>\Hnewline}}}
>>>

\<input line num\><<<
<!--l. \the\inputlineno-->%
>>>





It seems that the \verb+auto-text-indent="true"+ setting does not
export to rtf or ms word doc successfully from OOo at present. The
style p-indent thus displays as non-indented when the file is
exported from Ooo. Hence, it is replaced with 

\begin{verbatim}
fo:text-indent="0.4cm"
style:auto-text-indent="false"
\end{verbatim}


\<configure ooffice tex4ht\><<<
\NewConfigureOO{p-indent}
\ConfigureOO{p-indent}{%
<style:style style:name="First-line-indent"
                  style:family="paragraph"
                  style:parent-style-name="Text-body"
                  style:class="text">\Hnewline
<style:paragraph-properties  fo:margin-left="0cm"
                  fo:margin-right="0cm"
                  fo:text-indent="0.499cm"
                  style:auto-text-indent="false"/>
</style:style>\Hnewline |<First-line-indent rtl|> }
>>>



\<configure ooffice tex4ht\><<<
\NewConfigureOO{p-noindent}
\ConfigureOO{p-noindent}{%
<style:style style:name="Text-body"
                  style:family="paragraph"
                  style:parent-style-name="Standard"
                  style:next-style-name="First-line-indent"
                  style:class="text"
                  style:master-page-name="">\Hnewline
<style:paragraph-properties  fo:margin-top="0cm"
                  fo:margin-bottom="0cm"
                  fo:line-height="120\%"
                  fo:text-align="justify"
                  style:justify-single-word="false"
                  fo:hyphenate="true"
                  fo:hyphenation-remain-char-count="2"
                  fo:hyphenation-push-char-count="2"
                  fo:hyphenation-ladder-count="no-limit"
                  style:page-number="0"/>
</style:style>
\Hnewline
<style:style style:name="Like-Text-body"
                  style:family="paragraph"
                  style:parent-style-name="Standard"
                  style:next-style-name="First-line-indent"
                  style:class="text"
                  style:master-page-name="">\Hnewline
<style:paragraph-properties  fo:margin-top="0cm"
                  fo:margin-bottom="0cm"
                  fo:line-height="120\%"
                  fo:text-align="justify"
                  style:justify-single-word="false"
                  fo:hyphenate="true"
                  fo:hyphenation-remain-char-count="2"
                  fo:hyphenation-push-char-count="2"
                  fo:hyphenation-ladder-count="no-limit"
                  style:page-number="0"/>
</style:style>
\Hnewline  |<Text-body rtl|>}
>>>








\<configure ooffice tex4ht\><<<
\NewConfigureOO{p-nopar}
\ConfigureOO{p-nopar}
   {<style:style style:name="p-nopar" 
         style:family="paragraph" 
         style:parent-style-name="Text-body">\Hnewline
      <style:paragraph-properties  
         fo:text-align="justify" 
         style:auto-text-indent="false"
         fo:margin-top="0.3cm" 
         fo:margin-bottom="0.3cm"
         style:justify-single-word="false"/>\Hnewline
    </style:style>\Hnewline  |<p-nopar rtl|> }
>>>

%%%%%%%%%%%%%
\subsection{Page Layout}
%%%%%%%%%%%%%

The following doesn't catch because the style name is not in use.  Is
there a way to get styles to be specifies as a default without
referring to names.

\<oo2 automatic styles\><<<
\Configure{automatic-styles}
   {\csname a:page-layout\endcsname}
\NewConfigure{page-layout}{1}
\Configure{page-layout}{%
   <style:page-layout style:name="Standard-page-layout">\Hnewline
     <style:page-layout-properties
         fo:page-width="\the\paperwidth" |%"8.5in"|%
         fo:page-height="\the\paperheight" |%"11in"|%
         style:num-format="1"
         style:print-orientation="portrait"
         fo:margin-top="0.7874in"
         fo:margin-bottom="0.7874in"
         fo:margin-left="0.7874in"
         fo:margin-right="0.7874in"
         style:writing-mode="lr-tb"
         style:footnote-max-height="0in">\Hnewline
       <style:footnote-sep
         style:width="0.0071in"
         style:distance-before-sep="0.0398in"
         style:distance-after-sep="0.0398in"
         style:adjustment="left"
         style:rel-width="25\%"
         style:color="\#000000"/>\Hnewline
     </style:page-layout-properties>\Hnewline
     <style:header-style/>
     <style:footer-style/>
   </style:page-layout>\Hnewline
}
>>>

%   <style:style style:name="P1"
%              style:family="paragraph"
%              style:parent-style-name="Text-body">
%      <style:paragraph-properties fo:line-height="0.2201in"/>
%    </style:style>

\<styles.xml: master-styles\><<<
<style:master-page
          style:name="Standard"
          style:page-layout-name="Standard-page-layout"/>\Hnewline
>>>

\<styles.xml: automatic-styles\><<<
<style:page-layout style:name="Standard-page-layout"> 
   <style:page-layout-properties
      fo:page-width="\the\paperwidth" |%"8.5in"|%
      fo:page-height="\the\paperheight" |%"11in"|%
      fo:margin-left="1.0in"  
      fo:margin-right="1.0in"  
      style:print-orientation="portrait"  
    > 
    </style:page-layout-properties> 
</style:page-layout> 
>>>


%%%%%%%%%%%%%%%%%%%%%%%%%%%%%%%
\subsection{Captions of Figures}
%%%%%%%%%%%%%%%%%%%%%%%%%%%%%%%

\<ooffice captions\><<<
\Configure{caption}
   {\ht:everypar{\HtmlPar}%
    \ifvmode \EndP \par\ShowPar\else \HCode{<text:line-break/>}\fi 
    \HCode{\Hnewline  <text:span text:style-name="caption" >%
           <text:span text:style-name="caption-title" >}}
   {: }
   {\HCode{</text:span>}}
   {\HCode{</text:span>}}
\OOstyles{%
  <style:style style:name="caption-title" style:family="text">\Hnewline
     <style:text-properties fo:font-style="normal"
                       fo:font-weight="bold"/>\Hnewline
  </style:style>\Hnewline
}
>>>





%%%%%%%%%%%%%%%%%%%%%%%%%%%%%%%
\subsection{Vertical Spaces}
%%%%%%%%%%%%%%%%%%%%%%%%%%%%%%%

\<big skip\><<<
<text:p text:style-name="bigskip"></text:p>%
>>>


\<configure ooffice tex4ht\><<<
\OOstyles{%
  <style:style style:name="bigskip" style:family="paragraph"
         style:parent-style-name="Text-body">
    <style:paragraph-properties  
       fo:margin-top="0.2cm" 
       fo:margin-bottom="0.0cm" 
    />
  </style:style>
}
>>>

%%%%%%%%%%%%%%%%%%%%%%%%%%%%%%%
\subsection{Page Breaks}
%%%%%%%%%%%%%%%%%%%%%%%%%%%%%%%



\<configure ooffice tex4ht\><<<
\OOstyles{%
  <style:style style:name="clearpage" style:family="paragraph"
      style:parent-style-name="Text-body">
      <style:paragraph-properties  fo:break-before="page" />
  </style:style>
}
\def\Hclearpage{%
  \par \ifvmode
     \IgnorePar\EndP
     \HCode{|<new page|>}%
     \ShowPar\par
  \fi
}

>>>

\<new page\><<<
<text:p text:style-name="clearpage"></text:p>%
>>>



%%%%%%%%%%%%%%%%%%%%%%%%%%%%%%%
\subsection{Footnotes}
%%%%%%%%%%%%%%%%%%%%%%%%%%%%%%%


\<configure ooffice latex\><<<
|<footnote mark|>
|<footnote text|>
|<footnote thank|>
>>>

\<footnote text\><<<
\Configure{footnotetext}
   {\special{t4ht@)}\Tag{|<footnote id|>-tx}{}%
    |<check duplicate footnote text|>%
    \SaveEndP \NoFonts  \let\@textsuperscript\hbox
    \HCode{<\text:footnote\Hnewline\text:note:class
             text:id="}|<footnote id|>%
               \HCode{"><\text:footnote-citation>}%
    \IgnorePar \special{t4ht@(}%
   }
   {\EndNoFonts \RecallEndP
    \HCode{</\text:footnote-citation><\text:footnote-body>
    <text:p\Hnewline text:style-name="Footnote">}\NoFonts
%     |<footnotetext space|>%
    \SaveHtmlPar |<footnote pars|>\ShowPar \special{t4ht@)}%
   }
   {\RecallHtmlPar\EndNoFonts
     \HCode{</text:p></\text:footnote-body></\text:footnote>}%
    |<notice footnote text|>%
   }
>>>


\<footnotetext space\><<<
%<text:s/>
>>>

\<configure ooffice latex\><<<
\NewConfigureOO{Footnote}
\ConfigureOO{Footnote}{
<style:style 
    style:name="Footnote" 
    style:family="paragraph" 
    style:parent-style-name="Standard" 
    style:class="extra">\Hnewline
  <style:paragraph-properties  
             fo:margin-left="0.2in" 
             fo:margin-right="0in"  
             fo:text-indent="-0.2in" 
              style:auto-text-indent="false"/> 
</style:style>\Hnewline
}
>>>






%    \Odef\Link[##1]##2##3{\global\let\cur:EndLink\EndLink
%         \gdef\EndLink{\global\let\EndLink=\cur:EndLink}}


\<footnote pars\><<<
\Configure{HtmlPar}
   {\EndP \HCode{<text:p text:style-name="Footnote">}}
   {\EndP \HCode{<text:p text:style-name="Footnote">}}
   {\HCode{</text:p>}}
   {\HCode{</text:p>}}%
>>>




\<footnote mark\><<<
 \ifx \tmp:bx\:UnDef \csname newbox\endcsname \tmp:bx \fi  
\Configure{footnotemark}
   {\Tag{|<footnote id|>}{}%
    |<footnote mark without text|>}
   {|<end footnote mark without text|>%
   \special{t4ht@(}}
>>>

\<check duplicate footnote text\><<<
\ifTag{|<footnote id|>}
   {\expandafter\ifx\csname (|<footnote id|>)\endcsname\def
       \:warning{Multiple footnote texts for mark \FNnum}%
    \fi
   } {}%
>>>


\<notice footnote text\><<<
\expandafter\global\expandafter
   \let\csname (|<footnote id|>)\endcsname\def
>>>    


\<footnote mark without text\><<<
\ifTag{|<footnote id|>-tx}
   {\setbox\tmp:bx=\hbox\bgroup}
   {\NoFonts
    \HCode{<\text:footnote\Hnewline\text:note:class
             text:id="}|<footnote id|>%
               \HCode{"><\text:footnote-citation>}%
   }%
>>>

\<end footnote mark without text\><<<
\ifTag{|<footnote id|>-tx}
   {\egroup}
   {\EndNoFonts \HCode{</\text:footnote-citation><\text:footnote-body>
      <text:p\Hnewline text:style-name="Footnote">%
      </text:p></\text:footnote-body></\text:footnote>}%
   }%
>>>


\<footnote id\><<<
fn\FNnum x\minipageNum
>>>

\<thank id\><<<
fn\the\c@footnote x\minipageNum
>>>



\<configure ooffice footmisc\><<<
\Configure{footnotetext}
   {\Tag{|<footnote id|>-tx}{}%
    |<check duplicate footnote text|>%
    \NoFonts  \let\@textsuperscript\hbox
    \HCode{<\text:footnote\Hnewline\text:note:class
            text:id="}|<footnote id|>%
               \HCode{"><\text:footnote-citation>}%
    \let\fn:nbsp\:nbsp \let\:nbsp\empty  
   }
   {\EndNoFonts \let\:nbsp\fn:nbsp
    \HCode{</\text:footnote-citation><\text:footnote-body>%
    <text:p\Hnewline text:style-name="Footnote">|<footnotetext space|>}%
    \SaveHtmlPar  \Configure{HtmlPar}{}{}{}{}%
   }
   {\RecallHtmlPar  
    \HCode{</text:p></\text:footnote-body></\text:footnote>}%
    |<notice footnote text|>}
>>>




\<configure ooffice latex\><<<
\Configure{textsuperscript}
  {\HCode{<text:span text:style-name="textsuperscript">}}
  {\HCode{</text:span>}} 
\OOstyles{%
<style:style style:name="textsuperscript" style:family="text">\Hnewline
   <style:text-properties style:text-position="super 58\%"/>\Hnewline
</style:style>\Hnewline
}
\Configure{textsubscript}
  {\HCode{<text:span text:style-name="textsubscript">}}
  {\HCode{</text:span>}} 
\OOstyles{%
<style:style style:name="textsubscript" style:family="text">\Hnewline
   <style:text-properties style:text-position="sub 58\%"/>\Hnewline
</style:style>\Hnewline
}
>>>


%%%%%%%%%%%%%%%%%%%%%%%%%%%%%%%
\subsection{Etc}
%%%%%%%%%%%%%%%%%%%%%%%%%%%%%%%

\<configure ooffice latex\><<<
\Configure{centercr}
   {\ht:special{t4ht=<text:line-break/>}}
   {\ht:everypar{\ht:everypar{\HtmlPar}}}
>>>


 
\<configure ooffice latex\><<<
\def\:temp{T\HCode{<text:span text:style-name="TEX">}E\HCode{</text:span>}X}
\HLet\TeX\:temp
\OOstyles{\Hnewline 
    <style:style style:name="TEX" style:family="text"> 
      <style:text-properties style:text-position="-25 100\%"/> 
    </style:style> 
\Hnewline} 
>>>

\<configure ooffice latex\><<<
\def\:temp{L\HCode{<text:span text:style-name="LATEX">}A\HCode{</text:span>}\TeX}
\HLet\LaTeX\:temp
\OOstyles{\Hnewline 
    <style:style style:name="LATEX" style:family="text"> 
      <style:text-properties style:text-position="15 70\%"/> 
    </style:style> 
\Hnewline} 
>>>


\<configure ooffice-mml latex\><<<
\Configure{newline}
  {\ifmtext |<end mtext|>\HCode{</\a:mathml mtext><\a:mathml 
                                        mtext class="newline">}%
            |<start mtext|>%
   \else \HCode{<text:line-break\Hnewline/>}\fi}
>>>



 \<start mtext\><<<
\ht:special{t4ht@,&\#x00A0;}%
>>>

\<end mtext\><<<
\ht:special{t4ht@,}%
>>>
 




\<configure ooffice amsmath\><<<
\def\:temp{{\protect\AmSfont AMS}}
\HLet\AmS\:temp
>>>




%%%%%%%%%%%%%%%%%%%%%%%%%%%%%
\chapter{Classes}
%%%%%%%%%%%%%%%%%%%%%%%%%%%%


%%%%%%%%%%%%%%%%%%%%%%%%%%%%%
\section{Main Classes}
%%%%%%%%%%%%%%%%%%%%%%%%%%%%


\<configure ooffice article\><<<
|<ooffice title page|>
|<content styles|>
|<non chap tocs|>
|<ooffice bib with refname|>
|<ooffice begin-end env|>
|<basic ooffice sections|>
|<basic ooffice section styles|>
|<ooffice captions|>
|<ooffice index|>
|<non-chapter index|>
|<config lists of figures and tables|>
|<base font size|>
>>>

\<configure ooffice report\><<<
|<ooffice title page|>
|<content styles|>
|<chap tocs|>
|<ooffice bib with bibname|>
|<ooffice begin-end env|>
|<ooffice chapters|>
|<report chapters|>
|<basic ooffice sections|>
|<basic ooffice section styles|>
|<ooffice captions|>
|<ooffice index|>
|<chapter index|>
|<config lists of figures and tables|>
|<base font size|>
>>>

\<configure ooffice book\><<<
|<ooffice title page|>
|<content styles|>
|<chap tocs|>
|<ooffice bib with bibname|>
|<ooffice begin-end env|>
|<ooffice chapters|>
|<book chapters|>
|<basic ooffice sections|>
|<basic ooffice section styles|>
|<ooffice captions|>
|<ooffice index|>
|<chapter index|>
|<config lists of figures and tables|>
|<base font size|>
>>>



\<base font size\><<<
{\Configure{Needs}{Font\string_Size: #1}\ifcase  \@ptsize
   \or \Needs{11}\or \Needs{12}\else \fi}
>>>


\<configure ooffice scrbook\><<<
|<ooffice title page|>
|<content styles|>
|<chap tocs|>
|<scr chap tocs|>
|<ooffice bib with refname|>
|<ooffice begin-end env|>
|<scr ooffice chapters|>
|<basic scr ooffice sections|>
|<basic ooffice section styles|>
|<ooffice captions|>
|<ooffice index|>
|<chapter index|>
|<scrbook end units|>
|<config lists of figures and tables|>
|<base font size|>
>>>

\<configure ooffice scrreprt\><<<
|<ooffice title page|>
|<content styles|>
|<chap tocs|>
|<scr chap tocs|>
|<ooffice bib with bibname|>
|<ooffice begin-end env|>
|<scr ooffice chapters|>
|<basic scr ooffice sections|>
|<basic ooffice section styles|>
|<ooffice captions|>
|<ooffice index|>
|<chapter index|>
|<config lists of figures and tables|>
|<base font size|>
>>>

\<configure ooffice scrartcl\><<<
|<ooffice title page|>
|<content styles|>
|<ooffice bib with bibname|>
|<ooffice begin-end env|>
|<basic scr ooffice sections|>
|<basic ooffice section styles|>
|<ooffice captions|>
|<ooffice index|>
|<base font size|>
>>>



\<configure ooffice amsart\><<<
|<ooffice title page|>
|<content styles|>
|<non chap tocs|>
|<ooffice captions|>
|<basic ams ooffice sections|>
|<basic ooffice section styles|>
|<ooffice bib with bibname|>
|<base font size|>
>>>


\<configure ooffice amsproc\><<<
|<base font size|>
>>>

\<configure ooffice amsbook\><<<
|<base font size|>
>>>



\<configure ooffice slides\><<<
\Configure{maketitle}
   {\HCode{<!--begin maketitle-->}}
   {\HCode{<!--end maketitle-->}\par}
   {\ifvmode \IgnorePar\fi \EndP
    \HCode{<text:p text:style-name="Title|<cond rtl class|>" >}}
   {\HCode{</text:p>}}
\Configure{author date}
   {\ifvmode \IgnorePar\fi \EndP
    \HCode{<text:p text:style-name="author|<cond rtl class|>">}}
   {\HCode{</text:p>}}
   {\ifvmode \IgnorePar\fi \EndP
    \HCode{<text:p text:style-name="date|<cond rtl class|>">}}
   {\HCode{</text:p>}}
\NewConfigureOO{title}
\ConfigureOO{title}{%
<style:style style:name="Title"
             style:family="paragraph"
             style:parent-style-name="Heading"
             style:next-style-name="author"
             style:class="chapter">\Hnewline
<style:paragraph-properties  fo:margin-top="0.42cm"
                  fo:margin-bottom="0.51cm"
                  fo:font-size="20pt"
                  fo:font-weight="normal"
                  style:font-size-asian="18pt"
                  style:font-weight-asian="bold"
                  style:font-size-complex="18pt"
                  style:font-weight-complex="bold"
                  fo:text-align="center"
                  style:justify-single-word="false"/>
</style:style> \Hnewline <style:style style:name="Title-rtl"
             style:family="paragraph"
             style:parent-style-name="Heading-rtl"
             style:next-style-name="author-rtl"
             style:class="chapter">\Hnewline
<style:paragraph-properties  fo:margin-top="0.42cm"
                  fo:margin-bottom="0.51cm"
                  fo:font-size="20pt"
                  fo:font-weight="normal"
                  style:font-size-asian="18pt"
                  style:font-weight-asian="bold"
                  style:font-size-complex="18pt"
                  style:font-weight-complex="bold"
                  fo:text-align="center"
                  style:justify-single-word="false"
                  style:writing-mode="rl-tb" />
</style:style> \Hnewline
 }
\NewConfigureOO{author}
\ConfigureOO{author}{%
<style:style      style:name="author"
                  style:family="paragraph"
                  style:parent-style-name="Text-body"
                  style:next-style-name="date">\Hnewline
<style:paragraph-properties  fo:margin-top="0cm"
                  fo:margin-bottom="0.199cm"
                  fo:font-size="14pt"
                  fo:text-align="center"
                  style:justify-single-word="false"/>
</style:style>\Hnewline <style:style      style:name="author-rtl"
                  style:family="paragraph"
                  style:parent-style-name="Text-body-rtl"
                  style:next-style-name="date-rtl">\Hnewline
<style:paragraph-properties  fo:margin-top="0cm"
                  fo:margin-bottom="0.199cm"
                  fo:font-size="14pt"
                  fo:text-align="center"
                  style:justify-single-word="false"
                  style:writing-mode="rl-tb" />
</style:style>\Hnewline
}
\NewConfigureOO{date}
\ConfigureOO{date}{%
<style:style style:name="date"
                  style:family="paragraph"
                  style:parent-style-name="Text-body"
                  style:next-style-name="Text-body">\Hnewline
<style:paragraph-properties  fo:margin-top="0cm"
                  fo:margin-bottom="0.199cm"
                  fo:font-size="14pt"
                  fo:text-align="center"
                  style:justify-single-word="false"/>
</style:style>\Hnewline <style:style style:name="date-rtl"
                  style:family="paragraph"
                  style:parent-style-name="Text-body"
                  style:next-style-name="Text-body">\Hnewline
<style:paragraph-properties  fo:margin-top="0cm"
                  fo:margin-bottom="0.199cm"
                  fo:font-size="14pt"
                  fo:text-align="center"
                  style:justify-single-word="false"
                  style:writing-mode="rl-tb" />
</style:style>\Hnewline
}
>>>








%%%%%%%%%%%%%%%%%%%%%%%%%%%%%
\section{g-brief}
%%%%%%%%%%%%%%%%%%%%%%%%%%%%



\<configure ooffice g-brief\><<<
\Configure{g-brief}
   {\ifvmode \IgnorePar\fi \EndP |<open g-brief|>}
   {\ifvmode \IgnorePar\fi \EndP |<end open g-brief|>}
   {\ifvmode \IgnorePar\fi \EndP \par}
   {\ifvmode \IgnorePar\fi \EndP \par}
>>>


\<open g-brief\><<<
\let\sv:br =\\%
\let\sv:@imakebox=\@imakebox
\def\\{\HCode{</table:table-row>\Hnewline<table:table-row>}}%
\let\:clearpage\clearpage
\def\clearpage{\let\clearpage\:clearpage
   \ifvmode \IgnorePar\fi \EndP
   \HCode{<table:table table:style-name="date">
       <table:table-column/><table:table-column/>
       <table:table-column/><table:table-column/>
       <table:table-row>\Hnewline}%
   \IgnorePar
}%
\def\@imakebox[##1][##2]##3{%
  \HCode{<table:table-cell>\Hnewline
          <text:p text:style-name="makebox-##2">}%
  \bgroup \let\\=\sv:br \let\@imakebox\sv:@imakebox
          \let\par\sv:par
          ##3%
  \egroup
  \HCode{</text:p></table:table-cell>}%
  \def\par{\let\par\brf:par
     \ifvmode \IgnorePar\fi \EndP
     \HCode{</table:table-row></table:table>}}%
}%
>>>



\<end open g-brief\><<<
\let\@imakebox\sv:@imakebox
\let\\=\sv:br
>>>

\<configure ooffice g-brief\><<<
\OOstyles{%
<style:style style:name="makebox-r" \Hnewline
           style:family="paragraph" >\Hnewline
    <style:paragraph-properties   fo:text-align="end" 
           style:justify-single-word="false" />\Hnewline
</style:style>\Hnewline
}
>>>



\<open g-brief\><<<
\let\brf:par\par
\HTML:PAR{p-g-brief}{p-g-brief}%
>>>

\<end open g-brief\><<<
\HTML:PAR{p-\ifHCond nopar\else noindent\fi}%
         {p-\ifdim \parindent=\z@ no\fi indent}%
>>>


\<configure ooffice g-brief\><<<
\OOstyles{%
 <style:style style:name="p-g-brief"
         style:family="paragraph"
         style:parent-style-name="Text-body">\Hnewline
    <style:paragraph-properties 
       fo:text-align="justify"
       style:auto-text-indent="false"
       style:justify-single-word="false"
       fo:margin-top="0.7cm" 
       fo:margin-bottom="0.7cm"
    />\Hnewline
 </style:style>\Hnewline
}
>>>



\<configure ooffice g-briefNO\><<<
\Configure{letterfoot}
   {\ifvmode \IgnorePar\fi \EndP
    \bgroup
        \Configure{halignTB<>}{tabular}{\HCode{\Hnewline
             table:style-name="letterfoot"
             }<>\HAlign}%
        \Configure{table-column-style}
             {table:style-name="letterfoot\ar:cnt"}%
   }
   {}
   {\egroup} 
\OOstyles{%
<style:style style:name="letterfoot" 
             style:family="table">\Hnewline
<style:text-properties
                 fo:margin-top="2cm"
                 table:align="margins"/>\Hnewline
</style:style>\Hnewline
} 
\OOstyles{%
<style:style style:name="letterfoot1" 
             style:family="table-column">\Hnewline
<style:text-properties style:column-width="3cm"
style:rel-column-width="3030*"
    />\Hnewline
</style:style>\Hnewline
} 
>>>


The above styles don't work?




\<configure ooffice g-brief\><<<
\Configure{letterfoot}
   {\ifvmode \IgnorePar\fi \EndP
    \HCode{<text:p text:style-name="letterfoot">}%
    \bgroup
        \Configure{tabular}
           {}{} {}{\HCode{<text:line-break/>}} {}{~~}%
   }
   {}
   {\egroup \HCode{</text:p>}} 
\OOstyles{%
   {<style:style style:name="letterfoot" 
         style:family="paragraph" 
         style:parent-style-name="Text-body">\Hnewline
      <style:paragraph-properties  
         fo:text-align="justify" 
         style:auto-text-indent="false"
         fo:margin-top="2cm" 
         fo:margin-bottom="0cm"
         style:justify-single-word="false"/>\Hnewline
    </style:style>\Hnewline}
} 
>>>


The above styles don't work?






\<configure ooffice g-brief\><<<
\Configure{letterhead}
   {\ifvmode \IgnorePar\fi \EndP 
    \HCode{<text:section 
                  text:style-name="header-style" text:name="header">
          <text:p text:style-name="left-col">}}
   {\HCode{</text:p>}}
   {\ifvmode \IgnorePar\fi \EndP 
    \HCode{</text:section>
           <text:p text:style-name="p-retouradresse">}}
   {\ifvmode \IgnorePar\fi \EndP 
    \HCode{</text:p><text:p text:style-name="postage">}}
   {\HCode{</text:p>}}  
\Configure{retouradresse}
  {\HCode{<text:span text:style-name="retouradresse">}}
  {\HCode{</text:span>}} 
>>>

\<configure ooffice g-brief\><<<
\NewConfigureOO{g-brief-header}
\ConfigureOO{g-brief-header}{\Hnewline
<style:style style:name="header-style" style:family="section">\Hnewline
   <style:text-properties text:dont-balance-text-columns="false"
                     fo:margin-bottom="2cm">\Hnewline
      <style:columns    fo:column-count="2"
                        fo:column-gap="1in">\Hnewline
          <style:column fo:margin-left="0in"
                        fo:margin-right="0in"/>\Hnewline
          <style:column fo:margin-left="0in"
                        fo:margin-right="0in"/>\Hnewline
      </style:columns>\Hnewline      
   </style:text-properties>\Hnewline
</style:style>\Hnewline
}
>>>

\<configure ooffice g-briefNO\><<<
\OOstyles{%
<style:style style:name="header-style" style:family="section">\Hnewline
   <style:text-properties text:dont-balance-text-columns="false"
                     fo:margin-bottom="2cm">\Hnewline
      <style:columns    fo:column-count="2"
                        fo:column-gap="1in">\Hnewline
          <style:column fo:margin-left="0in"
                        fo:margin-right="0in"/>\Hnewline
          <style:column fo:margin-left="0in"
                        fo:margin-right="0in"/>\Hnewline
      </style:columns>\Hnewline      
   </style:text-properties>\Hnewline
</style:style>\Hnewline
}
>>>

\<configure ooffice g-brief\><<<
\OOstyles{%
<style:style style:name="p-retouradresse" \Hnewline
             style:family="paragraph" \Hnewline
             style:parent-style-name="Standard">\Hnewline
    <style:paragraph-properties  fo:margin-top="0.7cm" \Hnewline
                      fo:margin-bottom="0.7cm"/>\Hnewline
</style:style>\Hnewline
}
\OOstyles{%
<style:style style:name="postage" \Hnewline
             style:family="paragraph" \Hnewline
             style:parent-style-name="Standard">\Hnewline
    <style:paragraph-properties  fo:margin-top="0.5cm" \Hnewline
                      fo:margin-bottom="0.5cm"/>\Hnewline
</style:style>\Hnewline
}
>>>

The following configuration does not show.

\<configure ooffice g-brief\><<<
\OOstyles{%
<style:style style:name="tabular" 
             style:family="table"
             style:parent-style-name="header-style">\Hnewline
<\style:tableproperties style:rel-width="46\%"
                 fo:margin-left="1cm"
                 fo:margin-right="0cm"
                 table:align="margins"/>\Hnewline
</style:style>\Hnewline
}
>>>


\<configure ooffice g-brief\><<<
\Configure{betreff}
  {\HCode{<text:span text:style-name="betreff">}}
  {\HCode{</text:span>}}
\Configure{betrefftext}
  {\HCode{<text:span text:style-name="betrefftext">}}
  {\HCode{</text:span>}}
\Configure{gruss}
  {\HCode{<text:span text:style-name="gruss">}}
  {\HCode{</text:span>}}
\Configure{grussskip}
  {\HCode{<text:span text:style-name="grussskip">}}
  {\HCode{</text:span>}}
\Configure{unterschrift}
   {\HCode{<text:span text:style-name="unterschrift">}}
   {\HCode{</text:span>}}
\Configure{datumtext}
  {\HCode{<text:span text:style-name="datumtext">}}
  {\HCode{</text:span>}}
\Configure{datum}
  {\HCode{<text:span text:style-name="datum">}}
  {\HCode{</text:span>}}
\Configure{anlagen}
  {\HCode{<text:span text:style-name="anlagen">}}
  {\HCode{</text:span>}}
\Configure{verteiler}
   {\HCode{<text:span text:style-name="verteiler">}}
   {\HCode{</text:span>}}
\Configure{sprache}
  {\HCode{<text:span text:style-name="sprache">}}
  {\HCode{</text:span>}}
>>>

\<configure ooffice g-brief\><<<
\Configure{telefontex}
  {\HCode{<text:span text:style-name="telefontex">}}
  {\HCode{</text:span>}}
\Configure{telefaxtext}
  {\HCode{<text:span text:style-name="telefaxtext">}}
  {\HCode{</text:span>}}
\Configure{telextext}
  {\HCode{<text:span text:style-name="telextext">}}
  {\HCode{</text:span>}}
\Configure{telefax}
  {\HCode{<text:span text:style-name="telefax">}}
  {\HCode{</text:span>}} 
\Configure{telefon}
  {\HCode{<text:span text:style-name="telefon">}}
  {\HCode{</text:span>}} 
\Configure{telex}
  {\HCode{<text:span text:style-name="telex">}}
  {\HCode{</text:span>}} 
>>>

\<configure ooffice g-brief\><<<
\Configure{emailtext}
  {\HCode{<text:span text:style-name="emailtext">}}
  {\HCode{</text:span>}}
\Configure{email}
  {\HCode{<text:span text:style-name="email">}}
  {\HCode{</text:span>}} 
\Configure{httptext}
  {\HCode{<text:span text:style-name="httptext">}}
  {\HCode{</text:span>}}
\Configure{http}
  {\HCode{<text:span text:style-name="http">}}
  {\HCode{</text:span>}} 
\Configure{banktext}
  {\HCode{<text:span text:style-name="banktext">}}
  {\HCode{</text:span>}}
\Configure{bank}
  {\HCode{<text:span text:style-name="bank">}}
  {\HCode{</text:span>}} 
\Configure{konto}
  {\HCode{<text:span text:style-name="konto">}}
  {\HCode{</text:span>}} 
\Configure{blztext}
  {\HCode{<text:span text:style-name="blztext">}}
  {\HCode{</text:span>}}
\Configure{blz}
  {\HCode{<text:span text:style-name="blz">}}
  {\HCode{</text:span>}} 
>>>

\<configure ooffice g-brief\><<<
\Configure{ihrzeichentext}
  {\HCode{<text:span text:style-name="ihrzeichentext">}}
  {\HCode{</text:span>}}
\Configure{ihrzeichen}
  {\HCode{<text:span text:style-name="ihrzeichen">}}
  {\HCode{</text:span>}} 
\Configure{ihrschreibentext}
  {\HCode{<text:span text:style-name="ihrschreibentext">}}
  {\HCode{</text:span>}}
\Configure{ihrschreiben}
  {\HCode{<text:span text:style-name="ihrschreiben">}}
  {\HCode{</text:span>}} 
\Configure{meinzeichentext}
  {\HCode{<text:span text:style-name="meinzeichentext">}}
  {\HCode{</text:span>}}
\Configure{meinzeichen}
  {\HCode{<text:span text:style-name="meinzeichen">}}
  {\HCode{</text:span>}} 
\Configure{unserzeichentext}
  {\HCode{<text:span text:style-name="unserzeichentext">}}
  {\HCode{</text:span>}}
>>>

\<configure ooffice g-brief\><<<
\Configure{adresse}
  {\HCode{<text:span text:style-name="adresse">}}
  {\HCode{</text:span>}} 
\Configure{land}
  {\HCode{<text:span text:style-name="land">}}
  {\HCode{</text:span>}} 
\Configure{name}
  {\HCode{<text:span text:style-name="name">}}
  {\HCode{</text:span>}} 
\Configure{ort}
  {\HCode{<text:span text:style-name="ort">}}
  {\HCode{</text:span>}} 
\Configure{postvermerk}
  {\HCode{<text:span text:style-name="postvermerk">}}
  {\HCode{</text:span>}} 
\Configure{strasse}
  {\HCode{<text:span text:style-name="strasse">}}
  {\HCode{</text:span>}} 
\Configure{zusatz}
  {\HCode{<text:span text:style-name="zusatz">}}
  {\HCode{</text:span>}} 
>>>


\<configure oolatex g-briefNO\><<<
\Configure{anrede}
  {\ifvmode \IgnorePar\fi \EndP \HCode{<text:p text:style-name="anrede">}}
  {\HCode{</text:p>}}
>>>


%%%%%%%%%%%%%%%%%%
\section{Babel}
%%%%%%%%%%%%%%%%%%


\<configure ooffice babel\><<<
|<quotedblbase|>
\Configure{accent}\lower@umlaut\lower@umlaut{|<diaeresis codes|>{}{34}}
   {\a:accents{uml}{#1}}     {\b:accents{uml}{#1}{#2}}
>>>

% \Configure{accent}\lower@umlaut\lower@umlaut{A{A}E{E}I{I}O{O}U{U}Y%
%           {Y}a{a}e{e}i{i}\i{i}o{o}u{u}y{y}{}{34}}
%   {\a:accents{uml}{#1}}     {\b:accents{uml}{#1}{#2}}


\<configure ooffice babelNO\><<<
|<T1 old iso-8859-2 accents|>
|<old iso-8859-2 accents|>
>>>


\<T1 old iso-8859-2 accents\><<<
\expand:after{\Configure{accent}}\csname T1\string\`\expandafter\endcsname
   \csname T1\string\`\endcsname{|<grave iso-8859-2|>{}{}}
   {\HChar{#1}}   {\b:accents{grave}{#1}{#2}}
\expand:after{\Configure{accent}}\csname T1\string\'\expandafter\endcsname
   \csname T1\string\'\endcsname{|<acute iso-8859-2|>{}{}}
   {\HChar{#1}}   {\b:accents{acute}{#1}{#2}}
\expand:after{\Configure{accent}}\csname T1\string\^\expandafter\endcsname
   \csname T1\string\^\endcsname{|<circumflex iso-8859-2|>{}{}}
   {\HChar{#1}}   {\b:accents{circ}{#1}{#2}}
\expand:after{\Configure{accent}}\csname T1\string\~\expandafter\endcsname
   \csname T1\string\~\endcsname{|<tilde iso-8859-2|>{}{}}
   {\HChar{#1}}   {\b:accents{tilde}{#1}{#2}}
\expand:after{\Configure{accent}}\csname T1\string\"\expandafter\endcsname
   \csname T1\string\"\endcsname{|<diaeresis iso-8859-2|>{}{34}}
   {\HChar{#1}}     {\b:accents{uml}{#1}{#2}}
\expand:after{\expand:after{\Configure{accent}}%
   \csname T1\string\r\endcsname}%
   \csname T1\string\r\endcsname{|<ring iso-8859-2|>{}{}}
   {\HChar{#1}}   {\b:accents{ring}{#1}{#2}}
>>>



\<T1 old iso-8859-2 accents\><<<
\expand:after{\expand:after{\Configure{accent}}%
   \csname T1\string\c\endcsname}%
   \csname T1\string\c\endcsname{|<cedilla iso-8859-2|>{}{}}
   {\HChar{#1}}     {\b:accents{cedil}{#1}{#2}}
\expand:after{\expand:after{\Configure{accent}}%
   \csname T1\string\t\endcsname}%
   \csname T1\string\t\endcsname{{}{}}
   {\HChar{#1}}     {\b:accents{udot}{#1}{#2}}
\expand:after{\expand:after{\Configure{accent}}%
   \csname T1\string\H\endcsname}%
   \csname T1\string\H\endcsname{|<double acute iso-8859-2|>{}{}}
   {\HChar{#1}}{\b:accents{Huml}{#1}{#2}}
\expand:after{\expand:after{\Configure{accent}}%
   \csname T1\string\b\endcsname}%
   \csname T1\string\b\endcsname{{}{}}
   {\HChar{#1}}     {\b:accents{b}{#1}{#2}}
\expand:after{\expand:after{\Configure{accent}}%
   \csname T1\string\d\endcsname}%
   \csname T1\string\d\endcsname{{}{}}
   {\HChar{#1}}     {\b:accents{d}{#1}{#2}}
>>>



\<T1 old iso-8859-2 accents\><<<
\expand:after{\Configure{accent}}\csname T1\string\.\expandafter\endcsname
   \csname T1\string\.\endcsname
   {|<dot iso-8859-2|>{}{}} 
   {\HChar{#1}}{\b:accents{dot}{#1}{#2}}
\expand:after{\Configure{accent}}\csname T1\string\u\expandafter\endcsname
   \csname T1\string\u\endcsname
   {|<breve iso-8859-2|>{}{}}
   {\HChar{#1}}{\b:accents{breve}{#1}{#2}}
\expand:after{\Configure{accent}}\csname T1\string\vec\expandafter\endcsname
   \csname T1\string\vec\endcsname
   {|<vec iso-8859-2|>{}{}}
   {\HChar{#1}}{\b:accents{vec}{#1}{#2}}
\expandafter\let\csname T1\string\v\endcsname\:UnDef
\expand:after{\Configure{accent}}\csname T1\string\v\expandafter\endcsname
   \csname T1\string\v\endcsname
   {|<caron iso-8859-2|>{}{}}
   {\HChar{#1}}{\b:accents{check}{#1}{#2}} 
\expand:after{\Configure{accent}}\csname T1\string\=\expandafter\endcsname
   \csname T1\string\=\endcsname
   {|<bar iso-8859-2|>{}{}}
   {\HChar{#1}}{\b:accents{bar}{#1}{#2}}
>>>



\<old iso-8859-2 accents\><<<
\Configure{accent}\`\grave{|<grave iso-8859-2|>{}{}}
   {\HChar{#1}}   {\b:accents{grave}{#1}{#2}}
\Configure{accent}\'\acute{|<acute iso-8859-2|>{}{}}
   {\HChar{#1}}   {\b:accents{acute}{#1}{#2}}
\Configure{accent}\^\hat{|<circumflex iso-8859-2|>{}{}}
   {\HChar{#1}}   {\b:accents{hat}{#1}{#2}}
\Configure{accent}\~\tilde{|<tilde iso-8859-2|>{}{}}
   {\HChar{#1}}   {\b:accents{tilde}{#1}{#2}}
\Configure{accent}\"\ddot{|<diaeresis iso-8859-2|>{}{34}}
   {\HChar{#1}}   {\b:accents{uml}{#1}{#2}}
>>>

\<old iso-8859-2 accents\><<<
\Configure{accent}\c\c{|<cedilla iso-8859-2|>{}{}}
   {\HChar{#1}}     {\b:accents{cedil}{#1}{#2}}
\Configure{accent}\t\t{{}{}}
   {\HChar{#1}}     {\b:accents{udot}{#1}{#2}}
\Configure{accent}\H\H{|<double acute iso-8859-2|>{}{}}
   {\HChar{#1}}     {\b:accents{Huml}{#1}{#2}}
\Configure{accent}\b\b{{}{}}
   {\HChar{#1}}     {\b:accents{b}{#1}{#2}}
\Configure{accent}\d\d{{}{}}
   {\HChar{#1}}     {\b:accents{d}{#1}{#2}}
>>>


\<old iso-8859-2 accents\><<<
\Configure{accent}\.\.{|<dot above iso-8859-2|>{}{}}  
   {\HChar{#1}}      {\b:accents{dot}{#1}{#2}}
\Configure{accent}\dot\dot{|<dot above iso-8859-2|>{}{}}  
   {\HChar{#1}}      {\b:accents{dot}{#1}{#2}}
\Configure{accent}\u\breve{|<breve iso-8859-2|>{}{}}
   {\HChar{#1}}      {\b:accents{breve}{#1}{#2}}
\Configure{accent}\vec\vec{{}{}}
   {\HChar{#1}}      {\b:accents{vec}{#1}{#2}}
\Configure{accent}\v\v{|<caron iso-8859-2|>{}{}}
   {\HChar{#1}}      {\b:accents{check}{#1}{#2}} 
\Configure{accent}\check\check{|<caron iso-8859-2|>{}{}}
   {\HChar{#1}}      {\b:accents{check}{#1}{#2}} 
\Configure{accent}\=\bar{{}{}} 
   {\HChar{#1}}      {\b:accents{bar}{#1}{#2}}
>>>

\<configure ooffice frenchb\><<<
\Configure{frenchb-thinspace}{\HCode{\string&\:sharp x202F;}\setbox\tmp:bx=\hbox\bgroup}{\egroup}
\Configure{frenchb-nbsp}{\:nbsp\setbox\tmp:bx=\hbox\bgroup}{\egroup}
>>>

%%%%%%%%%%%%%%%%%%
\section{ElsArt}
%%%%%%%%%%%%%%%%%%

\<configure ooffice elsart\><<<
\ConfigureEnv{frontmatter}
  {\IgnorePar
   \HCode{<!--begin frontmatter-->%
          <text:section text:name="frontmatter">}\ShowPar}
  {\ifvmode\IgnorePar\fi \EndP  \IgnorePar
   \HCode{</text:section><!--end frontmatter-->}\ShowPar}
  {}{}
>>>
 
\<configure ooffice elsart\><<<
\Configure{thanks author date and} 
   {\ifvmode \IgnorePar\fi \EndP
    \HCode{<text:p text:style-name="Text-body|<cond rtl class|>">}}
   {\HCode{</text:p>}\IgnorePar}
   {\ifvmode \IgnorePar\fi \EndP
    \HCode{<text:p text:style-name="author|<cond rtl class|>">}\IgnorePar}
   {\HCode{</text:p>}\IgnorePar}
   {\ifvmode \IgnorePar\fi \EndP
    \HCode{<text:p text:style-name="date|<cond rtl class|>">}}
   {\HCode{</text:p>}\IgnorePar}
   {}
   {}%   {\HCode{<text:line-break/>}}
>>>

\<configure ooffice elsart\><<<
\Configure{address}
  {\IgnorePar \EndP
   \HCode{<text:p\Hnewline text:style-name="address">}\IgnorePar}
  {\IgnorePar\EndP \HCode{</text:p>}\IgnorePar}
\NewConfigureOO{address}
\ConfigureOO{address}{%
<style:style      style:name="address"
                  style:family="paragraph"
                  style:parent-style-name="Text-body">\Hnewline
<style:paragraph-properties  fo:margin-top="0cm"
                  fo:margin-bottom="0.199cm"
                  fo:font-size="12pt"
                  fo:font-style="italic"
                  fo:text-align="center"
                  style:justify-single-word="false"/>
</style:style>\Hnewline |<author rtl|>}
>>>

\<configure ooffice elsart\><<<
\Configure{abstract}
{\HCode{<text:p\Hnewline 
        text:style-name="abstract-title">}\par\IgnorePar}
  { \HCode{</text:p>}}
  {\IgnorePar\EndP}
\ConfigureOO{abstract-title}{\Hnewline
<style:style style:name="abstract-title"
                  style:family="paragraph"
                  style:parent-style-name="abstract"
                  style:next-style-name="abstract">\Hnewline
<style:paragraph-properties  fo:margin-top="0.499cm"
                  fo:margin-bottom="0cm"
                  fo:font-weight="bold"
                  fo:font-size="10pt"
                  fo:text-align="left"
                  fo:border-top="0.01cm solid \#000000"
                  style:justify-single-word="false"/>\Hnewline
</style:style>\Hnewline
}
\ConfigureOO{abstract}{\Hnewline
<style:style style:name="abstract"
                  style:family="paragraph"
                  style:parent-style-name="Text-body"
                  style:next-style-name="Text-body">\Hnewline
<style:paragraph-properties  fo:margin-left="0cm"
                  fo:margin-right="0cm"
                  fo:margin-top="0cm"
                  fo:margin-bottom="0.199cm"
                  fo:font-size="10pt"
                  fo:text-indent="0cm"
                  fo:text-align="left"
                  style:auto-text-indent="false"/>\Hnewline
</style:style>\Hnewline |<abstract rtl|> }
>>>


\<configure ooffice elsart\><<<
\Configure{keyword}
  {\IgnorePar\EndP
   \HCode{<text:p\Hnewline text:style-name="keyword">}\IgnorePar}
  {\IgnorePar\EndP \HCode{</text:p>}\IgnorePar}
\NewConfigureOO{keyword}
\ConfigureOO{keyword}{\Hnewline
<style:style style:name="keyword"
                  style:family="paragraph"
                  style:parent-style-name="Text-body"
                  style:next-style-name="Text-body">\Hnewline
<style:paragraph-properties  fo:margin-left="0cm"
                  fo:margin-right="0cm"
                  fo:margin-top="0cm"
                  fo:margin-bottom="0.199cm"
                  fo:font-size="10pt"
                  fo:font-style="italic"
                  fo:text-indent="0cm"
                  fo:text-align="left"
                  fo:border-bottom="0.01cm solid \#000000"
                  style:auto-text-indent="false"/>\Hnewline
</style:style>\Hnewline > }
>>>

\<configure ooffice elsart\><<<

>>>

\<configure ooffice elsart\><<<

>>>


%%%%%%%%%%%%%%%%%%
\section{Foils}
%%%%%%%%%%%%%%%%%%


\<configure ooffice foils\><<<
\Configure{foilhead} 
   {\IgnorePar\EndP \HCode{<!--start of foilhead-->}}   
   {\IgnorePar\EndP \HCode{<!--end of foilhead-->}}
   {\HCode{<text:h text:style-name="foilhead|<cond rtl class|>"
                   \text:outlinelevel="3"   >}%
    \IgnoreIndent}
   {\HCode{</text:h>\Hnewline}\par\ShowPar}
>>>

\<configure ooffice foils\><<<
\NewConfigureOO{foilhead}
\ConfigureOO{foilhead}{\Hnewline
<style:style style:name="foilhead"
                  style:family="paragraph"
                  style:parent-style-name="Heading"
                  style:next-style-name="Text-body"
                  style:class="text">\Hnewline
<style:paragraph-properties  fo:font-size="12pt"
                  fo:font-weight="bold"
                  style:font-size-asian="14pt"
                  style:font-weight-asian="bold"
                  style:font-size-complex="14pt"
                  style:font-weight-complex="bold"
                  fo:margin-top="0.42cm"
                  fo:margin-bottom="0.51cm"
                  fo:text-align="center"
                  style:justify-single-word="false"
/>
</style:style>
}
>>>










\<configure ooffice foils\><<<
\Configure{thank}
   {\setbox\tmp:bx=\hbox\bgroup}
   {\egroup}
   {\NoFonts   \leavevmode   \bgroup \protect\Configure{$}{}{}{}%
    \hbox{\HCode{<text:note\Hnewline text:note-class="footnote"
                  text:id="}|<thank id|>%
               \HCode{"><text:note-citation>}}%
   }
   {\egroup \EndNoFonts
    \HCode{</text:note-citation><text:note-body>
    <text:p\Hnewline text:style-name="Footnote" ><text:s/>}}
   {\HCode{</text:p></text:note-body></text:note>}}   
>>>



\<configure ooffice foils\><<<
\def\:temp#1{{\stepcounter{footnote}%
   \a:thank\@fnsymbol\c@footnote\b:thank
   \a:thanks
      \c:thank \@fnsymbol\c@footnote\d:thank #1\e:thank
   \b:thanks
}}
\HLet\thanks\:temp
>>>




\<configure ooffice foils\><<<
\Configure{maketitle}
   {\bgroup  
       \Configure{newpage}{}%
       \HCode{<!--begin maketitle-->}
       \Configure{tabular}{}{}{}{}{\ifnum \HRow>1
               \HCode{<text:line-break\Hnewline/>}\fi}{}%
       \ConfigureEnv{tabular}{}{}{}{}%
   }
   {\HCode{<!--end maketitle-->}\egroup \par}
   {\ifvmode \IgnorePar\fi \EndP
    \HCode{<text:p text:style-name="Title|<cond rtl class|>" >}}
   {\HCode{</text:p>}}
\OOstyles{%
   <style:style style:name="maketitle-tabular" 
                style:family="table">\Hnewline
   <\style:tableproperties 
                  fo:margin-top="0.42cm"
                  fo:margin-bottom="0.51cm"
                  fo:font-size="20pt"
                  fo:font-weight="normal"
                  fo:text-align="center"
                  style:justify-single-word="false"
       table:align="margins"/>\Hnewline
   </style:style>\Hnewline
}
\NewConfigureOO{title}
\ConfigureOO{title}{%
<style:style style:name="Title"
             style:family="paragraph"
             style:parent-style-name="Heading"
             style:next-style-name="author"
             style:class="chapter">\Hnewline
<style:paragraph-properties  fo:margin-top="0.42cm"
                  fo:margin-bottom="0.51cm"
                  fo:font-size="20pt"
                  fo:font-weight="normal"
                  style:font-size-asian="18pt"
                  style:font-weight-asian="bold"
                  style:font-size-complex="18pt"
                  style:font-weight-complex="bold"
                  fo:text-align="center"
                  style:justify-single-word="false"/>
</style:style> \Hnewline <style:style style:name="Title-rtl"
             style:family="paragraph"
             style:parent-style-name="Heading-rtl"
             style:next-style-name="author-rtl"
             style:class="chapter">\Hnewline
<style:paragraph-properties  fo:margin-top="0.42cm"
                  fo:margin-bottom="0.51cm"
                  fo:font-size="20pt"
                  fo:font-weight="normal"
                  style:font-size-asian="18pt"
                  style:font-weight-asian="bold"
                  style:font-size-complex="18pt"
                  style:font-weight-complex="bold"
                  fo:text-align="center"
                  style:justify-single-word="false"
                  style:writing-mode="rl-tb" />
</style:style> \Hnewline
 }
>>>



\<configure ooffice foils\><<<
\Configure{author date}
   {\ifvmode \IgnorePar\fi \EndP
    \HCode{<text:p text:style-name="author|<cond rtl class|>">}}
   {\HCode{</text:p>}}
   {\ifvmode \IgnorePar\fi \EndP
    \HCode{<text:p text:style-name="date|<cond rtl class|>">}}
   {\HCode{</text:p>}}
\NewConfigureOO{author}
\ConfigureOO{author}{%
<style:style      style:name="author"
                  style:family="paragraph"
                  style:parent-style-name="Text-body"
                  style:next-style-name="date">\Hnewline
<style:paragraph-properties  fo:margin-top="0cm"
                  fo:margin-bottom="0.2cm"
                  fo:font-size="14pt"
                  fo:text-align="center"
                  style:justify-single-word="false"/>
</style:style>\Hnewline <style:style      style:name="author-rtl"
                  style:family="paragraph"
                  style:parent-style-name="Text-body-rtl"
                  style:next-style-name="date-rtl">\Hnewline
<style:paragraph-properties  fo:margin-top="0cm"
                  fo:margin-bottom="0.2cm"
                  fo:font-size="14pt"
                  fo:text-align="center"
                  style:justify-single-word="false"
                  style:writing-mode="rl-tb" />
</style:style>\Hnewline
}
\NewConfigureOO{date}
\ConfigureOO{date}{%
<style:style style:name="date"
                  style:family="paragraph"
                  style:parent-style-name="Text-body"
                  style:next-style-name="Text-body">\Hnewline
<style:paragraph-properties  fo:margin-top="0.6cm"
                  fo:margin-bottom="0.2cm"
                  fo:font-size="14pt"
                  fo:text-align="center"
                  style:justify-single-word="false"/>
</style:style>\Hnewline <style:style style:name="date-rtl"
                  style:family="paragraph"
                  style:parent-style-name="Text-body"
                  style:next-style-name="Text-body">\Hnewline
<style:paragraph-properties  fo:margin-top="0cm"
                  fo:margin-bottom="0.199cm"
                  fo:font-size="14pt"
                  fo:text-align="center"
                  style:justify-single-word="false"
                  style:writing-mode="rl-tb" />
</style:style>\Hnewline
}
>>>




%%%%%%%%%%%%%%%%%%%%%%%%%%%%%%%%%%%%%%%%%%%%%%%%%%%%%%%%%%%%%%%%%%%%%%%% 
\chapter{Packages} 
%%%%%%%%%%%%%%%%%%%%%%%%%%%%%%%%%%%%%%%%%%%%%%%%%%%%%%%%%%%%%%%%%%%%%%%% 
 
%%%%%%%%%%%%%%%%%%
\section{Pstricks}
%%%%%%%%%%%%%%%%%%


\<configure ooffice pst-all\><<<
\ConfigureEnv{psmatrix}
   {\Picture*{}}
   {\EndPicture}
   {}{}
>>>
         
\<configure ooffice pstricks\><<<
\Configure{pspicture}
   {\ifvmode \ifinner\else \vfill\break\fi
    \leavevmode\fi
    \bgroup 
       \ifdim \pst@dimh<\z@ \pst@dimh=-\pst@dimh \fi
       \HCode{<!--width="\ifdim \wd\pst@hbox=\z@  \the\pst@dimg
                         \else\the\wd\pst@hbox\fi"  
                 height="\ifdim \ht\pst@hbox=\z@ \the\pst@dimh
                         \else \the\ht\pst@hbox\fi"-->}%
     \egroup
     \Picture+[\a:pspictureAlt]{ class="pspicture"}}
   {\EndPicture}
\NewConfigure{pspictureAlt}{1}
\Configure{pspictureAlt}{PIC}
>>>


%%%%%%%%%%%%%%%%%%
\section{Fancyvrb}
%%%%%%%%%%%%%%%%%%



\<configure ooffice fancyvrb\><<<
\HAssign\fancyvrbNo=0
\Configure{fancyvrb}
   {\IgnorePar \EndP \gHAdvance\fancyvrbNo by 1
    \gHAssign\fancyvrb:cnt0
    \HCode{<text:p text:style-name="fancyvrb\fancyvrbNo">}}
   {\EndP\HCode{</text:p>}}
   {\gHAdvance\fancyvrb:cnt by 1
    \ifnum \fancyvrb:cnt>1 
    \HCode{<text:line-break />}\fi} {}
   {}{}
>>>


\<configure ooffice fancyvrb\><<<
\Configure{BVerbatimInput}
   {\ifvmode \IgnorePar \fi \EndP 
    \gHAssign\HRow=0
    \HCode{<text:p text:style-name="BVerbatimInput">}}
   {\ifvmode \IgnorePar \fi \EndP \HCode{</text:p>}}
   {\gHAdvance\HRow by 1 \ifnum \HRow>1 \HCode{<text:line-break />}\fi}{}
>>>

%%%%%%%%%%%%%%%%%%
\section{Ulem}
%%%%%%%%%%%%%%%%%%



\<configure ooffice ulem\><<<
\Configure{uline}
     {\ifmmode \HCode{<\a:mathml mrow class="uline">}%
      \else \HCode{<text:span text:style-name="uline">}\fi}
     {\ifmmode \HCode{</\a:mathml mrow>}\else \HCode{</text:span>}\fi}
\NewConfigureOO{uline}
\ConfigureOO{uline}{|<oo uline style|>}
>>>

\<oo uline style\><<<
<style:style style:name="uline" style:family="text">
    <style:text-properties style:text-underline-style="solid"
                           style:text-underline-width="auto"
                           style:text-underline-color="font-color"
    />
</style:style>
>>>



\<configure ooffice ulem\><<<
\Configure{uuline}
     {\ifmmode \HCode{<\a:mathml mrow class="uuline">}%
      \else \HCode{<text:span text:style-name="uuline">}\fi}
     {\ifmmode \HCode{</\a:mathml mrow>}\else \HCode{</text:span>}\fi}
\NewConfigureOO{uuline}
\ConfigureOO{uuline}{|<oo uuline style|>}
>>>

\<oo uuline style\><<<
<style:style style:name="uuline" style:family="text">
    <style:text-properties style:text-underline-style="solid"
                           style:text-underline-width="auto"
                           style:text-underline-color="font-color"
    />
</style:style>
>>>


\<configure ooffice ulem\><<<
\Configure{uwave}
     {\ifmmode \HCode{<\a:mathml mrow class="uwave">}%
      \else \HCode{<text:span text:style-name="uwave">}\fi}
     {\ifmmode \HCode{</\a:mathml mrow>}\else \HCode{</text:span>}\fi}
\NewConfigureOO{uwave}
\ConfigureOO{uwave}{|<oo uwave style|>}
>>>

\<oo uwave style\><<<
<style:style style:name="uwave" style:family="text">
    <style:text-properties style:text-underline-style="wave"
                           style:text-underline-width="auto"
                           style:text-underline-color="font-color"
    />
</style:style>
>>>

%    <style:text-properties style:text-underline="single" />



\<configure ooffice ulem\><<<
\Configure{sout}
     {\ifmmode \HCode{<\a:mathml mrow class="sout">}%
      \else \HCode{<text:span text:style-name="sout">}\fi}
     {\ifmmode \HCode{</\a:mathml mrow>}\else \HCode{</text:span>}\fi}
\NewConfigureOO{sout}
\ConfigureOO{sout}{|<oo sout style|>}
>>>

\<oo sout style\><<<
<style:style style:name="sout" style:family="text">
   <style:text-properties style:text-line-through-style="solid"
                          style:text-underline-style="none"     />
</style:style>
>>>




\<configure ooffice ulem\><<<
\Configure{xout}
     {\ifmmode \HCode{<\a:mathml mrow class="xout">}%
      \else \HCode{<text:span text:style-name="xout">}\fi}
     {\ifmmode \HCode{</\a:mathml mrow>}\else \HCode{</text:span>}\fi}
\NewConfigureOO{xout}
\ConfigureOO{xout}{|<oo xout style|>}
>>>

\<oo xout style\><<<
<style:style style:name="xout" style:family="text">
   <style:text-properties style:text-line-through-style="solid"
                          style:text-underline-style="none"     />
</style:style>
>>>



%%%%%%%%%%%%%%%%%%
\section{German}
%%%%%%%%%%%%%%%%%%



\<configure ooffice german\><<<
|<quotedblbase|>
>>>

\<configure ooffice ngerman\><<<
|<quotedblbase|>
>>>





\<quotedblbase\><<<
\Configure{quotedblbase}{\leavevmode\ht:special{t4ht@+&{35}x201E;}x}
\Configure{quotesinglbase}{\leavevmode\ht:special{t4ht@+&{35}x201A;}x}
>>>


\<configure ooffice german\><<<
\Configure{accent}\grmn@OTumlaut
      \grmn@OTumlaut{|<diaeresis codes|>{}{34}}
   {\a:accents{uml}{#1}}     {\b:accents{uml}{#1}{#2}}
>>>


\<configure ooffice ngerman\><<<
\Configure{accent}\grmn@OTumlaut
      \grmn@OTumlaut{|<diaeresis codes|>{}{34}}
   {\a:accents{uml}{#1}}     {\b:accents{uml}{#1}{#2}}
>>>


\<diaeresis codes\><<<
H{1E26}h{1E27}W{1E84}w{1E85}X{1E8C}x{1E8D}%
t{1E97}A{00C4}E{00CB}I{00CF}O{00D6}U{00DC}%
a{00E4}e{00EB}i{00EF}o{00F6}u{00FC}y{00FF}%
Y{0178}\i{00EF}%
>>>



%%%%%%%%%%%%%%%%%%
\section{Hyperref}
%%%%%%%%%%%%%%%%%%




\<configure ooffice  hyperref\><<<
\ifx \@baseurl\@empty \else
   \Configure{@HEAD}
      {\HCode{<!--base href="\:baseurl" /-->}}
   \let\:baseurl\@baseurl
   \let\@baseurl=\@empty
\fi
>>>




%%%%%%%%%%%%%%%%%%%%%
\section{Listings}
%%%%%%%%%%%%%%%%%%%%%


\<configure ooffice listings\><<<
\Log:Note{to ensure proper alignments use
   fixed size fonts (see listings.dtx)}
\Configure{listings}
   {\ifvmode \IgnorePar\fi \EndP 
    {\everypar{}\leavevmode}\HCode{<!--l. \the\inputlineno-->}%
     \HCode{<text:p text:style-name="Preformatted-Text">}%
   }
   {{\everypar{}\leavevmode}\HCode{</text:p>}\par}
   {}{}
>>>

\<configure ooffice listings\><<<
\ConfigureEnv{lstlisting}
   {\ifvmode \IgnorePar\fi \EndP  
    \HCode{<!--l. \the\inputlineno lstlisting-->}%
    \HCode{<text:p\Hnewline text:style-name="Preformatted-Text">}%
    \bgroup \special{t4ht@(}%
       \Configure{listings}
         {{\everypar{}\leavevmode}}
         {{\everypar{}\leavevmode}}
         {\HCode{<text:s/><text:line-break/>}}{}%
   }
   {\special{t4ht@)}\egroup
    \ifvmode \IgnorePar\fi \EndP   \HCode{</text:p>}\par}
   {} {}
>>>

\<configure ooffice listings\><<<
\Configure{lstinputlisting}
   {\ifvmode \IgnorePar\fi \EndP
    \HCode{<!--l. \the\inputlineno lstinputlisting-->}%
    \HCode{<text:p\Hnewline text:style-name="Preformatted-Text">}%
    \bgroup \tt \special{t4ht@(}%
       \Configure{listings}
         {{\everypar{}\leavevmode}}
         {{\everypar{}\leavevmode}}
         {\HCode{<text:s/><text:line-break/>}} 
         {\HCode{<text:line-break/>}}%
   }
   {\special{t4ht@)}\egroup
    \ifvmode \IgnorePar\fi \EndP   \HCode{</text:p>}\par}
>>>



\<configure ooffice-mml latex\><<<
\def\:nbsp{\ifmtext |<unicode space|>\else\ifmathml 
    \Tg<\a:mathml mspace />\else |<oo space|>\fi\fi} 
>>> 
 
\<unicode space\><<< 
\leavevmode\special{t4ht@+\string&{35}x00A0{59}}x% 
>>> 

\<oo space\><<<
\ht:special{t4ht@+\string&{35}x00A0{59}}x%
>>>

Nonbraking spaces may go to aux files and need protect the colon and 
semicolons from catcode changes when loaded, e.g., in french.






%%%%%%%%%%%%%
\section{Linguistics}
%%%%%%%%%%%%%

%%%%%%%%%%%%%
\subsection{linguex}
%%%%%%%%%%%%%



\<configure ooffice linguex\><<<
\Configure{a.list}
  {\ConfigureList{list}%
   {\EndP\HCode{<text:list
         text:style-name="list|<cond rtl class|>"
         text:name="list"\Hnewline>}%
    \PushMacro\end:itm
    \global\let\end:itm=\empty
    \SaveHtmlPar \HTML:PAR{dd}{dd}%
   }
   {\PopMacro\end:itm \global\let\end:itm \end:itm
    \EndP
    \RecallHtmlPar
    \HCode{</text:list-item></text:list>}\ShowPar}
   {\end:itm \global\def\end:itm{\EndP\HCode{</text:list-item>}}%
    \HCode{<text:list-item><text:h
      text:style-name="dt|<cond rtl class|>"\Hnewline >}\bgroup \bf}
   {\egroup\HCode{</text:h>}\par\ShowPar}%
  }
>>>

%%%%%%%%%%%%%
\subsection{cgloss4e}
%%%%%%%%%%%%%



\<configure ooffice cgloss4e\><<<
\Configure{tsent}
   {\HCode{<table:table-cell table:style-name="tsent-td">%
           <text:p text:style-name="tsent-p">}%    
    \gHAdvance\tsent:cnt by 1
   }
   {\HCode{</text:p></table:table-cell>}}
   {\HCode{<text:line-break />}}
>>>

\<configure ooffice cgloss4e\><<<
\Configure{gll} 
   {\ifvmode \IgnorePar\fi \EndP 
    \bgroup 
      \ConfigureEnv{flushleft} 
       {\ifvmode \IgnorePar\fi \EndP} 
       {\ShowPar} 
       {\EndP\IgnorePar
        |<boxed env|>% 
        \HCode{<table:table table:style-name="gll-table"\Hnewline>%
              }\IgnorePar} 
       {\IgnorePar\EndP\HCode{</table:table-row></table:table>}%
        |<end boxed env|>}% 
     \Configure{HtmlPar}{}{}{}{}% 
     \gHAssign\tsent:cnt=0
   } 
   {\tmp:cnt=0
    \loop 
       \advance\tmp:cnt by 1
       \HCode{<table:table-column table:style-name="gll-col"/>}%
       \ifnum \tsent:cnt>\tmp:cnt
    \repeat
    \HCode{<table:table-row table:style-name="gll-tr">}%
   }
   {\egroup} 
>>>

%%%%%%%%%%%%%
\subsection{covington}
%%%%%%%%%%%%%



\<configure ooffice covington\><<<
\Configure{tsent}
   {\HCode{<table:table-cell table:style-name="tsent-td">%
           <text:p text:style-name="tsent-p">}%    
    \gHAdvance\tsent:cnt by 1
   }
   {\HCode{</text:p></table:table-cell>}}
   {\HCode{<text:line-break />}}
>>>


\<configure ooffice covington\><<<
\HAssign\gll:Cnt = 0
\Configure{gll} 
   {\ifvmode \IgnorePar\fi \EndP 
    \gHAdvance\gll:Cnt by 1
    \bgroup 
      \ConfigureEnv{flushleft} 
       {\ifvmode \IgnorePar\fi \EndP} 
       {\ShowPar} 
       {\EndP\IgnorePar
        |<boxed env|>% 
        \HCode{<table:table table:name="gll-\gll:Cnt"
                            table:style-name="gll-table"\Hnewline>}%
        \ifnum 0<0\LikeRef{gll-\gll:Cnt}
           |<set gll's table-column|>%
        \fi
        \HCode{<table:table-row table:style-name="gll-tr">}%
        \IgnorePar
       } 
       {\IgnorePar\EndP
        \csname :glend\endcsname
        \HCode{</table:table-row></table:table>}%
        |<end boxed env|>}% 
     \Configure{HtmlPar}{}{}{}{}% 
     \gHAssign\tsent:cnt=0
   } 
   {}
   {\Tag{gll-\gll:Cnt}{\tsent:cnt}\egroup} 
>>>



\<set gll's table-column\><<<
\tmp:cnt=0
\loop 
   \advance\tmp:cnt by 1
   \HCode{<table:table-column table:style-name="gll-col"/>}%
   \ifnum \LikeRef{gll-\gll:Cnt}>\tmp:cnt
\repeat
>>>

\<configure ooffice covington\><<<
\Configure{glt}
   {\expandafter\ifx\csname :glend\endcsname\relax\else
       \HCode{</text:p></table:table-cell>}%
    \fi
    \HCode{</table:table-row><table:table-row
             table:style-name="glt"><table:table-cell
             table:number-columns-spanned="\tsent:cnt"><text:p
              text:style-name="p-glt">}%
    \gdef\:glend{\global\let\:glend\relax
       \HCode{</text:p></table:table-cell>}}%
   }
\NewConfigureOO{p-glt}
\ConfigureOO{p-glt}{\Hnewline
<style:style style:name="p-glt" 
             style:family="paragraph"
             style:parent-style-name="Standard" >
     <style:paragraph-properties  fo:margin-top="0.2cm"/>
</style:style>
}
>>>

\<configure ooffice covington\><<<
\Configure{gln}
   {\expandafter\ifx\csname :glend\endcsname\relax\else
       \HCode{</text:p></table:table-cell>}%
    \fi
    \gHAdvance\tsent:cnt by 1
    \HCode{<table:table-cell><text:p
              text:style-name="p-gln">}%
    \gdef\:glend{\global\let\:glend\relax
       \HCode{</text:p></table:table-cell>}}%
   }
\NewConfigureOO{p-gln}
\ConfigureOO{p-gln}{\Hnewline
<style:style style:name="p-gln" 
             style:family="paragraph"
             style:parent-style-name="Standard" >
     <style:paragraph-properties  fo:text-align="end"/>
</style:style>
}
>>>

\<configure ooffice covington\><<<
\NewConfigureOO{gll-table}
\ConfigureOO{gll-table}{\Hnewline
<style:style style:name="gll-table" style:family="table"> 
    <style:table-properties 
          fo:margin-top="0.3cm"
          fo:margin-bottom="0.3cm"
    /> 
</style:style>
}
>>>
%%%%%%%%%%%%%%%%%%%%%%%%%%%%%%%%%%%%%%%%%%%%%%%%%%%%%%%%%%%%%%%%%%%%%%%%
\chapter{Right to Left}
%%%%%%%%%%%%%%%%%%%%%%%%%%%%%%%%%%%%%%%%%%%%%%%%%%%%%%%%%%%%%%%%%%%%%%%%



\<cond rtl class\><<<
\if@rl-rtl\fi
>>>


%%%%%%%%%%%%%%%%%%
\section{Title Page}
%%%%%%%%%%%%%%%%%%





\<Title rtl\><<<
<style:style style:name="Title-rtl"
             style:family="paragraph"
             style:parent-style-name="Heading-rtl"
             style:next-style-name="author-rtl"
             style:class="chapter">\Hnewline
<style:paragraph-properties  fo:margin-top="0.42cm"
                  fo:margin-bottom="0.51cm"
                  fo:font-size="20pt"
                  fo:font-weight="normal"
                  style:font-size-asian="18pt"
                  style:font-weight-asian="bold"
                  style:font-size-complex="18pt"
                  style:font-weight-complex="bold"
                  fo:text-align="center"
                  style:justify-single-word="false"
                  style:writing-mode="rl-tb" />
</style:style> \Hnewline
>>>

\<author rtl\><<<
<style:style      style:name="author-rtl"
                  style:family="paragraph"
                  style:parent-style-name="Text-body-rtl"
                  style:next-style-name="date-rtl">\Hnewline
<style:paragraph-properties  fo:margin-top="0cm"
                  fo:margin-bottom="0.199cm"
                  fo:font-size="14pt"
                  fo:text-align="center"
                  style:justify-single-word="false"
                  style:writing-mode="rl-tb" />
</style:style>\Hnewline
>>>



\<date rtl\><<<
<style:style style:name="date-rtl"
                  style:family="paragraph"
                  style:parent-style-name="Text-body"
                  style:next-style-name="Text-body">\Hnewline
<style:paragraph-properties  fo:margin-top="0cm"
                  fo:margin-bottom="0.199cm"
                  fo:font-size="14pt"
                  fo:text-align="center"
                  style:justify-single-word="false"
                  style:writing-mode="rl-tb" />
</style:style>\Hnewline 
>>>




\<abstract rtl\><<<
<style:style style:name="abstract-rtl"
                  style:family="paragraph"
                  style:parent-style-name="Text-body-rtl"
                  style:next-style-name="Text-body-rtl">\Hnewline
<style:paragraph-properties  fo:margin-left="3cm"
                  fo:margin-right="3cm"
                  fo:margin-top="0cm"
                  fo:margin-bottom="0.199cm"
                  fo:font-size="10pt"
                  fo:text-indent="0cm"
                  style:auto-text-indent="false"
                  style:writing-mode="rl-tb" />\Hnewline
</style:style>\Hnewline
>>>


\<configure ooffice amsart\><<<
\Configure{keywords}
  {\par\IgnorePar\EndP
   \HCode{<text:p\Hnewline text:style-name="keywords">}\ShowPar}
  {\ifvmode \IgnorePar\fi \EndP \HCode{</text:p>}}
\Configure{subjclass}
   {\par\IgnorePar
    \HCode{<text:p\Hnewline text:style-name="subjclass">}\par\ShowPar} 
   {\ifvmode \IgnorePar \fi\EndP \HCode{</text:p>}}
\Configure{@maketitle}
   {\ifvmode \IgnorePar\fi \EndP
    \HCode{<text:p\Hnewline text:style-name="submaketitle"></text:p>}\IgnoreIndent
    |<start nested pars|>} 
   {|<end nested pars|>\ifvmode \IgnorePar \fi\EndP  
    \HCode{<text:p\Hnewline text:style-name="end-submaketitle"></text:p>}}
>>>




\<configure ooffice amsart\><<<
\NewConfigureOO{submaketitle}
\ConfigureOO{submaketitle}{\Hnewline
<style:style style:name="submaketitle"
                  style:family="paragraph"
                  style:parent-style-name="Text-body"
                  style:next-style-name="Text-body">\Hnewline
<style:paragraph-properties  fo:margin-left="0cm"
                  fo:margin-right="0cm"
                  fo:margin-top="0cm"
                  fo:margin-bottom="0cm"
                  fo:border-bottom="0.01cm solid \#000000"
                  style:auto-text-indent="false"/>\Hnewline
</style:style>\Hnewline > 
<style:style style:name="end-submaketitle"
                  style:family="paragraph"
                  style:parent-style-name="Text-body"
                  style:next-style-name="Text-body">\Hnewline
<style:paragraph-properties  fo:margin-left="0cm"
                  fo:margin-right="0cm"
                  fo:margin-top="0cm"
                  fo:margin-bottom="0cm"
                  fo:border-top="0.01cm solid \#000000"
                  style:auto-text-indent="false"/>\Hnewline
</style:style>\Hnewline > }
>>>


\<boxed subjclass\><<<
\hbox{\HCode{<text:p><draw:frame 
                draw:name="subjclass"  
                 draw:style-name="subjclass"  
                 text:anchor-type="as-char"  
                 draw:z-index="0"
       >  
       <draw:text-box 
       >}}% 
>>>


\<end boxed subjclass\><<<
\hbox{\HCode{</draw:text-box>  
</draw:frame></text:p> }}%
>>>




\<configure ooffice amsart\><<<
\Configure{maketitle}
   {\ifvmode \IgnorePar\fi \EndP |<ams title for TITLE|>%
    \HCode{<!--begin maketitle-->}%
    \ConfigureEnv{center}
       {\ifvmode \IgnorePar\fi} {\ifvmode \IgnorePar\fi}
       {\ifvmode \IgnorePar\fi} {\ifvmode \IgnorePar\fi}%
    |<footnote for ams title|>%
   }
   {\ifvmode \IgnorePar\fi \EndP \HCode{<!--end maketitle-->}\par}
   {\NoFonts\IgnorePar \EndP 
    \HCode{<text:p text:style-name="Title\if@rl-rtl\fi">}\IgnorePar
    \bgroup \Configure{HtmlPar}{}{}{}{}}
   {\egroup \HCode{</text:p>}\IgnoreIndent\EndNoFonts}
>>>



\<configure ooffice amsart\><<<
\Configure{thanks author date and} 
   {\par\IgnorePar\EndP \HCode{<text:p text:style-name="thanks">}\par\ShowPar}
   {\ifvmode \IgnorePar\fi\EndP \HCode{</text:p>}}
   {\ifvmode \IgnorePar\fi\EndP
       \HCode{<text:p text:style-name="authors"><text:span text:style-name="author">}}
   {\ifvmode \IgnorePar\fi\EndP \HCode{</text:span></text:p>}}
   {\par\IgnorePar\EndP \HCode{<text:p text:style-name="date">}\par\ShowPar}
   {\ifvmode \IgnorePar\fi\EndP \HCode{</text:p>}}
   {\lowercase{\HCode{</text:span><text:span text:style-name="and">}}and~%
    \lowercase{\HCode{</text:span><text:span text:style-name="xxxx">}}}
   {}%{\HCode{<text:line-break />}}
>>>



\<configure ooffice amsart\><<<
\NewConfigureOO{authors}
\ConfigureOO{authors}{%
<style:style      style:name="authors"
                  style:family="paragraph"
                  >\Hnewline
<style:paragraph-properties  fo:margin-top="0cm"
                  fo:margin-bottom="0.199cm"
                  fo:font-size="14pt"
                  fo:text-align="center"
                  style:justify-single-word="false"/>
</style:style>
}
>>>

\<configure ooffice amsart\><<<
\NewConfigureOO{abstract-title}
\ConfigureOO{abstract-title}{\Hnewline
<style:style style:name="abstract-title" 
           style:family="paragraph"
           style:parent-style-name="abstract-title">
     <style:paragraph-properties
                fo:margin-left="0.3in"
               fo:margin-right="0.2799in"
                fo:text-indent="0in"
                 fo:margin-top="0in"
              fo:margin-bottom="0.0783in"
                 fo:text-align="start"
     style:justify-single-word="false" 
        style:auto-text-indent="false"
     />
</style:style>\Hnewline
}
>>>



\<configure ooffice amsart\><<<
\NewConfigureOO{keywords}
\ConfigureOO{keywords}{\Hnewline
<style:style style:name="keywords"
                  style:family="paragraph"
                  style:parent-style-name="Text-body"
                  style:next-style-name="Text-body">\Hnewline
<style:paragraph-properties  fo:margin-left="0cm"
                  fo:margin-right="0cm"
                  fo:margin-top="0cm"
                  fo:margin-bottom="0.199cm"
                  fo:font-size="10pt"
                  fo:font-style="italic"
                  fo:text-indent="0cm"
                  fo:text-align="left"
                  style:auto-text-indent="false"/>\Hnewline
</style:style>\Hnewline > }
>>>



\<configure ooffice amsart\><<<
\Configure{addresses}
   {\ifvmode \IgnorePar\fi \EndP
     \HCode{<text:p text:style-name="addresses"></text:p>}\ShowPar
     |<start nested pars|>}
   {|<end nested pars|>\ifvmode \IgnorePar\fi \EndP 
    \HCode{<text:p text:style-name="end-addresses"></text:p>}}
\Configure{address}
   {\HCode{<text:span text:style-name="address-title">}}
   {\ifvmode \IgnorePar\fi \HCode{</text:span><text:span text:style-name="address">}}
   {\HCode{</text:span>}}
\Configure{curraddr}
   {\HCode{<text:span text:style-name="curraddr-title">}}
   {\ifvmode \IgnorePar\fi \HCode{</text:span><text:span text:style-name="curraddr">}}
   {\HCode{</text:span>}}
\Configure{email}
   {\HCode{<text:span text:style-name="email-title">}}
   {\ifvmode \IgnorePar\fi \HCode{</text:span><text:span text:style-name="email">}}
   {\HCode{</text:span>}}
\Configure{urladdr}
   {\HCode{<text:span text:style-name="urladdr-title">}}
   {\ifvmode \IgnorePar\fi \HCode{</text:span><text:span text:style-name="urladdr">}}
   {\HCode{</text:span>}}
>>>

\<configure ooffice amsart\><<<
\NewConfigureOO{address}
\ConfigureOO{address}{
<style:style style:name="address" style:family="text">
    <style:text-properties fo:text-transform="capitalize"
    />
</style:style>
}
>>>



\<start nested pars\><<<
\bgroup
\HTML:PAR{nested-p-\ifHCond nopar\else noindent\fi}%
         {nested-p-\ifdim \parindent=\z@ no\fi indent}%
>>>

\<end nested pars\><<<
\ifvmode \IgnorePar \fi \EndP \egroup
>>>

%%%%%%%%%%%%%%%%%%
\section{Lists}
%%%%%%%%%%%%%%%%%%



\<oo+ itemize style rtl\><<<
<text:list-style style:name="Itemize-rtl">\Hnewline
  <text:list-level-style-bullet
      \list:text:level="1" 
      text:style-name="Bullet-Symbols"
      text:bullet-char="\string&\:sharp x2022;" >\Hnewline
     <style:text-properties 
        text:space-before="0.15in" 
        text:min-label-distance="0.05in"/>\Hnewline
  </text:list-level-style-bullet>\Hnewline
  <text:list-level-style-bullet
      \list:text:level="2"
      text:bullet-char="\string&\:sharp x25E6;" >\Hnewline
     <style:text-properties
       text:space-before="0.1972in"
       text:min-label-width="0.2965in"/>\Hnewline
  </text:list-level-style-bullet>\Hnewline
  <text:list-level-style-bullet
      \list:text:level="3"
      text:bullet-char="\string&\:sharp x25AA;" >\Hnewline
     <style:text-properties
       text:space-before="0.3937in"
       text:min-label-width="0.4965in"/>\Hnewline
  </text:list-level-style-bullet>\Hnewline
  <text:list-level-style-bullet
      \list:text:level="4" 
      text:bullet-char="\string&\:sharp x25AB;" >\Hnewline
     <style:text-properties 
       text:space-before="0.5909in" 
       text:min-label-width="0.6965in"/>\Hnewline
  </text:list-level-style-bullet>\Hnewline
</text:list-style>\Hnewline
>>>

\<oo+ itemize style rtl\><<<
<style:style style:name="Inside-itemize-rtl"
    style:family="paragraph" 
    style:list-style-name="Itemize-rtl">\Hnewline
 <style:paragraph-properties 
    fo:text-align="end"
    style:justify-single-word="false"
    style:writing-mode="rl-tb"/>\Hnewline
</style:style>\Hnewline
>>>






\<oo2+ enumerate style rtl\><<<
<text:list-style style:name="Enumerate-rtl">\Hnewline
  <text:list-level-style-number
       text:level="1"
       text:style-name="Numbering-Symbols"
       style:num-prefix="\csname b:enumerate:1\endcsname" 
       style:num-suffix="\csname c:enumerate:1\endcsname" >\Hnewline
    <style:list-level-properties
        text:space-before="0.15in"  
       style:num-format="\csname a:enumerate:1\endcsname"
        text:min-label-distance="0.05in"/>\Hnewline
  </text:list-level-style-number>\Hnewline
  <text:list-level-style-number
       text:level="2"
       text:style-name="Numbering-Symbols"
       style:num-prefix="\csname b:enumerate:2\endcsname" 
       style:num-suffix="\csname c:enumerate:2\endcsname">\Hnewline
    <style:list-level-properties
       text:space-before="0.35in"
       style:num-format="\csname a:enumerate:2\endcsname"
       text:min-label-distance="0.05in"/>\Hnewline
  </text:list-level-style-number>\Hnewline
  <text:list-level-style-number
       text:level="3"
       text:style-name="Numbering-Symbols"
       style:num-prefix="\csname b:enumerate:3\endcsname" 
       style:num-suffix="\csname c:enumerate:3\endcsname">\Hnewline
    <style:list-level-properties
       text:space-before="0.6in" 
       style:num-format="\csname a:enumerate:3\endcsname"
       text:min-label-distance="0.05in"/>\Hnewline
  </text:list-level-style-number>\Hnewline
  <text:list-level-style-number
       text:level="4"
       text:style-name="Numbering-Symbols"
       style:num-prefix="\csname b:enumerate:4\endcsname" 
       style:num-suffix="\csname c:enumerate:4\endcsname">\Hnewline
    <style:list-level-properties
       text:space-before="0.8in"  
       style:num-format="\csname a:enumerate:4\endcsname"
       text:min-label-distance="0.05in"/>\Hnewline
  </text:list-level-style-number>\Hnewline
</text:list-style>\Hnewline
>>>

\<shared oo+ enumerate style rtl\><<<
<style:style style:name="Inside-enumerate-rtl"
   style:family="paragraph" 
   style:list-style-name="Enumerate-rtl">\Hnewline
 <style:paragraph-properties 
      fo:text-align="end"
      style:justify-single-word="false" 
      style:writing-mode="rl-tb"/>\Hnewline
</style:style>\Hnewline
>>>




%%%%%%%%%%%%%%%%%%
\section{Sectioning}
%%%%%%%%%%%%%%%%%%



\<Heading 1 rtl\><<<
<style:style style:name="Heading-1-rtl"
                  style:family="paragraph"
                  style:parent-style-name="Heading-rtl"
                  style:next-style-name="Text-body-rtl"
                  style:class="text">\Hnewline
<style:paragraph-properties  fo:font-size="17.2999992370605pt"
                  fo:font-weight="bold"
                  style:font-size-asian="115\%"
                  style:font-weight-asian="bold"
                  style:font-size-complex="115\%"
                  style:font-weight-complex="bold"
                  fo:text-align="end" 
                  style:justify-single-word="false"
                  style:writing-mode="rl-tb" />
</style:style>
\Hnewline
>>>




\<Heading 2 rtl\><<<
<style:style style:name="Heading-2-rtl"
                  style:family="paragraph"
                  style:parent-style-name="Heading"
                  style:next-style-name="Text-body"
                  style:class="text">\Hnewline
<style:paragraph-properties  fo:font-size="14.3999996185303pt"
                  style:font-size-complex="14pt"
                  style:font-style-complex="italic"
                  style:font-weight-complex="bold"
                  fo:margin-top="0.42cm"
                  fo:margin-bottom="0cm"
                  fo:text-align="end" 
                  style:justify-single-word="false"
                  style:writing-mode="rl-tb" />
</style:style>
\Hnewline
>>>



\<Heading 3 rtl\><<<
<style:style style:name="Heading-3-rtl"
                  style:family="paragraph"
                  style:parent-style-name="Heading-rtl"
                  style:next-style-name="Text-body-rtl"
                  style:class="text">\Hnewline
<style:paragraph-properties  fo:font-size="12pt"
                  fo:font-weight="bold"
                  style:font-size-asian="14pt"
                  style:font-weight-asian="bold"
                  style:font-size-complex="14pt"
                  style:font-weight-complex="bold"
                  fo:text-align="end" 
                  style:justify-single-word="false"
                  style:writing-mode="rl-tb" />
</style:style>
\Hnewline
>>>


\<Heading 4 rtl\><<<
<style:style style:name="Heading-4-rtl"
                  style:family="paragraph"
                  style:parent-style-name="Heading-rtl"
                  style:next-style-name="Text-body-rtl"
                  style:class="text">\Hnewline
<style:paragraph-properties  fo:font-size="12pt"
                  fo:font-weight="bold"
                  fo:margin-top="0.42cm"
                  fo:margin-bottom="0cm"
                  style:font-size-complex="85\%"
                  style:font-style-complex="italic"
                  style:font-weight-complex="bold"
                  fo:text-align="end" 
                  style:justify-single-word="false"
                  style:writing-mode="rl-tb" 
/>
</style:style>
\Hnewline
>>>


\<Heading 5 rtl\><<<
<style:style style:name="Heading-5-rtl"
                  style:family="paragraph"
                  style:parent-style-name="Heading"
                  style:next-style-name="Text-body"
                  style:class="text">\Hnewline
<style:paragraph-properties  fo:margin-left="0cm"
                  fo:margin-right="0cm"
                  fo:margin-top="0.42cm"
                  fo:margin-bottom="0cm"
                  fo:font-size="12pt"
                  fo:font-style="italic"
                  fo:font-weight="normal"
                  style:font-size-complex="85\%"
                  style:font-weight-complex="bold"
                  fo:text-indent="0cm"
                  style:auto-text-indent="false"
                  fo:text-align="end" 
                  style:justify-single-word="false"
                  style:writing-mode="rl-tb" />
</style:style>
\Hnewline
>>>



\<Heading 6 rtl\><<<
<style:style style:name="Heading-6-rtl"
                  style:family="paragraph"
                  style:parent-style-name="Heading-rtl"
                  style:next-style-name="Text-body-rtl"
                  style:class="text">\Hnewline
<style:paragraph-properties  fo:font-size="75\%"
                  fo:font-weight="bold"
                  style:font-size-asian="75\%"
                  style:font-weight-asian="bold"
                  style:font-size-complex="75\%"
                  style:font-weight-complex="bold"
                  fo:text-align="end" 
                  style:justify-single-word="false"
                  style:writing-mode="rl-tb" />
</style:style>
\Hnewline
>>>



\<Heading 7 rtl\><<<
<style:style style:name="Heading-7-rtl"
                  style:family="paragraph"
                  style:parent-style-name="Heading-rtl"
                  style:next-style-name="Text-body-rtl"
                  style:class="text">\Hnewline
<style:paragraph-properties  fo:font-size="75\%"
                  fo:font-weight="bold"
                  style:font-size-asian="75\%"
                  style:font-weight-asian="bold"
                  style:font-size-complex="75\%"
                  style:font-weight-complex="bold"
                  fo:text-align="end" 
                  style:justify-single-word="false"
                  style:writing-mode="rl-tb" />
</style:style>
\Hnewline
>>>


\<Heading 8 rtl\><<<
<style:style style:name="Heading-8-rtl"
                  style:family="paragraph"
                  style:parent-style-name="Heading-rtl"
                  style:next-style-name="Text-body-rtl"
                  style:class="text">\Hnewline
<style:paragraph-properties  fo:font-size="75\%"
                  fo:font-weight="bold"
                  style:font-size-asian="75\%"
                  style:font-weight-asian="bold"
                  style:font-size-complex="75\%"
                  style:font-weight-complex="bold"
                  fo:text-align="end" 
                  style:justify-single-word="false"
                  style:writing-mode="rl-tb" />
</style:style>
\Hnewline
>>>



\<Heading 9 rtl\><<<
<style:style style:name="Heading-9-rtl"
                  style:family="paragraph"
                  style:parent-style-name="Heading"
                  style:next-style-name="Text-body"
                  style:class="text">\Hnewline
<style:paragraph-properties  fo:font-size="75\%"
                  fo:font-weight="bold"
                  style:font-size-asian="75\%"
                  style:font-weight-asian="bold"
                  style:font-size-complex="75\%"
                  style:font-weight-complex="bold"
                  fo:text-align="end" 
                  style:justify-single-word="false"
                  style:writing-mode="rl-tb" />
</style:style>
\Hnewline
>>>


\<Heading 10 rtl\><<<
<style:style style:name="Heading-10-rtl"
                  style:family="paragraph"
                  style:parent-style-name="Heading-rtl"
                  style:next-style-name="Text-body-rtl"
                  style:class="text">\Hnewline
<style:paragraph-properties  fo:font-size="75\%"
                  fo:font-weight="bold"
                  style:font-size-asian="75\%"
                  style:font-weight-asian="bold"
                  style:font-size-complex="75\%"
                  style:font-weight-complex="bold"/>
</style:style>
\Hnewline
>>>

\subsection{mdwlist}

\<configure ooffice mdwlist\><<<

\ConfigureList{itemize*}%
   {\EndP\HCode{|<input line num|><\text:unorderedlist
      text:style-name="Itemize|<cond rtl class|>"
        >\Hnewline}\def\end@Item{}\bgroup
     \Configure{HtmlPar}
        {\EndP\HCode{<text:p
            text:style-name="Inside-itemize|<cond rtl class|>"
                     >}}
        {\EndP\HCode{<text:p
            text:style-name="Inside-itemize|<cond rtl class|>"
                      >}}
        {\HCode{</text:p>\Hnewline}}
        {\HCode{</text:p>\Hnewline}}%
     \gHAdvance\BegEnd:D by 1
   }
   {\gHAdvance\BegEnd:D by -1
    \EndP\IgnorePar \egroup \HCode{</text:list-item>\Hnewline
      </\text:unorderedlist>\Hnewline}\ShowPar}
   {\EndP\end@Item\DeleteMark}
   {\HCode{<text:list-item>}\par\ShowPar
      \def\end@Item{\HCode{</text:list-item>\Hnewline}}}

\ConfigureList{enumerate*}%
   {\EndP\HCode{<\text:orderedlist
      text:style-name="Enumerate|<cond rtl class|>"
       >\Hnewline}\def\end@Item{}\bgroup
     \Configure{HtmlPar}
      {\EndP\HCode{<text:p
           text:style-name="Inside-enumerate|<cond rtl class|>"
            >}}
      {\EndP\HCode{<text:p
           text:style-name="Inside-enumerate|<cond rtl class|>"
             >}}
      {\HCode{</text:p>\Hnewline}}     {\HCode{</text:p>\Hnewline}}}
   {\EndP\IgnorePar \egroup \HCode{</text:list-item>\Hnewline
     </\text:orderedlist>\Hnewline}\ShowPar}
   {\EndP\end@Item\DeleteMark}
   {\HCode{<text:list-item>}\par\ShowPar
     \def\end@Item{\HCode{</text:list-item>\Hnewline}}}
>>>

%%%%%%%%%%%%%%%%%%
\section{Verbatim}
%%%%%%%%%%%%%%%%%%



\<Preformatted Text rtl\><<<
<style:style style:name="Preformatted-Text-rtl" 
             style:family="paragraph"
             style:parent-style-name="Standard">\Hnewline
   <style:paragraph-properties 
%      fo:font-family="monospace"
%      fo:wrap-option="no-wrap"
%      style:font-name="Courier"
      fo:margin-top="0.2cm" 
      fo:margin-bottom="0.2cm" 
      fo:text-indent="0cm"
                  fo:text-align="end" 
                  style:justify-single-word="false"
                  style:writing-mode="rl-tb" 
   />\Hnewline
   <style:text-properties fo:font-family="monospace"/>\Hnewline    
</style:style>\Hnewline  
>>>


%%%%%%%%%%%%%%%%%%
\section{Paragraphs}
%%%%%%%%%%%%%%%%%%


\<First-line-indent rtl\><<<
<style:style style:name="First-line-indent-rtl"
                  style:family="paragraph"
                  style:parent-style-name="Text-body"
                  style:class="text">\Hnewline
<style:paragraph-properties  fo:margin-left="0cm"
                  fo:margin-right="0cm"
                  fo:text-indent="0.499cm"
                  style:auto-text-indent="false"
                  fo:text-align="end" 
                  style:justify-single-word="false"
                  style:writing-mode="rl-tb"  />
</style:style>\Hnewline
>>>


\<Text-body rtl\><<<
<style:style style:name="Text-body-rtl"
                  style:family="paragraph"
                  style:parent-style-name="Standard"
                  style:next-style-name="First-line-indent"
                  style:class="text"
                  style:master-page-name="">\Hnewline
<style:paragraph-properties  fo:margin-top="0cm"
                  fo:margin-bottom="0cm"
                  fo:line-height="120\%"
                  style:justify-single-word="false"
                  fo:hyphenate="true"
                  fo:hyphenation-remain-char-count="2"
                  fo:hyphenation-push-char-count="2"
                  fo:hyphenation-ladder-count="no-limit"
                  style:page-number="0"
                  fo:text-align="end" 
                  style:writing-mode="rl-tb"  />
</style:style>
\Hnewline
<style:style style:name="Like-Text-body-rtl"
                  style:family="paragraph"
                  style:parent-style-name="Standard"
                  style:next-style-name="First-line-indent"
                  style:class="text"
                  style:master-page-name="">\Hnewline
<style:paragraph-properties  fo:margin-top="0cm"
                  fo:margin-bottom="0cm"
                  fo:line-height="120\%"
                  style:justify-single-word="false"
                  fo:hyphenate="true"
                  fo:hyphenation-remain-char-count="2"
                  fo:hyphenation-push-char-count="2"
                  fo:hyphenation-ladder-count="no-limit"
                  style:page-number="0"                  
                  fo:text-align="end" 
                  style:writing-mode="rl-tb"  />
</style:style>
\Hnewline
>>>



\<p-nopar rtl\><<<
    <style:style style:name="p-nopar-rtl" 
         style:family="paragraph" 
         style:parent-style-name="Text-body">\Hnewline
      <style:paragraph-properties  
         style:auto-text-indent="false"
         fo:margin-top="0.3cm" 
         fo:margin-bottom="0.3cm"
         style:justify-single-word="false"
         fo:text-align="end" 
         style:writing-mode="rl-tb"  />\Hnewline
    </style:style>\Hnewline
>>>


%%%%%%%%%%%%%%%%%%
\section{List-Based Environments}
%%%%%%%%%%%%%%%%%%



\<quote rtl\><<<
<style:style style:name="quote-rtl"
             style:family="paragraph"
             style:parent-style-name="Text-body-rtl"
             style:next-style-name="Text-body-rtl">\Hnewline
<style:paragraph-properties  fo:margin-left="1cm"
             fo:margin-right="1cm"
             fo:margin-top="0.199cm"
             fo:margin-bottom="0.199cm"
             fo:text-indent="0cm"
             style:auto-text-indent="false"
             fo:text-align="end" 
             style:writing-mode="rl-tb"  />\Hnewline
</style:style>
\Hnewline
>>>


\<quotation rtl\><<<
      <style:style style:name="quotation-rtl"
                   style:family="paragraph" 
                   style:parent-style-name="Standard">\Hnewline
        <style:paragraph-properties 
                   fo:margin-left="1cm"
                   fo:margin-right="1cm"
                   fo:margin-top="0cm"
                   fo:margin-bottom="0cm"
                   fo:text-indent="0.499cm"
                   style:auto-text-indent="true"
                   fo:text-align="end" 
                   style:writing-mode="rl-tb" />\Hnewline
      </style:style>\Hnewline
>>>





%%%%%%%%%%%%%%%%%%
\section{Math}
%%%%%%%%%%%%%%%%%%



\<mathml rlbabel\><<<
\if@rl\else\beginR\fi
>>>


%%%%%%%%%%%%%%%%%%
\chapter{Math}
%%%%%%%%%%%%%%%%%%


%%%%%%%%%%%%%%%%%%
\section{Envelops}
%%%%%%%%%%%%%%%%%%

\<configure mathml rlbabel\><<<
\Configure{[]}
  {\Configure{@math}{|<block display|>}\csname a:mathml[]\endcsname
   \CenteredFrametrue \DviMath \CenteredFramefalse 
   \Tg<\a:mathml mrow\Hnewline>$$\DisplayMathtrue }
  {$$\Tg</\a:mathml mrow>\EndDviMath\csname b:mathml[]\endcsname
   |<mathml rlbabel|> }
>>>


\<block display\><<<
% \a:mathml display="block"
>>>


\<inline display\><<<
%\a:mathml display="inline"
>>>



%%%%%%%%%%%%%%%%%%%%%%%%%%%%%%%
\section{Underline and Overline}
%%%%%%%%%%%%%%%%%%%%%%%%%%%%%%%


\<configure ooffice-mml latex\><<<
\Configure{overline}
   {\bgroup\ifmathml |<mathml overline|>\else
         \HCode{<text:span text:style-name="overline"  >}\fi}
   {\ifmathml |<end mathml overline|>\else \HCode{</text:span>}\fi\egroup}
\Configure{underline}
   {\bgroup
    \ifmathml |<mathml underline|>%
    \else |<nomath underline|>\HCode{<text:span text:style-name="underline"  >}\fi}
   {\ifmathml |<end mathml underline|>\else \HCode{</text:span>}\fi \egroup}
>>>



\<configure ooffice-mml ushort \><<<
\Configure{ushort}
   {\bgroup
    \ifmathml |<mathml underline|>%
    \else |<nomath underline|>\HCode{<text:span text:style-name="underline"  >}\fi}
   {\ifmathml |<end mathml underline|>\else \HCode{</text:span>}\fi \egroup}
\Configure{ushortw}
   {\bgroup
    \ifmathml |<mathml underline|>%
    \else |<nomath underline|>\HCode{<text:span text:style-name="underline"  >}\fi}
   {\ifmathml |<end mathml underline|>\else \HCode{</text:span>}\fi \egroup}
\Configure{ushortd}
   {\bgroup
    \ifmathml |<mathml underline|>%
    \else |<nomath underline|>\HCode{<text:span text:style-name="underline"  >}\fi}
   {\ifmathml |<end mathml underline|>\else \HCode{</text:span>}\fi \egroup}
\Configure{ushortdw}
   {\bgroup
    \ifmathml |<mathml underline|>%
    \else |<nomath underline|>\HCode{<text:span text:style-name="underline"  >}\fi}
   {\ifmathml |<end mathml underline|>\else \HCode{</text:span>}\fi \egroup}
\Configure{ushortdline}
   {\bgroup
    \ifmathml |<mathml underline|>%
    \else |<nomath underline|>\HCode{<text:span text:style-name="underline"  >}\fi}
   {\ifmathml |<end mathml underline|>\else \HCode{</text:span>}\fi \egroup}
>>>



\<mathml overline\><<<
\HCode{<\a:mathml mover  accent="false"\Hnewline
\mml:class="mml-overline"><\a:mathml mrow>}%
>>>

\<end mathml overline\><<<
\HCode{</\a:mathml mrow><\a:mathml mo\Hnewline
 accent="true">&\#x00AF;</\a:mathml mo>%    
</\a:mathml mover>}%
>>>


\<mathml underline\><<<
\HCode{<\a:mathml munder \a:mathml accentunder="false"\Hnewline
\mml:class="mml-underline"><\a:mathml mrow>}% 
>>>

\<nomath underline\><<<
\expandafter\everymath
\expandafter{\expandafter\everymath
             \expandafter{\the\everymath}}%
\let\o::@underline:\o:@@underline:
\def\o:@@underline:{\let\o:@@underline:\o::@underline:}%
>>>

\<end mathml underline\><<<
\HCode{</\a:mathml mrow><\a:mathml mo\Hnewline
accent="true">&\#x0332;</\a:mathml mo></\a:mathml munder>}%
>>>



\<configure ooffice fontmath\><<<
\NewConfigureOO{overline}
\ConfigureOO{overline}{|<oo overline style|>}
\NewConfigureOO{underline}
\ConfigureOO{underline}{|<oo underline style|>}
>>>

\<oo underline style\><<<
<style:style style:name="underline" style:family="text">
    <style:text-properties style:text-underline-style="solid"
                           style:text-underline-width="auto"
                           style:text-underline-color="font-color"
    />
</style:style>
>>>

Overline??

\<oo overline style\><<<
<style:style style:name="overline" style:family="text">
   <style:text-properties style:text-underline="dotted" />
</style:style>
>>>


%%%%%%%%%%%%%%%%%%
\section{mfenced}
%%%%%%%%%%%%%%%%%%

OO doesn't seem to hono mfenced 

\begin{verbatim}
\documentclass{article} 
\begin{document} 
\[ 
\sin\left(x\right)\] 
\end{document} 
\end{verbatim}

\<configure ooffice-mml latex\><<<
>>>

\<configure ooffice-mml plain\><<<
>>>






\<mfenced patches\><<<
\Configure{left}
   {}
   {\HCode{<\a:mathml mrow>\expandafter
       \ifx \csname \string#1:lr del\endcsname\relax 
                 <\a:mathml mo form="prefix" >#1</\a:mathml mo\space>\else
          \expandafter\ifx \csname \string#1:lr del\endcsname\empty \else
              <\a:mathml mo form="prefix" >\csname
                             \string#1:lr del\endcsname</\a:mathml mo>\fi
       \fi<\a:mathml mrow>}}
>>>





\<mfenced patches\><<<
\Configure{right}
   {\HCode{</\a:mathml mrow>\expandafter
       \ifx \csname \string#1:lr del\endcsname\relax 
            <\a:mathml mo form="postfix" >#1</\a:mathml mo>\else
          \expandafter\ifx \csname \string#1:lr del\endcsname\empty \else
            <\a:mathml mo form="postfix" >\csname
                             \string#1:lr del\endcsname</\a:mathml mo>\fi
       \fi
      }}
   {\HCode{</\a:mathml mrow>}}
>>>


%%%%%%%%%%%%%%%%%%%%%%%%%%%%%%%%%%%%%%%%%%%%%%%%%%%%%%%%%%%%%%%%%%%%%%%%
\chapter{???????????????????????}
%%%%%%%%%%%%%%%%%%%%%%%%%%%%%%%%%%%%%%%%%%%%%%%%%%%%%%%%%%%%%%%%%%%%%%%%



%%%%%%%%%%%%%%%%%%
\section{Long Tables}
%%%%%%%%%%%%%%%%%%


\<configure ooffice longtable\><<<
\Configure{halignTB<>}{longtable}{\HCode{\Hnewline
     table:style-name="longtable"
     }<>\HAlign}
\Configure{longtable}
   {\IgnorePar\EndP  \HCode{|<show input line no|>}%
    \halignTB{longtable}%
    |<list the grouping of columns|>}
   {|<insert Clr|>\HCode{</table:table>}}
   {\HCode{<table:table-row \Hnewline}\halignTR\HCode{ 
                              table:style-name="TBL-\TableNo-\HRow-">}}
   {\HCode{</table:table-row>}}
   {\HCode{<table:table-cell 
             \ifnum \HMultispan>1 table:number-columns-spanned="\HMultispan"\fi
           \space  table:style-name="TBL-\TableNo-\HRow-\HCol" ><text:p>}|<td save EndP|>%
    |<start array par box|>\PushStack\Table:Stck\TableNo}
   {\PopStack\Table:Stck\TableNo
    |<end array par box|>\HCode{</text:p></table:table-cell>}}
>>>






\<show input line no\><<<
<!--l. \the\inputlineno-->%
>>>



\<td save EndP\><<<
\SaveEndP 
>>>

\<td recall EndP\><<<
\RecallEndP 
>>>


%%%%%%%%%%%%%%%%%%
\section{Curve}
%%%%%%%%%%%%%%%%%%










\<configure ooffice curve\><<<
\ConfigureEnv{rubric} 
   {\bgroup |<rubric table|>} {\egroup} {} {}
>>>


\<rubric table\><<<
\Configure{table-column-style}
   {table:style-name="|<rubric col num|>"}
\Configure{longtable}
   {\IgnorePar\EndP  \HCode{|<show input line no|>}%
    \halignTB{longtable}%
    |<list the grouping of columns|>}
   {|<insert Clr|>\HCode{</table:table>}}
   {\HCode{<table:table-row \Hnewline}\halignTR\HCode{ 
                              table:style-name="TBL-\TableNo-\HRow-">}}
   {\HCode{</table:table-row>}}
   {\HCode{<table:table-cell 
             \ifnum \HMultispan>1 table:number-columns-spanned="\HMultispan"\fi
           \space  table:style-name="|<rubric cell num|>" ><text:p
                   text:style-name="|<rubric par num|>"  >}|<td save EndP|>%
    |<start array par box|>\PushStack\Table:Stck\TableNo}
   {\PopStack\Table:Stck\TableNo
    |<end array par box|>\HCode{</text:p></table:table-cell>}}
>>>

\<rubric col num\><<<
\ifcase \ar:cnt\relax
\or  %1
   keyCol%   
\or  %2
   markCol%
\else
   bodyCol%   TBL-\TableNo -\ar:cnt
\fi
>>>

\<rubric cell num\><<<
\ifcase \HMultispan\relax
\or  %1
   \ifcase \HCol\relax
   \or %1
         keyCell%
   \or %2
         markCell%
   \else
         TBL-\TableNo-\HRow-\HCol
   \fi
\or  %2
   \ifnum \HCol=2
         subrubricCell%
   \else
         TBL-\TableNo-\HRow-\HCol
   \fi
\or  %3
   \ifnum \HCol=1
         rubricheadCell%
   \else
         TBL-\TableNo-\HRow-\HCol
   \fi
\else    
   TBL-\TableNo-\HRow-\HCol
\fi
>>>


\<rubric par num\><<<
\ifcase \HMultispan\relax
\or  %1
   \ifcase \HCol\relax
   \or %1
         keyP%
   \or %2
         markP%
   \else
         TBL-\TableNo-\HRow-\HCol
   \fi
\or  %2
   \ifnum \HCol=2
         subrubricP%
   \else
         TBL-\TableNo-\HRow-\HCol
   \fi
\or  %3
   \ifnum \HCol=1
         rubricheadP%
   \else
         TBL-\TableNo-\HRow-\HCol
   \fi
\else    
   TBL-\TableNo-\HRow-\HCol
\fi
>>>



\<configure ooffice curve\><<<
\OOstyles{%
\Hnewline
<style:style  
          style:name="markCol"  
          style:family="table-column"  
          > \Hnewline
     <style:table-column-properties
          style:rel-column-width="180*"/> 
</style:style> \Hnewline
<style:style style:name="keyCol" style:family="table-column"> 
      <style:table-column-properties
          style:rel-column-width="1546*"/> 
</style:style> \Hnewline
<style:style style:name="bodyCol" style:family="table-column"> 
      <style:table-column-properties
          style:rel-column-width="6914*"/> 
</style:style> \Hnewline
}
>>>



\<configure ooffice curve\><<<
\Configure{maketitle} 
  {}{} 
  {\ifvmode \IgnorePar\fi \EndP 
   \HCode{<text:h text:style-name="curveTitle" text:outline-level="1"
             >}} 
  {\ifvmode \IgnorePar\fi \EndP \HCode{</text:h>}} 
  {\ifvmode \IgnorePar\fi \EndP 
   \HCode{<text:h text:style-name="curveSubTitle" text:outline-level="2"
             >}} 
  {\ifvmode \IgnorePar\fi \EndP \HCode{</text:h>}} 
>>>



\<configure ooffice curve\><<<
\OOstyles{%
\Hnewline
<style:style  
          style:name="curveTitle"  
          style:family="paragraph"  
          > \Hnewline
    <style:paragraph-properties  
          fo:text-align="center"  
          style:justify-single-word="false"
          fo:margin-top="0.4cm"
          fo:margin-bottom="0.1cm"
    /> \Hnewline
    <style:text-properties 
          fo:font-size="145\%"  
          fo:font-weight="bold"  
    /> \Hnewline
</style:style> \Hnewline
}
>>>



\<configure ooffice curve\><<<
\OOstyles{%
\Hnewline
<style:style  
          style:name="curveSubTitle"  
          style:family="paragraph"  
          > \Hnewline
    <style:paragraph-properties  
          fo:text-align="center"  
          style:justify-single-word="false"
          fo:margin-top="0.1cm"
          fo:margin-bottom="0.2cm"
    /> \Hnewline
    <style:text-properties 
          fo:font-size="130\%"  
          fo:font-weight="normal"  
          fo:font-style="italic"
    /> \Hnewline
</style:style> \Hnewline
}
>>>


\<configure ooffice curve\><<<
\Configure{makeheaders}
   {\ifvmode \IgnorePar\fi \EndP 
    \HCode{%
      <table:table table:name="makeheaders" > 
        <table:table-column table:number-columns-repeated="2"/> 
          <table:table-row> 
            <table:table-cell table:style-name="leftheader"
                              office:value-type="string"> 
              <text:p text:style-name="Pleft"  >}}
    {\HCode{</text:p> 
            </table:table-cell> 
            <table:table-cell table:style-name="rightheader" 
                              office:value-type="string"> 
              <text:p text:style-name="Pright"  >}}
    {\HCode{</text:p> 
            </table:table-cell> 
          </table:table-row> 
      </table:table>}}
>>>

\<configure ooffice curve\><<<
\OOstyles{%
<style:style 
          style:name="leftheader"
          style:family="table-cell"> 
      <style:table-cell-properties
          fo:padding="0.0382in" 
          fo:border="none"/> 
</style:style> 
<style:style 
          style:name="rightheader"
          style:family="table-cell"> 
    <style:table-cell-properties
          fo:padding="0.0382in"
          fo:border="none"/> i
</style:style> 
<style:style style:name="Pleft" 
             style:family="paragraph" > 
      <style:paragraph-properties 
            fo:text-align="left" 
            style:justify-single-word="false"/> 
      <style:text-properties
          fo:font-style="normal" /> 
</style:style> 
<style:style style:name="Pright" 
             style:family="paragraph" > 
      <style:paragraph-properties 
            fo:text-align="end" 
            style:justify-single-word="false"/> 
      <style:text-properties
          fo:font-style="normal" /> 
</style:style> 
}
>>>



\<configure ooffice curve\><<<
\Configure{rubrichead} 
  {\ifvmode \IgnorePar\fi \EndP 
   \HCode{<text:span text:style-name="rubrichead"  >}} 
  {\ifvmode \IgnorePar\fi \EndP \HCode{</text:span>}} 
>>>


\<configure ooffice curve\><<<
\OOstyles{%
\Hnewline
<style:style  
          style:name="rubrichead"  
          style:family="text"  
          > \Hnewline
    <style:text-properties 
          fo:font-size="110\%"  
          fo:font-weight="bold"  
    /> \Hnewline
</style:style> \Hnewline
}
>>>



\<configure ooffice curve\><<<
\OOstyles{%
\Hnewline
<style:style  
          style:name="rubricheadP"  
          style:family="paragraph"  
          > \Hnewline
    <style:paragraph-properties  
          fo:text-align="center"  
          style:justify-single-word="false"
          fo:margin-top="0.2cm"
          fo:margin-bottom="0.2cm"
    /> \Hnewline
</style:style> \Hnewline
}
>>>




\<configure ooffice curve\><<<
\Configure{subrubric} 
  {\ifvmode \IgnorePar\fi \EndP 
   \HCode{<text:span text:style-name="subrubric"  >}} 
  {\ifvmode \IgnorePar\fi \EndP \HCode{</text:span>}} 
>>>


\<configure ooffice curve\><<<
\OOstyles{%
\Hnewline
<style:style  
          style:name="subrubric"  
          style:family="text"  
          > \Hnewline
    <style:text-properties 
          fo:font-weight="normal"  
          fo:font-style="italic"
          fo:font-size="130\%"  
    /> \Hnewline
</style:style> \Hnewline
}
>>>

%%%%%%%%%%%%%%%%%%%%%%%%%%%%%%%%%%%%%%%%%%%%%%%%%%%%%%%%%%%%%%%%%%%%%%%%
\section{wrapfig}
%%%%%%%%%%%%%%%%%%%%%%%%%%%%%%%%%%%%%%%%%%%%%%%%%%%%%%%%%%%%%%%%%%%%%%%%

\<configure ooffice wrapfig\><<<
\pend:def\WFclear{%
\let\orig:par\par%
\def\par{\let\par\orig:par}%
}%
\append:def\WFclear{\let\par\orig:par}

\Configure{wrapfloat}
    {\ifvmode\IgnorePar\EndP
       \HCode{<text:p><text:span text:class-name="wrapfig-\WFplace">}%
       \def\endWrap{\ifvmode\IgnorePar\fi\EndP\HCode{</text:span></text:p>}\par}%
     \else
       \HCode{<text:span text:class-name="wrapfig-\WFplace">}\bgroup
       \def\endWrap{\egroup\HCode{</text:span>}}%
     \fi
    }
    {\endWrap}
>>>>

%%%%%%%%%%%%%%%%%%%%%%%%%%%%%%%%%%%%%%%%%%%%%%%%%%%%%%%%%%%%%%%%%%%%%%%%%
\section{subscript}
%%%%%%%%%%%%%%%%%%%%%%%%%%%%%%%%%%%%%%%%%%%%%%%%%%%%%%%%%%%%%%%%%%%%%%%%%

\<configure ooffice subscript\><<<
\Configure{textsubscript}
  {\HCode{<text:span text:style-name="textsubscript">}}
  {\HCode{</text:span>}} 
\OOstyles{%
<style:style style:name="textsubscript" style:family="text">\Hnewline
   <style:text-properties style:text-position="sub"/>\Hnewline
</style:style>\Hnewline
}
>>>

%%%%%%%%%%%%%%%%%%%%%%%%%%%%%%%%%%%%%%%%%%%%%%%%%%%%%%%%%%%%%%%%%%%%%%%%
\chapter{Notes}
%%%%%%%%%%%%%%%%%%%%%%%%%%%%%%%%%%%%%%%%%%%%%%%%%%%%%%%%%%%%%%%%%%%%%%%%




%%%%%%%%%%%%%%%%%%%%%%
\section{References}
%%%%%%%%%%%%%%%%%%%%%%

\begin{itemize}
\item Accessibility

\begin{itemize}

\item Peter Korn's Weblog,  Testing ODF document accessibility
\url{http://blogs.sun.com/korn/entry/testing_odf_document_accessibility_and}

\item
eDesigner,
\url{http://www.alphaworks.ibm.com/tech/adesigner}


\item
OASIS OpenDocument - Accessibility
\url{http://www.oasis-open.org/committees/tc_home.php?wg_abbrev=office-accessibility}
\end{itemize}


\item \href{http://books.evc-cit.info/odbook/book.html}{OASIS OpenDocument Essentials}

\item OpenDocument:
\href{http://www.oasis-open.org/committees/download.php/12572/OpenDocument-v1.0-os.pdf}{OASIS}

\item OpenDocument:
\href{http://www.oasis-open.org/specs/index.php}{OASIS} 



DTD?, relaxng schema provided

\item OpenOffice : 
  \href{http://xml.openoffice.org/source/browse/xml/xmloff/dtd/}{DTD},
  \href{http://books.evc-cit.info/book.php}{book},
  \href{http://books.evc-cit.info/odbook/book.html}{OASIS OpenDocument Essentials},
  \href{http://books.evc-cit.info/}{OpenOffice.org XML Essentials} book,
  \href{http://xml.openoffice.org/xml_specification.pdf}{OpenOffice.org XML 
   File Format
     1.0} Technical Reference Manual Version 2 December 2002,
  \href{http://www.xml.com/pub/a/2001/02/07/openoffice.html}{Adventures
    with OpenOffice and XML} (Matt Sergeant ),
  \href{http://xml.openoffice.org/}{home page},
  \href{http://www.ccsd1.k12.wy.us/Training/Help_Desk/StarOffice/816-7367.pdf}{user's
    guide},
  \href{http://xml.coverpages.org/OpenOfficeXMLSpec200207.pdf}{Technical
    reference},
  \href{http://ooo.ximian.com/lxr/source/xml/xmloff/dtd/}{some online
    dtds}

\item
Publication:
Maarten Wisse, "Hacking TEX4ht for XML Output: The Road towards a TEX
to Word Convertor", MAPS 28 (2002), pp. 28-35.
(\url{http://home.wanadoo.nl/pmwisse/hpmwissese3.html}.)

\item Samples: 
   \url{http://www.xml.com/2001/02/07/adventure.sxw},
   \url{http://cvs.berlios.de/cgi-bin/viewcvs.cgi/od2f/study/content.xml?annotate=1.2}

\item
\url{http://www.oasis-open.org/committees/download.php/10765/office-spec-1.0-cd-2.pdf}
\url{http://xml.coverpages.org/ni2005-01-04-a.htm}

\item
\url{http://docs.oasis-open.org/office/v1.1/OS/}

\item See NeoOffice/J for Macs

\item
Sun ODF Plugin 1.1 for Microsoft Office:
\url{http://www.sun.com/software/star/odf_plugin/}
\end{itemize}

%%%%%%%%%%%%%%%%%%% 
\section{To Do}
%%%%%%%%%%%%%%%%%%%

\begin{itemize}
\item Activate groups of columns?
\item Vertical Borders

Why the following doesn't work?

\begin{verbatim}
\OOstyles{%
  <style:style style:name="hline-cell" style:family="table-cell">
     <style:text-properties fo:padding="0.097cm"
                       fo:border-left="none"
                       fo:border-right="none"
                       fo:border-top="none"
                       fo:border-bottom="0.002cm solid \#000000"/>
  </style:style>
}
\end{verbatim}
  
\item Margins on sides of tables
\end{itemize}


%%%%%%%%%%%%%%%%%%%%%%%%%%%%%%%%
\section{Writer2LaTeX}
%%%%%%%%%%%%%%%%%%%%%%%%%%%%%%%%

\begin{verbatim}
Another thing with built-in styles in OOo: You use the style "Quotations",
but the definition is not included in styles.xml. This is not a problem
for OOo, but it will be a problem for other applications reading the file
(Writer2LaTeX, KWord,...). To fix that you should always include the
definition of a style.


w2l in-file
\end{verbatim}


%%%%%%%%%%%%%%%%%%%%%%%%%%%%%%%%
\section{Supporting Files}
%%%%%%%%%%%%%%%%%%%%%%%%%%%%%%%%

\begin{verbatim}
The .sxw file is a .JAR file (which is basically the same as a .ZIP file). 

The manifest.xml file is the difference between .ZIP and .JAR,
apparently. It makes searching the files easier, and the documentation
claims that this file is uncompressed, so that it can be easily
read. I have not tested this, but it seems like a good idea.

content.xml - the text of the document 
styles.xml - the styles used by the document 
meta.xml - various meta-information (author, etc) 
settings.xml - the Word Processor settings used. 
META-INF/manifest.xml - a list of all these files, plus any others used in the document 

\end{verbatim}

%%%%%%%%%%%%%%%%%%
\section{How-To Examples}
%%%%%%%%%%%%%%%%%%


%%%%%%%%%%%%%
\subsection{New Pages}
%%%%%%%%%%%%%



\<configure ooffice latex\><<<
\Configure{newpage}{\par \EndP \IgnorePar 
   \HCode{<text:p text:style-name="newpage"></text:p>}% 
   \par\ShowPar}  
\OOstyles{% 
  <style:style style:name="newpage" style:family="paragraph" > 
      <style:paragraph-properties fo:break-after="page"/> 
  </style:style> 
} 
>>>

 





\endinput
