% $Id$
% compile 3 times: latex tex4ht-info-html4   
%           or   xhlatex tex4ht-info-html4 "html,3,sections+"
%
% Copyright 2009-2017 TeX Users Group
% Copyright 2000-2009 Eitan M. Gurari
% Released under LPPL 1.3c+.
% See tex4ht-cpright.tex for license text.

\ifx \HTML\UnDef
   \def\HTML{infoht4}
   \def\CONFIG{\jobname}
   \def\MAKETITLE{\author{Eitan M. Gurari}}
   \def\next{\input mktex4ht.4ht   \endinput}
   \expandafter\next
\fi

% $Id$
% Common TeX definitions used only in the *-info.tex literate sources.
% Not installed.
% 
% Copyright 2009-2017 TeX Users Group
% Copyright 1996-2009 Eitan M. Gurari
% Released under LPPL 1.3c+.
% See tex4ht-cpright.tex for license text.

\expandafter\ifx \csname YES\HTML\endcsname\relax
    % begin comment. 21/07/2016 (dg)
    %   on first run \infoIVht expects \ConfigureHinput
    %   ( \def\infoIVht#1\ConfigureHinput{..} )
    %   so we feed it with "\ConfigureHinput" (no expansion here, merely
    %   a delimiter); the rest is slurped until the "//".
    % end
    \def\CleanComment{[0]\ConfigureHinput\id:gobble}
\else
    \let\saveCd=\<
    \def\<{\edef\FIRST{\the\inputlineno}\let\<\saveCd \saveCd}
    %
    % Eitan's commented-out definition started like this (and doesn't work):
    %\def\CleanComment#1tex4ht-info#2#3#4.#5>#6//{[\eatIV#4%     <--jobname
    %
    % Eitan's active definition started like this:
    %\def\CleanComment#1tex4ht-info#2#3#4.#5>#6//{[#4%     <--jobname
    %
    % However, that didn't work either.  #4 is not the jobname. 
    % The arguments when running htlatex tex4ht-info-mml.tex are these:
    % %#1<-
    %\CleanComment #1tex4ht-info#2#3#4.#5>#6//->
    %[\if ,\ifnum \FIRST =#6 .\else ,0\fi
    %#1<-
    %#2<--
    %#3<-m
    %#4<-ml
    %#5<-html#QPrTx1"\<infomml\
    %#6<-92\ifx \CodeId \:gobbleii \else ...\fi 
    %
    % As a result, when running  mzlatex hello.tex xhtml,info  there was
    % an error on the first line of infomml.4ht, which looked like this:
    % \ifx\infoIVht\UnDeF\def\infoIVht#1//{}\fi\infoIVht[ml0]28...//
    % That "ml" is not a number, so \ifnum fails.  This only happens
    % mzlatex and the info option, not htlatex.  We don't understand.
    %
    % This version avoids the spurious "ml" but mzlatex hello.tex still
    % fails, trying to process the \ConfigureHinput blocks as text.
    % Changing the bracketed number in infomml.4ht to small values seems
    % to make it pass, but can't see how to generate it.  The number
    % after the brackets (#6) changes also.
    % 
    % Since all this is only about the info option with mzlatex,
    % just leaving it failing for now.  Other things to do.
    \def\CleanComment#1tex4ht-info#2#3#4.#5>#6//{[1\empty %
                  \if,\ifnum \FIRST=#6 .\else ,0\fi\fi]#6//}
    \def\eatIV#1#2#3#4{}
\fi

\Comment{

\string\ifx\string\infoIVht\string\UnDeF\string\def\string\infoIVht#1//{}\string\fi\string\infoIVht\CleanComment}{//

}

\def\>>>#1<<<{\bgroup\csname no:catcodes\endcsname0{255}{12}%
   \csname no:catcodes\endcsname{13}{13}{13}% ^^M
   \def\temp##1>>>{\egroup
      \expandafter \def\csname #1\endcsname{##1}}\temp}

% $Id$
% A few common TeX definitions for literate sources.  Not installed in runtime.
% 
% Copyright 2009-2017 TeX Users Group
% Copyright 1996-2009 Eitan M. Gurari
%
% This work may be distributed and/or modified under the
% conditions of the LaTeX Project Public License, either
% version 1.3c of this license or (at your option) any
% later version. The latest version of this license is in
%   http://www.latex-project.org/lppl.txt
% and version 1.3c or later is part of all distributions
% of LaTeX version 2005/12/01 or later.
%
% This work has the LPPL maintenance status "maintained".
%
% The Current Maintainer of this work
% is the TeX4ht Project <http://tug.org/tex4ht>.
% 
% If you modify this program, changing the 
% version identification would be appreciated.

\newcount\tmpcnt  \tmpcnt\time  \divide\tmpcnt  60
\edef\temp{\the\tmpcnt}
\multiply\tmpcnt  -60 \advance\tmpcnt  \time

\edef\version{\the\year-\ifnum \month<10 0\fi
  \the\month-\ifnum \day<10 0\fi\the\day
   -\ifnum \temp<10 0\fi \temp
   :\ifnum \tmpcnt<10 0\fi\the\tmpcnt}

% a fixed-string version that can be enabled for debugging.
%\edef\versionDebug{000-00-00-00:00}
%\let\version\versionDebug

% #1 is the first year for Eitan.  The last year is always 2009.  RIP.
\def\CopyYear.#1.{#1-2009}

% command for write to terminal and the log file
% this version is used in the .4ht files build
% identical command is defined also in tex4ht-sty.tex, 
% it is used in TeX document compilation
\def\writesixteen#1{\immediate\write1616{#1}}



\expandafter\ifx \csname YES\HTML\endcsname\relax
\else
    \def\CleanComment#1tex4ht-info-html4#2#3#4.#5>#6//{[#4%     <--jobname
                  \if,\ifnum \FIRST=#6 .\else ,0\fi\fi]#6//}
\fi
%%%%%%%%%%%%%%%%%%%%%%%%%%%%%%%%%%%%%%%%%%%%%%%%%%%%%%%%%%%%%%%%%%%%%%%%
\chapter{INFO}
%%%%%%%%%%%%%%%%%%%%%%%%%%%%%%%%%%%%%%%%%%%%%%%%%%%%%%%%%%%%%%%%%%%%%%%%


\<infoht4\><<<
%%%%%%%%%%%%%%%%%%%%%%%%%%%%%%%%%%%%%%%%%%%%%%%%%%%%%%%%%%  
% infoht4.4ht                           |version %
% Copyright (C) |CopyYear.2000.       Eitan M. Gurari         %
%                                                        %
% This program can redistributed and/or modified under   %
% the terms of the LaTeX Project Public License          %
% Distributed from CTAN archives in directory            %
% macros/latex/base/lppl.txt; either version 1 of the    %
% License, or (at your option) any later version.        %
%                                                        %
% If you modify this program your changing its signature %
% with a directive of the following form will be         %
% appreciated.                                           %
%            \message{signature}                         %
%                                                        %
%                              gurari@cis.ohio-state.edu %
%                  http://www.cis.ohio-state.edu/~gurari %
%%%%%%%%%%%%%%%%%%%%%%%%%%%%%%%%%%%%%%%%%%%%%%%%%%%%%%%%%%
\immediate\write-1{version |version}
{          \catcode`\@=0 \catcode`\\=11 @relax
  @gdef@infoIVht[#1]#2//{%
    @ifnum #1>1
      @def@infoIVht[##1]##2//{%
        @ifnum ##1>1 @ifnum ##1<#1
           @bgroup 
             @no:catcodes0{255}{11}%
             @no:catcodes{91}{91}{12}% [
             @no:catcodes{47}{47}{12}% /
             @newlinechar13 %   
             @long@def@infoIVht####1\ifx\infoIVht####2infoIVht[####3//{%
               @def@infoIVht{******************************************}%
               @immediate@write-1{@infoIVht}%
               @immediate@write-1{****** @csname :Hin@endcsname.4ht}%
               @immediate@write-1{@infoIVht}%
               @bgroup
                @def@infoIVht{~~~~~~~~~~~~~~~~~~~~~~~~~~~~~~~~~~~~~~~~~*}%
                @let~=@space   @immediate@write-1{@infoIVht}%
               @egroup   
               @immediate@write-1{####1}%
               @bgroup
                @def@infoIVht{~~~~~~~~~~~~~~~~~~~~~~~~~~~~~~~~~~~~~~~~~*}%
                @let~=@space   @immediate@write-1{@infoIVht}%
               @egroup
               @immediate@write-1{@infoIVht}%
             @egroup}%
           @expandafter@expandafter@expandafter@infoIVht
     @fi@fi }%
  @fi }
}
>>>







\chapter{The Code}
                
\<configure infoht4 tex4ht\><<<
Cascade Style Sheets and Character Sets
---------------------------------------

\Configure{CssFile}.....................2

   #1  name for css file
   #2  comment in css file

\Configure{Css}

\Configure{charset}..................1
   
   #1  override information for the charset entry in the link element
       
   To be noticed, the configuration should be encountered early in the
   complation.

      Example:
   
        \Preamble{}
          \Configure{charset}{iso-8859-15}
        \begin{document} 
        \EndPreamble
   
   In the case of TeX, the configuration instruction should precede 
   the \Preamble command, and a declaration of the hook should also
   be provided.

      Example:
   
          \NewConfigure{charset}{1}
          \Configure{charset}{iso-8859-15}
        \Preamble{}
        \begin{document} 
        \EndPreamble

\Configure{dtd-lang}..................1

    #1 language specification for the DTD identifier

Paragraphs
----------

Three types of paragraphs are produced by \Configure{HtmlPar}:

   <p class="indent">....</p>
   <p class="noindent">....</p>
   <p class="nopar">....</p>

The third one handles degenerated cases which don't call for a
separate paragraph in the source code. For instance, an inline
display math code of the form

     ..........
     $$......$$
     ..........

translates into a markup similar to

     <p class="...">..........</p>
     <math>......</math>
     <p class="nopar">..........</p>

Frames
------

   May be requested with the command line option `frames'
   
   \Configure{frames}.....................2
   
      #1  The frames structure
      #2  Content for the frame of the table of contents
   
   Example:
   
      \Configure{frames}
         {\HorFrames[frameborder="no" border="0"
               framespacing="0" rows="*"]{*,3*}
           \Frame[ name="tex4ht-menu" ]{tex4ht-toc}
           \Frame[ name="tex4ht-main" ]{tex4ht-body}}
         {\let\contentsname=\empty \tableofcontents}
>>>

%%%%%%%%%%%%%%%
\section{latex}
%%%%%%%%%%%%%%%


\<configure infoht4 latex\><<<
Hypertext Title
---------------
\Configure{@HTML}.........................1

   Environment for setting \title material in <title>. The contributions
   are accumulative; an empty argument reinitializes the contribution.

Tables
------

Properties of tabular and array tables can be locally redefined with
instructions of the following forms.

   \Css{\#TBL-\TableNo-  {....}}           % within tables
   \Css{\#TBL-\TableNo-\HRow- {....}}      % within rows
   \Css{\#TBL-\TableNo-\HCol  {....}}      % within columns
   \Css{\#TBL-\TableNo-\HRow-\HCol {....}} % within cells
   \Css{\#TBL-\TableNo-<i>g  {....}}       % column group; <i> 

\Configure{enumerate}..............1
   Contribution into <ol>

>>>

%  \Configure{enumerate}{1aiA}

\>>>maketitleInfo<<<
Wrapper for the Document
------------------------

   \Tag{TITLE+}......................1

   #1  Content submitted to the <TITLE> element of the leading
       web page.

   This instruction should be used at most once in a document.
   In particular, it shouldn't be used when the \maketitle command
   is present, since the latter command invokes this feature.
>>>

%%%%%%%%%%%%%%%%%%%%%%%%%%%%%%%%%%%%%
\section{book}
%%%%%%%%%%%%%%%%%%%%%%%%%%%%%%%%





\<configure infoht4 book\><<<
Package options
---------------

   1   cut-off document into pages at the \part level
   2   cut-off document into pages at the \chapter and higher levels
   3   cut-off document into pages at the \section and higher levels
   4   cut-off document into pages at the \subsection and higher levels

       The chapter of \tableofcontents  is created only
       if \contentsname is not empty

  Macros sensitive to edef commands in sectioning titles
  should be neutralize within \Configure{@TITLE}{....}

|maketitleInfo
>>>


\section{report}

\<configure infoht4 report\><<<
Package options:

   1   cut-off document into pages at the \part level
   2   cut-off document into pages at the \section and higher levels
   3   cut-off document into pages at the \subsection and higher levels
   4   cut-off document into pages at the \subsubsection and higher levels

       The section of \tableofcontents  is created only

       if \contentsname is not empty

  Macros sensitive to edef commands in sectioning titles
  should be neutralize within \Configure{@TITLE}{....}

|maketitleInfo
>>>



\section{article}

\<configure infoht4 article\><<<
Package options:

   1   cut-off document into pages at the \part level
   2   cut-off document into pages at the \section and higher levels
   3   cut-off document into pages at the \subsection and higher levels
   4   cut-off document into pages at the \subsubsection and higher levels

       The section of \tableofcontents  is created only
       if \contentsname is not empty

  Macros sensitive to edef commands in sectioning titles
  should be neutralize within \Configure{@TITLE}{....}

|maketitleInfo
>>>

\<configure infoht4 amsart\><<<

|maketitleInfo
>>>


\endinput


