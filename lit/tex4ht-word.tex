% $Id$
% compile 3 times:     latex   tex4ht-word   
%                  or  xhlatex tex4ht-word "html,3,sections+"

% Copyright (C) 2009, 2010 TeX Users Group.
% Copyright (C) 2000--2009 Eitan M. Gurari
% Released under LPPL 1.3c+.
% See tex4ht-cpright.tex for license text.

%%%%%%%%%%%%%%%%%% load style files %%%%%%%%%%%%%%%%%%%%%%%%%%

\ifx \HTML\UnDef
   \def\HTML{htmlw}                   
   \def\CONFIG{\jobname}
   \def\MAKETITLE{\author{Eitan M. Gurari}}         
   \def\next{\input mktex4ht.4ht  \endinput}
   \expandafter\next
\fi

% $Id$
% A few common TeX definitions for literate sources.  Not installed in runtime.
% 
% Copyright 2009-2017 TeX Users Group
% Copyright 1996-2009 Eitan M. Gurari
%
% This work may be distributed and/or modified under the
% conditions of the LaTeX Project Public License, either
% version 1.3c of this license or (at your option) any
% later version. The latest version of this license is in
%   http://www.latex-project.org/lppl.txt
% and version 1.3c or later is part of all distributions
% of LaTeX version 2005/12/01 or later.
%
% This work has the LPPL maintenance status "maintained".
%
% The Current Maintainer of this work
% is the TeX4ht Project <http://tug.org/tex4ht>.
% 
% If you modify this program, changing the 
% version identification would be appreciated.

\newcount\tmpcnt  \tmpcnt\time  \divide\tmpcnt  60
\edef\temp{\the\tmpcnt}
\multiply\tmpcnt  -60 \advance\tmpcnt  \time

\edef\version{\the\year-\ifnum \month<10 0\fi
  \the\month-\ifnum \day<10 0\fi\the\day
   -\ifnum \temp<10 0\fi \temp
   :\ifnum \tmpcnt<10 0\fi\the\tmpcnt}

% a fixed-string version that can be enabled for debugging.
%\edef\versionDebug{000-00-00-00:00}
%\let\version\versionDebug

% #1 is the first year for Eitan.  The last year is always 2009.  RIP.
\def\CopyYear.#1.{#1-2009}

% command for write to terminal and the log file
% this version is used in the .4ht files build
% identical command is defined also in tex4ht-sty.tex, 
% it is used in TeX document compilation
\def\writesixteen#1{\immediate\write1616{#1}}

\<TeX4ht copyright\><<<
%
% This work may be distributed and/or modified under the
% conditions of the LaTeX Project Public License, either
% version 1.3c of this license or (at your option) any
% later version. The latest version of this license is in
%   http://www.latex-project.org/lppl.txt
% and version 1.3c or later is part of all distributions
% of LaTeX version 2005/12/01 or later.
%
% This work has the LPPL maintenance status "maintained".
%
% The Current Maintainer of this work
% is the TeX4ht Project <http://tug.org/tex4ht>.
% 
% If you modify this program, changing the 
% version identification would be appreciated.
>>>


%%%%%%%%%%%%%%%%%%%%%%%%%%%%%%%%%%%%%%%%%%%%%%%%%%%%%%%%%%%%%%%%%%%%%%%%
\section{Preamble}
%%%%%%%%%%%%%%%%%%%%%%%%%%%%%%%%%%%%%%%%%%%%%%%%%%%%%%%%%%%%%%%%%%%%%%%%

\<htmlw\><<<
% htmlw.4ht (|version), generated from |jobname.tex
% Copyright (C) 2009, 2010 TeX Users Group
% Copyright (C) |CopyYear.2000. Eitan M. Gurari
|<TeX4ht copywrite|>
>>>



%%%%%%%%%%%%%%%%%%%%%%%%%%%%%%%%%%%%%%%%%%%%%%%%%%%%%%%%%%%%%%%%%%%%%%%%
\section{General}
%%%%%%%%%%%%%%%%%%%%%%%%%%%%%%%%%%%%%%%%%%%%%%%%%%%%%%%%%%%%%%%%%%%%%%%%

\<configure htmlw tex4ht\><<<
% \Configure{Picture}{.wmf}
>>>

\<configure htmlw latex\><<<
\Css{body {text-align:both;}}
>>>

%%%%%%%%%%%%%%%%%%%%%%%%%%%%%%%%%%%%%%%%%%%%%%%%%%%%%%%%%%%%%%%%%%%%%%%%
\section{Char Set}
%%%%%%%%%%%%%%%%%%%%%%%%%%%%%%%%%%%%%%%%%%%%%%%%%%%%%%%%%%%%%%%%%%%%%%%%



\<configure htmlw babel\><<<
\def\:temp#1#2{%
   \def\:tempa{#1}\ifx \languagename\:tempa
      \def\A:charset{#2}\ifx \A:charset\empty\else
          \def\A:charset{charset=#2}%
   \fi\fi}
\:temp{afrikaans}{windows-1252}
\:temp{arabic}{windows-1256}
\:temp{austrian}{windows-1252}
\:temp{brazilian}{windows-1252}
\:temp{brazil}{windows-1252}
\:temp{catalan}{windows-1252}
\:temp{croatian}{windows-1251}
\:temp{czech}{windows-1251}
\:temp{danish}{windows-1252}
\:temp{dutch}{windows-1252}
\:temp{estonian}{windows-1251}
\:temp{finnish}{windows-1252}
\:temp{francais}{windows-1252}
\:temp{frenchb}{windows-1252}
\:temp{french}{windows-1252}
\:temp{galician}{windows-1252}
\:temp{germanb}{windows-1252}
\:temp{german}{windows-1252}
\:temp{greek}{windows-1253}
\:temp{hebrew}{windows-1255}
\:temp{hungarian}{windows-1250}
\:temp{ngerman}{windows-1252}
\:temp{norsk}{windows-1252}
\:temp{nynorsk}{windows-1252}
\:temp{polish}{windows-1252}
\:temp{polutonikogreek}{windows-1253}
\:temp{portuges}{windows-1252}
\:temp{portuguese}{windows-1252}
\:temp{romanian}{windows-1250}
\:temp{russian}{windows-1251}
\:temp{scottish}{windows-1252}
\:temp{slovak}{windows-1250}
\:temp{slovene}{windows-1250}
\:temp{spanish}{windows-1252}
\:temp{swedish}{windows-1252}
\:temp{ukrainian}{windows-1251}
\:temp{uppersorbian}{windows-1252}
\:temp{welsh}{windows-1252}
>>>


\<configure htmlw austrian\><<<
\def\A:charset{charset=windows-1252}
>>>
\<configure htmlw catalan\><<<
\def\A:charset{charset=windows-1252}
>>>
\<configure htmlw croatian\><<<
\def\A:charset{charset=windows-1251}
>>>
\<configure htmlw czech\><<<
\def\A:charset{charset=windows-1251}
>>>
\<configure htmlw danish\><<<
\def\A:charset{charset=windows-1252}
>>>
\<configure htmlw dutch\><<<
\def\A:charset{charset=windows-1252}
>>>
\<configure htmlw english\><<<
\def\A:charset{charset=windows-1251}
>>>
\<configure htmlw estonian\><<<
\def\A:charset{charset=windows-1251}
>>>
\<configure htmlw finnish\><<<
\def\A:charset{charset=windows-1252}
>>>
\<configure htmlw francais\><<<
\def\A:charset{charset=windows-1252}
>>>
\<configure htmlw galician\><<<
\def\A:charset{charset=windows-1252}
>>>
\<configure htmlw germanb\><<<
\def\A:charset{charset=windows-1252}
>>>
\<configure htmlw greek\><<<
\def\A:charset{charset=windows-1253}
>>>
\<configure htmlw hebrew\><<<
\def\A:charset{charset=windows-1255}
>>>
\<configure htmlw ngermanb\><<<
\def\A:charset{charset=windows-1252}
>>>
\<configure htmlw norsk\><<<
\def\A:charset{charset=windows-1252}
>>>
\<configure htmlw polish\><<<
\def\A:charset{charset=windows-1252}
>>>
\<configure htmlw portuges\><<<
\def\A:charset{charset=windows-1252}
>>>
\<configure htmlw romanian\><<<
\def\A:charset{charset=windows-1250}
>>>
\<configure htmlw russianb\><<<
\def\A:charset{charset=windows-1251}
>>>
\<configure htmlw scottish\><<<
\def\A:charset{charset=windows-1252}
>>>
\<configure htmlw slovak\><<<
\def\A:charset{charset=windows-1250}
>>>
\<configure htmlw slovene\><<<
\def\A:charset{charset=windows-1250}
>>>
\<configure htmlw spanish\><<<
\def\A:charset{charset=windows-1252}
>>>
\<configure htmlw swedish\><<<
\def\A:charset{charset=windows-1252}
>>>
\<configure htmlw ukraineb\><<<
\def\A:charset{charset=windows-1251}
>>>
\<configure htmlw usorbian\><<<
\def\A:charset{charset=windows-1252}
>>>
\<configure htmlw welsh\><<<
\def\A:charset{charset=windows-1252}
>>>


\<configure htmlw hebtex\><<<
\def\A:charset{charset=windows-1255}
>>>

\<configure htmlw exerquiz\><<<
\def\A:charset{charset=windows-1252}  
>>>


\<configure htmlw tex4ht\><<<
\def\A:charset{charset=windows-1252} 
>>>

%%%%%%%%%%%%%%%%%%%%%%%%%%%%%%%%%%%%%%%%%%%%%%%%%%%%%%%%%%%%%%%%%%%%%%%%
\section{Etc}
%%%%%%%%%%%%%%%%%%%%%%%%%%%%%%%%%%%%%%%%%%%%%%%%%%%%%%%%%%%%%%%%%%%%%%%%





\<configure htmlw alphanum\><<<
\Configure{toclvld}
   {}{}
   {\ifvmode \IgnorePar\fi \EndP\IgnorePar
    \HCode{<p class="MsoHeading6">}\TitleMark\space\HtmlParOff}
   {\HCode{</p>}\HtmlParOn \IgnoreIndent \par}
\Configure{toclvle}
   {}{}
   {\ifvmode \IgnorePar\fi \EndP\IgnorePar
    \HCode{<p class="MsoHeading7">}\TitleMark\space\HtmlParOff}
   {\HCode{</p>}\HtmlParOn \IgnoreIndent \par}
\Configure{toclvlf}
   {}{}
   {\ifvmode \IgnorePar\fi \EndP\IgnorePar
    \HCode{<p class="MsoHeading8">}\TitleMark\space\HtmlParOff}
   {\HCode{</p>}\HtmlParOn \IgnoreIndent \par}
\Configure{toclvlg}
   {}{}
   {\ifvmode \IgnorePar\fi \EndP\IgnorePar
    \HCode{<p class="MsoHeading9">}\TitleMark\space\HtmlParOff}
   {\HCode{</p>}\HtmlParOn \IgnoreIndent \par}
\Configure{toclvlh}
   {}{}
   {\ifvmode \IgnorePar\fi \EndP\IgnorePar
    \HCode{<p class="MsoHeading10">}\TitleMark\space\HtmlParOff}
   {\HCode{</p>}\HtmlParOn \IgnoreIndent \par}
\Configure{toclvli}
   {}{}
   {\ifvmode \IgnorePar\fi \EndP\IgnorePar
    \HCode{<p class="MsoHeading11">}\TitleMark\space\HtmlParOff}
   {\HCode{</p>}\HtmlParOn \IgnoreIndent \par}
\Configure{toclvlj}
   {}{}
   {\ifvmode \IgnorePar\fi \EndP\IgnorePar
    \HCode{<p class="MsoHeading12">}\TitleMark\space\HtmlParOff}
   {\HCode{</p>}\HtmlParOn \IgnoreIndent \par}
\Configure{toclvlk}
   {}{}
   {\ifvmode \IgnorePar\fi \EndP\IgnorePar
    \HCode{<p class="MsoHeading13">}\TitleMark\space\HtmlParOff}
   {\HCode{</p>}\HtmlParOn \IgnoreIndent \par}
\Configure{toclvll}
   {}{}
   {\ifvmode \IgnorePar\fi \EndP\IgnorePar
    \HCode{<p class="MsoHeading14">}\TitleMark\space\HtmlParOff}
   {\HCode{</p>}\HtmlParOn \IgnoreIndent \par}

\Css{p.MsoHeading6 {margin-top:12.0pt;
                    margin-right:0cm;
                    margin-bottom:3.0pt;
                    margin-left:0cm;
                    font-size:11.0pt;}}
\Css{p.MsoHeading7 {margin-top:12.0pt;
                    margin-right:0cm;
                    margin-bottom:3.0pt;
                    margin-left:0cm;
                    font-size:12.0pt;}}
\Css{p.MsoHeading8 {margin-top:12.0pt;
                    margin-right:0cm;
                    margin-bottom:3.0pt;
                    margin-left:0cm;
                    font-size:12.0pt;
                    font-style:italic;}}
\Css{p.MsoHeading9 {margin-top:12.0pt;
                    margin-right:0cm;
                    margin-bottom:3.0pt;
                    margin-left:0cm;
                    font-size:11.0pt;
                    font-family:Arial;}}
>>>



\begin{verbatim}
>> But they should go for h6, h7, h8 ... (which is not the case now).

EG> But HTML does not recognize h* tags besides h1--h5. What we can do is
EG> to set diffent CSS code there--however, I didn't see differents in
EG> fonts there. -eitan

This is true as far as plain html is concerned. But you can use higher
h? if you are planning to convert to *.doc because word interprets h1 as
heading 1 or h9 as heading 9 (presumably) as long as they are defined in
normal.dot.

EG> Do you know how can I inspect the content of that file? Under
EG> emacs it all looks like junk. -eitan

Which file do you mean? --- I suppose normal.dot. I do not think that
there is a possibility to view that file (apart from looking at it
with an hex editor).

But you can see the result quite easily if you produce an html-file
with several headings (lets say h1 to h9) and load that file into
word. After saving that file in a word format (*.doc) you will realize
that your headings are assigned to word's styles Heading 1 to Heading
9. Start then a new document with a different template. Let's say
thesis. Copy the hole text by Edit/Select All, Edit/Copy and Edit/Paste
into the new document. Then you will realize that all headings are
formated according to the heading styles of the thesis template.

This feature makes tex4ht even more interesting for converting
latex to word. Because you can write your texts with latex, convert
them to word and use then a template which is given to you by your
publisher. Unfortunately, German publishers do that quite often now
because nearly 90 % of all German lawyers are using word for writing
their essays these days.
\end{verbatim}





%%%%%%%%%%%%%%%%%%%%%%%%%%%%%%%%%%%%%%%%%%%%%%%%%%%%%%%%%%%%%%%%%%%%%%%%
\section{Plain Footnotes}
%%%%%%%%%%%%%%%%%%%%%%%%%%%%%%%%%%%%%%%%%%%%%%%%%%%%%%%%%%%%%%%%%%%%%%%%

\<configure htmlw plain\><<<
\:CheckOption{fn-in}\if:Option  
   |<inline footnote plain|>
   |<inline footnote css|>
\fi
>>>

\<inline footnote plain\><<<
\Configure{vfootnote}
  {|<mark inline vfootnote|>}
  {|<open inline vfootnote|>}
  {|<close inline vfootnote|>}
\Css{span.MsoFootnoteReference {vertical-align:super;}}
>>>

\<mark inline vfootnote\><<<
\Link{fn\FNnum}{fn\FNnum-bk}\HCode{<span class="MsoFootnoteReference">}%
  \ifx \FNmark\empty +\else \FNmark\fi\HCode{</span>}\EndLink
>>>



\<open inline vfootnote\><<<
|<open inline footnote box|>
   \Link{fn\FNnum-bk}{fn\FNnum}\HCode{<span class="MsoFootnoteReference">}%
   \ifx \FNmark\empty +\else \FNmark\fi
   \space\HCode{</span>}\EndLink
>>>

\<close inline vfootnote\><<<
|<close inline footnote box|>
>>>  

\<footnote par\><<<
\Configure{HtmlPar}
  {\EndP\HCode{<!--l. \the\inputlineno-->}\HCode{<p 
         \csname a:!P\endcsname class="MsoFootnoteText">}}
  {\EndP\HCode{<!--l. \the\inputlineno-->}\HCode{<p
         \csname a:!P\endcsname class="MsoFootnoteText">}}
  {\:xhtml{\Tg</p>}}
  {\:xhtml{\Tg</p>}}
>>>

%%%%%%%%%%%%%%%%%%%%%%%%%%%%%%%%%%%%%%%%%%%%%%%%%%%%%%%%%%%%%%%%%%%%%%%%
\section{Latex Footnotes}
%%%%%%%%%%%%%%%%%%%%%%%%%%%%%%%%%%%%%%%%%%%%%%%%%%%%%%%%%%%%%%%%%%%%%%%%


\<configure htmlw Preamble\><<<    
\:CheckOption{fn-out}\if:Option \else
   \Log:Note{for offline footnotes use
          command line option `fn-out'}
   \edef\Preamble{\Preamble,fn-in} 
\fi
>>>



\<configure htmlw latex\><<<
\:CheckOption{fn-in}\if:Option  
   |<inline footnote mark|>
   |<inline footnote text|>
   |<inline footnote css|>
\fi
>>>


\<inline footnote mark\><<<
\Configure{footnotemark}
   {\leavevmode |<inline fn mark /Link|>%
    \NoFonts
    \Tag{|<footnote id|>}{}}
   {\EndNoFonts
    |<end inline fn mark /Link|>%
    \expandafter\global\expandafter
       \let\csname (|<footnote id|>-bk)\endcsname\def}
>>>


\<inline footnote text\><<<
\Configure{footnotetext}
   {|<open inline footnote box|>%
    \leavevmode |<inline fn text /Link|>%
    \NoFonts
    \Tag{|<footnote id|>-bk}{}
   }
   {\EndNoFonts |<end inline fn text /Link|>}
   {\expandafter\global\expandafter
       \let\csname (|<footnote id|>)\endcsname\def
    |<close inline footnote box|>}
>>>



\<open inline footnote box\><<<
\SaveEndP\global\setbox\fn:box=\vtop\bgroup\ShowPar
   |<footnote par|>%
   \ifvoid\fn:box\else  \csname unvbox\endcsname\fn:box\fi
>>>

\<close inline footnote box\><<<
\EndP\egroup\RecallEndP
>>>  






\<inline fn text /Link\><<<
\ifTag{|<footnote id|>}
   {\expandafter\ifx\csname (|<footnote id|>)\endcsname\def
      \Link{|<footnote id|>-bk}{}%       
      \HCode{<span class="MsoFootnoteReference">}%
    \else
      \Link{|<footnote id|>-bk}{|<footnote id|>}%
      \HCode{<span class="MsoFootnoteReference">}%
    \fi
   }
   {\expandafter\ifx\csname (|<footnote id|>)\endcsname\def
    \else
      \Link{}{|<footnote id|>}%
      \HCode{<span class="MsoFootnoteReference">}%
    \fi
   }%
>>>

\<end inline fn text /Link\><<<
\ifTag{|<footnote id|>}
   {\space\HCode{</span>}\EndLink
   }
   {\expandafter\ifx\csname (|<footnote id|>)\endcsname\def
    \else
       \space\HCode{</span>}\EndLink
    \fi
   }%
>>>



\<mouseover inline fn mark /Link\><<<
|<start js|>%
\ifTag{|<footnote id|>-bk}
   {\expandafter\ifx\csname (|<footnote id|>-bk)\endcsname\def
      \Link[ |<mouseover-footnote arguments|>\Hnewline
                    ]{|<footnote id|>}{}%       
    \else
      \Link[ |<mouseover-footnote arguments|>\Hnewline
                    ]{|<footnote id|>}{|<footnote id|>-bk}%
    \fi
   }
   {\expandafter\ifx\csname (|<footnote id|>-bk)\endcsname\def
    \else \Link[ |<mouseover-footnote arguments|>\Hnewline
                    ]{}{|<footnote id|>-bk}\fi
   }%
>>>







\<inline fn mark /Link\><<<
\ifTag{|<footnote id|>-bk}
   {\expandafter\ifx\csname (|<footnote id|>-bk)\endcsname\def
      \Link{|<footnote id|>}{}%
      \HCode{<span class="MsoFootnoteReference">}%       
    \else
      \Link{|<footnote id|>}{|<footnote id|>-bk}%
      \HCode{<span class="MsoFootnoteReference">}%
    \fi
   }
   {\expandafter\ifx\csname (|<footnote id|>-bk)\endcsname\def
    \else
      \Link{}{|<footnote id|>-bk}
      \HCode{<span class="MsoFootnoteReference">}%
    \fi
   }%
>>>


\<end inline fn mark /Link\><<<
\ifTag{|<footnote id|>-bk}
   {\HCode{</span>}\EndLink}
   {\expandafter\ifx\csname (|<footnote id|>-bk)\endcsname\def
    \else \HCode{</span>}\EndLink\fi
   }%
>>>




\<inline footnote css\><<<
\Css{span.footnotetext{ font-size:80\%; font-style:italic; } }
\Css{div.footnotes{border-bottom:0;}}
>>>


\<ams footnotes\><<<
\:CheckOption{fn-in}\if:Option  
   \Css{span.MsoFootnoteReference {vertical-align:super;}}
\fi
>>>


\<footnote id\><<<
fn\FNnum x\minipageNum
>>>


\<configure htmlw amsart\><<<
|<ams footnotes|>
>>>

\<configure htmlw amsproc\><<<
|<ams footnotes|>
>>>

\<configure htmlw amsbook\><<<
|<ams footnotes|>
>>>


\endinput
