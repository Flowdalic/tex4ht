% $Id$
% compile 3 times: latex tex4ht-svg   
%           or   xhlatex tex4ht-svg "html,3,sections+"
%
% Copyright 2009-2017 TeX Users Group
% Copyright 2000-2009 Eitan M. Gurari
% Released under LPPL 1.3c+.
% See tex4ht-cpright.tex for license text.

%%%%%%%%%%%%%%%%%% load style files %%%%%%%%%%%%%%%%%%%%%%%%%%

\ifx \HTML\UnDef
   \def\HTML{svg-option,html4-svg}                   
   \def\CONFIG{\jobname}
   \def\MAKETITLE{\author{Eitan M. Gurari}}         
   \def\next{\input mktex4ht.4ht  \endinput}
   \expandafter\next
\fi

% $Id$
% A few common TeX definitions for literate sources.  Not installed in runtime.
% 
% Copyright 2009-2017 TeX Users Group
% Copyright 1996-2009 Eitan M. Gurari
%
% This work may be distributed and/or modified under the
% conditions of the LaTeX Project Public License, either
% version 1.3c of this license or (at your option) any
% later version. The latest version of this license is in
%   http://www.latex-project.org/lppl.txt
% and version 1.3c or later is part of all distributions
% of LaTeX version 2005/12/01 or later.
%
% This work has the LPPL maintenance status "maintained".
%
% The Current Maintainer of this work
% is the TeX4ht Project <http://tug.org/tex4ht>.
% 
% If you modify this program, changing the 
% version identification would be appreciated.

\newcount\tmpcnt  \tmpcnt\time  \divide\tmpcnt  60
\edef\temp{\the\tmpcnt}
\multiply\tmpcnt  -60 \advance\tmpcnt  \time

\edef\version{\the\year-\ifnum \month<10 0\fi
  \the\month-\ifnum \day<10 0\fi\the\day
   -\ifnum \temp<10 0\fi \temp
   :\ifnum \tmpcnt<10 0\fi\the\tmpcnt}

% a fixed-string version that can be enabled for debugging.
%\edef\versionDebug{000-00-00-00:00}
%\let\version\versionDebug

% #1 is the first year for Eitan.  The last year is always 2009.  RIP.
\def\CopyYear.#1.{#1-2009}

% command for write to terminal and the log file
% this version is used in the .4ht files build
% identical command is defined also in tex4ht-sty.tex, 
% it is used in TeX document compilation
\def\writesixteen#1{\immediate\write1616{#1}}

\<TeX4ht copyright\><<<
%
% This work may be distributed and/or modified under the
% conditions of the LaTeX Project Public License, either
% version 1.3c of this license or (at your option) any
% later version. The latest version of this license is in
%   http://www.latex-project.org/lppl.txt
% and version 1.3c or later is part of all distributions
% of LaTeX version 2005/12/01 or later.
%
% This work has the LPPL maintenance status "maintained".
%
% The Current Maintainer of this work
% is the TeX4ht Project <http://tug.org/tex4ht>.
% 
% If you modify this program, changing the 
% version identification would be appreciated.
>>>


%%%%%%%%%%%%%%%%%%%%%%%%%%%%%%%%%%%%%%%%%%%%%%%%%%%%%%%%%%%%%%%%%%%%%%%%
\chapter{Preamble}
%%%%%%%%%%%%%%%%%%%%%%%%%%%%%%%%%%%%%%%%%%%%%%%%%%%%%%%%%%%%%%%%%%%%%%%%

\<svg-option\><<<
% svg-option.4ht (|version), generated from |jobname.tex
% Copyright 2009-2017 TeX Users Group
% Copyright |CopyYear.2001. Eitan M. Gurari
|<TeX4ht copywrite|>
>>>

% 2017-05-05 Michal
% Change DTD and other stuff only for svg-inline
% this is legacy option which will hold the original `svg` option behaviour
\<configure svg-option tex4ht\><<<    
\:CheckOption{svg-inline}\if:Option
\Configure{VERSION}
  {\IgnorePar\HCode{<?xml version="1.0" |<xml encoding|> ?> \Hnewline}}
\Configure{DOCTYPE}{\HCode
  {<!DOCTYPE html PUBLIC
     "-//W3C//DTD XHTML 1.1 plus MathML 2.0 plus SVG 1.1//|<dtd lang|>"\Hnewline
     "http://www.w3.org/2002/04/xhtml-math-svg/xhtml-math-svg.dtd">\Hnewline
   <!--http://www.w3.org/TR/XHTMLplusMathMLplusSVG/-->\Hnewline
}}
>>>

\<xml encoding\><<<
 encoding="\expandafter\ifx \csname a:charset\endcsname\relax
         \expandafter\:encoding\A:charset
   \else \expandafter\:encoding\a:charset\fi"
>>>

\<configure svg-option tex4ht\><<<    
\Configure{@DOCTYPE}
  {<!ENTITY \% svg.dtd PUBLIC "-//W3C//DTD SVG 20010719//|<dtd lang|>"\Hnewline
     "http://www.w3.org/TR/2001/PR-SVG-20010719/DTD/svg10.dtd">\Hnewline
   \%svg.dtd; \Hnewline}
>>>


\<configure html4-svg tex4ht\><<<    
% hmtl-svg.4ht (|version), generated from |jobname.tex
% Copyright 2009-2017 TeX Users Group
% Copyright |CopyYear.2001. Eitan M. Gurari
|<TeX4ht copywrite|>
\Configure{@DOCTYPE}
  {<!ENTITY \% misc "ins || del || script || noscript || svg">\Hnewline}
>>>


\<dtd lang\><<<
\expandafter
\ifx \csname a:dtd-lang\endcsname\relax EN\else
  \csname a:dtd-lang\endcsname
\fi
>>>

\<configure svg-option tex4ht\><<<    
\Configure{@HTML}
  {\Hnewline xmlns:svg="http://www.w3.org/2000/svg"\Hnewline }
\fi
>>>



%%%%%%%%%%%%%%%%%%
\chapter{Code}
%%%%%%%%%%%%%%%%%%

%%%%%%%%%%%%%%%%%%
\section{DVI Images}
%%%%%%%%%%%%%%%%%%





\<configure svg-option tex4ht\><<<    
\Configure{Picture}{.svg}
\:CheckOption{svg-obj} \if:Option 
   |<object svg-obj|>
\else   \:CheckOption{svg-} \if:Option 
      |<object svg|>
\else 
      \Log:Note{for external SVG files
             try the command line options `svg-obj' and 'svg-'}
\:CheckOption{svg-inline}\if:Option
      |<internal svg|>
\fi
\fi\fi
>>>



\<object svg-obj\><<<
\Configure{IMG}
  {\special{t4ht=<object type="image/svg+xml"  data="}}
  {\special{t4ht=" name="}}
  {" }
  {\special{t4ht=" }}
  {\special{t4ht=></object>}}
>>>





\<object svg\><<<
\Configure{IMG}
  {\special{t4ht=<object type="image/svg+xml"><img src="}}
  {\special{t4ht=" alt="}}
  {" }
  {\special{t4ht=" }}
  {\special{t4ht=/></object>}}
>>>


\<internal svg\><<<
\Configure{IMG}
  {\special{t4ht=<!-- src="}}
  {\special{t4ht=" alt="}}
  {" }
  {\special{t4ht=" }}
  {\special{t4ht=-->}%
   \openin15=\PictureFile \relax
   \ifeof15  \:warning{\PictureFile\space is not available}%
   \else     \closein15  \special{t4ht*<\PictureFile}\fi
  }
>>>




%%%%%%%%%%%%%%%%%%
\section{Include graphics}
%%%%%%%%%%%%%%%%%%

\<configure svg-option graphics\><<<
\Configure{graphics*} 
   {svg} 
   {{\Configure{Needs}{File: \csname Gin@base\endcsname.svg}\Needs{}}%
     \special{t4ht=<object type="image/svg+xml"  
                          data="\Gin@base.svg" 
                          name="picture \Gin@base"
                         class="graphics"></object><!--tex4ht:graphics |<graphics dim|>-->}}
>>>

\<graphics dim\><<<
\csname a:Gin-dim\endcsname
>>>

%%%%%%%%%%%%%%%%%%
\section{Support for th4:draw}
%%%%%%%%%%%%%%%%%%




\<configure svg-option th4\><<<    
\:CheckOption{draw} \if:Option \:CheckOption{th4} \if:Option 
   \Configure{Fig}
      {\Svg}
      {\EndSvg}
   \:CheckOption{svg-} \if:Option 
      |<embed svg draw|>
   \else
      |<internal svg draw|>
   \fi
\fi \fi
>>>



\<external svg draw\><<<    
\HAssign\:svgN = 0
\def\Svg{%
  \gHAdvance\:svgN by 1
  \HCode{<embed src="\jobname\:svgN.svg"\Hnewline
     name="svg\:svgN" type="image/svg+xml"
%  height="600" width="600"
     \Hnewline pluginspage="http://www.adobe.com/svg/viewer/install/">}%   
  |<open embeded file|>%
  |<open embeded svg|>%
  \bgroup\Canvas \x:SUBOff  \x:SUPOff 
     \let\Picture=\empty \everymath{}\everydisplay{}%
}
\def\EndSvg{\EndCanvas\egroup
   |<close embeded svg|>%
   |<close embeded file|>}
>>>





\<internal svg draw\><<<    
\def\Svg{\HCode{<svg:svg>\Hnewline
                <svg:g style="stroke:black;  stroke-width:1;
                      stroke-opacity:1;">\Hnewline}%
  \bgroup\Canvas \x:SUBOff  \x:SUPOff 
     \let\Picture=\empty \everymath{}\everydisplay{}%
}
\def\EndSvg{\EndCanvas\egroup
   \HCode{</svg:g></svg:svg>}}
>>>




\<embed svg draw\><<<    
\let\:svg=\empty
\NewConfigure{Canvas}[4]{\ht:special{t4ht"%
   *%
   *d\Hnewline<!--width="\%.1f#4" %
   *D height="\%.1f#4" %
   *y above-baseline="\%.1f#4" -->%
   **\Hnewline<\:svg text x="\%.1f" y="\%.1f" >%
   *</\:svg text>%
   *\Hnewline<\:svg rect x="\%.1f#4" y="\%.1f#4"  
                   width="\%.1f#4"  height="\%.1f#4" />%
   *#1*#2*#1*#2*#3}}
\Configure{Canvas}{0.0000152587890625}{0.0}{0.5}{}
>>>


65536 scaled units = 1 pt. 1 / 65536 = 0.0000152587890625.

\<\><<<    
\NewConfigure{Canvas}[4]{\ht:special{t4ht"%
   *%
   *d\Hnewline<!--width="\%.0f#4;" %
   *D height="\%.0f#4;" -->\Hnewline %
%   **<svg:text x="\%.0f" y="\%.0f" >%
%   *</svg:text>\Hnewline %
   **<svg:foreignObject x="\%.0f" y="\%.0f" >%
   *</svg:foreignObject>\Hnewline %
%   *\Hnewline<svg:rect x="\%.0f#4" y="\%.0f#4"  
%                   width="\%.0f#4"  height="\%.0f#4"
%                   fill="black" stroke-width="0" />%
   *\Hnewline<svg:polyline points="\%.0f#4 \%.0f#4 \%.0f#4 \%.0f#4" 
                           style="stroke-width:\%.0f#4" />%
   *#1*#2*#1*#2*#3*2}}
>>>





\<open embeded file\><<<
\ht:special{t4ht>\jobname\:svgN.svg}%
\HCode{<?xml version="1.0" encoding="iso-8859-1"?>\Hnewline
<!DOCTYPE svg  PUBLIC "-//W3C//DTD SVG 1.0//|<dtd lang|>"\Hnewline
              "http://www.w3.org/TR/2001/PR-SVG-20010719/DTD/svg10.dtd">
\Hnewline}
>>>

\<close embeded file\><<<
\ht:special{t4ht<\jobname\:svgN.svg}%
>>>

\<open embeded svg\><<<
\HCode{<\:svg svg>\Hnewline
       <\:svg g style="stroke-width:0; fill:black;">\Hnewline}%
>>>

\<close embeded svg\><<<
\HCode{</\:svg g></\:svg svg>}%
>>>






%%%%%%%%%%%%%%%%%%
\chapter{Notes}
%%%%%%%%%%%%%%%%%%


Requires two compilations (e.g., with \verb!mzlatex try "html,svg"!)
for importing the SVG code.




\endinput

