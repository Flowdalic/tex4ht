% $Id$
%      latex tex4ht-html32
% or xhlatex tex4ht-html32 "html,3,sections+"
%
% Copyright (C) 2009-2016 TeX Users Group
% Copyright (C) 1996-2009 Eitan M. Gurari
% Released under LPPL 1.3c+.
% See tex4ht-cpright.tex for license text.

\ifx \HTML\UnDef
   \def\HTML{html32,html32-math} 
   \def\CONFIG{\jobname}
   \def\PREAMBLE{%
      \usepackage{url}%
   }
   \def\MAKETITLE{\author{Eitan M. Gurari}}
   \def\next{\input mktex4ht.4ht  \endinput}
   \expandafter\next
\fi



% $Id$
% A more few common TeX definitions for literate sources.  Not installed
% in runtime.  These are only used in a few files, compared to those in
% common.tex.  Do not know if any harm would come from including them always.
% 
% Copyright 2009, 2010 TeX Users Group
% Copyright 1996--2009 Eitan M. Gurari
%
% This work may be distributed and/or modified under the
% conditions of the LaTeX Project Public License, either
% version 1.3c of this license or (at your option) any
% later version. The latest version of this license is in
%   http://www.latex-project.org/lppl.txt
% and version 1.3c or later is part of all distributions
% of LaTeX version 2005/12/01 or later.
%
% This work has the LPPL maintenance status "maintained".
%
% The Current Maintainer of this work
% is the TeX4ht Project <http://tug.org/tex4ht>.
% 
% If you modify this program, changing the 
% version identification would be appreciated.

\let\AltxModifyShowCode=\ModifyShowCode
\def\ModifyShowCode{%
   \def\by{by}\def\={=}\AltxModifyShowCode}

\let\pReModifyOutputCode=\ModifyOutputCode
\def\ModifyOutputCode{%
   \def\by{}\def\={}%
   \pReModifyOutputCode}

% $Id$
% A few common TeX definitions for literate sources.  Not installed in runtime.
% 
% Copyright 2009-2017 TeX Users Group
% Copyright 1996-2009 Eitan M. Gurari
%
% This work may be distributed and/or modified under the
% conditions of the LaTeX Project Public License, either
% version 1.3c of this license or (at your option) any
% later version. The latest version of this license is in
%   http://www.latex-project.org/lppl.txt
% and version 1.3c or later is part of all distributions
% of LaTeX version 2005/12/01 or later.
%
% This work has the LPPL maintenance status "maintained".
%
% The Current Maintainer of this work
% is the TeX4ht Project <http://tug.org/tex4ht>.
% 
% If you modify this program, changing the 
% version identification would be appreciated.

\newcount\tmpcnt  \tmpcnt\time  \divide\tmpcnt  60
\edef\temp{\the\tmpcnt}
\multiply\tmpcnt  -60 \advance\tmpcnt  \time

\edef\version{\the\year-\ifnum \month<10 0\fi
  \the\month-\ifnum \day<10 0\fi\the\day
   -\ifnum \temp<10 0\fi \temp
   :\ifnum \tmpcnt<10 0\fi\the\tmpcnt}

% a fixed-string version that can be enabled for debugging.
%\edef\versionDebug{000-00-00-00:00}
%\let\version\versionDebug

% #1 is the first year for Eitan.  The last year is always 2009.  RIP.
\def\CopyYear.#1.{#1-2009}

% command for write to terminal and the log file
% this version is used in the .4ht files build
% identical command is defined also in tex4ht-sty.tex, 
% it is used in TeX document compilation
\def\writesixteen#1{\immediate\write1616{#1}}

\<TeX4ht copyright\><<<
%
% This work may be distributed and/or modified under the
% conditions of the LaTeX Project Public License, either
% version 1.3c of this license or (at your option) any
% later version. The latest version of this license is in
%   http://www.latex-project.org/lppl.txt
% and version 1.3c or later is part of all distributions
% of LaTeX version 2005/12/01 or later.
%
% This work has the LPPL maintenance status "maintained".
%
% The Current Maintainer of this work
% is the TeX4ht Project <http://tug.org/tex4ht>.
% 
% If you modify this program, changing the 
% version identification would be appreciated.
>>>


\def\.{\string\a:mathml:\space}


%%%%%%%%%%%%%%%%%%%%%%%%%%%%%%%%%%%%%%%%%%%%%%%%%%%%%%%%%%%%%%%%%%%%%%%%
\chapter{The Calling Tree for 4ht Files}
%%%%%%%%%%%%%%%%%%%%%%%%%%%%%%%%%%%%%%%%%%%%%%%%%%%%%%%%%%%%%%%%%%%%%%%%


\<0,32,4 tex4ht\><<<
\if:latex  |<Hinclude latex|>
\else      |<Hinclude plain|>  \fi
>>>


\<0,32,4 plain\><<<
|<Hinclude plain lib|>
|<Hinclude plain + latex lib|>
>>>

\<0,32,4 latex\><<<
|<Hinclude latex lib|>
|<Hinclude plain + latex lib|>
>>>

%%%%%%%%%%%%%%%%%%%%%%%%%%%%%%%%%%%%%%%%%%%%%%%%%%%%%%%%%%%%%%%%%%%%%%%%
\chapter{Structures}
%%%%%%%%%%%%%%%%%%%%%%%%%%%%%%%%%%%%%%%%%%%%%%%%%%%%%%%%%%%%%%%%%%%%%%%%

%%%%%%%%%%%%%%%%%%%%%%%%%%%%%%%%%%%%%%%%%%%%%%%%%%%%%%%%%%%%%%%%%%%%%%%%
\section{Front}
%%%%%%%%%%%%%%%%%%%%%%%%%%%%%%%%%%%%%%%%%%%%%%%%%%%%%%%%%%%%%%%%%%%%%%%%


\<book-report-article make title\><<<
\Configure{maketitle}
   {|<title for TITLE|>%
    \HCode{<div align="center">}}{\HCode{</div>}}
   {\IgnorePar\HCode{<h2 class="maketitleHead">}\IgnorePar}
   {\HCode{</h2>}\IgnoreIndent}
\Configure{thanks author date and} 
   {\HCode{<div align="left"><i>}}{\HCode{</i></div>}}
   {\HCode{<div align="center">}}{\HCode{</div>}}
   {\HCode{<div align="center">}}{\HCode{</div>}}
   {\:nbsp\:nbsp\:nbsp}{\HCode{<br\xml:empty>}}
\Configure{thank}
   {\HCode{<sup>}\Link{tk-\the\c@footnote}{}} 
   {\EndLink\HCode{</sup>}}
   {\HCode{<br\xml:empty>}\protect\Link{}{tk-\the\c@footnote}\protect
     \EndLink\HCode{<sup>}}
   {\HCode{</sup>}} {}
>>>





\<configure aa 3.2/4.0t\><<<
\Configure{subtitle institute}
   {\HCode{<br\xml:empty><span class="subtitle">}}{\HCode{</span>}}
   {\HCode{<div class="institute">}}{\HCode{</div>}}
   {\Tg<sup>}{\Tg</sup>}
   {\HCode{<br\xml:empty>}}

\Configure{maketitle}
   {\HCode{<div align="center" class="maketitle">}}
   {\HCode{</div>}}
   {\NoFonts\IgnorePar\HCode{<h2 class="maketitleHead">}\IgnorePar}
   {\HCode{</h2>}\IgnoreIndent\EndNoFonts}
\Configure{thanks author date and}{}{}
   {\HCode{<div class="author" align="center">}}{\HCode{</div>}}
   {\HCode{<div class="date" align="center">}}{\HCode{</div>}}
   {\SPAN:{and}and\EndSPAN:}
   {\HCode{<br\xml:empty>}}
>>>



\<configure aa 3.2/4.0t\><<<
\ConfigureEnv{abstract}
   {\:xhtml{\IgnorePar\EndP}\HCode {<div class="abstract">}}
   {\HCode{</div>}}{}{}

\Css{div.abstract{text-align:center;}}

\Configure{makeheadbox}
   {\HCode{<table class="makeheadbox"
       width="100\%"><tr><td><table><tr><td>}}
   {\HCode{</td></tr><tr><td>}}
   {\HCode{</td></tr><tr><td>}}
   {\HCode{</td></tr></table></td><td class="AALogo" width="10\%">}}
   {\HCode{</td></tr></table>}}
>>>

% \Css{.AALogo{font-size:120\%;font-weight: bold; text-align:right;}}

\<32,4 amsbook\><<<
\Configure{authors}{\HCode{<center>}}{\HCode{</center>}}
\Configure{title}
  {\HCode{<center>}\IgnorePar\HCode{<h2 class="titleHead">}}
  {\HCode{</h2>}\IgnoreIndent\HCode{</center>}}
\Configure{addresses}{\HCode{<center>}\IgnorePar}
                     {\IgnorePar\HCode{</center>}}
\Configure{date}{\HCode{<center>}}{\HCode{</center>}}
\Configure{keywords}{\HCode{<center>}}{\HCode{</center>}}
\Configure{abstract}{\HCode{<div><\tbl:XV><tr><td\Hnewline
   >}\IgnorePar}{\end:env}
\ConfigureEnv{abstract}{}{}{}{}
\Configure{thanks}{\HCode{<center>}}{\HCode{</center>}}
\Configure{subjclass}
  {\ShowPar\par\HCode{<center>}\bgroup
      \Configure{HtmlPar}{\HCode{<br\xml:empty>}}
             {\HCode{<br\xml:empty>}}{}{}}
  {\egroup\HCode{</center>}}
\Configure{translators}{\HCode{<center>}}{\HCode{</center>}}

>>>




\<title for TITLE\><<<
{\Configure{maketitle}{}{}{}{}%
\a:NoSection |<disable latex fonts|>\more:no \let\thanks|=\:gobble
\let\\|=\empty \def\TeX{TeX}%
\def\gobble:font##1##2{##2}\:TITLE: \no:fonts
\Tag{TITLE+}{\@title}}
>>>

\<disable latex fonts\><<<
\let\leavevmode|=\empty \let\not@math@alphabet|=\:gobbleII
\def\text@command##1{##1}\let\selectfont|=\empty
\def\check@icl ##1\check@icr{}%
>>>


%%%%%%%%%%%%%%%%%%%%%%%%%%%%%%%%%%%%%%%%%%%%%%%%%%%%%%%%%%%%%%%%%%%%%%%%
\section{Footnotes}
%%%%%%%%%%%%%%%%%%%%%%%%%%%%%%%%%%%%%%%%%%%%%%%%%%%%%%%%%%%%%%%%%%%%%%%%


%%%%%%%%%%%%%%%%%%%%%%%%%%%%%%%%%%%%%%%%%%%%%%%%%%%%%%%%%%%%%%%%%%%%%%%%
\subsection{Plain}
%%%%%%%%%%%%%%%%%%%%%%%%%%%%%%%%%%%%%%%%%%%%%%%%%%%%%%%%%%%%%%%%%%%%%%%%



\<32,4 plain\><<<
\Configure{vfootnote}
   {\HPageButton[fn\FNnum]{\FNmark}}
   {\BeginHPage[fn\FNnum]{ }}
   {\EndHPage{}}
>>>


%%%%%%%%%%%%%%%%%%%%%%%%%%%%%%%%%%%%%%%%%%%%%%%%%%%%%%%%%%%%%%%%%%%%%%%%
\subsection{LaTeX}
%%%%%%%%%%%%%%%%%%%%%%%%%%%%%%%%%%%%%%%%%%%%%%%%%%%%%%%%%%%%%%%%%%%%%%%%


\<latex footnotes\><<<
|<footnote mark|>
|<footnote text|>
>>>



\<footnote text\><<<
\Configure{footnotetext}
   {|<open footnote page|>%
    \leavevmode \Link{}{|<footnote id|>}%
    \NoFonts
   }
   {\EndNoFonts \EndLink}
   {|<close footnote page|>}
>>>

\<footnote mark\><<<
\Configure{footnotemark}
   {\leavevmode |<footnotemark link|>\NoFonts
    \Tag{|<footnote id|>}{}}
   {\EndNoFonts |<end footnotemark link|>}
>>>


\<open footnote page\><<<
\ifTag{|<footnote id|>}
   {\expandafter\ifx\csname (|<footnote id|>)\endcsname\def
       \:warning{Multiple footnote texts for mark \FNnum}%
    \else  \HPage{}\fi
   }
   {}%
>>>

\<close footnote page\><<<
\ifTag{|<footnote id|>}
   {\expandafter\ifx\csname (|<footnote id|>)\endcsname\def
    \else  \EndHPage{}\fi
   }
   {}%
\expandafter\global\expandafter
   \let\csname (|<footnote id|>)\endcsname\def
>>>    

\<footnotemark link\><<<
\ifTag{|<link tag to footnotetext|>|<footnote id|>}
   {\Link{|<footnote id|>}{}}
   {}%
>>>

\<end footnotemark link\><<<
\ifTag{|<link tag to footnotetext|>|<footnote id|>}
   {\EndLink}
   {}%
>>>



\<footnote id\><<<
fn\FNnum x\minipageNum
>>>

\<link tag to footnotetext\><<<
)Q>>>


%%%%%%%%%%%%%%%%%%%%%%%%%%%%%%%%%%%%%%%%%%%%%%%%%%%%%%%%%%%%%%%%%%%%%%%%
\subsection{AMS}
%%%%%%%%%%%%%%%%%%%%%%%%%%%%%%%%%%%%%%%%%%%%%%%%%%%%%%%%%%%%%%%%%%%%%%%%


\<ams footnotes\><<<
|<ams footnote mark|>
|<ams footnote text|>
>>>



\<ams footnote text\><<<
\Configure{footnotetext}
   {|<open footnote page|>%
    \HCode{<sup>}\Link{}{|<footnote id|>}%
    \NoFonts
   }
   {\EndNoFonts \EndLink \HCode{</sup>}}
   {|<close footnote page|>}
>>>

\<ams footnote mark\><<<
\Configure{footnotemark}
   {\HCode{<sup>}|<footnotemark link|>\NoFonts
    \Tag{|<footnote id|>}{}}
   {\EndNoFonts |<end footnotemark link|>\HCode{</sup>}}
>>>







%%%%%%%%%%%%%%%%%%%%%%%%%%%%%%%%%%%%%%%%%%%%%%%%%%%%%%%%%%%%%%%%%%%%%%%%
\chapter{Floating}
%%%%%%%%%%%%%%%%%%%%%%%%%%%%%%%%%%%%%%%%%%%%%%%%%%%%%%%%%%%%%%%%%%%%%%%%

%%%%%%%%%%%%%%%%%
\section{Wrapfig}
%%%%%%%%%%%%%%%%%

\<configure html32 wrapfig\><<<
\Configure{wrapfloat}
    {\getWFplace
     \ifvmode\IgnorePar\EndP
       \HCode{<div \WFplace>}%
       \def\endWrap{\ifvmode\IgnorePar\fi\EndP\HCode{</div>}\par}%
     \else
       \HCode{<span \WFplace>}\bgroup
       |<config span caption|>%
       \def\endWrap{\egroup\HCode{</span>}}%
     \fi
    }
    {\endWrap}
\def\getWFplace{%
   \let\:tempa=\empty
   \def\:temp##1r##2//{\if !##2!\else
                       \def\:tempa{align="left"}\fi}%
   \expandafter\:temp\WFplace l//%
   \def\:temp##1r##2//{\if !##2!\else
                       \def\:tempa{align="right"}\fi}%
   \expandafter\:temp\WFplace r//%
   \let\WFplace=\:tempa }
>>>

\<config span caption\><<<
\Configure{caption}
   {\HCode{<br\xml:empty><span class="caption"><b>}}
   {\HCode{</b>}: }{}{\HCode{</span><br\xml:empty>}}%
>>>



%%%%%%%%%%%%%%%%%%%%%%%%%%%%%%%%%%%%%%%%%%%%%%%%%%%%%%%%%%%%%%%%%%%%%%%%
\chapter{Etc}
%%%%%%%%%%%%%%%%%%%%%%%%%%%%%%%%%%%%%%%%%%%%%%%%%%%%%%%%%%%%%%%%%%%%%%%%






\<0,32,4 preambles\><<<
|<date utility|>
|<cascade style sheets|>
\Configure{Preamble}
   {|<default cascade style sheets|>} {}
>>>


\<default cascade style sheets\><<<
{\ifdim \lastskip>\z@ \unskip\fi  \IgnorePar\parindent\z@
\leavevmode}%
\immediate\write16{--- file \aa:CssFile\space ---}%
\ht:special{t4ht>\aa:CssFile}\ht:special{t4ht=\Hnewline /* css.sty */}%
\ht:special{t4ht<\aa:CssFile}%
>>>









\verb'\special' are like \verb'\hbox', and they so they may introduve empty lines in
vertical mode. That might be a problem if we don't want empty lines at
the start of the files. Hence, in latex we give them special treatment.



\<cascade style sheets\><<<
\ScriptCommand{\CssFile}{%
  \immediate\write16{--- file \aa:CssFile\space ---}%
  \def\FontSize##1##2{\:Context{##1}\ht:special{t4ht;\%##2}\%}%
  \def\FontName##1{\:Context{##1}\ht:special{t4ht;=}}%
  \def\:Context##1{\ht:special{t4ht>\jobname.tmp}##1\ht:special
     {t4ht>\aa:CssFile}}%
  \ht:special{t4ht>\jobname.tmp}\ht:special{t4ht>\aa:CssFile}\bb:CssFile
  \hfil\break\NoFonts}{\EndNoFonts
  \ht:special{t4ht<\aa:CssFile}\ht:special{t4ht<\jobname.tmp}}
\let\Css:File|=\CssFile
\def\CssFile{\futurelet\:temp\Css:Fl}
\def\Css:Fl{\ifx [\:temp  \expandafter\Css:fl
   \else \expand:after{\Css:File \space}\fi}
\def\Css:fl[#1]{\Css:File\space \css:files #1,,|<par del|>}
\def\css:files#1,#2|<par del|>{\def\:temp{#1}\ifx \:temp\empty
   \else \def\:temp{\in:css#1.|<par del|>\css:files#2,,|<par del|>}\fi
   \:temp }
\def\in:css#1.#2|<par del|>{\def\:temp{#2}\ifx \:temp\empty \input #1.css
   \else \inc:ss#1.#2|<par del|>\fi}
\def\inc:ss#1.|<par del|>{\input #1 }
\NewConfigure{CssFile}[2]{\def\aa:CssFile{#1}\def\bb:CssFile{#2}}
>>>

\verb'\CssFile[file-name,filename.ext,..]...\EndCssFile'.

Default file, just in case the user doesn't provide one. If
the user does, the following file will be overwritten.

Can't use below \verb'\a:CssFile' and \verb'\b:CssFile', because
\verb'\ScriptFile{\CssFile}' also needs them.



\verb'\Css' changes its definition upon reachin \verb'\CssFile'.  The 
first definition is needed within the sty files, and the info is
sent to the lg file (where else it can be sent?).


\<cascade style sheets\><<<
\def\Css#1{{\def\:temp{\Configure{Needs}}%
   \expandafter\:temp\expandafter{\aa:Css}\Needs{#1}}}
\let\send:css|=\Css
\ScriptCommand{\Css}{\HCode{<style
   type="text/css">\Hnewline}\NoFonts}{\EndNoFonts\HCode{</style>}}
\let\loc:css|=\Css
\def\Css{\futurelet\:temp\:Css}
\def\:Css{\ifx \:temp\bgroup \expandafter\send:css
   \else \expandafter\loc:css\fi}
>>>



\<date utility\><<<
\tmp:cnt|=\time  \divide\tmp:cnt |by 60
\edef\:temp{\the\tmp:cnt}
\multiply\tmp:cnt |by -60 \advance\tmp:cnt |by \time
\edef\:today{\the\year-\ifnum \month<10 0\fi
  \the\month-\ifnum \day<10 0\fi\the\day 
   \space\ifnum \:temp<10 0\fi \:temp 
   :\ifnum \tmp:cnt<10 0\fi\the\tmp:cnt :00}
>>>


\<date utility\><<<
\:CheckOption{hooks++} \if:Option
    \else \:CheckOption{hooks+}  
          \if:Option \else \:CheckOption{hooks}\fi
    \fi
\if:Option
   \Configure{hooks}
      {\HCode{<strong class="hooks">&lt;}}{\HCode{&gt;</strong>}}{}{}  
\fi
>>>






The following provides a faster version
than \verb'\LinkCommand\Link{a,href,name,}' for the \verb'\Link'
command

\<32,4 tex4ht\><<<
\Configure{Link}{a}{href=}{name=}{}
>>>



\section{article}

\<config book-report-article 3.2\><<<
|<book-report-article make title|>
|<book-report-article caption 3.2|>
|<latex report,... config 3.2|>
|<latex config div|>
|<latex config like div 3.2|>
>>>



\<32,4 report,book\><<<
\ConfigureEnv{description}{\IgnorePar}{}{}{}

>>>



\section{alsart}



\<configure html32 elsart\><<<
|<32,4 elsart|>
\Configure{abstract}{\HCode{<\tbl:XV{abstract}><tr><td\Hnewline
   >}}{\end:env}
\Configure{keyword}{\HCode{<\tbl:XV{keyword}><tr><td\Hnewline
   >}}{\end:env}
\Configure{title}
   {\IgnorePar\EndP\Tg<h2 class="title">\NoFonts}
   {\EndNoFonts\Tg</h2>}
>>>


\<32,4 elsart\><<<
\ConfigureEnv{frontmatter}
  {\EndP\IgnorePar
   \HCode{<\tbl:XV{frontmatter}><tr><td><div align="center"\Hnewline>}}
  {\IgnorePar\EndP\HCode{</div>}\end:TTT\IgnorePar}
  {}{}
>>>

\section{report}



\<32,4 report\><<<
\Configure{chapter}{}{}
   {\IgnorePar\EndP\HCode{<h2 class="chapterHead">}%
    \chaptername \ \thechapter\HCode{<br\xml:empty>}}
   {\HCode{</h2>}\IgnoreIndent}
\Configure{chapterTITLE+}{\thechapter\space#1}
>>>


\<32,4 report,book\><<<
\Configure{appendix}{}{}
   {\IgnorePar\EndP\HCode{<h2 class="appendixHead">}
      \appendixname \ \thechapter\HCode{<br\xml:empty>}}
   {\HCode{</h2>}\IgnoreIndent}
\Configure{appendixTITLE+}{\thechapter\space#1}
\Configure{likechapter}{}{}
   {\IgnorePar\EndP\HCode{<h2 class="likechapterHead">}}
   {\HCode{</h2>}\IgnoreIndent}

\Configure{endchapter}{likechapter,appendix,part,likepart}
\Configure{endlikechapter}{chapter,appendix,part,likepart}
\Configure{endappendix}{chapter,likechapter,part,likepart}
>>>

\section{book}







\<32,4 book\><<<
\Configure{chapter}{}{}
   {\IgnorePar\EndP\HCode{<h2 class="chapterHead">}
     \if@mainmatter \chaptername \ \thechapter\HCode{<br\xml:empty>}\fi}
   {\HCode{</h2>}\IgnoreIndent}
\Configure{chapterTITLE+}{\if@mainmatter\thechapter\space\fi#1}
\renewcommand\thechapter {\if@mainmatter\@arabic\c@chapter\fi}
>>>






%%%%%%%%%%%%%%%%%%%%%%%
\section{latex.ltx}
%%%%%%%%%%%%%%%%%%%%


\<latex options 1, 2, 3\><<<
\:CheckOption{7}     \if:Option
    \expandafter\ifx \csname @chapter\endcsname\relax
         |<cut toc: part|>      |%cut toc before cutat|%
         |<cutat: part|>
         |<cut toc: sec|>       
         |<cutat: sec|>    
         |<cut toc: subsec|>  
         |<cutat: subsection|>
         |<cut toc: subsubsec|>  
         |<cutat: subsubsection|>
         |<cut toc: paragraph|>  
         |<cutat: paragraph|>
         |<cut toc: subparagraph|>  
         |<cutat: subparagraph|>
         |<tocat: part|>
         |<tocat: section|>
         |<tocat: subsection|>
         |<tocat: subsubsection|>
         |<tocat: paragraph|>
    \else
         |<cut toc: part|>      |%cut toc before cutat|%
         |<cutat: part|>
         |<cut toc: ch|>
         |<cutat: chapter|>
         |<cut toc: sec|>
         |<cutat: sec|>
         |<cut toc: subsec|>
         |<cutat: subsection|>
         |<cut toc: subsubsec|>  
         |<cutat: subsubsection|>
         |<cut toc: paragraph|>  
         |<cutat: paragraph|>
         |<cut toc: subparagraph|>  
         |<cutat: subparagraph|>
         |<tocat: part|>     
         |<tocat: ch|>        
         |<tocat: section|>   
         |<tocat: subsection|>
         |<tocat: subsubsection|>
         |<tocat: paragraph|>
    \fi
\else\:CheckOption{6}     \if:Option
    \expandafter\ifx \csname @chapter\endcsname\relax
         |<cut toc: part|>      |%cut toc before cutat|%
         |<cutat: part|>
         |<cut toc: sec|>       
         |<cutat: sec|>    
         |<cut toc: subsec|>  
         |<cutat: subsection|>
         |<cut toc: subsubsec|>  
         |<cutat: subsubsection|>
         |<cut toc: paragraph|>  
         |<cutat: paragraph|>
         |<tocat: part|>
         |<tocat: section|>
         |<tocat: subsection|>
         |<tocat: subsubsection|>
    \else
         |<cut toc: part|>      |%cut toc before cutat|%
         |<cutat: part|>
         |<cut toc: ch|>
         |<cutat: chapter|>
         |<cut toc: sec|>
         |<cutat: sec|>
         |<cut toc: subsec|>
         |<cutat: subsection|>
         |<cut toc: subsubsec|>  
         |<cutat: subsubsection|>
         |<cut toc: paragraph|>  
         |<cutat: paragraph|>
         |<tocat: part|>     
         |<tocat: ch|>        
         |<tocat: section|>   
         |<tocat: subsection|>
         |<tocat: subsubsection|>
    \fi
\else \:CheckOption{5}     \if:Option
    \expandafter\ifx \csname @chapter\endcsname\relax
         |<cut toc: part|>      |%cut toc before cutat|%
         |<cutat: part|>
         |<cut toc: sec|>       
         |<cutat: sec|>    
         |<cut toc: subsec|>  
         |<cutat: subsection|>
         |<cut toc: subsubsec|>  
         |<cutat: subsubsection|>
         |<tocat: part|>
         |<tocat: section|>
         |<tocat: subsection|>
    \else
         |<cut toc: part|>      |%cut toc before cutat|%
         |<cutat: part|>
         |<cut toc: ch|>
         |<cutat: chapter|>
         |<cut toc: sec|>
         |<cutat: sec|>
         |<cut toc: subsec|>
         |<cutat: subsection|>
         |<cut toc: subsubsec|>  
         |<cutat: subsubsection|>
         |<tocat: part|>     
         |<tocat: ch|>        
         |<tocat: section|>   
         |<tocat: subsection|>   
    \fi
\else\:CheckOption{4}     \if:Option
    \expandafter\ifx \csname @chapter\endcsname\relax
         |<cut toc: part|>      |%cut toc before cutat|%
         |<cutat: part|>
         |<cut toc: sec|>       
         |<cutat: sec|>    
         |<cut toc: subsec|>  
         |<cutat: subsection|>
         |<tocat: part|>
         |<tocat: section|>
    \else
         |<cut toc: part|>      |%cut toc before cutat|%
         |<cutat: part|>
         |<cut toc: ch|>
         |<cutat: chapter|>
         |<cut toc: sec|>
         |<cutat: sec|>
         |<cut toc: subsec|>
         |<cutat: subsection|>
         |<tocat: part|>     
         |<tocat: ch|>        
         |<tocat: section|>   
    \fi
\else\:CheckOption{3}     \if:Option
    \expandafter\ifx \csname @chapter\endcsname\relax
         |<cut toc: part|>     |%cut toc before cutat|%
         |<cutat: part|>
         |<cut toc: sec|>
         |<cutat: sec|>
         |<cut toc: subsec|>
         |<cutat: subsection|>
         |<tocat: part|>     
         |<tocat: section|>   
    \else
         |<cut toc: part|>      |%cut toc before cutat|%
         |<cutat: part|>
         |<cut toc: ch|>
         |<cutat: chapter|>
         |<cut toc: sec|>
         |<cutat: sec|>
         |<tocat: part|>      
         |<tocat: ch|>     
    \fi
\else\:CheckOption{2}     \if:Option
    \expandafter\ifx \csname @chapter\endcsname\relax 
         |<cut toc: part|>      |%cut toc before cutat|%
         |<cutat: part|>
         |<cut toc: sec|>
         |<cutat: sec|> 
         |<tocat: part|>    
    \else
         |<cut toc: part|>      |%cut toc before cutat|%
         |<cutat: part|>
         |<cut toc: ch|>
         |<cutat: chapter|>
         |<tocat: part|>     
    \fi
\else\:CheckOption{1}     \if:Option
         |<cut toc: part|>      |%cut toc before cutat|%
         |<cutat: part|>                         
\else
    \Log:Note{for automatic sectioning 
        pagination, use the command line option 
                     `1', `2', `3', '4', '5', '6', or '7'}%
\fi \fi \fi \fi \fi \fi \fi
>>>




\<cut toc: part\><<<
\:CheckOption{notoc*}     \if:Option
   \Configure{tableofcontents*}
      {part,chapter,appendix}
\else
   |<notoc* note|>
   \Configure{tableofcontents*}
      {part,likepart,chapter,likechapter,appendix}
\fi
>>>



\<cutat: part\><<<
\CutAt{part}
\CutAt{likepart}
>>>




\<tocat: part\><<<
\:CheckOption{notoc*}     \if:Option
\else
  \:CheckOption{nominitoc}     \if:Option
   \else
      |<note nominitoc|>
     \TocAt*{part,/likepart,chapter,likechapter,appendix,%
          section,likesection}
     \TocAt*{likepart,/part,chapter,likechapter,appendix,%
          section,likesection}
  \fi
\fi
>>>

\<note nominitoc\><<<
\Log:Note{to eliminate mini tables of
          contents, use the command line option `nominitoc'}
>>>


\<cut toc: ch\><<<
\:CheckOption{notoc*}     \if:Option
  \Configure{tableofcontents*}{part,chapter,%
     appendix,section\expandafter\ifx 
     \csname @chapter\endcsname\relax ,subsection\fi}
\else
  |<notoc* note|>
  \Configure{tableofcontents*}{part,likepart,chapter,likechapter,%
     appendix,section,likesection\expandafter\ifx 
     \csname @chapter\endcsname\relax ,subsection,likesubsection\fi}
\fi
>>>





\<cutat: chapter\><<<
\CutAt{chapter,likechapter,appendix,part}
\CutAt{likechapter,appendix,part}
\CutAt{appendix,chapter,likechapter,part}
>>>

\<cut toc: sec\><<<
\:CheckOption{notoc*}     \if:Option
  \Configure{tableofcontents*}{part,chapter,appendix,section%
        \expandafter\ifx \csname @chapter\endcsname\relax
     ,subsection\fi}
\else
  |<notoc* note|>
  \Configure{tableofcontents*}{part,likepart,chapter,likechapter,%
     appendix,section,likesection%
        \expandafter\ifx \csname @chapter\endcsname\relax
     ,subsection,likesubsection\fi}
\fi
>>>


\<tocat: section\><<<
\:CheckOption{notoc*}     \if:Option
  \:CheckOption{nominitoc}     \if:Option
   \else
      |<note nominitoc|>
     \TocAt*{section,/likesection,/chapter,/likechapter,/appendix,/part,%
             subsection,subsubsection}
     \TocAt*{likesection,/section,/chapter,/likechapter,/appendix,/part,%
             subsection,subsubsection}
  \fi
\else
  \:CheckOption{nominitoc}     \if:Option
   \else
      |<note nominitoc|>
     \TocAt*{section,/likesection,/chapter,/likechapter,/appendix,/part,%
          subsection,likesubsection,subsubsection,likesubsubsection}
     \TocAt*{likesection,/section,/chapter,/likechapter,/appendix,/part,%
          subsection,likesubsection,subsubsection,likesubsubsection}
  \fi
\fi
>>>

\<cutat: sec\><<<
\CutAt{section,likesection,chapter,likechapter,appendix,part}
\CutAt{likesection,chapter,likechapter,appendix,part}
>>>

\<tocat: ch\><<<
\:CheckOption{notoc*}     \if:Option
  \:CheckOption{nominitoc}     \if:Option
   \else
      |<note nominitoc|>
     \TocAt*{chapter,/likechapter,/appendix,/part,%
             section,subsection}
     \TocAt*{likechapter,/appendix,/chapter,/part,%
             section,subsection}
     \TocAt*{appendix,/chapter,/likechapter,/part,%
             section,subsection}
   \fi
\else
  \:CheckOption{nominitoc}     \if:Option
   \else
      |<note nominitoc|>
     \TocAt*{chapter,/likechapter,/appendix,/part,%
             section,likesection,subsection,likesubsection}
     \TocAt*{likechapter,/appendix,/chapter,/part,%
             section,likesection,subsection,likesubsection}
     \TocAt*{appendix,/chapter,/likechapter,/part,%
             section,likesection,subsection,likesubsection}
   \fi
\fi
>>>


\<cut toc: subsec\><<<
\:CheckOption{notoc*}     \if:Option
  \Configure{tableofcontents*}{part,chapter,%
    appendix,section,subsection}
\else
  |<notoc* note|>
  \Configure{tableofcontents*}{part,likepart,chapter,likechapter,%
    appendix,section,likesection,likesubsection,subsection}
\fi
>>>


\<cutat: subsection\><<<
\CutAt{subsection,section,likesection,%
                  chapter,likechapter,appendix,part}
\CutAt{likesubsection,section,likesection,%
                  chapter,likechapter,appendix,part}
>>>



\<tocat: subsection\><<<
\:CheckOption{notoc*}     \if:Option
  \:CheckOption{nominitoc}     \if:Option
   \else
      |<note nominitoc|>
     \TocAt*{subsection,/likesubsection,/section,/likesection,%
             /chapter,/likechapter,%
             /appendix,/part,%
             subsubsection,paragraph}
     \TocAt*{likesubsection,/subsection,/likesection,%
             /section,/chapter,/likechapter,/appendix,/part,%
             subsubsection,paragraph}
  \fi
\else
  \:CheckOption{nominitoc}     \if:Option
   \else
      |<note nominitoc|>
     \TocAt*{subsection,/likesubsection,/section,/likesection,%
          /chapter,/likechapter,/appendix,/part,%
          subsubsection,likesubsubsection,%
          paragraph}
     \TocAt*{likesubsection,/subsection,%
          /likesection,/section,/chapter,/likechapter,/appendix,/part,%
          subsubsection,likesubsubsection,%
          paragraph}
  \fi
\fi
>>>


\<cut toc: subsubsec\><<<
\:CheckOption{notoc*}     \if:Option
  \Configure{tableofcontents*}{part,chapter,%
    appendix,section,subsection,subsubsection}
\else
  |<notoc* note|>
  \Configure{tableofcontents*}{part,likepart,chapter,likechapter,%
    appendix,section,likesection,%
    likesubsection,subsection,likesubsubsection,subsubsection}
\fi
>>>




\<tocat: subsubsection\><<<
\:CheckOption{notoc*}     \if:Option
  \:CheckOption{nominitoc}     \if:Option
   \else
      |<note nominitoc|>
     \TocAt*{subsubsection,/likesubsubsection,/subsection,%
             /likesubsection,/section,/likesection,%
             /chapter,/likechapter,%
             /appendix,/part,%
             paragraph,subparagraph}
     \TocAt*{likesubsubsection,/subsubsection,/likesubsection,%
             /subsection,/likesection,%
             /section,/chapter,/likechapter,/appendix,/part,%
             paragraph,subparagraph}
  \fi
\else
  \:CheckOption{nominitoc}     \if:Option
   \else
      |<note nominitoc|>
     \TocAt*{subsubsection,/likesubsubsection,%
          /subsection,/likesubsection,/section,/likesection,%
          /chapter,/likechapter,/appendix,/part,%
          paragraph,subparagraph}
     \TocAt*{likesubsubsection,/subsubsection,/likesubsection,/subsection,%
          /likesection,/section,/chapter,/likechapter,/appendix,/part,%
          paragraph,subparagraph}
  \fi
\fi
>>>






\<cut toc: paragraph\><<<
\:CheckOption{notoc*}     \if:Option
  \Configure{tableofcontents*}{part,chapter,%
    appendix,section,subsection,subsubsection,paragraph}
\else
  |<notoc* note|>
  \Configure{tableofcontents*}{part,likepart,chapter,likechapter,%
    appendix,section,likesection,%
    likesubsection,subsection,likesubsubsection,subsubsection,%
    paragraph}
\fi
>>>


\<tocat: paragraph\><<<
\:CheckOption{notoc*}     \if:Option
  \:CheckOption{nominitoc}     \if:Option
   \else
      |<note nominitoc|>
     \TocAt*{paragraph,/subsubsection,/likesubsubsection,/subsection,%
             /likesubsection,/section,/likesection,%
             /chapter,/likechapter,%
             /appendix,/part,%
             subparagraph}
  \fi
\else
  \:CheckOption{nominitoc}     \if:Option
   \else
      |<note nominitoc|>
     \TocAt*{paragraph,/subsubsection,/likesubsubsection,%
          /subsection,/likesubsection,/section,/likesection,%
          /chapter,/likechapter,/appendix,/part,%
          subparagraph}
  \fi
\fi
>>>








\<cut toc: subparagraph\><<<
\:CheckOption{notoc*}     \if:Option
  \Configure{tableofcontents*}{part,chapter,%
    appendix,section,subsection,subsubsection,%
    paragraph,subparagraph}
\else
  |<notoc* note|>
  \Configure{tableofcontents*}{part,likepart,chapter,likechapter,%
    appendix,section,likesection,%
    likesubsection,subsection,likesubsubsection,subsubsection,%
    paragraph,subparagraph}
\fi
>>>


\<cutat: subsubsection\><<<
\CutAt{subsubsection,subsection,likesubsection,section,likesection,%
                  chapter,likechapter,appendix,part}
\CutAt{likesubsubsection,subsection,likesubsection,section,likesection,%
                  chapter,likechapter,appendix,part}
>>>

\<cutat: paragraph\><<<
\CutAt{paragraph,subsubsection,likesubsubsection,subsection,likesubsection,%
       section,likesection,chapter,likechapter,appendix,part}
>>>

\<cutat: subparagraph\><<<
\CutAt{subparagraph,paragraph,subsubsection,likesubsubsection,%
       subsection,likesubsection,%
       section,likesection,chapter,likechapter,appendix,part}
>>>












                                              %%%%%%%%%%%%%%%%%%%%%%%
                                              % ltplain.dtx
                                              %%%%%%%%%%%%%%%%%%%%%%%

\subsection{obeylines}



                                              %%%%%%%%%%%%%%%%%%%%%%%
                                              % ltspace.dtx
                                              %%%%%%%%%%%%%%%%%%%%%%%

\subsection{Spaces}

\<32,4 latex\><<<
\Configure{hspace}{}{}{\:nbsp}
>>>



                                              %%%%%%%%%%%%%%%%%%%%%%%
                                              % ltlogos.dtx
                                              %%%%%%%%%%%%%%%%%%%%%%%

\subsection{Logos}








                                              %%%%%%%%%%%%%%%%%%%%%%%
                                              % ltoutenc.dtx
                                              %%%%%%%%%%%%%%%%%%%%%%%



\<0,32,4 plain,latex accents\><<<
\:CheckOption{new-accents}     \if:Option
   |<new accents|>
\else
   |<old accents |>
   |<old plain,latex accents|>
\fi
\let\^^_|=\v
>>>


\<0,32,4 latex\><<<
|<0,32,4 plain,latex accents|>
\let\@acci|=\' \let\@accii|=\` \let\@acciii|=\=       
>>>

\<0,32,4 plain\><<<
|<0,32,4 plain,latex accents|>
>>>


\<new accents\><<<
\:CheckOption{accent-}     \if:Option
  \Configure{HAccent}\acute{AEIOUYaeiouy{}}{\Picture+{}}{\EndPicture}
  \Configure{HAccent}\bar{}{\Picture+{}}{\EndPicture}
  \Configure{HAccent}\breve{}{\Picture+{}}{\EndPicture}
  \Configure{HAccent}\check{}{\Picture+{}}{\EndPicture}
  \Configure{HAccent}\ddot{AEIOUYaeiouy{}}{\Picture+{}}{\EndPicture}
  \Configure{HAccent}\dot{}{\Picture+{}}{\EndPicture}
  \Configure{HAccent}\grave{AEIOUaeiou{}}{\Picture+{}}{\EndPicture}
  \Configure{HAccent}\hat{AEIOUaeiou{}}{\Picture+{}}{\EndPicture}
  \Configure{HAccent}\tilde{AOaoNn{}}{\Picture+{}}{\EndPicture}
  \Configure{HAccent}\vec{}{\Picture+{}}{\EndPicture}
  \Configure{HAccent}\widehat{}{\Picture+{}}{\EndPicture}
  \Configure{HAccent}\widetilde{}{\Picture+{}}{\EndPicture}
\fi
\:CheckOption{mathaccent-}     \if:Option
  \Configure{HAccent}\"{AEIOUYaeiouy{}}{\Picture+{}}{\EndPicture}
  \Configure{HAccent}\'{AEIOUYaeiouy{}}{\Picture+{}}{\EndPicture}
  \Configure{HAccent}\.{}{\Picture+{}}{\EndPicture}
  \Configure{HAccent}\={}{\Picture+{}}{\EndPicture}
  \Configure{HAccent}\H{}{\Picture+{}}{\EndPicture}
  \Configure{HAccent}\^{AEIOUaeiou{}}{\Picture+{}}{\EndPicture}
  \Configure{HAccent}\`{AEIOUaeiou{}}{\Picture+{}}{\EndPicture}
  \Configure{HAccent}\b{}{\Picture+{}}{\EndPicture}
  \Configure{HAccent}\c{Cc{}}{\Picture+{}}{\EndPicture}
  \Configure{HAccent}\d{}{\Picture+{}}{\EndPicture}
  \Configure{HAccent}\t{}{\Picture+{}}{\EndPicture}
  \Configure{HAccent}\u{}{\Picture+{}}{\EndPicture}
  \Configure{HAccent}\v{}{\Picture+{}}{\EndPicture}
  \Configure{HAccent}\~{AOaoNn{}}{\Picture+{}}{\EndPicture}
\fi
>>>

\<new accents\><<<
\Configure{accent}{*}
   {<!--tex4ht:accent\Hnewline font="}{" char="}{" type="}{"-->}
   {<!--tex4ht:end accent-->}
\Configure{mathaccent}{*}
   {<!--tex4ht:mathaccent\Hnewline font="}{" char="}{" type="}{"-->}
   {<!--tex4ht:end mathaccent-->}
\Configure{accented}{*}
   {<!--tex4ht:accented\Hnewline font="}{" char="}{" type="}{"-->}
   {<!--tex4ht:end accented-->}
\Configure{accenting}{*}
   {<!--tex4ht:accenting\Hnewline-->}
   {<!--tex4ht:end accenting-->}
>>>


\<old accents\><<<
\Configure{accent}\`\grave{A{A}E{E}I{I}O{O}U{U}%
                    a{a}e{e}i{i}\i{i}o{o}u{u}{}{}}
   {\a:accents{grave}{#1}}   {\b:accents{grave}{#1}{#2}}
\Configure{accent}\'\acute{A{A}E{E}I{I}O{O}U{U}Y%
           {Y}a{a}e{e}i{i}\i{i}o{o}u{u}y{y}{}{}}
   {\a:accents{acute}{#1}}   {\b:accents{acute}{#1}{#2}}
\Configure{accent}\^\hat{A{A}E{E}I{I}O{O}U{U}a{a}%
                      e{e}i{i}\i{i}o{o}u{u}{}{}}
   {\a:accents{circ}{#1}}   {\b:accents{hat}{#1}{#2}}
\Configure{accent}\~\tilde{A{A}O{O}a{a}o{o}N{N}n{n}{}{}}
   {\a:accents{tilde}{#1}}   {\b:accents{tilde}{#1}{#2}}
\Configure{accent}\"\ddot{A{A}E{E}I{I}O{O}U{U}Y%
           {Y}a{a}e{e}i{i}\i{i}o{o}u{u}y{y}{}{34}}
   {\a:accents{uml}{#1}}     {\b:accents{uml}{#1}{#2}}
>>>





The following are also placed under accents configuration.

\<old accents\><<<
\Configure{accent}\c\c{C{C}c{c}{}{}}
   {\a:accents{cedil}{#1}}     {\b:accents{cedil}{#1}{#2}}
\Configure{accent}\t\t{{}{}}
   {\a:accents{udot}{#1}}     {\b:accents{udot}{#1}{#2}}
\Configure{accent}\H\H{{}{}} {}{\b:accents{Huml}{#1}{#2}}
>>>

The following originally have been defined to be parameter-less.



\<old accents\><<<
\Configure{accent}\.\dot{{}{}}  {}{\b:accents{dot}{#1}{#2}}
\Configure{accent}\u\breve{{}{}}{}{\b:accents{breve}{#1}{#2}}
\Configure{accent}\vec\vec{{}{}}{}{\b:accents{vec}{#1}{#2}}
\Configure{accent}\v\check{{}{}}{}{\b:accents{check}{#1}{#2}} 
\Configure{accent}\=\bar{{}{}}  {}{\b:accents{bar}{#1}{#2}}
>>>


%  \= macron

\<old accents\><<<
\Configure{accent}\widetilde\widetilde{{}{}} 
   {}{\b:accents{widetilde}{#1}{#2}}
\Configure{accent}\widehat\widehat{{}{}} 
   {}{\b:accents{widehat}{#1}{#2}}
>>>


\verb'\vec', \verb'\widetilde', and \verb'\widehat' are for math mode.
\verb'\b', \verb'\c', \verb'\d', \verb'\t', and \verb'\H' are for text mode.





%%%%%%%%%%%%%
\subsection{accents from html4}
%%%%%%%%%%%%%



\<latex accents\><<<
\Configure{add accent}{T1:2}
  {}{\ht:special{t4ht@+\string&\#x005E;}x}
  {}{}
>>>




\<babel accents\><<<
|<optional iso-8859-2 accents|>
>>>



\<old plain,latex accents\><<<
|<old iso-8859-1 accents|>
|<OT1 old iso-8859-1 accents|>
|<optional iso-8859-2 accents|>
>>>


\<old iso-8859-1 accents\><<<
\Configure{accent}\widetilde\widetilde{{}{}} 
   {\a:accents{widetilde}{#1}} {\b:accents{widetilde}{#1}{#2}}
\Configure{accent}\widehat\widehat{{}{}} 
   {\a:accents{widehat}{#1}} {\b:accents{widehat}{#1}{#2}}
>>>



\<configure html32 latex\><<<
\ifOption{charset=iso-8859-7}
   {|<T1 greek ldf iso-8859-7|>}
   {}
\:CheckOption{new-accents}     \if:Option
\else
   |<T1 old iso-8859-1 accents|>
\fi
\let\^^_|=\v
>>>



\<T1 old iso-8859-1 accents\><<<
\expand:after{\Configure{accent}}\csname T1\string\`\expandafter\endcsname
   \csname T1\string\`\endcsname{|<grave codes|>{}{}}
   {\a:accents{grave}{#1}}   {\b:accents{grave}{#1}{#2}}
\expand:after{\Configure{accent}}\csname T1\string\'\expandafter\endcsname
   \csname T1\string\'\endcsname{|<acute codes|>{}{}}
   {\a:accents{acute}{#1}}   {\b:accents{acute}{#1}{#2}}
\expand:after{\Configure{accent}}\csname T1\string\^\expandafter\endcsname
   \csname T1\string\^\endcsname{|<circumflex codes|>{}{}}
   {\a:accents{circ}{#1}}   {\b:accents{circ}{#1}{#2}}
\expand:after{\Configure{accent}}\csname T1\string\~\expandafter\endcsname
   \csname T1\string\~\endcsname{|<tilde codes|>{}{}}
   {\a:accents{tilde}{#1}}   {\b:accents{tilde}{#1}{#2}}
\expand:after{\Configure{accent}}\csname T1\string\"\expandafter\endcsname
   \csname T1\string\"\endcsname{|<diaeresis codes|>{}{34}}
   {\a:accents{uml}{#1}}     {\b:accents{uml}{#1}{#2}}
\expand:after{\Configure{accent}}\csname T1\string\r\endcsname
   \mathring{|<ring codes|>{}{}}
   {\a:accents{ring}{#1}}   {\b:accents{ring}{#1}{#2}}
>>>



\<T1 old iso-8859-1 accents\><<<
\expand:after{\expand:after{\Configure{accent}}%
   \csname T1\string\c\endcsname}%
   \csname T1\string\c\endcsname{|<cedilla codes|>{}{}}
   {\a:accents{cedil}{#1}}     {\b:accents{cedil}{#1}{#2}}
\expand:after{\expand:after{\Configure{accent}}%
   \csname T1\string\t\endcsname}%
   \csname T1\string\t\endcsname{{}{}}
   {\a:accents{udot}{#1}}     {\b:accents{udot}{#1}{#2}}
\expand:after{\expand:after{\Configure{accent}}%
   \csname T1\string\H\endcsname}%
   \csname T1\string\H\endcsname{{}{}} 
   {\a:accents{Huml}{#1}} {\b:accents{Huml}{#1}{#2}}
\expand:after{\expand:after{\Configure{accent}}%
   \csname T1\string\b\endcsname}%
   \csname T1\string\b\endcsname{{}{}}
   {\a:accents{b}{#1}}     {\b:accents{b}{#1}{#2}}
\expand:after{\expand:after{\Configure{accent}}%
   \csname T1\string\d\endcsname}%
   \csname T1\string\d\endcsname{{}{}}
   {\a:accents{d}{#1}}     {\b:accents{d}{#1}{#2}}
>>>



\<T1 old iso-8859-1 accents\><<<
\expand:after{\Configure{accent}}\csname T1\string\.\expandafter\endcsname
   \csname T1\string\.\endcsname
   {|<dot above codes|>{}{}}
   {\a:accents{dot}{#1}} {\b:accents{dot}{#1}{#2}}
\expand:after{\Configure{accent}}\csname T1\string\u\expandafter\endcsname
   \csname T1\string\u\endcsname
   {|<breve codes|>{}{}}
   {\a:accents{breve}{#1}} {\b:accents{breve}{#1}{#2}}
\expand:after{\Configure{accent}}\csname T1\string\vec\expandafter\endcsname
   \csname T1\string\vec\endcsname
   {|<vec iso-8859-1|>{}{}}
   {\a:accents{vec}{#1}} {\b:accents{vec}{#1}{#2}}
\expand:after{\Configure{accent}}\csname T1\string\v\expandafter\endcsname
   \csname T1\string\v\endcsname
   {|<caron codes|>{}{}}
   {\a:accents{check}{#1}} {\b:accents{check}{#1}{#2}} 
\expand:after{\Configure{accent}}\csname T1\string\=\expandafter\endcsname
   \csname T1\string\=\endcsname
   {|<macron codes|>{}{}}  
   {\a:accents{bar}{#1}} {\b:accents{bar}{#1}{#2}}
>>>



\<OT1 old iso-8859-1 accents\><<<
\expand:after{\Configure{accent}}\csname OT1\string\`\endcsname
   \grave{|<grave codes|>{}{}}
   {\a:accents{grave}{#1}}   {\b:accents{grave}{#1}{#2}}
\expand:after{\Configure{accent}}\csname OT1\string\'\endcsname
   \acute{|<acute codes|>{}{}}
   {\a:accents{acute}{#1}}   {\b:accents{acute}{#1}{#2}}
\expand:after{\Configure{accent}}\csname OT1\string\^\endcsname
   \hat{|<circumflex codes|>{}{}}
   {\a:accents{circ}{#1}}   {\b:accents{circ}{#1}{#2}}
\expand:after{\Configure{accent}}\csname OT1\string\~\endcsname
   \tilde{|<tilde codes|>{}{}}
   {\a:accents{tilde}{#1}}   {\b:accents{tilde}{#1}{#2}}
\expand:after{\Configure{accent}}\csname
           OT1\string\"\expandafter\endcsname
   \csname OT1\string\"\endcsname{|<diaeresis codes|>{}{34}}
   {\a:accents{uml}{#1}}     {\b:accents{uml}{#1}{#2}}
\Configure{accent}\ddot\ddot{|<diaeresis codes|>{}{34}}
   {\a:accents{uml}{#1}}     {\b:accents{uml}{#1}{#2}}
\expand:after{\Configure{accent}}\csname OT1\string\r\endcsname
   \mathring{|<ring codes|>{}{}}
   {\a:accents{ring}{#1}}   {\b:accents{ring}{#1}{#2}}
>>>


\<OT1 old iso-8859-1 accents\><<<
\expand:after{\expand:after{\Configure{accent}}%
   \csname OT1\string\c\endcsname}%
   \csname OT1\string\c\endcsname{|<cedilla codes|>{}{}}
   {\a:accents{cedil}{#1}}     {\b:accents{cedil}{#1}{#2}}
\expand:after{\expand:after{\Configure{accent}}%
   \csname OT1\string\t\endcsname}%
   \csname OT1\string\t\endcsname{{}{}}
   {\a:accents{udot}{#1}}     {\b:accents{udot}{#1}{#2}}
\expand:after{\expand:after{\Configure{accent}}%
   \csname OT1\string\H\endcsname}%
   \csname OT1\string\H\endcsname{{}{}}
   {\a:accents{Huml}{#1}} {\b:accents{Huml}{#1}{#2}}
\expand:after{\expand:after{\Configure{accent}}%
   \csname OT1\string\b\endcsname}%
   \csname OT1\string\b\endcsname{{}{}}
   {\a:accents{b}{#1}}     {\b:accents{b}{#1}{#2}}
\expand:after{\expand:after{\Configure{accent}}%
   \csname OT1\string\d\endcsname}%
   \csname OT1\string\d\endcsname{{}{}}
   {\a:accents{d}{#1}}     {\b:accents{d}{#1}{#2}}
>>>




\<OT1 old iso-8859-1 accents\><<<
\expand:after{\Configure{accent}}\csname OT1\string\.\endcsname
   \dot{|<dot above codes|>{}{}}
   {\a:accents{dot}{#1}} {\b:accents{dot}{#1}{#2}}
\expand:after{\Configure{accent}}\csname OT1\string\u\endcsname
   \breve{|<breve codes|>{}{}}
   {\a:accents{breve}{#1}} {\b:accents{breve}{#1}{#2}}
\expand:after{\Configure{accent}}\csname OT1\string\vec\endcsname
   \vec{|<vec iso-8859-1|>{}{}}
   {\a:accents{vec}{#1}} {\b:accents{vec}{#1}{#2}}
\expand:after{\Configure{accent}}\csname OT1\string\v\endcsname
   \check{|<caron codes|>{}{}}
   {\a:accents{check}{#1}} {\b:accents{check}{#1}{#2}} 
\expand:after{\Configure{accent}}\csname OT1\string\=\endcsname
   \bar{|<bar iso-8859-1|>{}{}}
   {\a:accents{bar}{#1}} {\b:accents{bar}{#1}{#2}}
>>>



\<OT1 old iso-8859-1 accents\><<<
|<ot1enc.def unicode|>
>>>



\<ot1enc.def unicode\><<<
\expandafter\def
   \csname OT1\string\l\endcsname{\ht:special{t4ht@+\string&{35}x0142{59}}x}
\expandafter\def
   \csname OT1\string\L\endcsname{\ht:special{t4ht@+\string&{35}x0141{59}}x}
>>>




\<optional iso-8859-2 accents\><<<
\def\:temp{charset=iso-8859-2}
\ifx \a:charset\:UnDef
      \ifx  \A:charset\:temp \let\:temp=\def \fi
\else \ifx  \a:charset\:temp \let\:temp=\def \fi
\fi
\ifx \:temp\def
   \:CheckOption{new-accents}     \if:Option \else
      |<old iso-8859-2 accents|>
\fi \fi
>>>


\<old iso-8859-2 accents\><<<
|<T1 old iso-8859-2 accents|>
>>>



\<old iso-8859-2 accents\><<<
\Configure{accent}\`\grave{|<grave codes|>{}{}}
   {\a:accents{grave}{#1}}   {\b:accents{grave}{#1}{#2}}
\Configure{accent}\'\acute{|<acute codes|>{}{}}
   {\a:accents{acute}{#1}}   {\b:accents{acute}{#1}{#2}}
\Configure{accent}\^\hat{|<circumflex codes|>{}{}}
   {\a:accents{hat}{#1}}   {\b:accents{hat}{#1}{#2}}
\Configure{accent}\~\tilde{|<tilde codes|>{}{}}
   {\a:accents{tilde}{#1}}   {\b:accents{tilde}{#1}{#2}}
\Configure{accent}\"\ddot{|<diaeresis codes|>{}{34}}
   {\a:accents{uml}{#1}}   {\b:accents{uml}{#1}{#2}}
>>>


\<old iso-8859-2 accents\><<<
\Configure{accent}\c\c{|<cedilla codes|>{}{}}
   {\a:accents{cedil}{#1}}     {\b:accents{cedil}{#1}{#2}}
\Configure{accent}\t\t{{}{}}
   {\a:accents{udot}{#1}}     {\b:accents{udot}{#1}{#2}}
\Configure{accent}\H\H{|<double acute iso-8859-2|>{}{}} 
  {\a:accents{Huml}{#1}}     {\b:accents{Huml}{#1}{#2}}
\Configure{accent}\b\b{{}{}}
   {\a:accents{b}{#1}}     {\b:accents{b}{#1}{#2}}
\Configure{accent}\d\d{{}{}}
   {\a:accents{d}{#1}}     {\b:accents{d}{#1}{#2}}
>>>


\<old iso-8859-2 accents\><<<
\Configure{accent}\.\.{|<dot above codes|>{}{}}  
   {\a:accents{dot}{#1}}      {\b:accents{dot}{#1}{#2}}
\Configure{accent}\dot\dot{|<dot above codes|>{}{}}  
   {\a:accents{dot}{#1}}      {\b:accents{dot}{#1}{#2}}
\Configure{accent}\u\breve{|<breve codes|>{}{}}
   {\a:accents{breve}{#1}}      {\b:accents{breve}{#1}{#2}}
\Configure{accent}\vec\vec{{}{}}
   {\a:accents{vec}{#1}}      {\b:accents{vec}{#1}{#2}}
\Configure{accent}\v\v{|<caron codes|>{}{}}
   {\a:accents{check}{#1}}      {\b:accents{check}{#1}{#2}} 
\Configure{accent}\check\check{|<caron codes|>{}{}}
   {\a:accents{check}{#1}}      {\b:accents{check}{#1}{#2}} 
\Configure{accent}\=\bar{{}{}} 
   {\a:accents{bar}{#1}}      {\b:accents{bar}{#1}{#2}}
>>>


\<configure html4 romanian\><<<
\def\A:charset{charset=iso-8859-2}
\:CheckOption{new-accents}     \if:Option \else
   |<old iso-8859-2 accents|>
\fi
>>>


\<configure html4 croatian\><<<
\def\A:charset{charset=iso-8859-2}
\:CheckOption{new-accents}     \if:Option \else
   |<old iso-8859-2 accents|>
\fi
>>>


\<configure html4 slovak\><<<
\def\A:charset{charset=iso-8859-2}
\:CheckOption{new-accents}     \if:Option \else
   |<old iso-8859-2 accents|>
\fi
>>>


\<configure html4 slovene\><<<
\def\A:charset{charset=iso-8859-2}
\:CheckOption{new-accents}     \if:Option \else
   |<old iso-8859-2 accents|>
\fi
>>>



\<T1 old iso-8859-2 accents\><<<
\expand:after{\Configure{accent}}\csname T1\string\`\expandafter\endcsname
   \csname T1\string\`\endcsname{|<grave codes|>{}{}}
   {\a:accents{grave}{#1}}   {\b:accents{grave}{#1}{#2}}
\expand:after{\Configure{accent}}\csname T1\string\'\expandafter\endcsname
   \csname T1\string\'\endcsname{|<acute codes|>{}{}}
   {\a:accents{acute}{#1}}   {\b:accents{acute}{#1}{#2}}
\expand:after{\Configure{accent}}\csname T1\string\^\expandafter\endcsname
   \csname T1\string\^\endcsname{|<circumflex codes|>{}{}}
   {\a:accents{circ}{#1}}   {\b:accents{circ}{#1}{#2}}
\expand:after{\Configure{accent}}\csname T1\string\~\expandafter\endcsname
   \csname T1\string\~\endcsname{|<tilde codes|>{}{}}
   {\a:accents{tilde}{#1}}   {\b:accents{tilde}{#1}{#2}}
\expand:after{\Configure{accent}}\csname T1\string\"\expandafter\endcsname
   \csname T1\string\"\endcsname{|<diaeresis codes|>{}{34}}
   {\a:accents{uml}{#1}}     {\b:accents{uml}{#1}{#2}}
\expand:after{\expand:after{\Configure{accent}}%
   \csname T1\string\r\endcsname}%
   \csname T1\string\r\endcsname{|<ring codes|>{}{}}
   {\a:accents{ring}{#1}}   {\b:accents{ring}{#1}{#2}}
>>>




\<T1 old iso-8859-2 accents\><<<
\expand:after{\expand:after{\Configure{accent}}%
   \csname T1\string\c\endcsname}%
   \csname T1\string\c\endcsname{|<cedilla codes|>{}{}}
   {\a:accents{cedil}{#1}}     {\b:accents{cedil}{#1}{#2}}
\expand:after{\expand:after{\Configure{accent}}%
   \csname T1\string\t\endcsname}%
   \csname T1\string\t\endcsname{{}{}}
   {\a:accents{udot}{#1}}     {\b:accents{udot}{#1}{#2}}
\expand:after{\expand:after{\Configure{accent}}%
   \csname T1\string\H\endcsname}%
   \csname T1\string\H\endcsname{|<double acute iso-8859-2|>{}{}}
   {\a:accents{Huml}{#1}}{\b:accents{Huml}{#1}{#2}}
\expand:after{\expand:after{\Configure{accent}}%
   \csname T1\string\b\endcsname}%
   \csname T1\string\b\endcsname{{}{}}
   {\a:accents{b}{#1}}     {\b:accents{b}{#1}{#2}}
\expand:after{\expand:after{\Configure{accent}}%
   \csname T1\string\d\endcsname}%
   \csname T1\string\d\endcsname{{}{}}
   {\a:accents{d}{#1}}     {\b:accents{d}{#1}{#2}}
>>>



\<T1 old iso-8859-2 accents\><<<
\expand:after{\Configure{accent}}\csname T1\string\.\expandafter\endcsname
   \csname T1\string\.\endcsname
   {|<dot iso-8859-2|>{}{}} 
   {\a:accents{dot}{#1}}{\b:accents{dot}{#1}{#2}}
\expand:after{\Configure{accent}}\csname T1\string\u\expandafter\endcsname
   \csname T1\string\u\endcsname
   {|<breve codes|>{}{}}
   {\a:accents{breve}{#1}}{\b:accents{breve}{#1}{#2}}
\expand:after{\Configure{accent}}\csname T1\string\vec\expandafter\endcsname
   \csname T1\string\vec\endcsname
   {|<vec iso-8859-2|>{}{}}
   {\a:accents{vec}{#1}}{\b:accents{vec}{#1}{#2}}
\expandafter\let\csname T1\string\v\endcsname\:UnDef
\expand:after{\Configure{accent}}\csname T1\string\v\expandafter\endcsname
   \csname T1\string\v\endcsname
   {|<caron codes|>{}{}}
   {\a:accents{check}{#1}}{\b:accents{check}{#1}{#2}} 
\expand:after{\Configure{accent}}\csname T1\string\=\expandafter\endcsname
   \csname T1\string\=\endcsname
   {|<bar iso-8859-2|>{}{}}
   {\a:accents{bar}{#1}}{\b:accents{bar}{#1}{#2}}
>>>







\<grave codes\><<<
A{00C0}E{00C8}I{00CC}N{01F8}O{00D2}U{00D9}W{1E80}%
Y{1EF2}a{00E0}e{00E8}i{00EC}n{01F9}o{00F2}u{00F9}%
w{1E81}y{1EF3}\i{00EC}%
>>>


\<acute codes\><<<
A{00C1}C{0106}D{010E}E{00C9}G{01F4}I{00CD}L{0139}N{0143}%
O{00D3}R{0154}S{015A}U{00DA}Y{00DD}Z{0179}a{00E1}c{0107}%
d{010F}e{00E9}g{01F5}i{00ED}l{013A}n{0144}o{00F3}r{0155}%
s{015B}u{00FA}y{00FD}z{017A}j{FFFD}J{FFDD}\i{00ED}\j{FFFD}%
>>>





\<circumflex codes\><<<
A{00C2}E{00CA}I{00CE}O{00D4}U{00DB}a{00E2}e{00EA}%
i{00EE}o{00F4}u{00FB}C{0108}c{0109}G{011C}g{011D}%
H{0124}h{0125}J{0134}j{0135}S{015C}s{015D}W{0174}%
w{0175}Y{0176}y{0177}Z{1E90}z{1E91}\i{00EE}\j{0135}%
>>>



\<tilde codes\><<<
A{00C3}N{00D1}O{00D5}a{00E3}n{00F1}o{00F5}I{0128}%
i{0129}U{0168}u{0169}V{1E7C}v{1E7D}E{1EBC}e{1EBD}%
Y{1EF8}y{1EF9}\i{0129}%
>>>






\<diaeresis codes\><<<
H{1E26}h{1E27}W{1E84}w{1E85}X{1E8C}x{1E8D}%
t{1E97}A{00C4}E{00CB}I{00CF}O{00D6}U{00DC}%
a{00E4}e{00EB}i{00EF}o{00F6}u{00FC}y{00FF}%
Y{0178}\i{00EF}%
>>>




\<ring codes\><<<
A{00C5}a{00E5}U{016E}u{016F}%
>>>







\<cedilla codes\><<<
K{0136}k{0137}L{013B}l{013C}N{0145}n{0146}%
R{0156}r{0157}S{015E}s{015F}T{0162}t{0163}%
E{0228}e{0229}D{1E10}d{1E11}H{1E28}h{1E29}%
C{00C7}c{00E7}G{0122}g{0123}%
>>>






\<dot above codes\><<<
C{010A}c{010B}E{0116}e{0117}G{0120}g{0121}%
I{0130}Z{017B}z{017C}A{0226}a{0227}O{022E}%
o{022F}B{1E02}b{1E03}D{1E0A}d{1E0B}F{1E1E}%
f{1E1F}H{1E22}h{1E23}M{1E40}m{1E41}N{1E44}%
n{1E45}P{1E56}p{1E57}R{1E58}r{1E59}S{1E60}%
s{1E61}T{1E6A}t{1E6B}W{1E86}w{1E87}X{1E8A}%
x{1E8B}Y{1E8E}y{1E8F}%
>>>









\<breve codes\><<<
g{011F}I{012C}i{012D}O{014E}o{014F}%
U{016C}u{016D}A{0102}a{0103}E{0114}%
e{0115}G{011E}\i{012D}%
>>>





\<vec iso-8859-1\><<<
>>>

\<bar iso-8859-1\><<<
>>>




\<caron codes\><<<
C{010C}c{010D}D{010E}d{010F}%
E{011A}e{011B}L{013D}l{013E}N{0147}n{0148}R{0158}r{0159}%
S{0160}s{0161}T{0164}t{0165}Z{017D}z{017E}A{01CD}a{01CE}I{01CF}%
i{01D0}O{01D1}o{01D2}U{01D3}u{01D4}G{01E6}g{01E7}K{01E8}k{01E9}%
j{01F0}H{021E}h{021F}\i{01D0}\j{01F0}%
>>>




                                              %%%%%%%%%%%%%%%%%%%%%%%
                                              % ltfssini.dtx
                                              %%%%%%%%%%%%%%%%%%%%%%%

\subsection{tt Font}

\<0,32,4 latex\><<<
\ifx \ttfamily\:UnDef \else \Configure{tt}{\ttfamily} \fi
>>>



                                              %%%%%%%%%%%%%%%%%%%%%%%
                                              % ltxref.dtx
                                              %%%%%%%%%%%%%%%%%%%%%%%

\subsection{Cross References}


\<32,4 latex\><<<
\Configure{ref}{\Link}{\EndLink}{}
>>>







                                              %%%%%%%%%%%%%%%%%%%%%%%
                                              % ltmiscen.dtx
                                              %%%%%%%%%%%%%%%%%%%%%%%

\subsection{Miscellaneous Environments}


\<32,4 plain,latex\><<<
\def\end:TTT{\EndP\HCode{</td></tr></table>}} 
>>>

\<32 latex\><<<
\ConfigureEnv{center}{\EndP}{\par \ShowPar}
  {\EndP\IgnorePar
   \HCode{<\tbl:XV{center}><tr><td><div align="center"\Hnewline>}}
  {\IgnorePar\EndP\HCode{</div>}\end:TTT\IgnorePar}
\ConfigureEnv{flushleft}{\par\leavevmode\IgnorePar}{\par \ShowPar}
                     {\start:fenv{left}}{\end:fenv}
\ConfigureEnv{flushright}{\par\leavevmode\IgnorePar}{\par \ShowPar}
                      {\start:fenv{right}}{\end:fenv}
>>>



\<32,4 latex\><<<
\ConfigureEnv{verbatim}{\env:verb{verbatim}}{\endenv:verb}{}{}
\ConfigureEnv{verbatim*}{\env:verb{verbatim}}{\endenv:verb}{}{}
\def\env:verb#1{\ifvmode \IgnorePar \fi\EndP\HCode{<\tbl:XV{#1}><tr
  class="#1"><td\Hnewline 
  class="#1">}\HCode{<pre class="#1">}\EndNoFonts}
\def\endenv:verb{\NoFonts
  \ifvmode \IgnorePar\fi \EndP
  \ht:special{t4ht=</pre>}\end:TTT \ShowPar\:xhtml{\par}}
\def\start:fenv#1{\:xhtml{\IgnorePar\EndP}\HCode{<table
    \Hnewline width="100\%"
    \:zbsp{flush#1}><tr><td><div align="#1"\Hnewline>}}
\def\end:fenv{\IgnorePar\HCode{</div>}\end:TTT\IgnorePar}
>>>











                                              %%%%%%%%%%%%%%%%%%%%%%%
                                              % ltmath.dtx
                                              %%%%%%%%%%%%%%%%%%%%%%%

\subsection{Math Setup}





\<configure html32-math latex\><<<
\:CheckOption{pic-eqnarray}  \if:Option      
   |<PIC eqnarray Config|>
\fi
>>>


\<PIC eqnarray Config\><<<
\ConfigureEnv{eqnarray}
     {\IgnorePar\EndP\Tg<div class="pic-eqnarray">\Picture*{}}
     {\EndPicture\Tg</div>}{}{}
\Css{div.pic-eqnarray {text-align:center;}}  
\ConfigureEnv{eqnarray*}
     {\IgnorePar\EndP\Tg<div class="pic-eqnarray-star">\Picture*{}}
     {\EndPicture\Tg</div>}{}{}
\Css{div.pic-eqnarray-star {text-align:center;}}  
>>>








                                              %%%%%%%%%%%%%%%%%%%%%%%
                                              % lttab.dtx
                                              %%%%%%%%%%%%%%%%%%%%%%%

\subsection{Tabbing, Tabular and Array Environments}




\<32,4 picmath latex\><<<
\:CheckOption{pic-array}  \if:Option
   \ConfigureEnv{array}
     {\IgnorePar\EndP\Tg<div class="pic-array">\Picture*{}$$}
     {$$\EndPicture\Tg</div>}{}{}
   \Css{div.pic-array {text-align:center;}}  
\fi
>>>


\<32,4 latex\><<<
\:CheckOption{pic-tabular}  \if:Option
   \ConfigureEnv{tabular}
     {\IgnorePar\EndP\Tg<div class="pic-tabular">\Picture*{}}
     {\EndPicture\Tg</div>}{}{}
   \Css{div.pic-tabular {text-align:center;}}  
\fi
>>>




\<vspace body for array/tabular\><<<
\append:def\vspc:s{\h:HBorder}%
\def\:tempb{\ifnum \tmp:cnt<\ar:cnt 
    \advance\tmp:cnt by 1 \append:def\vspc:s{\i:HBorder}%
    \expandafter\:tempb
  \fi }
\tmp:cnt|=0 \:tempb
\append:def\vspc:s{\j:HBorder}\global\let\vspc:s|=\vspc:s
>>>


\<0,32,4 latex\><<<
\Configure{hline}{\ifx \ar:cnt\:UnDef 
   \else\o:noalign:{|<hline body for array/tabular|>}\fi}
\Configure{//[]}{\ifx \ar:cnt\:UnDef 
   \else\o:noalign:{|<vspace body for array/tabular|>}\fi}
>>>



\<hline body for array/tabular\><<<
\append:def\hline:s{\a:HBorder}%
\def\:tempb{\ifnum \tmp:cnt<\ar:cnt 
    \advance\tmp:cnt by 1 \append:def\hline:s{\b:HBorder}%
    \expandafter\:tempb
  \fi }
\tmp:cnt|=0 \:tempb
\append:def\hline:s{\c:HBorder}\global\let\hline:s|=\hline:s
>>>





\<configure clear noalign\><<<
\Configure{noalign}{}{}
>>>

\<configure tabular noalign\><<<
\Configure{noalign}%
  {\f:tabular\d:tabular \HCode{<tr><td colspan="\ar:cnt">}}
  {\HCode{</td></tr>}\pend:def\TableNo{0}\c:tabular\e:tabular}%
>>>






\verb'\AllColMargins' Return a binary string in which 1 represents
a column, and 0 represents a `@'. \verb'\ColMargins' retrieves the
zeros before the 1's that represent the current and following 1's.











\subsection{The option @()}








\<configuring @()\><<<
\Configure{@{}}{}
>>>






We force border around the full table whenever a vertical line is
requested, because it makes the tables better looking within the
existing capabilities.



Currently, we either have empty \verb'\VBorder', or one defined to
\verb'\def\VBorder{border="1"}'.



When \verb'\putVBorder' is call  in \verb'\VBorder' 
we have a sequence of the form
\verb'<COLGROUP><COL ...">...</COLGROUP>...' with the last 
tag possibly missing.








   







%%%%%%%%%%%%%%%%%%%%%% to be placed %%%%%%%%%%%%%%%%%%%%%%%%%
\subsection{to be placed}
%%%%%%%%%%%%%%%%%%%
















\<0,32,4 plain,latex\><<<
\Configure{ }{\:nbsp}
>>>
















\<0,32,4 latex\><<<
\Configure{framebox}
   {\Picture+[]{ \a:@Picture{framebox}}} {\EndPicture}
\Configure{InsertTitle}{\let\label|=\lb:l
   \let\ref|=\o:ref \Configure{ref}{}{}{}}
\Configure{AfterTitle}{\let\index|=\:index
       \let\ref|=\:ref  \let\label|=\lb:l }
\Configure{NoSection}
  {\let\sv:index|=\index \let\sv:label|=\label \let\sv:ref|=\ref
   \let\sv:newline|=\newline \def\newline{ }%
   \let\sv:setfontsize|=\@setfontsize  \let\@setfontsize|=\:gobbleIII
   \let\index|=\@gobble  \let\label|=\@gobble  \let\ref|=\@gobble
  }
  {\let\index|=\sv:index \let\label|=\sv:label \let\ref|=\sv:ref
   \let\newline|=\sv:newline    \let\@setfontsize|=\sv:setfontsize 
  }
\Configure{oalign}{\Picture+{ \a:@Picture{oalign}}}{\EndPicture}

\Configure{TocLink}
  {\Link{#2}{#3}{\Configure{ref}{}{}{}\Configure{cite}{}{}{}{}#4}\EndLink}
>>>








\<configure html32-math latex\><<<
\Configure{stackrel}{\Picture+{ class="stackrel"}\mathrel}
                    {\limits ^}{\EndPicture}
>>>







Don't put \verb'\:xhtml{\IgnorePar\EndP}' on the following; \verb'\PicDisplay'
should handle it.


\<4 picmath th4,latex\><<<
\Configure{[]} 
   {\PicDisplay $$\everymath{}\everydisplay{}}
   {$$\EndPicDisplay |<try inline par|>}
\Configure{()}{\protect\PicMath$}{$\protect\EndPicMath}
>>>



\<0,32,4 latex\><<<
\Configure{picture}
    {\Picture+[PICT]{}}
    {\EndPicture}
>>>

\<32,4 latex\><<<
\Configure{cite}{}{}{\Link}{\EndLink}
\Configure{bibitem}{\Link}{\EndLink}
\ConfigureEnv{minipage}
   {\IgnorePar\EndP
    \HCode{<center class="minipage">}%
    {\ifnum 0=`}\fi
   }
   {\ifnum 0=`{\fi}%
    \EndP\HCode{</center>}}
   {}{}
\Configure{fbox}
  {\leavevmode\Picture+[]{ \a:@Picture{fbox}}} {\EndPicture}
>>>



\<shared sec div config\><<<
\Configure{endsection}
     {likesection,chapter,likechapter,appendix,part,likepart}
\Configure{endlikesection}
     {section,chapter,likechapter,appendix,part,likepart}
>>>





\<shared subsec div config\><<<
\Configure{endsubsection}
   {likesubsection,section,likesection,chapter,%
      likechapter,appendix,part,likpart}
\Configure{endlikesubsection}
   {subsection,section,likesection,chapter,%
      likechapter,appendix,part,likpart}
>>>





\<shared sub end div config\><<<
\Configure{endsubsubsection}
   {likesubsubsection,subsection,likesubsection,section,%
      likesection,chapter,likechapter,appendix,part,likpart}
\Configure{endlikesubsubsection}
   {subsubsection,subsection,likesubsection,section,%
      likesection,chapter,likechapter,appendix,part,likpart}
\Configure{endparagraph}
   {likeparagraph,subsubsection,likesubsubsection,subsection,%
    likesubsection,section,%
    likesection,chapter,likechapter,appendix,part,likpart}
\Configure{endlikeparagraph}
   {paragraph,subsubsection,likesubsubsection,subsection,%
    likesubsection,section,%
    likesection,chapter,likechapter,appendix,part,likpart}
\Configure{endsubparagraph}
   {likesubparagraph,likeparagraph,subsubsection,likesubsubsection,%
    subsection,likesubsection,section,%
    likesection,chapter,likechapter,appendix,part,likpart}
\Configure{endlikesubparagraph}
   {subparagraph,likeparagraph,subsubsection,likesubsubsection,%
    subsection,likesubsection,section,%
    likesection,chapter,likechapter,appendix,part,likpart}
>>>

\<latex shared div config\><<<
\ConfigureEnv{thebibliography}{\IgnorePar}{\IgnorePar\par}{}{}
\ifx \part\:UnDef \else
   |<latex shared part config|>
\fi
>>>











\<shared part div config\><<<
\Configure{endpart}{likepart}
\Configure{endlikepart}{endpart}
>>>



\<latex shared part config\><<<
\Configure{part}{}{}
   {\IgnorePar \IgnorePar\HCode{<h1 class="partHead">}%
    \partname \ \thepart\HCode{<br\xml:empty>}}
   {\HCode{</h1>}\IgnoreIndent}
\Configure{partTITLE+}{\thepart\space #1}
>>>


\<latex shared likepart config\><<<
\Configure{likepart}{}{}
   {\IgnorePar\IgnorePar\HCode{<h1 class="likepartHead">}}
   {\HCode{</h1>}\IgnoreIndent}
>>>













It is better to put the LI in the third field to avoid extra space 
to the following text.













\<save configure tableofcontents\><<<
\let\sv:atoc|=\a:tableofcontents
\let\sv:btoc|=\b:tableofcontents
\let\sv:ctoc|=\c:tableofcontents
\let\sv:dtoc|=\d:tableofcontents
\let\sv:etoc|=\e:tableofcontents
>>>


\<recall configure tableofcontents\><<<
\let\a:tableofcontents|=\sv:atoc
\let\b:tableofcontents|=\sv:btoc
\let\c:tableofcontents|=\sv:ctoc
\let\d:tableofcontents|=\sv:dtoc
\let\e:tableofcontents|=\sv:etoc
>>>




Earlier we had 
\verb'\:CheckOption{no-halign} \if:Option \else |<pic array|> \fi', 
and the same for pic tabular. Why?


\<no wrap\><<<
nowrap="nowrap"   >>>



\<32 plain,latex\><<<
\Configure{displaylines}
   {\HCode{<table \Hnewline border="0" width="100\%">}}
   {\HCode{</table>}}
   {\HCode{<tr><td\Hnewline valign="bottom" align="center" |<no wrap|> >}}
   {\HCode{</td></tr>}}
>>>



\<picmath plain,latex\><<<
\let\A:displaylines|=\a:displaylines
\let\B:displaylines|=\b:displaylines
\let\C:displaylines|=\c:displaylines
\let\D:displaylines|=\d:displaylines
\Configure{displaylines}
   {\ifmmode
       \def\A:displaylines{\Picture*{}}%
       \def\B:displaylines{\EndPicture}%  
    \fi 
    \A:displaylines} 
   {\B:displaylines} {\C:displaylines}{\D:displaylines}
>>>


%%%%%%%%%%%%%%%%%%
\section{Long Tables}
%%%%%%%%%%%%%%%%%%

\<configure html32 tex4ht\><<<
\HAssign\TableNo=0
\newif\ifHCond
>>>

\<configure html32 longtable\><<<
\:CheckOption{old-longtable}\if:Option 
  |<config old longtable|>
\else
  |<config new longtable|>
\fi
\:CheckOption{pic-longtable}  \if:Option
   \ConfigureEnv{longtable}
     {\IgnorePar\EndP
      \Tg<div class="pic-longtable">\Picture*{}}
     {\EndPicture\Tg</div>}{}{}
%   \Css{div.pic-longtable {text-align:center;}}
\else
   \Log:Note{for pictorial longtable,
                      use the command line option `pic-longtable'}
\fi
>>>


\<config new longtable\><<<   
\Configure{halignTB<>}{longtable}{\HCode{id="TBL-\TableNo"
    class="longtable" cellpadding="5" \VBorder}<>\HAlign}
\Configure{longtable}
   {\IgnorePar\EndP
    \gHAdvance\TableNo by 1
     \HCode{|<show input line no|><div class="longtable">}%
    \halignTB{longtable}%
 }
   {\HCode{</table></div>}}
   {\HCode{<tr \Hnewline}\halignTR\HCode{ id="TBL-\TableNo-\HRow-">}}
   {\r:HA}
   {\HCode{<td \ifnum \HMultispan>1 colspan="\HMultispan" \fi}%
    \halignTD \HCode{ id="TBL-\TableNo-\HRow-\HCol"
    \Hnewline class="td}|<tabular td align|>%
    \HCode{">}|<td save EndP|>%
    |<start array par box|>\PushStack\Table:Stck\TableNo}
   {\PopStack\Table:Stck\TableNo |<end array par box|>\d:HA}
\Configure{longtableparbox}{\IgnorePar\leavevmode\ShowPar\par}
>>>

\<config old longtable\><<<
\let\a:longtable=\a:tabular
\let\b:longtable=\b:tabular
\let\c:longtable=\c:tabular
\let\d:longtable=\d:tabular
\let\e:longtable=\e:tabular
\let\f:longtable=\f:tabular
>>>


\<show input line no\><<<
<!--l. \the\inputlineno-->%
>>>

\<tabular td align\><<<
\NoFonts
           \bgroup
              \ifx \ttfamily\:UnDef \else \ttfamily\fi
              \ColMargins
           \egroup 
\EndNoFonts
>>>

\<td save EndP\><<<
\SaveEndP 
>>>

\<td recall EndP\><<<
\RecallEndP 
>>>
\<start array par box\><<<
\par  \ShowPar
>>>

\<end array par box\><<<
\IgnorePar \EndP
>>>

\<html latex array/tabular Config 3.2\><<<<
\let\AllColMargins|=\empty
\Configure{VBorder}{\global\let\VBorder|=\empty}{\gdef\VBorder{border="1"}}{}{}

  \Configure{HBorder}
     {<tr\Hnewline class="hline">}
         {<td><hr\xml:empty></td>}   {</tr>}
     {<tr\Hnewline class="cline">}
     {<td></td>} {<td><hr\xml:empty></td>} {</tr>}
     {<tr\Hnewline class="vspace" style="font-size:\HBorderspace">} 
         {<td\Hnewline>\string&nbsp;</td>} {</tr>}
%\Configure{putHBorder}{\HCode{\HBorder}}
\Configure{putHBorder}{\HCode{}}
\HAssign\Next:TableNo|=0 \global\let\TableNo=\Next:TableNo
% \Css{.hline hr, .cline hr{  height : 1px; margin:0px; }}
>>>






\<html latex array/tabular Config 3.2\><<<<
\def\ColMargins{}
\def\nosp:hbr#11#2//{\ifnum \tmp:cnt>1 \advance\tmp:cnt by -1
   \nosp:hbr#2//\else\nosp:gt#11#2//\fi}
\def\nosp:gt#11#21#3//{%
   \def\:temp{#1}\ifx \:temp\empty 1\else 0\fi
   \def\:temp{#2}\ifx \:temp\empty 1\else 0\fi}
% \Css{div.td00{ margin-left:0pt; margin-right:0pt; }}
% \Css{div.td01{ margin-left:0pt; margin-right:5pt; }}
% \Css{div.td10{ margin-left:5pt; margin-right:0pt; }}
% \Css{div.td11{ margin-left:5pt; margin-right:5pt; }}
>>>




\<html latex array/tabular Config 3.2\><<<<
\Configure{multicolumn}
   {\let\col:Css\empty
%    \Configure{VBorder} 
%       {\let\VBorder|=\empty}{\def\VBorder{border="1"}}{}{}
   }
   {}
   {\ifvmode\IgnorePar\fi     
    \HCode{<div class="multicolumn" }\HColAlign\HCode{>}%
   }
   {\ifvmode\IgnorePar\fi \EndP\HCode{</div>}}
>>>

The \verb'<TABLE>' is needed as a grouping mechanism for \verb'<CENTER>'.


%%%%%%%%%%%%%%%%%%
\section{Ams Math}
%%%%%%%%%%%%%%%%%%


\<configure html32-math amsmath\><<<
\Configure{overset} {\Picture+{ \a:@Picture{}}} {\EndPicture}
\Configure{underset} {\Picture+{ \a:@Picture{}}} {\EndPicture}
>>>


\<configure html32-math amsmath\><<<
\Configure{xrightarrow} {\Picture+{ \a:@Picture{}}} {\EndPicture}
\Configure{xleftarrow} {\Picture+{ \a:@Picture{}}} {\EndPicture}
\Configure{genfrac}
  {\Picture+{}\bgroup} {}{}{}{}{\egroup\EndPicture}
>>>


%%%%%%%%%%%%%%%%%%%%%
\section{Shared}
%%%%%%%%%%%%%%%%%%%





\<0,32,4 article,report,book\><<<
|<html latex tocs|>          |%should appear before TocAt|%
|<latex options 1, 2, 3|>
>>>










\<0,32,4 article,report,book\><<<
|<latex shared div config|>
|<shared sec div config|>
|<shared subsec div config|>
|<shared sub end div config|>
\ifx \part\:UnDef \else
   |<shared part div config|>
   |<latex shared likepart config|>
\fi
>>>




\section{aa}





\subsection{Sizes of Fonts}



pages should honor the base font sizes the
readers choose for their browsers. Hence, under this assumption, all
tex4ht should do is just assure appropriate relative dimensions for
fonts of other sizes. To meet this end, I modified latex.4ht to
automatically include

   \verb'{\Configure{Needs}{Font\string_Size: #1}\Needs{1...}}'

when options 11pt and 12pt are listed in \verb'\documentclass'.


\<32,4 article,report,book\><<<
|<base font size|>
>>>


\<ams art,proc,book\><<<
|<base font size|>
|<ams footnotes|>
>>>>



\<base font size\><<<
{\Configure{Needs}{Font\string_Size: #1}\ifcase  \@ptsize
   \or \Needs{11}\or \Needs{12}\else \fi}
>>>


\section{plain}







\<picmath plain,latex\><<<
\Configure{$$}{\:xhtml{\EndP}\PicDisplay}{\EndPicDisplay}      
   {\everymath{}\everydisplay{}}  
>>>



\<0,32,4 plain\><<<
\Configure{settabs}[1.5]{}{}{}{}{}
\Configure{line}{\HCode{<br\xml:empty>}}
>>>





























































\<math plain,fontmathNO\><<<
\Configure{big}{\HCode{<span class="big">}}{\HCode{</span>}}{}{}
\Configure{Big}{\HCode{<span class="bbig">}}{\HCode{</span>}}{}{}
\Configure{bigg}{\HCode{<span class="bigg">}}{\HCode{</span>}}{}{}
\Configure{Bigg}{\HCode{<span class="bbigg">}}{\HCode{</span>}}{}{}
>>>







\<under/over line css\><<<
\Configure{underline}
   {\HCode{<span class="underline">}} {\HCode{</span>}}
\Configure{overline}
   {\HCode{<span class="overline">}} {\HCode{</span>}}
\Css{.underline{ text-decoration:underline; }}
\Css{.overline{ text-decoration:overline; }}
>>>








The \verb'100%' is to allow centering of stuff on the page.
The table is a grouping mechanism to protect internal stuff
from external centering operations.




The \verb'special{t4ht=' is superior to \verb'\HCode{' because it can
prevent \verb'<p>'s from entering before \verb'<NOBR>'.




The hbox is for avoiding the start of a new paragraph, if in vmode.


\<configure html32 latex\><<<
\ifOption{charset=iso-8859-7}
   {|<T1 greek ldf iso-8859-7|>}
   {}
\:CheckOption{new-accents}     \if:Option
\else
   |<T1 old iso-8859-1 accents|>
\fi
\let\^^_|=\v
>>>



\<32,4 plain,latex\><<<
\:CheckOption{new-accents}     \if:Option \else 
  \Configure{accents}
    {\expandafter\ifx \csname #1-num\endcsname\relax
       \ht:special{t4ht@+\string&{35}x#2{59}}x%
     \else
       \ht:special{t4ht@+\string&\#x#2;}X%
     \fi
    }
    {\Picture+{ \a:@Picture{#1}}#2{#3}\EndPicture}
    \expandafter\let\csname ring-num\endcsname=\def
\fi
>>>


\<configure html32 plain\><<<
\:CheckOption{new-accents}     \if:Option \else 
   \Configure{accents}
      {\ht:special{t4ht@+\string&{35}x#2{59}}x}
      {\Picture+{ \:Picture:{#1}}#2{#3}\EndPicture}
\fi
>>>


\<32,4 plain,latexNO\><<<
\:CheckOption{new-accents}     \if:Option \else 
  \Configure{accents}
    {\ht:special{t4ht@+\string&#2#1;}#2}
    {\Picture+{ \a:@Picture{#1}}#2{#3}\EndPicture}
\fi
>>>

\<32,4 latexPRE-CSS\><<<
\:CheckOption{new-accents}     \if:Option \else 
   \Configure{textscaccent}
      {\Tg<small class="small-caps">}{\Tg</small>}
\fi
>>>








\section{amsart}







\section{amsproc}




\section{amsbook}








\section{babel.sty}






\<configure html32 babel\><<<
|<0,32,4 babel|>
>>>



\<0,32,4 babel.def\><<<
\Configure{quotedblbase}{\leavevmode\special{t4ht@+&{35}8222;}x}
\Configure{quotesinglbase}{\leavevmode\special{t4ht@+&{35}8218;}x}
>>>





\<configure html32 frenchb\><<<
\def\A:charset{charset=iso-8859-1}
\Configure{frenchb-thinspace}
   {\ht:special{t4ht@?unhskip}}
   {}
\Configure{frenchb-nbsp}
   {\ht:special{t4ht@?unhskip}}{}
\Configure{system-nbsp}
   {\ht:special{t4ht@[unhskip}}
   {\ht:special{t4ht@]unhskip}}
\Configure{frenchb-thinspace}
   {\ht:special{t4ht@[unhskip}}
   {\ht:special{t4ht@]unhskip}}
\Configure{@TITLE}{\Configure{frenchb-thinspace}{}{}}
>>>













































































































%%%%%%%%%%%%%%%%%%%%%%%%%
\section{Scientific Word}
%%%%%%%%%%%%%%%%%%%%%%%%%

\<configure html32 tcilatex\><<<
\Configure{GRAPHICSPS}
   {\Picture+[PICT]{}}  {\EndPicture}
\Configure{GRAPHICSHP}
   {\Picture+[PICT]{}}  {\EndPicture}
>>>

\<configure html32 seslideb\><<<
\ConfigureEnv{center}
  {\IgnorePar \par \EndP \HCode{<div class="center"\Hnewline>}}
  {\ifvmode\IgnorePar\fi\EndP\HCode{</div>}}
  {} {}
{\Configure{Needs}{Font\string_Size: #1}%
   \expandafter\Needs\expandafter{\csname normalsize@fsize\endcsname}}
>>>

\<configure html32 seslideb\><<<
\ConfigureToc{swSlide}
   {} {\relax  }  {}  { }
\:CheckOption{0}     
   \if:Option  \else    \:CheckOption{1} \fi
\if:Option 
   \:CheckOption{1} \if:Option
      \CutAt{swSlide} 
      \Configure{crosslinks}{}{}{}{}{}{}{}{}
   \fi
   |<seslideb toc|>
   |<seslideb page break|>
\else
   \Log:Note{pagination may be obtained
       through the option `0' or `1', at locations marked with
       \noexpand\csname PageBreak\string\endcsname}
   |<handle my missing sw fonts|>
\fi
>>>

\<handle my missing sw fonts\><<<
\Configure{swSlide}{}{}{\bgroup\bf}{\egroup}
>>>

\<seslideb toc\><<<
\def\:temp{%
   \ifOption{1}{\par\IgnorePar\EndP\HCode{<hr />}\par}{}
   \HAssign\TocN=1
   \def\swTitle{\gHAdvance\TocN by 1 \TocN}\:TableOfContents[swSlide]
   \let\swTitle=\relax 
   \ifOption{0}{\par\IgnorePar\EndP\HCode{<hr />}\par}{}
   \let\swSlide=\o:swSlide: \let\o:swSlide:=\empty \swSlide}
\HLet\swSlide=\:temp   
>>>


\<seslideb page break\><<<
\Configure{swSlide}
  {\gHAdvance\swSlideN by 1 \Link{}{s-\swSlideN}\EndLink }
  {\rightline{
       {\HAdvance\swSlideN by -1 \Link{s-\swSlideN}{}\swSlideN\EndLink}
       \Link[\jobname.html]{}{}\HCode{&lt;&gt;}\EndLink{}
       {\HAdvance\swSlideN by 1 
         \ifTag{)Qs-\swSlideN}{\Link{s-\swSlideN}{}\swSlideN\EndLink}{}}
     }
     \ifOption{0}{\par \IgnorePar\EndP\HCode{<hr />}\par}{}
   }
   {\IgnorePar\bgroup \bf }
   {\egroup \ShowPar\IgnoreIndent\par}
\HAssign\swSlideN=1
>>>




%%%%%%%%%%%%%%%%%
\section{Babel}
%%%%%%%%%%%%%%%

\<0,32,4 babel\><<<
|<0,32,4 babel.def|>
\ifx \@begindocumenthook\:UnDef\else
   \:CheckOption{new-accents}     \if:Option \else
      \def\:temp{russian}\ifx \languagename\:temp
         |<russian|>
      \fi
\fi\fi
>>>

We had also \verb'\append:def\@begindocumenthook{\HLet\"|=\ddot}' in
babel. It gets russian and brazil into infinite loop.  Why it was
inserted.


\<russian\><<<
\Configure{accent}\"\ddot{A{A}E{E}I{I}O{O}U{U}Y%
           {Y}a{a}e{e}i{i}\i{i}o{o}u{u}y{y}�{e}{}{34}}
   {\a:accents{uml}{#1}}    
   {\def\:temp{>}\def\:tempa{#2}\ifx \:temp\:tempa\HCode{�}%
    \else \def\:temp{<}\ifx \:temp\:tempa\HCode{�}%
    \else \b:accents{uml}{#1}{#2}\fi\fi}
>>> 






\section{babel.sty}

\<32,4 tex4ht\><<<
\NewConfigure{charset}[1]{\def\a:charset{#1}}
\def\:temp#1charset=#2,#3|<par del|>{%
   \if !#2!\else \Configure{charset}{#2}\fi}
\expandafter\:temp\Preamble charset=,|<par del|>
>>>





\<configure html32 CJK\><<<
\Configure{charset}{charset=big5}
>>>












\section{moreverb}




\<32,4 moreverb\><<<
\ConfigureEnv{verbatimtab}{\env:verb{verbatim}}{\endenv:verb}{}{}
\ConfigureEnv{verbatimtab*}{\env:verb{verbatim}}{\endenv:verb}{}{}
\ConfigureEnv{boxedverbatim}
   {\env:verb{boxedverbatim}}{\endenv:verb}{}{}
\ConfigureEnv{boxedverbatim*}
   {\env:verb{boxedverbatim}}{\endenv:verb}{}{}
>>>




\section{color}



\<configure html32 color\><<<
\Configure{HColor}{red}{\#FF0000}
\Configure{HColor}{rgb 1 0 0}{\#FF0000}
\Configure{HColor}{blue}{\#0000FF}
\Configure{HColor}{rgb 0 0 1}{\#0000FF}
\Configure{HColor}{green}{\#00FF00}
\Configure{HColor}{rgb 0 1 0}{\#00FF00}
\Configure{HColor}{white}{\#FFFFFF}
\Configure{HColor}{gray 1}{\#FFFFFF}
\Configure{HColor}{rgb 1 1 1}{\#FFFFFF}
\Configure{HColor}{black}{\#000000}
\Configure{HColor}{gray 0}{\#000000}
\Configure{HColor}{rgb 0 0 0}{\#000000}
\Configure{HColor}{cyan}{\#00FFFF}
\Configure{HColor}{cmyk 1 0 0 0}{\#00FFFF}
\Configure{HColor}{magenta}{\#FF00FF}
\Configure{HColor}{cmyk 0 1 0 0}{\#FF00FF}
\Configure{HColor}{yellow}{\#FFFF00}
\Configure{HColor}{cmyk 0 0 1 0}{\#FFFF00}
\Configure{color}{\:gobble} 
>>>




\<configure html32 color\><<<
\Configure{HColor:gray}{%
   \int:of\:tempa{255}{#1}%
   \Configure{HColor}{}{rgb(\:tempa,\:tempa,\:tempa)}%
}
\Configure{HColor:rgb}{%
   \int:of\:tempa{255}{#1}%
   \int:of\:tempb{255}{#2}%
   \int:of\:tempc{255}{#3}%
   \Configure{HColor}{}{rgb(\:tempa,\:tempb,\:tempc)}%
}
\Configure{HColor:cmyk}{%
   \int:of\:Cyan{255}{#1}%
   \int:of\:Magenta{255}{#2}%
   \int:of\:Yellow{255}{#3}%
   \int:of\:Black{255}{#4}%
   \:cmyk\:tempa\:Cyan
   \:cmyk\:tempb\:Magenta
   \:cmyk\:tempc\:Yellow
   \Configure{HColor}{}{rgb(\:tempa,\:tempb,\:tempc)}%
}
|<HColor util|>
>>>


\<HColor util\><<<
\def\int:of#1#2#3{%
   \tmp:dim=#3pt \tmp:dim=#2\tmp:dim
   \edef\:temp{\tmp:cnt\the\tmp:dim//}%
   \def#1##1//{}\afterassignment#1\:temp
   \edef#1{\the\tmp:cnt}%
}
>>>


\<HColor util\><<<
\def\:cmyk#1#2{%
   \tmp:cnt=255 \advance\tmp:cnt by -\:Black
   \multiply\tmp:cnt by#2 \advance\tmp:cnt by \:Black 
   \advance\tmp:cnt by -255 \tmp:cnt=-\tmp:cnt
   \ifnum \tmp:cnt<0 \tmp:cnt=0 \fi
   \edef#1{\the\tmp:cnt}%
}
>>>






\<plain tex classes\><<<
\Configure{MathClass}{1}{}{}{}{
   \mathchar"1360
   \mathchar"1357
   \mathchar"1356
   \mathchar"1355
   \mathchar"1354
   \mathchar"1353
   \mathchar"1352
   \mathchar"1351
   \mathchar"1350
   \mathchar"134E
   \mathchar"134C
   \mathchar"134A
   \mathchar"1348
   \mathchar"1346
   \mathchar"1273
}
>>>


\subsection{2: Binary Operations}

\begin{verbatim}
\mathchardef\triangleleft="212F
\mathchardef\triangleright="212E
\mathchardef\bigtriangleup="2234
\mathchardef\bigtriangledown="2235
\mathchardef\wedge="225E \let\land=\wedge
\mathchardef\vee="225F \let\lor=\vee
\mathchardef\cap="225C
\mathchardef\cup="225B
\mathchardef\ddagger="227A
\mathchardef\dagger="2279
\mathchardef\sqcap="2275
\mathchardef\sqcup="2274
\mathchardef\uplus="225D
\mathchardef\amalg="2271
\mathchardef\diamond="2205
\mathchardef\bullet="220F
\mathchardef\wr="226F
\mathchardef\div="2204
\mathchardef\odot="220C
\mathchardef\oslash="220B
\mathchardef\otimes="220A
\mathchardef\ominus="2209
\mathchardef\oplus="2208
\mathchardef\mp="2207
\mathchardef\pm="2206
\mathchardef\circ="220E
\mathchardef\bigcirc="220D
\mathchardef\setminus="226E % for set difference A\setminus B
\mathchardef\cdot="2201
\mathchardef\ast="2203
\mathchardef\times="2202
\mathchardef\star="213F
\mathcode`\*="2203 % \ast
\mathcode`\+="202B
\mathcode`\-="2200
\end{verbatim}

\<plain tex classes\><<<
\Configure{MathClass}{2}{}{}{}{
*-+/
\mathchar"212F
\mathchar"212E
\mathchar"2234
\mathchar"2235
\mathchar"225E 
\mathchar"225F 
\mathchar"225C
\mathchar"225B
\mathchar"227A
\mathchar"2279
\mathchar"2275
\mathchar"2274
\mathchar"225D
\mathchar"2271
\mathchar"2205
\mathchar"220F
\mathchar"226F
\mathchar"2204
\mathchar"220C
\mathchar"220B
\mathchar"220A
\mathchar"2209
\mathchar"2208
\mathchar"2207
\mathchar"2206
\mathchar"220E
\mathchar"220D
\mathchar"226E 
\mathchar"2201
\mathchar"2203
\mathchar"2202
\mathchar"213F
}
>>>


\subsection{3: Relational Operations}





The catcode is needed because 303A is \verb':'.

\<plain tex classes\><<<
\Configure{MathClass}{3}{}{}{}{
   \mathchar"3128
   \mathchar"3129
   \mathchar"312A
   \mathchar"312B
   \mathchar"315E
   \mathchar"315F
   \mathchar"3210
   \mathchar"3211
   \mathchar"3212
   \mathchar"3213
   \mathchar"3214
   \mathchar"3215
   \mathchar"3216
   \mathchar"3217
   \mathchar"3218
   \mathchar"3219
   \mathchar"321A
   \mathchar"321B
   \mathchar"321C
   \mathchar"321D
   \mathchar"321E
   \mathchar"321F
   \mathchar"3220
   \mathchar"3221
   \mathchar"3224
   \mathchar"3227
   \mathchar"3232
   \mathchar"3233
   \mathchar"3236
   \mathchar"3237
   \mathchar"323F
   :=><
   \mathchar"322F
   \mathchar"3276
   \mathchar"3277
   \mathchar"326B
   \mathchar"326A
   \mathchar"3261
   \mathchar"3260
   \mathchar"3225
   \mathchar"3226
   \mathchar"322D
   \mathchar"322E
   \mathchar"322C
   \mathchar"3228
   \mathchar"3229
}
>>>







\begin{verbatim}
\mathcode`\>="313E
\mathcode`\<="313C
\mathcode`\=="303D
\mathcode`\:="303A
\mathchardef\leq="3214 \let\le=\leq
\mathchardef\geq="3215 \let\ge=\geq
\mathchardef\succ="321F
\mathchardef\prec="321E
\mathchardef\approx="3219
\mathchardef\succeq="3217
\mathchardef\preceq="3216
\mathchardef\supset="321B
\mathchardef\set="321A
\mathchardef\supseteq="3213
\mathchardef\seteq="3212
\mathchardef\in="3232
\mathchardef\ni="3233 \let\owns=\ni
\mathchardef\gg="321D
\mathchardef\ll="321C
\mathchardef\not="3236
\mathchardef\leftrightarrow="3224
\mathchardef\leftarrow="3220 \let\gets=\leftarrow
\mathchardef\rightarrow="3221 \let\to=\rightarrow
\mathchardef\mapstochar="3237 \def\mapsto{\mapstochar\rightarrow}
\mathchardef\sim="3218
\mathchardef\simeq="3227
\mathchardef\perp="323F
\mathchardef\equiv="3211
\mathchardef\asymp="3210
\mathchardef\smile="315E
\mathchardef\frown="315F
\mathchardef\leftharpoonup="3128
\mathchardef\leftharpoondown="3129
\mathchardef\rightharpoonup="312A
\mathchardef\rightharpoondown="312B
\mathchardef\propto="322F
\mathchardef\sqsubseteq="3276
\mathchardef\sqsupseteq="3277
\mathchardef\parallel="326B
\mathchardef\mid="326A
\mathchardef\dashv="3261
\mathchardef\vdash="3260
\mathchardef\nearrow="3225
\mathchardef\searrow="3226
\mathchardef\nwarrow="322D
\mathchardef\swarrow="322E
\mathchardef\Leftrightarrow="322C
\mathchardef\Leftarrow="3228
\mathchardef\Rightarrow="3229
\end{verbatim}


\subsection{4/5: Delimiters}

\begin{verbatim}
\mathcode`\(="4028
\mathcode`\)="5029
\mathcode`\[="405B
\mathcode`\]="505D
\mathcode`\{="4266
\mathcode`\}="5267
\delcode`\(="028300
\delcode`\)="029301
\delcode`\[="05B302
\delcode`\]="05D303
\def\lmoustache{\delimiter"437A340 } % top from (, bottom from )
\def\rmoustache{\delimiter"537B341 } % top from ), bottom from (
\def\lgroup{\delimiter"462833A } % extensible ( with sharper tips
\def\rgroup{\delimiter"562933B } % extensible ) with sharper tips
\def\backslash{\delimiter"26E30F } % for double coset G\backslash H
\def\rangle{\delimiter"526930B }
\def\langle{\delimiter"426830A }
\def\rbrace{\delimiter"5267309 } \let\}=\rbrace
\def\lbrace{\delimiter"4266308 } \let\{=\lbrace
\def\rceil{\delimiter"5265307 }
\def\lceil{\delimiter"4264306 }
\def\rfloor{\delimiter"5263305 }
\def\lfloor{\delimiter"4262304 }
\def\arrowvert{\delimiter"26A33C } % arrow without arrowheads
\def\Arrowvert{\delimiter"26B33D } % double arrow without arrowheads
\def\bracevert{\delimiter"77C33E } % the vertical bar that extends braces
\def\Vert{\delimiter"26B30D } \let\|=\Vert         How should these be treated?
\def\vert{\delimiter"26A30C }                       "   "       "    "   "
\def\uparrow{\delimiter"3222378 }
\def\downarrow{\delimiter"3223379 }
\def\updownarrow{\delimiter"326C33F }
\def\Uparrow{\delimiter"322A37E }
\def\Downarrow{\delimiter"322B37F }
\def\Updownarrow{\delimiter"326D377 }
\end{verbatim}

The comamnds \verb'\Configure{MathClass}{4}...'
and \verb'\Configure{MathClass}{5}...'
are for unmatched delimiters, and the comamnd
\verb'\Configure{MathDelimiters}{(}{)}' is for matched ones.




\<plain tex classes\><<<
\Configure{MathClass}{4}{}{}{}{}
\Configure{MathDelimiters}{(}{)}
\Configure{MathDelimiters}{[}{]}
\Configure{MathDelimiters}{\mathchar"4262}{\mathchar"5263}
\Configure{MathDelimiters}{\mathchar"4264}{\mathchar"5265}
\Configure{MathDelimiters}{\mathchar"4266}{\mathchar"5267}
\Configure{MathDelimiters}{\mathchar"4268}{\mathchar"5269}
\Configure{MathDelimiters}{\mathchar"4300}{\mathchar"5301}
\Configure{MathDelimiters}{\mathchar"4302}{\mathchar"5303}
\Configure{MathDelimiters}{\mathchar"4304}{\mathchar"5305}
\Configure{MathDelimiters}{\mathchar"4306}{\mathchar"5307}
\Configure{MathDelimiters}{\mathchar"4308}{\mathchar"5309} 
\Configure{MathDelimiters}{\mathchar"430A}{\mathchar"530B}
>>>



\begin{verbatim}
\mathcode`\?="503F
\end{verbatim}






\subsection{6: Punctuation Marks}

\begin{verbatim}
\mathchardef\ldotp="613A % ldot as a punctuation mark
\mathchardef\cdotp="6201 % cdot as a punctuation mark
\mathchardef\colon="603A % colon as a punctuation mark
\mathcode`\;="603B
\mathcode`\,="613B
\end{verbatim}







\<plain tex classes\><<<
\Configure{MathClass}{6}{}{}{}{
\mathchar"613A 
\mathchar"6201 
\mathchar"603A 
?; ,
}
>>>




\subsection{Questions}

\begin{verbatim}


\delcode`\<="26830A
\delcode`\>="26930B
\delcode`\|="26A30C
\delcode`\\="26E30F

% N.B. { and } should NOT get delcodes; otherwise parameter grouping fails!

\def\mathhexbox#1#2#3{\leavevmode
  \hbox{$\m@th \mathchar"#1#2#3$}}
\def\dag{\mathhexbox279}
\def\ddag{\mathhexbox27A}
\def\S{\mathhexbox278}
\def\P{\mathhexbox27B}
\end{verbatim}






\subsection{Type 4: Math Open}


latex.ltx, fontmath.ltx, plain.tex







\section{fleqn.sty}





\section{amsppt.sty}


\<32 amsppt, 32,4 vanilla\><<<
\Configure{title}
   {\IgnorePar\EndP\HCode{<div class="title">}\begingroup\bf}
   {\endgroup\IgnorePar\HCode{</div>}}
\Css{div.title {margin-top: 0.5em;}}
\Configure{author}
   {\IgnorePar\EndP\HCode{<br\xml:empty><center>}\IgnorePar\par}
   {\IgnorePar\EndP\HCode{</center>}}

>>>

\<32 amsppt\><<<
\Configure{affil}{\IgnorePar\HCode{<br\xml:empty><center>}\IgnorePar}
                  {\IgnorePar\HCode{</center>}}
\Configure{abstract} {\HCode{<br\xml:empty><center>}} {\HCode{</center>}}
   {\IgnorePar\HCode{<table cellpadding="15"><tr><td>}\IgnorePar\par}
   {\IgnorePar\HCode{</td></tr></table>}\IgnorePar\par}
\Configure{date}{\IgnorePar\HCode{<center>}\IgnorePar}
   {\IgnorePar\HCode{</center>}}
|<32 amsppt, 32,4 vanilla|>
>>>


\<32,4 amsppt\><<<
\Configure{specialhead}{}{}
   {\IgnorePar\EndP\HCode{<h1 class="amsspecialheadHead">}}
   {\HCode{</h1>}\IgnoreIndent}
\ConfigureToc{specialhead}
   {\HCode{<center>}\ignorespaces}{ }
   {}{\HCode{</center>}}
\Configure{head}{}{}
   {\IgnorePar\EndP\HCode{<h2 class="amsheadHead">}}
   {\HCode{</h2>}\IgnoreIndent}
\ConfigureToc{head}
   {\ignorespaces}{ }{}{\HCode{<br\xml:empty>}}
\Configure{subhead}{}{}
   {\IgnorePar\EndP\HCode{<h3 class="amssubheadHead">}}
   {\HCode{</h3>}\IgnoreIndent}
\ConfigureToc{subhead} 
   {\:nbsp\:nbsp\:nbsp\:nbsp\ignorespaces} { }
   {} {\HCode{<br\xml:empty>}}
\Configure{subsubhead}{}{}
   {\IgnorePar\EndP\HCode{<h4 class="amssubsubheadHead">}}
   {\HCode{</h4>}\IgnoreIndent}
\ConfigureToc{subsubhead}
  {\:nbsp\:nbsp\:nbsp\:nbsp\:nbsp%
       \:nbsp\:nbsp\:nbsp\ignorespaces} { } {} {\HCode{<br\xml:empty>}}
\Configure{block}
   {\IgnorePar\EndP
       \HCode{<table cellpadding="15"><tr><td>}\IgnorePar\par}
   {\IgnorePar\HCode{</td></tr></table>}\IgnorePar\par}
\Configure{caption}
   {\:xhtml{\IgnorePar\EndP}\HCode{<center>}}{}{\HCode{</center>}}
\Configure{roster}
    {\IgnorePar\EndP\HCode{<table>}\let\end:item|=\empty}
    {\IgnorePar\end:item\HCode{</table>}\IgnorePar\par} 
    {\IgnorePar\end:item \HCode{<tr valign="top"><td>}
                        \def\end:item{\EndP\HCode{</td></tr>}}}
    {\:nbsp\EndP\HCode{</td><td>}\ShowPar}
    {\IgnorePar\EndP\HCode{<table>}\let\end:iitem|=\empty}
    {\IgnorePar\end:iitem\HCode{</table>}\IgnorePar\par}
    {\IgnorePar\end:iitem\HCode{<tr valign="top"><td>}}
    {\HCode{</td><td>}
      \def\end:iitem{\EndP\HCode{</td></tr>}}}
|<bib in amsppt.sty|>
>>>


\<bib in amsppt.sty\><<<
\Configure{vol}{\HCode{<strong>}}{\HCode{</strong>}}
\Configure{book}{\HCode{<em>}}{\HCode{</em>}}
\Configure{paper}{\HCode{<em>}}{\HCode{</em>}}
\Configure{Refs}{\IgnorePar\EndP\HCode{<table class="Refs">}}
                {\HCode{</table>}}
\Configure{ref}{\HCode{<tr valign="top"><td align="right">}}
               {\EndP\HCode{</td></tr>}}{}
\Configure{keyformat}{}{\EndP\HCode{</td><td>}}
>>>



\section{amsmath.sty}





\<configure html32-math amsmath\><<<
\Configure{equation}
  {\Configure{gather}
     {\HCode{<\tbl:XV{equation}><tr><td><center>}\IgnorePar}
     {\end:TTT\IgnorePar\par}
     {}{}
     {\ifnum\HCol=2 \IgnorePar\HCode{</center></td><td width="5\%">}\fi}
     {}
  }{}{}
>>>





\<configure html32-math amsmath\><<<
\Configure{equation}
  {\NoHtmlEnv \Configure{gather}
     {\HCode{<\tbl:XV{equation}><tr><td><center>}\IgnorePar}
     {\end:TTT\IgnorePar\par}
     {\Configure{$}{\PicMath}{\EndPicMath}{}}
     {}
     {\ifnum\HCol=2 \IgnorePar\HCode{</center></td><td width="5\%">}\fi}
     {}
  }{}{}
>>>


Equations in amsmath.sty are defined in term of gather, and 
gather is a one parameter macro.  Unlike laktex where the body is
read within the environment, in gather it is read at the \verb'\begin{equation}' point under the conditions that exist there.


\<32,4 pic amsmath\><<<
\Configure{substack}{\Picture+{}}{\EndPicture}
>>>




\section{amstex.sty (amstex1)}




\section{amstex.tex}


\<amstex.tex m:env\><<<
|<amsmath / amstex1 m:env|>
>>>


\<NO\><<<
\def\m:env#1{\:xhtml{\IgnorePar\EndP}\HCode{<center><table class="#1"
   border="0" cellpadding="0" cellspacing="15"><tr><td>}}
\def\endm:env{\HCode
  {</td></tr></table></center>}\IgnorePar}
>>>

\<amsmath / amstex1 m:env\><<<
\def\m:env#1{\relax\ifmmode\else\par\fi\:xhtml{\IgnorePar\EndP}%
  \HCode{<center class="#1"><table class="#1"\Hnewline
   border="0" cellpadding="0" cellspacing="15"><tr><td>}}
\def\endm:env{\:xhtml{\IgnorePar\EndP}%
   \HCode{</td></tr></table></center>}\IgnorePar
   \ifmmode\else\par\fi}
>>>













% \Configure{topaligned}{\m:env{topaligned}}{\endm:env}
% \Configure{botaligned}{\m:env{botaligned}}{\endm:env}







\<32,4 picmath amstex.tex\><<<
\Configure{frac}{\Picture+{}}{\EndPicture}
\Configure{dfrac}{\Picture+{}}{\EndPicture}
\Configure{tfrac}{\Picture+{}}{\EndPicture}
\Configure{binom}{\Picture+{}}{\EndPicture}
\Configure{dbinom}{\Picture+{}}{\EndPicture}
\Configure{tbinom}{\Picture+{}}{\EndPicture}
\Configure{boxed}{\Picture+{}}{\EndPicture}
>>>






\section{vanilla}






\<32,4 vanilla\><<<
\Configure{matrix}
   {\EndP\HCode{<center><table\Hnewline
        border="0" cellpadding="0" cellspacing="15" class="matrix">}}
   {\HCode{</table></center>}\IgnorePar}
   {\HCode{<tr\Hnewline valign="top">}}{\HCode{</tr>}}
   {\HCode{<td>}}   {\HCode{</td>}}
\Configure{align}
   {\EndP\HCode{<center><table\Hnewline
        border="0" cellpadding="0" cellspacing="15" class="align">}}
   {\HCode{</table></center>}\IgnorePar}
   {\HCode{<tr\Hnewline valign="top">}}{\HCode{</tr>}}
   {\HCode{<td>}}   {\HCode{</td>}}
>>>


\section{slidesec}




\section{ltugboat}




%%%%%%%%%%%%%%%%%%%
\section{tex4ht}
%%%%%%%%%%%%%%%%%%%









\<0,32,4 tex4ht\><<<
\Configure{HVerbatim+}{\z@}{\:nbsp}
\:CheckOption{jpg} \if:Option
   \Configure{Picture}{.jpg}  
\else
   \Log:Note{for jpg bitmaps of pictures, use the `jpg'
       command line option. |<ch bitmaps|>}
\fi
\:CheckOption{gif} \if:Option 
   \Configure{Picture}{.gif}  
\else
   \Log:Note{for gif bitmaps of pictures, use the `gif'
       command line option. |<ch bitmaps|>}
\fi
>>>



\<ch bitmaps\><<<
(Character bitmaps are controled only by `g' records of tex4ht.env and `-g'
switches of tex4ht.c)
>>>




\<32,4 th4\><<<
\:CheckOption{javascript}
   \if:Option  \else\:CheckOption{th4}\fi
\if:Option 
   \Configure{JavaScript}
      {\HCode{<script type="text/JavaScript" ><!--\Hnewline}}
      {\HCode{//-->\Hnewline </script>}}
\fi
>>>







\verb'\Hnewline' is needed at end of file to avoid loosing the
last line under some applications.


















\<configure html32 latex\><<<
\Configure{htf}{4}{+}{<small\Hnewline
   class="}{}{}{}{}{small-caps">}{</small>}
\Configure{htf}{6}{+}{<u\Hnewline
   class="}{}{}{}{}{underline">}{</u>}
\Configure{htf}{8}{+}
   {<sup class="htf"><strong>}{}{}{}{}{}{</strong></sup>}
\Configure{htf}{10}{+}
   {<tt>}{}{}{}{}{}{</tt>}
\Configure{htf}{14}{+}{<i>}{}{}{}{}{}{</i>}
>>>

\verb'\Configure{htf}{0}{+}{<!--span  class="}{\%s}{-\%s}{--\%d}{}{"-->}{<!--/span-->}' caused netscape to
loose spaces between comments.
\verb'\Configure{htf}{0}{+}{<!--span\Hnewline class="}{\%s}{-\%s}{ - -\%d}{}{"-->}{<!--/span-->}' caused netscape
to misbehave on \verb'<pre>'




























\section{th4}


\<0 th4\><<<
\Configure{Chapter}
   {}{}    {Chapter  \theChapterCounter} {}
\Configure{Appendix}
   {}{}    {Appendix  \theChapterCounter} {}
\Configure{LikeChapter}
   {}{}    {} {}
>>>

\<32,4 th4\><<<
\Configure{Columns}
  {\IgnorePar\EndP
      \HCode{<table \Hnewline cellspacing="15"><tr valign="top">}}
  {\HCode{</tr></table>}}
  {\HCode{<td>}\ColMag{1.03}}
  {\IgnorePar\EndP\HCode{</td>}}
>>>



\<32,4 th4\><<<
\:CheckOption{index}\if:Option 
   \Configure{index}
     {\bgroup
         \Configure{Columns}
           {\IgnorePar\EndP
               \HCode{<table \Hnewline class="index" width="100\%"
                           cellspacing="15"><tr valign="top">}}
           {\HCode{</tr></table>}}
           {\HCode{<td>}\ColMag{1.1}}
           {\IgnorePar\EndP\HCode{</td>}}
        \Columns{2}\IndexFonts} 
     {\EndColumns \egroup}
     {\bgroup\IgnorePar\EndP\expandafter\ifx \csname prev:A\endcsname\relax
            \else \hfil\break \Tg<br\xml:empty>\par\IgnorePar \fi \IndexSec}
     {\egroup~~~~}
     {\bgroup\hfil\break\Tg<br\xml:empty>~~~}{\egroup~~~~}
     {~}{}
   \def\Idx:ch{0}
   \def\IndexSec#1{%
      \tmp:cnt=`#1\relax
      \ifnum \tmp:cnt>`Z\advance\tmp:cnt by -32 \fi
      \ifnum \tmp:cnt<`A\else \ifnum \tmp:cnt>`Z \else
          \ifnum \Idx:ch<\tmp:cnt
          \bgroup
            \Configure{centerline}
               {\HCode{<div\Hnewline class="IndexSec">}}{\HCode{</div>}}
             \leftline{\bf \char\tmp:cnt }%
             \global\let\prev:A|=\:UnDef
             \xdef\Idx:ch{\the\tmp:cnt}%
          \egroup
          \fi
      \fi \fi #1%
   }
   \Css{.IndexSec {margin-top:1em; margin-bottom:0.5em;}}
\fi
>>>

\<32,4 th4\><<<
\Configure{Chapter}
   {}{}   
   {\IgnorePar\EndP\HCode{<h2 class="ChapterHead">}%
      Chapter  \theChapterCounter \HCode{<br\xml:empty>}}
   {\HCode{</h2>}\IgnoreIndent\IgnorePar}
\Configure{Appendix}
   {}{}   
   {\IgnorePar\EndP\HCode{<h2 class="AppendixHead">}%
         Appendix  \theChapterCounter \HCode{<br\xml:empty>}}
   {\HCode{</h2>}\IgnoreIndent\IgnorePar}
\Configure{LikeChapter}
   {}{}   
   {\IgnorePar\EndP\HCode{<h2 class="LikeChapterHead">}\noindent
     \bgroup \def\uppercase##1{##1}}
   {\egroup \HCode{</h2>}\IgnoreIndent\IgnorePar }
>>>






\<32,4 th4\><<<
\Configure{UList}
  {\IgnorePar\EndP\def\:tempB{disc}%
   \ifx\:tempA\:tempB \else \def\:tempB{square}\fi
   \ifx\:tempA\:tempB \else \def\:tempB{circle}\fi
   \hbox{\IgnorePar\EndP\HCode{<ul
   \ifx\:tempA\:tempB type="\:tempA" \fi \:UL:>}}}
  {\everypar{}\EndP\HCode{</li></ul>}}
  {\ifnum \ListCounter>1  \EndP\HCode{</li>}\fi \hfil\break \HCode{<li>}}
\Configure{OList}
  {\IgnorePar\EndP\hbox{\HCode{<ol 
   \ifx \:temp\empty \else  type="\:temp" \fi 
   \:OL:>}}}
  {\everypar{}\EndP\HCode{</li></ol>}}
  {\ifnum \ListCounter>1  \EndP\HCode{</li>}\fi \hfil\break \HCode{<li>}}
>>>




\<32,4 th4\><<< 
\Configure{HTable}
  {\everypar{}\EndP\HCode{<table \Hnewline\TABLE:\:HTable:>}%
       \def\BR{\HCode{<br\xml:empty>}}}
  {\HCode{</table>}}
  {\HCode{<tr \:TR>}}{\HCode{</tr>}}
  {\everypar{}\HCode{<\TD:typ\TD:more\Hnewline>}}
  {\everypar{}\HCode{</\TD:typ>}}
\let\:HTable:|=\empty
>>>




\<32,4 th4\><<< 
\Configure{Item}{}{\par}%
\Configure{DList}
   {\IgnorePar\EndP\HCode{<dl \:DL:>}}
   {\everypar{}\EndP\HCode{\End:dd</dl>}}
   {\IgnorePar\EndP\ifnum \ListCounter>1  \HCode{</dd>}\fi
    \HCode{<dt>}}
   {\HCode{</dt><dd>}\ShowPar \def\End:dd{</dd>}\hfil\break}
\Configure{buttonList}{}{}
  {}{.\ #1 }{\ListCounter}
>>>


\<32,4 th4\><<< 
\Configure{Part}{}{}{%
  \html:rightskip
  \bgroup
     \html:rightskip  \everypar{} 
     \IgnorePar\EndP\HCode{<h1 class="PartHead">}}
  {\HCode{</h1>}\IgnoreIndent \egroup   \IgnoreIndent}
\Configure{LikeSection}{}{}
  {\IgnorePar  \EndP\HCode{<h3 class="LikeSectionHead">}}
  {\HCode{</h3>}\IgnoreIndent
    |<addr for Tag and Ref of Sec|>%
    \par \IgnoreIndent
  }
\ConfigureMark{Section}{\theSection}
\Configure{Section}
  {}{}
  {\IgnorePar\EndP\HCode{<h3 class="SectionHead">}%
     \gHAdvance\SectionCounter |by 1
     \TitleMark\space
  }{\HCode{</h3>}\IgnoreIndent
     |<addr for Tag and Ref of Sec|>%
     \par \IgnoreIndent
  }
\Configure{SubSection}
  {}{}
  {\par \IgnorePar\EndP\HCode{<h3 class="SubSectionHead">}}
  {\HCode{</h3>}\IgnoreIndent \ShowPar}
>>>

\<32,4 th4\><<< 
\ConfigureToc{Chapter}
  {\HCode{<span class="ChapterToc">}}
  {~}
  {}
  {\HCode{</span><br\xml:empty>}}
\ConfigureToc{Section}
  {\HCode{<span class="SectionToc">}~~~}
  {~}
  {}
  {\HCode{</span><br\xml:empty>}}
\ConfigureToc{LikeSection}
  {}
  {\HCode{<span class="LikeSectionToc">}~~~}
  {}
  {\HCode{</span><br\xml:empty>}}
\ConfigureToc{SubSection}
  {}
  {\HCode{<span class="SubSectionToc">}~~~~~~}
  {}
  {\HCode{</span><br\xml:empty>}}
>>>





\section{seminar}






\section{slides}








\section{amsthm.sty}





\section{colortbl.sty}





\section{epsfig}



\<0,32,4 epsfig\><<<
\Configure{epsfig} {\Picture+[epsfig]{}}{\EndPicture}
>>>

\section{psfig}



\<0,32,4 psfig\><<<
\Configure{psfig} {\Picture+[psfig]{}}{\EndPicture}
>>>

\section{graphics}



\<0,32,4 graphics\><<<
\Configure{graphics}{\Picture+[PIC]{}}{\EndPicture}
\Configure{graphics*}
   {gif}
   {\Picture[pict]{\csname Gin@base\endcsname.gif}}
\Configure{graphics*}
   {png}
   {\Picture[pict]{\csname Gin@base\endcsname.png}}
\Configure{graphics*}
   {jpeg}
   {\Picture[pict]{\csname Gin@base\endcsname.jpeg}}
\Configure{graphics*}
   {jpg}
   {\Picture[pict]{\csname Gin@base\endcsname.jpg}}
>>>





\section{foils}



\<32,4 foils\><<<
\Configure{foilheads} {}{} 
   {\IgnorePar\EndP\HCode{<h1 class="foilheadsHead">}}
   {\HCode{</h1>}\IgnorePar}
\ConfigureEnv{Theorem}{\par\leavevmode}{\ShowPar}{}{}
\ConfigureEnv{Lemma}{\par\leavevmode}{\ShowPar}{}{}
\ConfigureEnv{Corollary}{\par\leavevmode}{\ShowPar}{}{}
\ConfigureEnv{Corollary*}{\par\leavevmode}{\ShowPar}{}{}
\ConfigureEnv{Proposition}{\par\leavevmode}{\ShowPar}{}{}
\ConfigureEnv{Definition}{\par\leavevmode}{\ShowPar}{}{}
\ConfigureEnv{Proof}{\par\leavevmode}{\ShowPar}{}{}
\ConfigureEnv{thebibliography}{\par\leavevmode}{\ShowPar}{}{}
\:CheckOption{1}  \if:Option 
    \CutAt{foilheads}
    \ConfigureToc{foilheads}{}{ *\ }{}{}
    \Configure{tableofcontents*}{foilheads}
\fi
>>>

\section{index}



\<configure html32 index\><<<
\Configure{NoSection}
  {\let\sv:index|=\p@index \let\sv:label|=\label \let\sv:ref|=\ref
   \let\sv:newline|=\newline \def\newline{ }%
   \def\p@index[##1]{\@gobble}\let\label|=\@gobble  \let\ref|=\@gobble
  }
  {\let\p@index|=\sv:index \let\label|=\sv:label \let\ref|=\sv:ref
   \let\newline|=\sv:newline
  }
>>>



\section{}




\section{ntheorem}



\<32,4 ntheorem\><<<
\ConfigureEnv{Anmerkung}
   {\HCode{<div class="Anmerkung">}} {\HCode{</div>}} {}{}
\ConfigureEnv{Beispiel}
   {\HCode{<div class="Beispiel">}} {\HCode{</div>}} {}{}
\ConfigureEnv{Bemerkung}
   {\HCode{<div class="Bemerkung">}} {\HCode{</div>}} {}{}
\ConfigureEnv{Beweis}
   {\HCode{<div class="Beweis">}} {\HCode{</div>}} {}{}
\ConfigureEnv{Corollary}
   {\HCode{<div class="Corollary">}} {\HCode{</div>}} {}{}
\ConfigureEnv{Definition}
   {\HCode{<div class="Definition">}} {\HCode{</div>}} {}{}
\ConfigureEnv{Example}
   {\HCode{<div class="Example">}} {\HCode{</div>}} {}{}
\ConfigureEnv{Korollar}
   {\HCode{<div class="Korollar">}} {\HCode{</div>}} {}{}
\ConfigureEnv{Lemma}
   {\HCode{<div class="Lemma">}} {\HCode{</div>}} {}{}
\ConfigureEnv{Proof}
   {\HCode{<div class="Proof">}} {\HCode{</div>}} {}{}
\ConfigureEnv{Proposition}
   {\HCode{<div class="Proposition">}} {\HCode{</div>}} {}{}
\ConfigureEnv{Remark}
   {\HCode{<div class="Remark">}} {\HCode{</div>}} {}{}
\ConfigureEnv{Satz}
   {\HCode{<div class="Satz">}} {\HCode{</div>}} {}{}
\ConfigureEnv{Theorem}
   {\HCode{<div class="Theorem">}} {\HCode{</div>}} {}{}
\ConfigureEnv{anmerkung}
   {\HCode{<div class="anmerkung">}} {\HCode{</div>}} {}{}
\ConfigureEnv{beispiel}
   {\HCode{<div class="beispiel">}} {\HCode{</div>}} {}{}
\ConfigureEnv{bemerkung}
   {\HCode{<div class="bemerkung">}} {\HCode{</div>}} {}{}
\ConfigureEnv{beweis}
   {\HCode{<div class="beweis">}} {\HCode{</div>}} {}{}
\ConfigureEnv{corollary}
   {\HCode{<div class="corollary">}} {\HCode{</div>}} {}{}
\ConfigureEnv{definition}
   {\HCode{<div class="definition">}} {\HCode{</div>}} {}{}
\ConfigureEnv{example}
   {\HCode{<div class="example">}} {\HCode{</div>}} {}{}
\ConfigureEnv{korollar}
   {\HCode{<div class="korollar">}} {\HCode{</div>}} {}{}
\ConfigureEnv{lemma}
   {\HCode{<div class="lemma">}} {\HCode{</div>}} {}{}
\ConfigureEnv{proof}
   {\HCode{<div class="proof">}} {\HCode{</div>}} {}{}
\ConfigureEnv{proposition}
   {\HCode{<div class="proposition">}} {\HCode{</div>}} {}{}
\ConfigureEnv{remark}
   {\HCode{<div class="remark">}} {\HCode{</div>}} {}{}
\ConfigureEnv{satz}
   {\HCode{<div class="satz">}} {\HCode{</div>}} {}{}
\ConfigureEnv{theorem}
   {\HCode{<div class="theorem">}} {\HCode{</div>}} {}{}
>>>





\section{hyperref}

\<configure html32 hyperref\><<<
\Configure{Form}
   {\IgnorePar\EndP\leavevmode \Tg<form \Hnewline \Attributes>}
   {\IgnorePar\EndP\Tg</form>}
>>>

\<configure html32 hyperref\><<<
|<hyperref shared|>
|<hyperref TextField|>
|<hyperref multiline|>
|<hyperref password|>
|<hyperref radio|>
|<hyperref on...|>
\NewConfigure{::action}{1}
\Configure{::action}
   {\edef\Attributes{\Attributes\space action="\AttributeVal"}}
\NewConfigure{::method}{1}
\Configure{::method}
   {\edef\Attributes{\Attributes\space method="\AttributeVal"}}
\NewConfigure{PushButton::}{1}
\Configure{PushButton::}
   {\leavevmode\Tg<input type="button" \Attributes\space/>}
\NewConfigure{Reset::}{1}
\Configure{Reset::}
  {\leavevmode\Tg<input type="reset" \Attributes\space/>}
\NewConfigure{Submit::}{1}
\Configure{Submit::}
  {\leavevmode\Tg<input type="submit" \Attributes\space/>}
\NewConfigure{CheckBox::}{2}
\Configure{CheckBox::}
  {\leavevmode\Tg<input type="checkbox" \Attributes\space/>}{}
\NewConfigure{CheckBox::checked}{2}
\Configure{CheckBox::checked}
  {\leavevmode\Tg<input type="checkbox" checked="checked"
      \Attributes\space/>}{}
\HAssign\form:id=0
>>>


\<hyperref shared\><<<
\NewConfigure{::value}{1}
\Configure{::value}
   {\edef\Attributes{\Attributes\space value="\AttributeVal"}}
\NewConfigure{::name}{1}
\Configure{::name}
   {\edef\Attributes{\Attributes\space name="\AttributeVal"}}
\NewConfigure{::default}{1}
\Configure{::default}
   {\let\::default=\AttributeVal}
\def\get:int#1.#2//{\tmp:cnt=#1 }
>>>

\<\><<<
\NewConfigure{::borderwidth}{1}
\Configure{::borderwidth}
   {\Css{div\#form-\form:id {border-width: \AttributeVal;
                              border-style:solid;}}}
\NewConfigure{::bordercolor}{1}
\Configure{::bordercolor}
   {\expandafter\get:colors\AttributeVal//%
      \Css{div\#form-\form:id {border-color:\AttributeVal}}}
\def\get:colors#1 #2 #3//%
   \get:color{#2}\edef\AttributeVal{\AttributeVal,
        \the\tmp:cnt\%}%
   \get:color{#3}\edef\AttributeVal{rgb(\AttributeVal,
        \the\tmp:cnt\%)}%
}
\def\get:color#1{%
   \tmp:dim=#1pt \multiply\tmp:dim by 100
   \expandafter\get:int\the\tmp:dim//}
>>>









\<hyperref on...\><<<
\def\:tempc#1{%
  \NewConfigure{::#1}{1}%
  \Configure{::#1}%
    {\edef\Attributes{\Attributes\space #1="\AttributeVal"}}}
\:tempc{onclick}
\:tempc{onblur}
\:tempc{onchange}
\:tempc{onclick}
\:tempc{ondblclick}
\:tempc{onfocus}
\:tempc{onkeydown}
\:tempc{onkeypress}
\:tempc{onkeyup}
\:tempc{onmousedown}
\:tempc{onmousemove}
\:tempc{onmouseout}
\:tempc{onmouseover}
\:tempc{onmouseup}
\:tempc{onselect}
>>>



\<hyperref TextField\><<<
\NewConfigure{TextField::}{2}
\Configure{TextField::}{}{ \Tg<input type="text" \Attributes />}
\NewConfigure{TextField::width}{1}
\Configure{TextField::width}
   {\tmp:dim=\AttributeVal   \divide\tmp:dim by 6
    \expandafter\get:int\the\tmp:dim//%
    \edef\Attributes{\Attributes\space size="\the\tmp:cnt"}}
\NewConfigure{TextField::default}{1}
\Configure{TextField::default}
   {\edef\Attributes{\Attributes\space value="\AttributeVal"}}
>>>


\<hyperref multiline\><<<
\NewConfigure{TextField::multiline}{2}
\Configure{TextField::multiline}
  {}
  { \Tg<textarea
        \Attributes>\expandafter\set:ln\multiline:value,|<par del|>%
  \global\let\multiline:value=\empty \Tg</textarea>}

\let\multiline:value=\empty
\def\set:ln#1,#2|<par del|>{#1%
  \def\:temp{#2}\ifx \:temp\empty \else
     \hfil\break  \def\:temp{\set:ln#2|<par del|>}%
  \fi \:temp}

\NewConfigure{multiline::value}{1}
\Configure{multiline::value}
  {\let\multiline:value=\AttributeVal}

\NewConfigure{multiline::width}{1}
\Configure{multiline::width}
   {\tmp:dim=\AttributeVal   \divide\tmp:dim by 6
    \expandafter\get:int\the\tmp:dim//%
    \edef\Attributes{\Attributes\space cols="\the\tmp:cnt"}}
>>>


\<hyperref password\><<<
\NewConfigure{TextField::password}{2}
\Configure{TextField::password}
  {}{\Tg<input type="password" \Attributes />}
>>>

\<hyperref shared\><<<
\def\Default:Checked#1{%
   \let\:temp=\relax
   \let\:tempa=\relax
   \edef\:temp{\def\:temp####1#1#1####2//{\def\:temp{####2}}%
                   \:temp \AttributeVal #1=#1#1//%
       \def\:tempa####1=####2//{\def\noexpand\AttributeVal{####1}}%
           \:tempa\AttributeVal=//}%
   \:temp}
>>>

\<hyperref radio\><<<
\NewConfigure{ChoiceMenu::radio}{5}
\Configure{ChoiceMenu::radio}
   {\IgnorePar\EndP\leavevmode
      \Tg<div id="form-\form:id">\gHAdvance\form:id by 1 }
   { }{\IgnorePar\EndP\Tg</div>}
   {\Default:Checked\radio::default
    \Tg<input\Hnewline type="radio" 
            \ifx \:temp\empty\else checked="checked" \fi
            \Attributes\space />}
   {}
\NewConfigure{radio::default}{1}
\Configure{radio::default}
   {\let\radio::default=\AttributeVal}
>>>


\<hyperref radio\><<<
\NewConfigure{ChoiceMenu::combo}{5}
\Configure{ChoiceMenu::combo}
   {}
   {~\Tg<select\Hnewline \Attributes \Hnewline size="1">}
   {\Tg</select>}
   {\Tg<option \ifx\::default\AttributeVal selected="selected"\fi
        \Hnewline>} 
   {\Tg</option>}
\NewConfigure{combo::default}{1}
\Configure{combo::default}
   {\let\radio::default=\AttributeVal}
>>>





\<hyperref radio\><<<
\NewConfigure{ChoiceMenu::popdown}{5}
\Configure{ChoiceMenu::popdown}
   {}{\HCode{\Hnewline <select \Attributes \Hnewline size="1">}}
   {\Tg</select>}
   {\Tg<option \ifx\::default\AttributeVal selected="selected"\fi
        \Hnewline>} {\Tg</option>}
>>>

The \verb'size="1"' makes the select a popout memnu


\<hyperref radio\><<<
\NewConfigure{ChoiceMenu::}{5}
\Configure{ChoiceMenu::}
   {}{\HCode{\Hnewline <select\Hnewline \Attributes>}}{\Tg</select>}
   {\Tg<option \ifx\::default\AttributeVal selected="selected"\fi
        \Hnewline>}  {\Tg</option>}
\NewConfigure{::menulength}{1}
\Configure{::menulength}
   {\edef\Attributes{\Attributes\space size="\AttributeVal"}}
>>>



\section{web}



\section{exerquiz}




\<32,4 exerquiz\><<<
\Configure{Form}{}{}
\Configure{@HEAD}{\input exerqz.4ht }
\Css{.onClick {color:green;}}
\Configure{TextField::}{}{%
   \IgnorePar \EndP
   \HCode{<form action="." name="form\quiz@total"><input
       type="text"\Hnewline  \Attributes /></form>}%
}
\Configure{javascript}{JavaScript:}
>>>

\<32,4 exerquiz\><<<
\ifx \eq@sqrtmsg\:UnDef
   \def\eq@sqrtmsg{"Right!"}
\fi
\ifx \eq@sqwgmsg\:UnDef
   \def\eq@sqwgmsg{"Wrong!"}
\fi
\Configure{shortquiz}
   {(\alph{quizno})}
   {alert(\eq@sqrtmsg,3);}
   {alert(\eq@sqwgmsg,3);}
\Configure{quiz}
   {(\alph{quizno})}
   {qthis=this;
     ProcessQuestion (\ANS,"\alph{quizno}",\thequestionno,
      1,"\eq@bqlabel",\Quiz:N)}
   {InitializeQuiz("\quiz@total",
      \ifeq@nocorrections0\else1\fi,\Quiz:N,\LikeRef{ans-\Quiz:N},
      "(",")")}
   {QuizEnd("\:bqlabel",\thequestionno,"\quiz@total",\Quiz:N)}
   {Corrections("\eq@RC","\eq@AC",\Quiz:N)}
\Configure{quiz*}
   {[]}
   {qthis=this;
     ProcessQuestion (\ANS,"[]",\thequestionno,
      0,"\eq@bqlabel",\Quiz:N)}
   {InitializeQuiz("\quiz@total",
      \ifeq@nocorrections0\else1\fi,\Quiz:N,
      \LikeRef{ans-\Quiz:N},"","")}
>>>

\<-NOPE\><<<
\Configure{quiz*}
   {qthis=this;
    ProcessQuestion(\ANS,"\alph{quizno}",\thequestionno,
      0,"\eq@bqlabel",\Quiz:N)}
>>>
 

\<32,4 exerquiz\><<<
\ConfigureEnv{shortquiz}
   {\IgnorePar\EndP\leavevmode} {} {}{}
\Configure{ReturnTo}{\begin{flushright}}{\end{flushright}}
>>>



\<configure html32 exerquiz\><<<
\ConfigureList{questions}%
   {\HCode{<ol type="1"\Hnewline>}}
   {\HCode{</ol>}\ShowPar}
   {\DeleteMark}
   {\HCode{<li>}\AnchorLabel}
>>>







\<exerqz\><<<
% exerqz.4ht (|version), generated from |jobname.tex
% Copyright (C) 2009-2010 TeX Users Group
% Copyright (C) |CopyYear.1999. Eitan M. Gurari & Donald P. Story
|<exerqz's vars|>
|<predefined exerquiz javascript|>
>>>



\<predefined exerquiz javascript\><<<
\JavaScript-$
var QuizInitialized;
var EndQuizPushed;
var CurrentQuizNo;
var Responses;
var ResponsesAddr;
var Cor;
var CorAddr;
var qthis;
var prev_notify;
function InitializeQuiz(qtfield,mark,quizN,ansN,lbrc,rbrc) {
  Score=0;
  QuizInitialized=1;
  CurrentQuizNo=quizN;
  eval( 'document.form'+qtfield+'.'+qtfield+'.value="$eqScore";' )  
  RightWrong=new Array();
  |<hide sol|>
  Responses=new Array();
  ResponsesAddr=new Array();
  |<hide cor|>  
  Cor=new Array();
  CorAddr=new Array();
  EndQuizPushed=0;
  for(var i=1; i<=ansN; i++){
     RightWrong[i]=0;
  }
}
\EndJavaScript
>>>



\<show sol\><<<
if( ResponsesAddr[probno] != null ){
  if (notify == 0 ) {
    ResponsesAddr[probno].value=Responses[probno];
  } else {
    ResponsesAddr[probno].value="("+Responses[probno]+")";
} }
qthis.value = "#";  ResponsesAddr[probno]=qthis;
>>>

\<hide sol\><<<
for(var i in Responses){
  if (prev_notify == 0 ) {
    ResponsesAddr[i].value=Responses[i];
  } else {
    ResponsesAddr[i].value="("+Responses[i]+")"; 
} }
>>>

% if( Responses != null ){


\<record cor\><<<
var k=Cor.length;
eval('Cor[k]=thisform'+quizN+'.ans'+quizN+'x'+i+'.value');
eval('CorAddr[k]=thisform'+quizN+'.ans'+quizN+'x'+i);
>>>


\<hide cor\><<<
for(var i in Cor){
  CorAddr[i].value=Cor[i];
} 
>>>

% if( Cor != null ){



\<predefined exerquiz javascript\><<<
\JavaScript
function href(addr) { top.location.href=addr; }
\EndJavaScript
>>>

% function href(addr) { window.navigate(addr); }




\<predefined exerquiz javascript\><<<
\JavaScript
function Corrections(lbl1,lbl2,quizN) {
  if ( (EndQuizPushed == 1) && ( CurrentQuizNo == quizN ) ){
    for(var i in RightWrong){
      if( (RightWrong[i]==0) ){
        |<record cor|>
        eval('thisform'+quizN+'.ans'+quizN+'x'+i+'.value= "*"');
  } }
} } 
\EndJavaScript
>>>





\<32,4 exerquiz\><<<
\immediate\write16{%
***********************************************************\Hnewline
The `\eq@CA' button fails under Netscape, \Hnewline
due to a code of the following form.\Hnewline
\Hnewline
<html><head><title>?</title>\Hnewline
<script\space type="text/JavaScript"\space ><!--\space \Hnewline
\space \space \space function\space f()\space {\Hnewline
\space \space \space \space \space formxx.inputxx.value="BBB";\Hnewline
\space \space \space}\Hnewline
//-->\space \Hnewline
</script>\space \space </head>\space <body>\Hnewline
\Hnewline
<form\space id="formxx">\space \space \Hnewline
\space \space \space <input\space type="text"\space \space name="foo"\space id="inputxx"\space \space value="AAA"\space />\Hnewline
\space \space \space <input\space value="\eq@CA"\space type="button"\space \space \space onClick="f()"\space />\Hnewline
\space \space \space <input\space value="CLEAR"\space type="reset"\space \space \space onClick="clear()"\space />\Hnewline
</form>\Hnewline
    \Hnewline
</body> \Hnewline
</html> \Hnewline
\Hnewline
If you know how to fix the function f() above for Netscape,\Hnewline
without changing the id attribute names, please consider\Hnewline
emailing the fix to tex4ht@tug.org. Thanks\Hnewline
***********************************************************}
>>>


\<predefined exerquiz javascript\><<<
\JavaScript
function LinkTo(addr) {
}
\EndJavaScript
>>>

\<exerqz's vars\><<<
\def\eqXInitQuizMsg{\hbox{%
   \let\noexpand|=\string
   \csname eq@InitQuizMsg\endcsname}}
\expandafter\ifx \csname eq@InitQuizMsg\endcsname\relax
    \expandafter\def\csname eq@InitQuizMsg\endcsname{
        "You must initialize the Quiz! Click on "+bqlabel}
\fi
\def\eqXQuizTotalMsg{\hbox{%
   \let\noexpand|=\string
   \def\thequestionno{"+thequestionno+"}%
   \csname eq@QuizTotalMsg\endcsname}}
\expandafter\ifx \csname eq@QuizTotalMsg\endcsname\relax
    \expandafter\def\csname eq@QuizTotalMsg\endcsname{"Score: '
         +Score +' out of '+thequestionno+'"}
\fi
\def\eqXMadeChoice{\hbox{%
   \let\noexpand|=\string
   \csname eq@MadeChoice\endcsname}}
\expandafter\ifx \csname eq@MadeChoice\endcsname\relax
    \expandafter\def\csname eq@MadeChoice\endcsname{
            "You have already made a choice. Your choice was ("
            +Responses[probno]+")."
            +" Do you want to change it?"}
\fi
\expandafter\ifx \csname eqScore\endcsname\relax
    \def\eqScore{Score:}
\fi
>>>

The \verb'Wollen Sie dies \noexpand\344ndern?' is a problem because it
takes the \verb'\344' into \verb'44'. A \verb'\string' will properly 
produce \verb'\344'; hence, the above dirty trick.

\begin{verbatim}
Doesn't IE escape in the way that Acrobat JavaScript does?

Another possibility is to use String.fromCharCode()

Convert Octal \344 to decimal 228, then use
String.fromCharCode(228)



   
          How to deliver browser specific content using JavaScript
                                      

<SCRIPT LANGUAGE="JavaScript">
<!--
  if( -1 != navigator.userAgent.
      indexOf ("AOL") )
  {
    // load America Online version
    location.href="aol.htm";
  }
  else
  if( -1 != navigator.userAgent.
      indexOf ("MSIE") )
  {
    // load Microsoft Internet
    // Explorer version
    location.href="msie.htm";
  }
  else
  if( -1 != navigator.userAgent.
      indexOf ("Mozilla") )
  {
    // load Netscape version
    location.href="netscape.htm";
  }
  else
  {
    // load other version
    location.href="other.htm";
  }
-->
</SCRIPT>

\end{verbatim}



\<predefined exerquiz javascript\><<<
\JavaScript-$
function QuizEnd(bqlabel,thequestionno,quiztotal,quizN){
  if ((QuizInitialized !=1) |||| ( CurrentQuizNo!= quizN )){   
     alert($eqXInitQuizMsg,3);
  } else {
     eval( 'document.form'+quiztotal+'.'+quiztotal+
       '.value=$eqXQuizTotalMsg');
         QuizInitialized=-1;
         EndQuizPushed=1; 
} }
\EndJavaScript

\JavaScript-$
function  ProcessQuestion
  (key,letterresp,probno,notify,bqlabel,quizN) {
    if ((QuizInitialized !=1) |||| ( CurrentQuizNo!= quizN )){   
      alert($eqXInitQuizMsg,3);
    } else {
      |<function ProcUserResp(key,letterresp,probno,notify)|> 
      prev_notify = notify;
}   }
\EndJavaScript
>>>





\<function ProcUserResp(key,letterresp,probno,notify)\><<<
if (Responses[probno] == null) {
   if (key==1) {
      Score++;
      RightWrong[probno]=1;
   }
   else
      RightWrong[probno]=0;
   |<show sol|>
   Responses[probno]=letterresp;
}
else {
   if (notify==0)
      User=true;
   else
      User=confirm($eqXMadeChoice);
   if (User) {
      if (RightWrong[probno]==1) {
          if (key==0) {
            Score -= 1;
            RightWrong[probno]=0;
            |<show sol|>
            Responses[probno]=letterresp;
         }
      }
      else {
          if (key==1) {
            Score++;
            RightWrong[probno]=1;
            |<show sol|>
            Responses[probno]=letterresp;
         }
         else {
            RightWrong[probno]=0;
            |<show sol|>
            Responses[probno]=letterresp;
         }
      }
   }
}
>>>









%%%%%%%%%%%%%%%%%%%%%%%%%%%
%%%%%%%%%%%%%%%%%%%%%% to be placed %%%%%%%%%%%%%%%%%%%%%%%%%
\subsection{to be placed}




\<0 th4,latex\><<<
\Configure{()}{$}{$}
\Configure{[]}{\:xhtml{\IgnorePar\EndP}$$}{$$}
>>>
















































\section{tex4ht}



\<title for hypertext page\><<<
\Configure{TITLE+}{\HCode{\jobname.\:html}}
>>>

\<0,32,4 latex\><<<
\ifTag{TITLE+}{\Configure{TITLE+}{\LikeRef{TITLE+}}}{}
>>>





















\<0,32,4 tex4ht\><<<
|<0,32,4 preambles|>
\ifx \a:FontCss:\:UnDef
   \Configure{FontCss}{Font\string_Css##1}
                   {Font\string_Css\string_Plus\space##1}
\fi
\expandafter\ifx \csname aa:Css\endcsname\relax
   \Configure{Css}{Css: ##1}
\fi
\:CheckOption{edit} \if:Option 
   \Configure{edit}{\HCode{<strong>&lt;}}{\HCode{&gt;</strong>}}
                {<strong>&lt;}{&gt;</strong>}
\fi
\:CheckOption{hooks++} \if:Option
\else \:CheckOption{hooks+}  \if:Option
\else \:CheckOption{hooks}  \if:Option
\fi\fi\fi
\if:Option
   \Configure{hooks}
      {\HCode{<strong class="hooks">&lt;}}{\HCode{&gt;</strong>}}{}{}  
\fi


\Configure{ExitHPage}{exit}{exit }{}
\Configure{TocLink}{\Link{#2}{#3}#4\EndLink}
\Configure{MiniHalign}{\hlg:a}{\hlg:b}\hlg:c\hlg:d{\hlg:e}\hlg:f
\:CheckOption{no-halign} \if:Option \else
  \Configure{noalign-}{}{}
\fi
\Configure{PictureAlt*+}
   {\let\sv:HtmlPar|=\HtmlPar   \let\HtmlPar|=\empty
     |<postscript for /Picture|>%
     |<tex halign and cr/crcr|>%
     \NoFonts\csname PauseMathClass\endcsname \SUBOff \SUPOff
     \let\HCode|=\:gobble     |%\offinterlineskip|%
     \let\EndPicture|=\empty}
   {\let\HCode|=\:HCode
     \let\EndPicture|=\:UnDef \let\HtmlPar|=\sv:HtmlPar \SUBOn \SUPOn
     \csname EndPauseMathClass\endcsname \EndNoFonts
     |<tex4ht halign and cr/crcr|>%
     |<delay postscript|>}
>>>

\<postscript for /Picture\><<<
\def\PsCode##1{{\ht:special{\PsCodeSpecial##1}}}%
>>>

\<tex halign and cr/crcr\><<<
\iffalse{\fi   
\let\sv:halign|=\halign
\let\sv:cr|=\cr
\let\sv:crcr|=\crcr
\iffalse}\fi 
\RecallTeXcr \let\halign |=\TeXhalign
>>>


\<tex4ht halign and cr/crcr\><<<
\iffalse{\fi   
\let\halign|=\sv:halign
\let\cr|=\sv:cr
\let\crcr|=\sv:crcr
\iffalse}\fi 
>>>


\<delay postscript\><<<
\let\PsCode|=\relax
>>>


\<32,4 tex4ht\><<<
  \Configure{writetoc}{}
>>>


\<0,32,4 tex4ht\><<<
\Configure{CutAtTITLE+}{}
\Configure{HPageTITLE+}{}
\Configure{AtBeginDocument} 
  {\edef\recallcatcodes{%
      \catcode`\noexpand\_|=\the\catcode`\_
      \catcode`\noexpand\^|=\the\catcode`\^ }%
   \catcode`\_=8\catcode`\^=7}
  {\recallcatcodes}
>>>



\<0,32,4 tex4ht\><<<
\Configure{crosslinks}{[}{]
   }{next}{prev}{prev-|<tail|>}{front}{tail}{up}
\:CheckOption{next}     \if:Option   
   \Configure{next+}{\ShowPar\par\noindent [}{]}
\fi
\Configure{TocAt*}{}{}
\Configure{TocAt}{}{}

\Configure{halignTB}{\HCode{<table }}{\HCode{>}}
\def\t:HA{\HCode{</table>}}
\def\R:HA{\HCode{<tr \Hnewline valign="baseline">}}
\def\r:HA{\HCode{</tr>}}
\def\D:HA{\HCode{<td \ifnum \HMultispan>1 colspan="\HMultispan" \fi}%
   \halignTD \HCode{\Hnewline>}}
\def\d:HA{\HCode{</td>}}
\Configure{HVerbatim+}{\z@}{\:nbsp}
\Configure{CssFile}{\jobname.css}
  {/* \aa:CssFile\space from \jobname.tex (TeX4ht) */}
\Configure{Picture+}{}{}
\Configure{Picture*}{}{}
\Configure{Needs}{l. 
   \the\inputlineno\space--- needs --- #1 ---}
\Configure{Needs-}{l.
   \the\inputlineno\space--- needs --- #1 ---}
|<yes css|>
>>>
\<0,32,4 tex4ht\><<<
\Configure{moveright}{\leavevmode\endgraf }
\Configure{HChar}{x}
>>>




\<yes css\><<<
   \def\:SPAN#1#2{\HCode{<span class="#1">}#2\HCode{</span>}} 
   \def\SPAN:#1{\HCode{<span class="#1">}}
   \def\EndSPAN:{\HCode{</span>}} 
   \def\DIV:#1{\HCode{<div class="#1">}}
   \def\EndDIV:{\HCode{</div>}} 
>>>

\<no css\><<<
   \def\:SPAN#1#2{#2}
   \let\SPAN:|=\:gobble  \let\EndSPAN:|=\empty
   \let\DIV:|=\:gobble  \let\EndDIV:|=\empty
>>>





\section{Interpretation for the Entries}




Use \verb'\ ', and not \verb'~', in style files, because some
users redefine the latter macro.

\<html latex tocs\><<<
\def\tocpart#1#2#3{\par\toc:SPAN{partToc}{#2}\par}%
\def\toclikepart#1#2#3{\par\toc:SPAN{likepartToc}{#2}\par}%
\expandafter\ifx \csname @chapter\endcsname\relax 
   \def\tocsection#1#2#3{\par
       \toc:SPAN{sectionToc}{\def\:temp{#1}\ifx \:temp\empty\else
            #1 \fi #2}\par}
   \def\toclikesection#1#2#3{\par\toc:SPAN{likesectionToc}{#2}\par}%
   \def\tocsubsection#1#2#3{\par\ \toc:num{subsection}{#1}{#2}\par}
   \def\toclikesubsection#1#2#3{\par\ \toc:SPAN{likesubsectionToc}{#2}\par}
   \def\tocsubsubsection#1#2#3{\par
      \ \ \toc:num{subsubsection}{#1}{#2}\par}
   \def\toclikesubsubsection#1#2#3{\par
      \ \ \toc:SPAN{likesubsubsectionToc}{#2}\par}
   \def\tocparagraph#1#2#3{\par\ \ \toc:num{paragraph}{#1}{#2}\par}
   \def\toclikeparagraph#1#2#3{\par\ \ \toc:SPAN{likeparagraphToc}{#2}\par}
   \def\tocsubparagraph#1#2#3{\par
      \ \ \ \ \toc:num{subparagraph}{#1}{#2}\par}
   \def\toclikesubparagraph#1#2#3{\par
      \ \ \ \ \toc:SPAN{likesubparagraphToc}{#2}\par}
\else
   \def\tocchapter#1#2#3{\par\toc:SPAN{chapterToc}{#1 #2}\par}
   \def\toclikechapter#1#2#3{\par\toc:SPAN{likechapterToc}{#2}\par}%
   \def\tocappendix#1#2#3{\par\toc:SPAN{appendixToc}{#1 #2}\par}
   \def\tocsection#1#2#3{\par\ \toc:num{section}{#1}{#2}\par}
   \def\toclikesection#1#2#3{\par\ \toc:SPAN{likesectionToc}{#2}\par}
   \def\tocsubsection#1#2#3{\par\ \ \toc:num{subsection}{#1}{#2}\par}
   \def\toclikesubsection#1#2#3{\par
      \ \ \toc:SPAN{likesubsectionToc}{#2}\par}
   \def\tocsubsubsection#1#2#3{\par
      \ \ \ \toc:num{subsubsection}{#1}{#2}\par}
   \def\toclikesubsubsection#1#2#3{\par
      \ \ \ \toc:SPAN{likesubsubsectionToc}{#2}\par}
   \def\tocparagraph#1#2#3{\par\ \ \ \toc:num{paragraph}{#1}{#2}\par}
   \def\toclikeparagraph#1#2#3{\par
      \ \ \ \toc:SPAN{likeparagraphToc}{#2}\par}
   \def\tocsubparagraph#1#2#3{\par
      \ \ \ \ \ \toc:num{subparagraph}{#1}{#2}\par}
   \def\toclikesubparagraph#1#2#3{\par
      \ \ \ \ \ \toc:SPAN{likesubparagraphToc}{#2}\par}
\fi
\def\toc:num#1#2#3{\def\:temp{#1#2}\toc:SPAN{#1Toc}{\ifx \:temp\empty \else 
   #2 \fi #3}}
\def\toc:SPAN#1#2{\HCode{<span class="#1">}#2\HCode{</span>}} 
>>>








\section{amstex}


\<configure picmath0 amstex\><<<
\:CheckOption{no-matrix} \if:Option \else
\:CheckOption{pic-matrix}   \if:Option
      |<pic amstex.tex matrix 0.0|> 
\else
      |<tabular amstex.tex matrix 0.0|> 
\fi\fi
\:CheckOption{no-align} \if:Option \else
\:CheckOption{pic-align}   \if:Option
      |<pic amstex.tex align 0.0|> 
\else
      |<tabular amstex.tex align 0.0|> 
\fi\fi
>>>



















%%%%%%%%%%%%%%%%%%%%%%%%%%%%%%%%%%%%%%%%%%%%%%%%%%%%%%%%%%%%%%%
\chapter{HTML3.2}
%%%%%%%%%%%%%%%%%%%%%%%%%%%%%%%%%%%%%%%%%%%%%%%%%%%%%%%%%%%%%%%


\<html32\><<<
% html32.4ht (|version), generated from |jobname.tex
% Copyright (C) 2009-2010 TeX Users Group
% Copyright (C) |CopyYear.1997. Eitan M. Gurari
|<TeX4ht copywrite|>
>>>



\section{article, report, book}

\<config book-report-article 3.2\><<<
|<32,4 article,report,book|>
|<0,32,4 article,report,book|>
>>>



\<32 latex\><<<
\ConfigureList{description}%
   {\EndP\HCode{<dl class="description">}%
      |<save end:itm|>\global\let\end:itm=\empty}
   {|<recall end:itm|>\EndP\HCode{</dd></dl>}\ShowPar}
   {\end:itm \global\def\end:itm{\EndP\Tg</dd>}\HCode{<dt
        class="description">}\bgroup \bf}
   {\egroup\EndP\HCode{</dt><dd\Hnewline class="description">}}
>>>


\<32,4 article,report,book\><<<
|<quotations 32,4|>
\Configure{listof}{}{}{}{\HCode{<br\xml:empty>}}{}{}
>>>

\<quotations 32,4\><<<
\ConfigureEnv{quotation}{}{}{\start:env{quotation}}{\end:env}
>>>

\<config book-report-article 3.2\><<<
\ConfigureList{thebibliography}%
   {\HCode{<dl class="thebibliography">}}
   {\HCode{</dl>}\ShowPar}
   {\HCode{<dt  class="thebibliography">}\bgroup \bf}
   {\egroup\HCode{<dd\Hnewline class="thebibliography">}}
\:CheckOption{/bib} \if:Option \else
   \ConfigureList{thebibliography}%
    {\HCode{<table border="0" cellpadding="0"
                                cellspacing="0"><tr><td>}}%
    {\HCode{</td></tr></table>}}%
    {\HCode{</td></tr><tr valign="top"><td>}}%
    {\AnchorLabel \:nbsp\HCode{</td><td>}}
\fi
>>>

\<config book-report-article 3.2\><<<
\ConfigureEnv{quote}{}{}{\start:env{quote}}{\end:env}
>>>


\<config book-report-article 3.2\><<<
\Configure{theindex}
   {\ifvmode \IgnorePar\fi \EndP
    \HCode{<div class="theindex">}\let\end:theidx|=\empty}
   {\end:theidx\HCode{</div>}}
   {} {\hfil\break\HCode{<br\xml:empty>}}
   {\ \ \ \ } {\hfil\break\HCode{<br\xml:empty>}}
   {\ \ \ \ \ \ \ \ } {\hfil\break\HCode{<br\xml:empty>}}
   {\hbox{\end:theidx\HCode{<p class="theindex">}}%
    \def\end:theidx{\HCode{</p>}}}
>>>


\<32,4 report,book\><<<
\Configure{@begin}{theindex}{\ifx \indexname\empty \else
        \chapter*{\indexname}\fi}
>>>

\<configure html32 article\><<<
\Configure{@begin}{theindex}{\ifx \indexname\empty \else
        \section*{\indexname}\fi}
>>>




\<latex config div\><<<
\ConfigureMark{section}
   {\ifnum \c:secnumdepth>\c@secnumdepth  \expandafter\:gobble
    \else \thesection\fi}
\Configure{section}{}{}
   {\ifvmode\IgnorePar\fi \EndP \HCode{<h3 class="sectionHead">}%
    \TitleMark\space}
   {\HCode{</h3>}\IgnoreIndent \par}
\Configure{sectionTITLE+}{\ifnum \c:secnumdepth>\c@secnumdepth
      \ifnum \c@secnumdepth > 0
      \thesection\space
    \fi\fi #1}
>>>


\<latex config like div 3.2\><<<
\Configure{likesection}{}{}
   {\ifvmode\IgnorePar\fi \EndP \HCode{<h3 class="likesectionHead">}}
   {\HCode{</h3>}\IgnoreIndent \ShowPar \par}
\Configure{likesubsection}{}{}
   {\bgroup \IgnorePar\HCode{<h4 class="likesubsectionHead">}}
   {\HCode{</h4>}\IgnoreIndent\egroup}
\Configure{likesubsubsection}{}{}
   {\bgroup\IgnorePar\HCode{<h5 class="likesubsubsectionHead">}}
   {\HCode{</h5>}\IgnoreIndent\egroup}
>>>


\<latex config div\><<<
\ConfigureMark{subsection}
   {\ifnum \c:secnumdepth>\c@secnumdepth  \expandafter\:gobble
    \else \thesubsection \fi}
\Configure{subsection}{}{}
   {\bgroup \IgnorePar\HCode{<h4 class="subsectionHead">}%
    \TitleMark\space}
   {\HCode{</h4>}\IgnoreIndent\egroup}
\Configure{subsectionTITLE+}{\ifnum \c:secnumdepth>\c@secnumdepth
      \ifnum \c@secnumdepth > 0
      \thesubsection\space
    \fi\fi #1}
>>>


\<latex config div\><<<
\ConfigureMark{subsubsection}
   {\ifnum \c:secnumdepth>\c@secnumdepth  \expandafter\:gobble
    \else \thesubsubsection \fi}
\Configure{subsubsection}{}{}
   {\bgroup \IgnorePar\HCode{<h5 class="subsubsectionHead">}%
    \TitleMark\space}
   {\HCode{</h5>}\IgnoreIndent\egroup}
\Configure{subsubsectionTITLE+}{\ifnum \c:secnumdepth>\c@secnumdepth
      \ifnum \c@secnumdepth > 0
      \thesubsubsection\space
    \fi\fi #1}
>>>



\<latex config div\><<<
\Configure{paragraph}{}{}
  {\ShowPar\IgnoreIndent\HCode{<strong class="paragraphHead">}}
  {\HCode{</strong>}\IgnorePar}
\Configure{subparagraph}{}{}
  {\ShowPar\IgnoreIndent\HCode{<strong class="subparagraphHead">}}
  {\HCode{</strong>}\IgnorePar}
>>>






\<config book-report-article 3.2\><<<
|<config sections 3.2|>
>>>



\section{article}

\<configure html32 article\><<<
|<config book-report-article 3.2|> 
|<config report / article 3.2|>
>>>




\section{report}

\<configure html32 report\><<<
|<config book-report-article 3.2|> 
     |<config report / article 3.2|>
|<32,4 report,book|>
|<32,4 report|>
>>>




\<config report / article 3.2\><<<
\ConfigureEnv{abstract}{\HCode{<\tbl:XV{abstract}><tr><td\Hnewline
   >}}{\end:env}{}{}
>>>


\section{book}

\<configure html32 book\><<<
|<config book-report-article 3.2|> 
|<32,4 report,book|>
|<32,4 book|>
>>>



\section{amsart}



\<ams art,proc\><<<
\Configure{endsection}
     {part}
\Configure{endsubsection}
   {section,part}
\Configure{endsubsubsection}
   {subsection,section,part}
\Configure{endparagraph}
   {subsubsection,subsection,section,part}
>>>


\<configure html32 amsart\><<<
|<config sections 3.2|>
|<book-report-article caption 3.2|>
|<latex report,... config 3.2|>
|<latex shared div config|>
|<ams art,proc,book|>
|<latex config div|>
|<ams art,proc|>
>>>


\<ams art,proc,book\><<<
\Configure{maketitle}
   {\ifvmode \IgnorePar\fi \EndP |<title for TITLE|>%
    \HCode{<div  class="maketitle">}%
    \ConfigureEnv{center}
       {\ifvmode \IgnorePar\fi} {\ifvmode \IgnorePar\fi}
       {\ifvmode \IgnorePar\fi} {\ifvmode \IgnorePar\fi}%
    |<footnote for ams title|>%
   }
   {\ifvmode \IgnorePar\fi \EndP \HCode{</div>}}
   {\NoFonts\IgnorePar \EndP 
    \HCode{<h2>}\IgnorePar}
   {\HCode{</h2>}\IgnoreIndent\EndNoFonts}
>>>

\<footnote for ams title\><<<
\Configure{footnotetext}
   {\HCode{<sup>}\HPage{\FNnum}\HCode{<div>}\NoFonts}
   {\EndNoFonts}
   {\HCode{</div>}\EndHPage{}\HCode{</sup>}}%
>>>



\<ams art,proc\><<<
\Configure{|<thanks author date and|>} 
   {\par\IgnorePar\EndP \HCode{<div class="thanks">}\par\ShowPar}
   {\ifvmode \IgnorePar\fi\EndP \HCode{</div>}}
   {\ifvmode \IgnorePar\fi\EndP
       \HCode{<div class="authors"><span class="author">}}
   {\ifvmode \IgnorePar\fi\EndP \HCode{</span></div>}}
   {\par\IgnorePar\EndP \HCode{<div class="date">}\par\ShowPar}
   {\ifvmode \IgnorePar\fi\EndP \HCode{</div>}}
   {\lowercase{\HCode{</span><span class="and">}}and~%
    \lowercase{\HCode{</span><span>}}}
   {\HCode{<br\xml:empty>}}
>>>



\<config book-report-article 3.2\><<<
\Configure{|<thanks author date and|>} 
   {\ifvmode \IgnorePar\fi\EndP \HCode{<div class="thanks" >}}
   {\ifvmode \IgnorePar\fi\EndP \HCode{</div>}}
   {\ifvmode \IgnorePar\fi\EndP \HCode{<div class="author" >}}
   {\ifvmode \IgnorePar\fi\EndP \HCode{</div>}}
   {\ifvmode \IgnorePar\fi\EndP \HCode{<div class="date" >}}
   {\ifvmode \IgnorePar\fi\EndP \HCode{</div>}}
   {\HCode{<br class="and" \xml:empty>}}
   {\HCode{<br \xml:empty>}}
>>>

\<thanks author date and\><<<
thanks author date and>>>






\<book-report-article caption 3.2\><<<
   |<makeketitle config 3.2|>
>>>

\<makeketitle config 3.2\><<<
\Configure{caption}{\HCode{\if:nopar  \else <br\xml:empty>\fi
   <div align="center"><table\Hnewline
    ><tr valign="bottom"><td |<no wrap|> ><strong>}}
   {} {\HCode{</strong></td><td \Hnewline>}} 
   {\HCode{</td></tr></table></div>}}
>>>








\section{aa}

\<configure html32 aa\><<<
|<config sections 3.2|>
|<book-report-article caption 3.2|>
|<latex report,... config 3.2|>
|<latex config div|>
|<latex config like div 3.2|>
|<configure aa 3.2/4.0t|>
|<latex shared div config|>
|<shared sec div config|>
|<shared subsec div config|>
|<shared sub end div config|>
\ifx \part\:UnDef \else
   |<shared part div config|>
   |<latex shared likepart config|>
\fi
>>>




\<latex report,... config 3.2\><<<
\ConfigureToc{lof}  {\empty}{\ }{}{\HCode{<br\xml:empty>}}
\ConfigureToc{lot}  {\empty}{\ }{}{\HCode{<br\xml:empty>}}
>>>





\section{pictex}

\<configure html32 pictex\><<<
\Configure{pictex}{\HCode{<div align="center"
   >}}{\HCode{</div>}}{+[PICT]}
>>>

\section{array}

\<configure html32 array\><<<
|<array.sty Configure 3.2|>
|<html latex array/tabular Config 3.2|>
>>>


\<html latex array/tabular Config 3.2\><<<
\Configure{array}
   {\halignTB{array}}
   \t:HA
   \R:HA\r:HA\D:HA\d:HA   
\Configure{halignTB<>}{array}{<>\HAlign}
\Configure{tabular}
   {\halignTB{tabular}}
   \t:HA
   \R:HA\r:HA\D:HA\d:HA
\Configure{halignTB<>}{tabular}{\HCode{cellpadding="5" \VBorder}<>\HAlign}
\ConfigureEnv{array}{\IgnorePar\HCode{<div align="center">}}
                    {\HCode{</div>}\ShowPar}{}{}
\ConfigureEnv{tabular}{\IgnorePar\HCode{<div align="center">}}
                      {\HCode{</div>}\ShowPar}{}{}
>>>
\<array.sty Configure 3.2\><<<
\Configure{VBorder}{\let\VBorder|=\empty}{\gdef\VBorder{border="1"}}{}{}
>>>



\section{latex.ltx}

\<configure html32 latex\><<<
|<0,32,4 plain,latex|>
|<32,4 plain,latex|>
|<32 plain,latex|>
|<32 picmath th4,latex|>
|<0,32,4 latex|>
|<32,4 latex|>
|<32 latex|> 
|<latex footnotes|>
% \:CheckOption{javahelp} \if:Option 
%   \input javahelp.4ht
% \else
%   \Log:Note{for a JavaHelp output format,
%         use the command line option `javahelp'}
% \fi
>>>

                                              %%%%%%%%%%%%%%%%%%%%%%%
                                              % ltplain.dtx
                                              %%%%%%%%%%%%%%%%%%%%%%%

\subsection{obeylines}



\<32 plain,latex\><<<
\:CheckOption{/obeylines} \if:Option
   |<delayed obeylines confg|>
\else  |<obeylines confg|> \fi
>>>


\<obeylines confg\><<<
\Configure{obeylines}
   {} {} {\hbox{\HCode{<br\xml:empty>}}}
>>>

Typically, \verb'\obeylines' appears in a separate line before the
content. The following option is introduced to avoid an extra leading
empty line.

\<delayed obeylines confg\><<<
\Configure{obeylines}
   {\def\Line:Break{\def\Line:Break{\hbox{\HCode{<br\xml:empty>}}}}} {}
   {\Line:Break}
>>>


                                              %%%%%%%%%%%%%%%%%%%%%%%
                                              % ltlogos.dtx
                                              %%%%%%%%%%%%%%%%%%%%%%%

\subsection{Logos}


If we redefine the following we get LaTeX in hypertext, and protection
within titles.  Without this definition we get LATEX.

\<32 latex\><<<
|</TeX for 3.2|>
\def\:temp{LaTeX}
\HLet\LaTeX|=\:temp
>>>

\</TeX for 3.2\><<<
\def\:temp{TeX}
\HLet\TeX|=\:temp
>>>



                                              %%%%%%%%%%%%%%%%%%%%%%%
                                              % ltmisc.dtx
                                              %%%%%%%%%%%%%%%%%%%%%%%

\subsection{Miscellaneous Environments}




\<32 latex\><<<
\Configure{centercr}{\ht:special{t4ht=<br\xml:empty>}}
   {\everypar{\everypar{\HCode{<p>}}}}
>>>

\<32 latex\><<<
\def\start:env#1{\IgnorePar\HCode{<\tbl:XV{#1}><tr><td\Hnewline>}}
\def\tbl:XV#1{table \Hnewline border="0" width="100\%"}
\def\end:env{\IgnorePar \end:TTT\ShowPar}
>>>

 cellpadding="0"    cellspacing="15"

                                              %%%%%%%%%%%%%%%%%%%%%%%
                                              % ltmath.dtx
                                              %%%%%%%%%%%%%%%%%%%%%%%

\subsection{Math Setup}


\<32 latex\><<<
\:CheckOption{pic-eqnarray}  \if:Option      
\else |<TABLE eqnarray Config 3.2|>\fi
>>>


\<TABLE eqnarray Config 3.2\><<<
\Configure{eqnarray}{\HCode{<table\Hnewline>}}{\HCode{</table>}}
    {\HCode{<tr valign="middle">}}{\HCode{</tr>}}
    {\HCode{\ifnum \HCol=4 <td\Hnewline width="10"></td>\fi
       <td\Hnewline |<no wrap|>  align="\ifcase\HCol \or right\or center\or 
               left\else right\fi" \Hnewline>}}{\HCode{</td>}}
\ConfigureEnv{eqnarray}{\HCode{<div align="center">}}
                       {\HCode{</div>}}{}{}
\ConfigureEnv{eqnarray*}{\HCode{<div align="center">}}
                        {\HCode{</div>}}{}{}
>>>




                                              %%%%%%%%%%%%%%%%%%%%%%%
                                              % lttab.dtx
                                              %%%%%%%%%%%%%%%%%%%%%%%

\subsection{Tabbing, Tabular and Array Environments}





\<32 latex\><<<
\Configure{VBorder}{\let\VBorder|=\empty}{\def\VBorder{border="1"}}{}{}
|<html latex array/tabular Config 3.2|>
>>>




\<PICT dot tabbing\><<<
\:CheckOption{pic-tabbing'} \if:Option
  \edef\:temp{\LikeRef{|<tabbing tag|>.}}%
  \def\:tempa{.}\ifx \:temp\:tempa 
      \ConfigureEnv{tabbing}{\Picture*{}}{\EndPicture}{}{}
  \fi
\fi 
>>>


\<32 latex\><<<
\:CheckOption{pic-tabbing}  \if:Option
    \ConfigureEnv{tabbing}{\Picture*{}}{\EndPicture}{}{}
\else 
    |<TABLE tabbing Config 3.2|>   
    |<PICT dot tabbing|>
\fi
>>>



\<TABLE tabbing Config 3.2\><<<
\Configure{tabbing}[1.5]{\IgnorePar\leavevmode \ht:special{t4ht=<table 
     \Hnewline\:zbsp{tabbing}><tr \Hnewline valign="bottom">}}
   {\ht:special{t4ht=</tr></table>}}
   {\gt:tab \ht:special{t4ht=<td \:tempa\Hnewline>}}
   {\ht:special{t4ht=</td>}}
>>>   






%%%%%%%%%%%%%%%%%%%%%% to be placed %%%%%%%%%%%%%%%%%%%%%%%%%
\subsection{to be placed}


\<32 picmath th4,latex\><<<
\Configure{[]} 
  {\PicDisplay$$\everymath{}\everydisplay{}}
  {$$\EndPicDisplay}
\Configure{()}{\protect\PicMath$}{$\protect\EndPicMath}
>>>


\<32 latex\><<<
\ifx\bf\:UnDef 
   \def\bf{\normalfont\bfseries}
\fi
\ConfigureList{trivlist}%
   {\HCode{<dl>}} {\HCode{</dl>}\ShowPar}
   {\HCode{<dt>}\bgroup \bf}{\egroup\HCode{<dd\Hnewline>}}
\ConfigureList{list}%
   {\HCode{<dl>}} {\HCode{</dl>}\ShowPar}
   {\HCode{<dt>}\bgroup \bf}{\egroup\HCode{<dd\Hnewline>}}
>>>




\<32 latex\><<<
\ConfigureEnv{enumerate}
   {}{|<try env inline par|>}{}{}
\ConfigureList{enumerate}%
      {|<enumerate I|>}
      {|<enumerate II|>}
      {|<enumerate III|>}
      {|<enumerate IV-|>}
>>>


\<enumerate I\><<<
\EndP\HCode{<ol 
      class="enumerate\expandafter\the\csname @enumdepth\endcsname"
     >}|<save end:itm|>\global\let\end:itm=\empty
>>>

\<enumerate II\><<<
|<recall end:itm|>%
\EndP\HCode{</li></ol>}\ShowPar
>>>


\<enumerate III\><<<
\end:itm \gdef\end:itm{\EndP\Tg</li>}\DeleteMark
>>>

\<enumerate IV-\><<<
\HCode{<li class="enumerate">}\AnchorLabel
>>>




\<32 latex\><<<
\ConfigureEnv{itemize}
   {\ifvmode \IgnorePar\fi \EndP}{|<try env inline par|>}
   {}{}
\ConfigureList{itemize}%
   {\EndP\HCode{<ul class="itemize\expandafter\the
          \csname @itemdepth\endcsname">}%
       |<save end:itm|>\global\let\end:itm=\empty}
   {|<recall end:itm|>\ifvmode \IgnorePar\fi 
    \EndP\HCode{</li></ul>}\ShowPar}
   {\end:itm \global\def\end:itm{\EndP\Tg</li>}\DeleteMark}
   {\HCode{<li class="itemize">}}
>>>


\<save end:itm\><<<
\PushMacro\end:itm
>>>


\<recall end:itm\><<<
\PopMacro\end:itm \global\let\end:itm \end:itm 
>>>



\<try env inline par\><<<
\ShowPar 
>>>




\<32 latex\><<<
\NewConfigure{enumerate}[1]{\c:enu #1|<par del|>}
\def\c:enu#1#2#3#4#5|<par del|>{%
   \def\OLStyle{\ifcase \@enumdepth \or
      type="#1"\or type="#2"\or type="#3"\or type="#4"\else\fi #5}}
>>>

\<32 latex\><<<
\Configure{enumerate}{1aiA}
>>>


% \Configure{tableofcontents}{}{}{\ShowPar}{\HCode{<br\xml:empty>}}{}

\<32 latex\><<<
\Configure{tableofcontents}
   {\IgnorePar\EndP\HCode{<div class="tableofcontents">}\IgnorePar}
   {}
   {\IgnorePar\EndP\HCode{</div>}\ShowPar}
   {\HCode{<br\xml:empty>}}   {}
>>>

\<32 latex\><<<
\Configure{TocAt}
   {|<save configure tableofcontents|>%
    \Configure{tableofcontents}
       {\IgnorePar\EndP\HCode{<div class="\sec:typ TOCS">}}
       {}{\IgnorePar\HCode{</div>}\ShowPar}{\HCode{<br\xml:empty>}}{}%
    \ifvmode \IgnorePar\fi \EndP
   }
   {|<recall configure tableofcontents|>\par\ShowPar}
\Configure{TocAt*}
   {|<save configure tableofcontents|>%
    \Configure{tableofcontents}
       {\IgnorePar\EndP\HCode{<div class="\sec:typ TOCS">}}
       {}{\IgnorePar\HCode{</div>}\ShowPar}{\HCode{<br\xml:empty>}}{}%
    \ifvmode \IgnorePar\fi \EndP
   }
   {|<recall configure tableofcontents|>\par\ShowPar}
>>>


% \Configure{TocAt}
%    {|<save configure tableofcontents|>%
%      \Configure{tableofcontents}
%      {\IgnorePar}{}{\ShowPar} {\HCode{<br\xml:empty>}}{}}
%    {|<recall configure tableofcontents|>}
% \Configure{TocAt*}
%    {|<save configure tableofcontents|>%
%       \Configure{tableofcontents}
%       {\IgnorePar}{}{\ShowPar}{\HCode{<br\xml:empty>}}{}}
%    {|<recall configure tableofcontents|>}
% 



\<32 latex\><<<
\Configure{newtheorem}
   {\HCode{<div class="newtheorem"><b class="head">}}
   {\HCode{</b>}}
   {\HCode{</div>}}
>>>

\<32 latex\><<<
\Configure{verbatim}{\:nbsp}{\a:sp}
\Configure{verb}{\HCode{<code>}}{\HCode{</code>}}
>>>

\<32 latex\><<<
\def\env:verb#1{\ifvmode \IgnorePar \fi\EndP
   \HCode{<\tbl:XV{#1}><tr><td\Hnewline
   >}\HCode{<pre>}\EndNoFonts}
>>>



\<32 latex\><<<
\Configure{marginpar}
   {\HCode{<table \Hnewline align="right"><tr><td \Hnewline><u><small>}}
   {\HCode{</small></u></td></tr></table>}}
>>>

\<32 latex\><<<
\Configure{equation}
  {\:xhtml{\IgnorePar\EndP}%
     \HCode{<\tbl:XV{equation}><tr><td><div align="center">}\IgnorePar
   \Configure{$$}{\PicDisplay}{\EndPicDisplay}
     {\everymath{}\everydisplay{}}
  }
  {\IgnorePar\HCode{</div></td><td width="5\%">}}
  {\end:TTT\IgnorePar\par}
>>>


\<32 latex\><<<
\ConfigureEnv{picture}
    {\ifvmode \IgnorePar\leavevmode\HCode{<p align="center">}\fi}
    {}{}{}
>>>


\<32 latex\><<<
\Configure{float}{}
   {\ifvmode \IgnorePar\fi \EndP 
    \HCode{<hr\xml:empty><div align="center"\Hnewline><table><tr><td\Hnewline>}}
   {\ifvmode \IgnorePar\fi \EndP 
     \HCode{</td></tr></table></div><hr\xml:empty>}\csname par\endcsname}
>>>


\<32 latex\><<<
\Configure{newline}{\HCode{<br\xml:empty>}}
>>>






\section{plain latex}



\<32 plain,latex\><<<
\def\:zbsp#1{cellpadding="0" border="0" cellspacing="0"\Hnewline}
>>>




\<pic plain/latex math 3.2NO\><<<
\Configure{big}{\HCode{<big>}}{\HCode{</big>}}
\Configure{Big}{\HCode{<big><big>}}{\HCode{</big></big>}}
\Configure{bigg}{\HCode{<big><big><big>}}
   {\HCode{</big></big></big>}}
\Configure{Bigg}{\HCode{<big><big><big><big>}}
   {\HCode{</big></big></big></big>}}
>>>

\<pic plain math 3.2\><<<
\Contribute{underbrace}{align="middle"}
>>>


\<pic plain/latex math 3.2\><<<
\Configure{underline}
  {\Tg<u>}{\Tg</u>}
\newbox\tmp:bx
\Configure{overline}{\Picture+{ 
   \a:@Picture{overline}}\setbox\tmp:bx|=\hbox
       \bgroup\everypar{}}{\egroup\o:overline:{\box\tmp:bx}\EndPicture}
>>>


\<pic plain/latex math 3.2\><<<
\NewConfigure{@root}{1}
\Configure{@root}{align="middle" }
>>>





\section{plain}

\<configure html32 plain\><<<
|<0,32,4 plain,latex|>
|<32 plain,latex|>
|<0,32,4 plain|>
|<32,4 plain|>
|</TeX for 3.2|>
|<32 plain|> 
\:CheckOption{plain-} \if:Option \else
  \Configure{item}{}{}{\par\leavevmode}{}
\fi
>>>



\<32 plain,latex\><<<
\Configure{centerline}{\HCode{<div align="center"\Hnewline 
      class="centerline">}}{\HCode{</div>}}
\Configure{leftline}{\HCode{<p\Hnewline  class="leftline">}}{}
\Configure{rightline}{\HCode{<p align="right"\Hnewline
                                class="rightline">}}{}
>>>





\<picmath plain,latex\><<<
\Configure{pmatrix}
  {\ifvmode
      \def\end:pmatrix{\IgnorePar\HCode{</div>}\end:TTT}%
      \:xhtml{\IgnorePar\EndP}\HCode
        {<\tbl:XV{pmatrix}><tr><td><div align="center"\Hnewline>}%
   \else \HCode{<span class="pmatrix">}%
      \def \end:pmatrix{\HCode{</span>}}%
   \fi}
  {\end:pmatrix}
>>>






\<32 plain\><<<
\:CheckOption{pic-eqalign}  \if:Option      
   \:CheckOption{no-halign} \if:Option \else      
   \fi
\else  |<TABLE eqalign shared Configure 3.2|>
\fi
>>>





\<TABLE eqalign shared Configure 3.2\><<<
\Configure{eqalign}
   {\HCode{<div align="center"><table>}}
   {\HCode{</table></div>}}
   {\HCode{<tr \Hnewline 
         valign="midlle">}}{\IgnorePar\HCode{</tr>}}
   {\HCode{\ifnum \HCol=3 <td\Hnewline width="30"></td>\fi
        <td align="\ifnum \HCol=2
         left\else right\fi" |<no wrap|> \Hnewline>}}
   {\HCode{</td>}}
\def\:eqalign:{\Configure{noalign}
  {\HCode{<tr><td class="noalign" colspan="2">}}%
  {\HCode{</td></tr>}}}
>>>


\<TABLE eqalign shared Configure 3.2\><<<
\Configure{eqalignno}
   {\HCode{<div align="center"><table>}}
   {\HCode{</table></div>}}
   {\HCode{<tr \Hnewline 
         valign="bottom">}}{\IgnorePar\HCode{</tr>}}
   {\HCode{\ifnum \HCol=3 <td\Hnewline width="10"></td>\fi
        <td align="\ifnum \HCol=2
         left\else right\fi" |<no wrap|> \Hnewline>}}
   {\HCode{</td>}}
\def\:eqalignno:{\Configure{noalign}
  {\HCode{<tr><td class="noalign" colspan="2">}}%
  {\HCode{</td></tr>}}}
>>>




\<TABLE eqalign shared Configure 3.2\><<<
\Configure{leqalignno}
   {\HCode{<div align="center"><table>}}
   {\HCode{</table></div>}}
   {\HCode{<tr \Hnewline 
         valign="middle">}}{\IgnorePar\HCode{</tr>}}
   {\HCode{\ifnum \HCol=3 <td\Hnewline width="10"></td>\fi
        <td align="\ifnum \HCol=2
         left\else right\fi" |<no wrap|> \Hnewline>}}
   {\HCode{</td>}}
\def\:leqalignno:{\Configure{noalign}
  {\HCode{<tr><td class="noalign" colspan="2">}}%
  {\HCode{</td></tr>}}}
>>>







\<32 plain\><<<
\Configure{settabs}[1.5]{\IgnorePar\leavevmode \ht:special{t4ht=<table 
     \Hnewline\:zbsp{settabs}><tr \Hnewline valign="bottom">}}
   {\ht:special{t4ht=</tr></table>}}
   {\gt:tab \ht:special{t4ht=<td \:tempa\Hnewline>}}
   {\ht:special{t4ht=</td>}}
>>>   



\<32 plain\><<<
\Configure{narrower}{\ifvmode \IgnorePar\fi
   \HCode{<table cellpadding="15" class="narrower"><tr
      class="narrower"><td class="narrower">}\ifvmode \IgnorePar\fi}
   {\ifvmode \IgnorePar\fi
    \HCode{</td></tr></table>}\ifvmode \IgnorePar\fi}
>>>



\<32 plain\><<<
\Configure{proclaim}
    {\IgnorePar\HCode{<div class="proclaim"><strong class="proclaim">}}
    {\HCode{</strong>}}    {\IgnorePar\HCode{</div>}}
\Configure{beginsection}
  {}{}
  {\ifvmode \IgnorePar\fi
   \HCode{<h3 class="beginsection">}} 
  {\HCode{</h3>}\par\ShowPar}
\ConfigureToc{beginsection} 
     {} {\relax}  {}  { }
>>>





\<32 plain\><<<
\Configure{TableOfContents}   {}{}{\ShowPar}{\HCode{<br\xml:empty>}}{}
>>>



\<32 plain\><<<
\Configure{insert}
  {\IgnorePar\HCode{<hr\xml:empty>}\IgnorePar}
  {\IgnorePar\HCode{<hr\xml:empty>}\IgnorePar}
>>>




\section{amsmath}



\<32,4 pic amsmath\><<<
|<amsmath / amstex1 m:env|>
\ConfigureEnv{eqxample}{\m:env{eqxample}}{\endm:env}{}{}   
>>>








\<configure html32-math amsmath\><<<
\ConfigureEnv{align*}{\m:env{align*}}{\endm:env}{}{}   
\ConfigureEnv{align}{\m:env{align}}{\endm:env}{}{}   
\ConfigureEnv{alignat*}{\m:env{alignat*}}{\endm:env}{}{}   
\ConfigureEnv{alignat}{\m:env{alignat}}{\endm:env}{}{}   
\ConfigureEnv{flalign*}{\m:env{flalign*}}{\endm:env}{}{}   
\ConfigureEnv{flalign}{\m:env{flalign}}{\endm:env}{}{}
\ConfigureEnv{xalignat*}{\m:env{xalignat*}}{\endm:env}{}{}   
\ConfigureEnv{xalignat}{\m:env{xalignat}}{\endm:env}{}{}   
\ConfigureEnv{xxalignat}{\m:env{xxalignat}}{\endm:env}{}{}   
>>>




\<32,4 pic amsmath\><<<
\ConfigureEnv{gather*}{\m:env{gather*}}{\endm:env}{}{}   
\ConfigureEnv{gathered}{\m:env{gathered}}{\endm:env}{}{}   
\ConfigureEnv{matrix}{\m:env{matrix}}{\endm:env}{}{}   
\ConfigureEnv{quotation}{\m:env{quotation}}{\endm:env}{}{}   
\Configure{equations}{*}{}
>>>




A `\verb'\begin{multline}...\end{multline}' is not a standard environment
in the sense that the environment as a whole is read in one piece and
then processed, instead of reading it piecewise and process it as it
goes.  That is, we have a behavior similar to that in verbatim
environments. The behavior is due to multline being implemented in
terms of \verb'\gather@#1{..}'.  Hence, for the picture environment, we
need to change early the catcodes of `\verb'_' and `\verb'^'.

\<32,4 pic amsmath\><<<
\ConfigureEnv{multline}
  {\:xhtml{\IgnorePar\EndP}%
    \HCode{<\tbl:XV{multline}><tr><td>}\Picture*{}\ExtractHLabel
  }
  {\EndPicture \HCode{</td><td width="5\%">}\PutHLabel\end:TTT}
  {}{}  {}{}
\ConfigureEnv{multline*}
  {\:xhtml{\IgnorePar\EndP}%
    \HCode{<\tbl:XV{multline-star}><tr><td>}\Picture*{}%
  }
  {\EndPicture \end:TTT}
  {}{}
|<extract amsmath labels|>
>>>


\<extract amsmath labels\><<<
\def\ExtractHLabel{%
   \def\tagform@##1{{\xdef\:HLabel{\noexpand\tagform@{##1}}}}}
\def\PutHLabel{\:HLabel}
>>>





\<configure html32-math amsmath\><<<
|<32,4 pic amsmath|>
>>>

\<32,4 pic amsmath\><<<
\ConfigureEnv{gather}
  {\:xhtml{\IgnorePar\EndP}%
    \HCode{<\tbl:XV{gather}><tr><td 
       class="gather1">}\Picture*{}\ExtractHLabel
  }
  {\EndPicture \HCode{</td><td width="5\%">}\PutHLabel\end:TTT}
  {}{}
\ConfigureEnv{gather*}
  {\:xhtml{\IgnorePar\EndP}%
    \HCode{<\tbl:XV{gather-star}><tr><td>}\Picture*{}%
  }
  {\EndPicture \end:TTT}
  {}{}
\Css{td.gather-star, td.gather1 {text-align:center; }}
>>>

\section{amstex.sty}

\<configure html32 amstex1\><<<
\ConfigureEnv{aligned}{\m:env{aligned}}{\endm:env}{}{}
|<32,4 amstex1|>
>>>


\<32,4 amstex1\><<<
|<amsmath / amstex1 m:env|>
\ConfigureEnv{equation*}{\m:env{equation*}}{\endm:env}{}{}
\ConfigureEnv{equation}{\m:env{equation}}{\endm:env}{}{}   
\Configure{eqn}{\HCode{</td><td>}}
\ConfigureEnv{align}{\m:env{align}}{\endm:env}{}{}   
\ConfigureEnv{align*}{\m:env{align*}}{\endm:env}{}{}   
\ConfigureEnv{alignat}{\m:env{alignat}}{\endm:env}{}{}   
\ConfigureEnv{alignat*}{\m:env{alignat*}}{\endm:env}{}{}   
\ConfigureEnv{xalignat}{\m:env{xalignat}}{\endm:env}{}{}   
\ConfigureEnv{xxalignat}{\m:env{xxalignat}}{\endm:env}{}{}   
\ConfigureEnv{xalignat*}{\m:env{xalignat*}}{\endm:env}{}{}   
\ConfigureEnv{aligned}{\m:env{aligned}}{\endm:env}{}{}
\ConfigureEnv{alignedat}{\m:env{alignedat}}{\endm:env}{}{}
\ConfigureEnv{gather}{\m:env{gather}}{\endm:env}{}{}   
\ConfigureEnv{gather*}{\m:env{gather*}}{\endm:env}{}{}   
\ConfigureEnv{gathered}{\m:env{gathered}}{\endm:env}{}{}   
\ConfigureEnv{matrix}{\m:env{matrix}}{\endm:env}{}{}   
\ConfigureEnv{multline}{\m:env{multline}}{\endm:env}{}{}   
\ConfigureEnv{multline*}{\m:env{multline*}}{\endm:env}{}{}   
\ConfigureEnv{pmatrix}{\m:env{pmatrix}}{\endm:env}{}{}
\ConfigureEnv{bmatrix}{\m:env{bmatrix}}{\endm:env}{}{}
\ConfigureEnv{vmatrix}{\m:env{vmatrix}}{\endm:env}{}{}
\ConfigureEnv{Vmatrix}{\m:env{Vmatrix}}{\endm:env}{}{}
\Configure{gather}{\Picture*{}}{\EndPicture}
>>>



\section{amstex.tex}

\<configure html32-math amstex\><<<
|<32,4 picmath amstex.tex|>
  \:CheckOption{no-matrix} \if:Option \else
  \:CheckOption{pic-matrix}   \if:Option
         |<pic amstex.tex matrix 3.2|> 
  \else
         |<tabular amstex.tex matrix 3.2|> 
  \fi\fi
  \:CheckOption{no-align} \if:Option \else
  \:CheckOption{pic-align}   \if:Option
         |<pic amstex.tex align 3.2|> 
  \else
         |<tabular amstex.tex align 3.2|> 
  \fi\fi
  \:CheckOption{no-cases} \if:Option \else
     \:CheckOption{pic-cases}   \if:Option
           |<pic amstex.tex cases 3.2|> 
  \else
         |<nonpic amstex.tex cases 3.2|> 
  \fi\fi

>>>


\<pic amstex.tex cases 3.2\><<<
\Configure{cases}{\m:env{cases}}{\endm:env}
>>>

\<nonpic amstex.tex cases 3.2\><<<
\Configure{cases}{\m:env{cases}}{\endm:env}
>>>



\<pic amstex.tex align 3.2\><<<
\Configure{align}{\m:env{align}}{\endm:env}
>>>


\<tabular amstex.tex align 3.2\><<<
\Configure{align}
   {\HCode{<table\Hnewline class="align">}}  {\HCode{</table>}}
   {\HCode{<tr\Hnewline>}}   {\HCode{</tr>}}
   {\HCode{<td>}}   {\HCode{</td>}}
>>>





\<pic amstex.tex matrix 3.2\><<<
\Configure{matrix}{\m:env{matrix}}{\endm:env}
>>>

\<tabular amstex.tex matrix 3.2\><<<
\Configure{matrix}
   {\HCode{<table\Hnewline class="matrix">}}  {\HCode{</table>}}
   {\HCode{<tr\Hnewline>}}   {\HCode{</tr>}}
   {\HCode{<td>}}   {\HCode{</td>}}
>>>




\section{vanilla}

\<configure html32 vanilla\><<<
|<32,4 vanilla|>
|<32 amsppt, 32,4 vanilla|>
\Configure{heading}
   {}{}{\IgnorePar\EndP\HCode{<h2 class="heading">}}{\HCode{</h2>}}
\ConfigureToc{heading}
  {}{\HCode{<span class="heading">}}{}{\HCode{</span><br\xml:empty>}}
\Configure{subheading}
  {}{}{\EndP\HCode{<h3 class="subheading">}}{.\HCode{</h3>}}
\ConfigureToc{subheading}
  {}{\HCode{<span class="subheading">}}{}{\HCode{</span><br\xml:empty>}}
\Configure{demo}
    {\IgnorePar\EndP\HCode{<div class="demo"><span class="demo">}}
    {\HCode{</span>}}    {\IgnorePar\EndP\HCode{</div>}}
\Configure{aligned}
   {\EndP\HCode{<center><table\Hnewline
        border="0" cellpadding="0" cellspacing="15" class="aligned">}}
   {\HCode{</table></center>}\IgnorePar}
   {\HCode{<tr\Hnewline valign="top">}}{\HCode{</tr>}}
   {\HCode{<td>}}   {\HCode{</td>}}
>>>

\section{minitoc}

\<configure html32 minitoc\><<<
\Configure{minitoc}{\HCode{<div>}}{\HCode{</div>}}{}{}
\Configure{parttoc}{\HCode{<div>}}{\HCode{</div>}}{}{}
\Configure{secttoc}{\HCode{<div>}}{\HCode{</div>}}{}{}
\Configure{minilof}{\HCode{<div>}}{\HCode{</div>}}{}{}
\Configure{partlof}{\HCode{<div>}}{\HCode{</div>}}{}{}
\Configure{sectlof}{\HCode{<div>}}{\HCode{</div>}}{}{}
\Configure{minilot}{\HCode{<div>}}{\HCode{</div>}}{}{}
\Configure{partlot}{\HCode{<div>}}{\HCode{</div>}}{}{}
\Configure{sectlot}{\HCode{<div>}}{\HCode{</div>}}{}{}
>>>

\section{fancyvrb}

\<configure html32 fancyvrb\><<<
\Configure{fancyvrb}
   {\HCode{<div class="fancyvrb">}}  {\HCode{</div>}}
   {}  {\HCode{<br\xml:empty>}}
   {\ \ }{}
>>>








\section{url}

\<configure html32 url\><<<
\Configure{url}{\Link[#1]{}{}#1\EndLink}
>>>



\section{amsfonts}

\<configure html32 amsfonts\><<<  
\:CheckOption{fonts} \if:Option
  \Configure{mathbb}{\Protect\HCode{<b>}}
                    {\Protect\HCode{</b>}}
  \Configure{mathfrak}{\Protect\HCode{<span class="mathfrak">}}
                    {\Protect\HCode{</span>}}
\fi
>>>





\section{amsppt}

\<configure html32 amsppt\><<<
|<sectioning in amsppt.sty|>
|<32,4 amsppt|>
|<32 amsppt|>
\ifx \EnditemitemList\:UnDef
\Configure{itemitem}{}{}
   {\par\leavevmode\:nbsp\:nbsp\:nbsp}{}
\fi
>>>

\section{fontmath.ltx}

\<configure html32-math fontmath\><<<
|<32,4 picmath: plain, fontmath, amsmath, amstex1|>
|<math plain,fontmath|>
>>>





\<configure html32 fontmath\><<<
\Configure{mathit}{\Protect\HCode{<i>}}{\Protect\HCode{</i>}}
\Configure{mathbf}{\Protect\HCode{<b>}}{\Protect\HCode{</b>}}
\Configure{mathtt}{\Protect\HCode{<tt>}}{\Protect\HCode{</tt>}}
\Configure{mathsf}{}{}
\Configure{mathrm}{}{}
\Configure{textbf}{\Protect\HCode{<b>}}{\Protect\HCode{</b>}}
\Configure{textit}{\Protect\HCode{<i>}}{\Protect\HCode{</i>}}
\Configure{textrm}{}{}
\Configure{textup}{}{}
\Configure{textsc}{}{}
\Configure{textsf}{}{}  
\Configure{textsl}{\Protect\HCode{<i>}}{\Protect\HCode{</i>}}
\Configure{texttt}{\Protect\HCode{<tt>}}{\Protect\HCode{</tt>}}
\Configure{emph}{\Protect\HCode{<em>}}{\Protect\HCode{</em>}}
>>>

\section{emulateapj}

\<configure html32 emulateapj\><<<
\Configure{affil}{\HCode{<center>}}{\HCode{</center>}}
\Configure{author}{\HCode{<center>}}{\HCode{</center>}}
\Configure{keywords}{\HCode{<center><div>}}{\HCode{</div></center>}}
\Configure{subjectheadings}
   {\HCode{<center><div>}}{\HCode{</div></center>}}
\Configure{slugcomment}
   {\HCode{<center><i>}} {\HCode{</i></center>}}
\Configure{subtitle}{}{}
\Configure{submitted}{}{\HCode{<br\xml:empty>}}
\Configure{title}
   {\HCode{<h1 class="titleHead">}}
   {\HCode{</h1>}}
>>>



\section{slidesec}

\<configure html32 slidesec\><<<
\ConfigureToc{tocslidesection}  {\empty}{\ }{}{\HCode{<br\xml:empty>}}
>>>



\section{seminar}

\<configure html32 seminar\><<<
\ConfigureEnv{slide}
   {\HCode{<hr />}}  {\rightline{\the\c@slide}}{}{}
>>>


\section{tex4ht}


\<configure html32 Preamble\><<<  
\Configure{PROLOG}{DOCTYPE}
\:CheckOption{no-DOCTYPE} \if:Option
   \Configure{PROLOG}{}
\else
  \Log:Note{to remove the DOCTYPE declaration
          use the command line option `no-DOCTYPE'}   
\fi
>>>



\<dtd lang\><<<
\expandafter
\ifx \csname a:dtd-lang\endcsname\relax EN\else
  \csname a:dtd-lang\endcsname
\fi >>>


\<configure html32 tex4ht\><<<
|<0,32,4 tex4ht|>
|<32,4 tex4ht|>
|<32 tex4ht|>
|<title for hypertext page|>
\ifx \a:DOCTYPE\relax
   \Configure{DOCTYPE}{\IgnorePar\HCode{<!DOCTYPE 
      html PUBLIC "-//W3C//DTD HTML 3.2//|<dtd lang|>"
        \Hnewline\space\space
        "http://www.w3.org/pub/WWW/MarkUp/Wilbur/HTML32.dtd">\Hnewline
      }}      
   |<xml html32|>
\fi
\def\:gobbleM#1->{} 
\Configure{@HEAD}
   {\HCode{<!--\space\expandafter\:gobbleM\meaning
                   \Preamble\space-->\Hnewline}}
\immediate\write-1{TeX4ht package options:
    \expandafter\:gobbleM\meaning\Preamble}
>>>


\<xml html32\><<<
\:CheckOption{xmldtd} \if:Option
   \Configure{DOCTYPE}
     {\HCode{<!DOCTYPE html \xhtml:DOCTYPE>            \Hnewline
             <!--http://www.w3.org/TR/xhtml1/DTD/xhtml1-transitional.dtd-->
             \Hnewline}}
   \def\xhtml:DOCTYPE{PUBLIC
       "-//W3C//DTD XHTML 1.0 Transitional//|<dtd lang|>"\Hnewline
       \space\space
       "http://www.w3.org/TR/xhtml1/DTD/xhtml1-transitional.dtd"}
\fi
>>>



\<configure html32 tex4ht\><<<
\Configure{HtmlPar}
   {\EndP\HCode{<!--l. \the\inputlineno-->}\HCode{<p>}}
   {\EndP\HCode{<!--l. \the\inputlineno-->}\HCode{<p>}}
   {\:xhtml{\Tg</p>}}
   {\:xhtml{\Tg</p>}}
\ifx \a:HTML\:UnDef
   \Configure{HTML}
     {\IgnorePar\HCode{<html \a:@HTML\Hnewline >}}
     {\HCode{\Hnewline</html>\Hnewline}}
\fi
\ifx \a:HEAD\:UnDef
   \Configure{HEAD}
      {\IgnorePar\NoFonts\HCode
         {<head>|<src note|>\Hnewline}}
      {\HCode{</head>}\EndNoFonts}
\fi
\ifx \a:BODY\:UnDef
   \Configure{BODY}
      {\IgnorePar\HCode{<body\Hnewline >}\ShowPar}
      {\ifvmode\IgnorePar\fi \EndP\HCode{\Hnewline </body>}}
\fi
\ifx \a:TITLE\:UnDef
   \Configure{TITLE}{\Protect\IgnorePar
      \HCode{<title>}}{\HCode{</title>\Hnewline}}
\fi
|<no css|>
>>>



\<src note\><<<
<!--\FileName\space from \jobname.tex
(TeX4ht)-->%
>>>

\<4 src note\><<<
<!--\FileName\space from \jobname.tex -->%
>>>




\<configure html32 tex4ht\><<<
\Configure{crosslinks+}
   {\IgnorePar\EndP\HCode{<p class="noindent">}}
   {\HCode{</p>}\par\ShowPar}
   {\IgnorePar\EndP\HCode{<p>}}
   {\HCode{</p>}\par\ShowPar}
\Configure{MkHalign}
   {\halignTB{MkHalign}}
   \t:HA
   \R:HA\r:HA\D:HA\d:HA   
\Configure{halignTD} {}{}
   {<}{\HCode{align="left" |<no wrap|> }}
   {-}{\HCode{align="center" |<no wrap|> }}
   {>}{\HCode{align="right" |<no wrap|> }}
   {^}{\HCode{valign="top" |<no wrap|> }}
   {=}{\HCode{valign="baseline" |<no wrap|> }}
   {||}{\HCode{valign="middle" |<no wrap|> }}
   {_}{\HCode{valign="bottom" |<no wrap|> }}
   {p}{\HCode{align="left"}}
   {}
\def\R:HA{\HCode{<tr \Hnewline valign="middle">}}
\Configure{halign}
   {\halignTB{halign}}
   \t:HA
   \R:HA\r:HA\D:HA\d:HA
\Configure{pic-halign}{}
\Configure{IMG}
   {\ht:special{t4ht=<img\Hnewline src="}}
   {\ht:special{t4ht=" alt="}}
   {" }
   {\ht:special{t4ht=" }}
   {\ht:special{t4ht=\xml:empty>}}
\Configure{Picture*}{}{}
\Configure{@Picture}{\:class}
\def\:class#1{%
       \expandafter\ifx\csname :#1:\endcsname\relax\else
       \csname :#1:\endcsname\fi}
>>>



\<contribute to picmath of 3.2\><<<
\NewConfigure{@neq}{1}
\Configure{@neq}{align="middle"}
\NewConfigure{@buildrelover}{1}
\Configure{@buildrelover}{align="middle"}
\NewConfigure{@doteq}{1}
\Configure{@doteq}{align="middle"}
\NewConfigure{@underbrace}{1}
\NewConfigure{@frac}{1}
\Configure{@frac}{align="middle"}
\NewConfigure{@left}{1}
\Configure{@left}{align="middle"}
>>>






\<configure html32 tex4ht\><<<
\Configure{htf}{1}{+}{<img
   src="}{" alt="}{}{}{}{}{"\xml:empty>}
\Configure{htf}{3}{+}{<img
   src="}{" alt="}{}{}{}{}{" align="middle"\xml:empty>}
\Configure{htf}{4}{+}{<small>}{}{}{}{}{}{</small>}
\Configure{htf}{6}{+}{<u>}{}{}{}{}{}{</u>}
>>>








\section{th4}

\<configure html32 th4\><<<
\Configure{Verbatim}{\HCode{<pre>}} {\HCode{</pre>}} {}{\:nbsp} 
|<32,4 th4|>
|<32 th4|>
>>>





\section{amsthm.sty}

\<configure html32 amsthm\><<<
\ConfigureEnv{proof}{\par\leavevmode}{\par\ShowPar}{}{}
>>>

\section{foils}

\<configure html32 foils\><<<
|<32,4 foils|>
>>>





\section{epsfig}

\<configure html32 epsfig\><<<
|<0,32,4 epsfig|>
>>>

\section{psfig}

\<configure html32 psfig\><<<
|<0,32,4 psfig|>
>>>

\section{graphics}

\<configure html32 graphics\><<<
|<0,32,4 graphics|>
>>>






\section{moreverb}

\<configure html32 moreverb\><<<
|<32,4 moreverb|>
>>>



\section{xy}



\<configure html32 xy\><<<
|<32,4 xy|>
>>>

\<32,4 xy\><<<
\Configure{xypic}
   {\Picture*{}} {\EndPicture}
>>>






\section{pb-diagram}




\<configure html32 pb-diagram\><<<
\ConfigureEnv{diagram}
   {\Picture*{}$} {$\EndPicture} {}{}
>>>




\section{amscd}




\<configure html32 amscd\><<<
\ConfigureEnv{CD}
   {\Picture*{}$} {$\EndPicture} {}{}
>>>













\section{ntheorem}

\<configure html32 ntheorem\><<<
|<32,4 ntheorem|>
>>>




%%%%%%%%%%%%%%%%%%%%%%%%%%%%%%%%%%%%%%%%%%%%%%%%%%%%%%%%%%%%%%%%%%%%%%%%
\chapter{Picmath 3.2 \& 4}
%%%%%%%%%%%%%%%%%%%%%%%%%%%%%%%%%%%%%%%%%%%%%%%%%%%%%%%%%%%%%%%%%%%%%%%%


\<html32-math\><<<
%%%%%%%%%%%%%%%%%%%%%%%%%%%%%%%%%%%%%%%%%%%%%%%%%%%%%%%%%%  
% html32-math.4ht (|version), generated from |jobname.tex
% Copyright (C) 2009-2010 TeX Users Group
% Copyright (C) |CopyYear.1999. Eitan M. Gurari
|<TeX4ht copywrite|>
>>>


\section{tex4ht}




\<try inline par\><<<
\ShowPar\par{\HCondtrue\noindent}%
>>>





\<configure html32-math tex4ht\><<<
|<32,4 picmath tex4ht|>
\Configure{PicMath}{}{}{}{}
|<contribute to picmath of 3.2|>
>>>

\<32,4 picmath tex4ht\><<<
\:CheckOption{no_^} 
\if:Option \else \:CheckOption{no_}\fi
\if:Option \else
   \Configure{SUB}
      {\HCode{<sub>}}{\HCode{</sub>}}
\fi 
\:CheckOption{no_^} 
\if:Option \else \:CheckOption{no^}\fi
\if:Option \else
   \Configure{SUP}
      {\HCode{<sup>}}{\HCode{</sup>}}
\fi
\:CheckOption{no_^} 
 \if:Option \else \:CheckOption{no_}\fi
 \if:Option \else \:CheckOption{no^}\fi
\if:Option \else
   \Configure{SUBSUP}
      {\HCode{<sub>}}{\HCode{</sub><sup>}}{\HCode{</sup>}}
\fi
\Configure{left}
  {\Picture+{ \a:@Picture{left}}}
  {\aftergroup\EndPicture   }
\Configure{mathchoice}{\PictureOff}{\PictureOn}
>>>






\verb'\endgraf' is safer than \verb'\par', because the latter may be redefined.
For instance, see p 262 in texbook.







\<configure html32-math tex4ht\><<<
\Configure{PicDisplay}{\HCode{<center>}}{\HCode{</center>}}{}{} 
>>>








\section{plain}



\<configure html32-math plain\><<<
|<picmath plain,latex|>
|<math plain,fontmath|>
|<picmath plain|>
|<32,4 picmath plain|>
>>>

\<32,4 picmath plain\><<<
\Configure{sqrt}
   {\Picture+{ \a:@Picture{sqrt}}}
   {\EndPicture}
|<32,4 picmath: plain, fontmath, amsmath, amstex1|>
>>>


\section{latex}












\<configure html32-math latex\><<<
|<picmath plain,latex|>
|<picmath latex|>
|<32,4 picmath latex|>
|<32 picmath th4,latex|>
>>>


\<picmath plain,latex\><<<
|<pic plain/latex math 3.2|> 
\def\A:root#1\b:root#2\c:root{\o:root:#1\of{#2}\c:root}
\Configure{root}
   {\Picture+{ \a:@Picture{root}}\A:root}
   {}
   {\EndPicture}
\Configure{mathpalette}
    {\Picture+{ \a:@Picture{mathpalette}}} {\EndPicture}
>>>


\<32,4 picmath latex\><<<
\Configure{pmatrix} {\Picture+{ \a:@Picture{}}} {\EndPicture}
\Configure{bordermatrix} {\Picture+{ \a:@Picture{}}} {\EndPicture}
\Configure{frac}
   {\Picture+{ \a:@Picture{frac}}\bgroup}
   {} {} 
   {\egroup\EndPicture}
>>>




\<32,4 picmath latex\><<<
\Configure{sqrtsign}
   {\Picture+{ \a:@Picture{sqrt}}}
   {\EndPicture}
\Configure{matrix}
   {\Picture+{ \a:@Picture{matrix}}}  {\EndPicture}
   {}{}{}{}
>>>




\<picmath plain,latex\><<<
\Configure{L}                     {\pic:sym{L}}
\Configure{l}                     {\pic:sym{l}}
\def\pic:sym#1{\Protect\Picture+{ \a:@Picture{#1}}\csname
   o:#1:\endcsname\Protect\EndPicture}
>>>

\<math plain,fontmath\><<<
\Configure{Longrightarrow}        {\pic:sym{Longrightarrow}}
\Configure{bowtie}                {\pic:sym{bowtie}}
\Configure{cong}                  {\pic:sym{cong}}
\Configure{ddots}                 {\pic:sym{ddots}}
\Configure{doteq}                 {\pic:sym{doteq}}
\Configure{hookleftarrow}         {\pic:sym{hookleftarrow}}
\Configure{hookrightarrow}        {\pic:sym{hookrightarrow}}
\Configure{longmapsto}            {\pic:sym{longmapsto}}
\Configure{mapsto}                {\pic:sym{mapsto}}
\Configure{models}                {\pic:sym{models}}
\Configure{neq}                   {\pic:sym{neq}}
\Configure{notin}                 {\pic:sym{notin}}
\Configure{vdots}                 {\pic:sym{vdots}}
\Configure{angle}                 {\pic:sym{angle}}
\Configure{rightleftharpoons}     {\pic:sym{rightleftharpoons}}
\Configure{leftrightharpoons}     {\pic:sym{leftrightharpoons}}
>>>


\<picmath latex\><<<
\Configure{mathellipsis}          {...}
>>>


\<picmath plain\><<<
\Configure{ldots}                 {...}
\Configure{cdots}                 {\pic:sym{cdots}}
>>>


\<configure html32 fontmath\><<<
\Configure{cdots}                 {\pic:sym{cdots}}
>>>


\<configure html32 amsmath\><<<
\Configure{@cdots}                {\pic:sym{@cdots}}
\Configure{iint}                  {\pic:sym{iint}}
\Configure{iiint}                 {\pic:sym{iiint}}
\Configure{iiiint}                {\pic:sym{iiint}}
\Configure{idotsint}              {\pic:sym{tsint}}
\Configure{doteq}                 {\pic:sym{tsint}}
>>>







\section{amsmath}

\<configure html32-math amsmath\><<<
|<32,4 picmath amsmath,amstex1|>
|<32,4 picmath amsmath|>
>>>



\<32,4 picmath amsmath,amstex1\><<<
|<32,4 picmath: plain, fontmath, amsmath, amstex1|>
\Configure{dotsc}                 {\pic:sym{dotsc}}
\Configure{dotso}                 {\pic:sym{dotso}}
>>>




\<32,4 picmath amsmath\><<<
\Configure{overset} {\Picture+{ \a:@Picture{}}} {\EndPicture}
\Configure{underset} {\Picture+{ \a:@Picture{}}} {\EndPicture}
>>>




\<32,4 picmath amsmath\><<<
\Configure{xrightarrow} {\Picture+{ \a:@Picture{}}} {\EndPicture}
\Configure{xleftarrow} {\Picture+{ \a:@Picture{}}} {\EndPicture}
>>>



\section{amstex1 (amstex.sty)}

\<configure html32-math amstex1\><<<
|<32,4 picmath amsmath,amstex1|>
>>>




\section{th4}




\<configure html32-math th4\><<<
|<32 picmath th4,latex|>
>>>












  













  



  



  







  

     

\<temp hcode accents\><<<
\HCode{&\expandafter \ifx\csname U#2#1\endcsname\relax
                 #2#1\else \#x\csname U#2#1\endcsname\fi;}%
>>>


\<xmlns\><<<
xmlns="http://www.w3.org/1999/xhtml"
>>>















\subsection{TeX Engine}







The \verb'\trap:base' is to catch empty bases of exponents like, e.g.,
in \verb'$a^{^b}$'.











\<?\><<<
\def\MathRow#1{%
   \Configure{\expandafter\:gobble\string#1*}{*}%
      {<|.mrow\Hnewline 
         class="\expandafter\:gobble\string#1">}{</|.mrow>}%
      {\Configure{\expandafter\:gobble\string#1}{}{}{}{}}#1}%
>>>


\<recall dvimath par\><<<
\sv:ignore
>>>



\<sv dvimath par\><<<
\edef\sv:ignore{\if:nopar  
    \noexpand\IgnorePar\else \noexpand\ShowPar\fi}%
>>>




The \verb'\MathRow' requests a \verb'<|.mrow\Hnewline>...</|.mrow>', instead of the contributions
of \verb'\mathop', \verb'\mathrel',...., for the next parameter.

















\subsection{latex.ltx}


















Definitions like \verb'\def\mathbf#1{\a:mathbf#1\b:mathbf}'
can't be done on a global level, because \verb'\mathbf' is just
a name of a font. So, for instance, \verb'\bf' expands to \verb'\mathbf',
and so  \verb'$\bf R$' indirectly brings up the latter command.





\subsection{plain.sty}







\subsection{Palin + LaTeX}

The default \verb'\left' and \verb'\right' in their default definition
with tex produce multi-part delimiters, from cmex, on large
subformulas. Hence, the `'.' below is needed.













\subsection{Amsmath}


 The \verb'\HCode{}' in \verb'\sideset' is for catching superscripts and subscripts






\section{Eqnarray}





Had `BASELINE' before `MIDDLE', but changed to conform with math
in page 252-- in intro to theory book.


















\section{Accents through `accents' Configurations}





Why originally the accents are defined within a group? (knuth answer
this in the texbook.)







\section{PsTricks}




\<configure html32 pstricks\><<<
\Configure{pspicture}
   {\ifvmode \ifinner\else \vfill\break\fi
    \leavevmode\fi \Picture+{ class="pspicture"}}
   {\EndPicture\HCode{<!--width="\the\wd\pst@hbox"  
         height="\the\ht\pst@hbox"-->}}
>>>








\section{Fractions}











\<config mathml amstex1\><<<
\ConfigureEnv{aligned}{}{}{}{}
\Configure{aligned}
   {\HCode{<|.mtable\Hnewline class="aligned">}}
   {\HCode{</|.mtable>}}
   {\HCode{<|.mtr\Hnewline>}}   {\HCode{</|.mtr>}}
   {\HCode{<|.mtd>}}   {\HCode{</|.mtd>}}
>>>


















>











%%%%%%%%%%%%%%%%%%%%%%%%%%%%%%%%%%%%%%%%%%%%%%%%%%%%%%%%%%%%%%%%
\chapter{Sty Files}
%%%%%%%%%%%%%%%%%%%%%%%%%%%%%%%%%%%%%%%%%%%%%%%%%%%%%%%%%%%%%%%%


%%%%%%%%%%%%%%%%%%%
\section{ProTex}
%%%%%%%%%%%%%%%%%%%

\<configure html32 ProTex\><<<
\:CheckProtexOption{[[]]}\if:Option
   |<frame protex code|>
\fi
>>>

\<frame protex code\><<<
\Configure{FrameCode}
   {\ifvmode \IgnorePar\fi |<lynx separator|>\EndP
        \HCode{<div class="ShowCode">\ifx \:test\:minus  
                 \else<div class="head">\fi}\par\IgnorePar}
   {\ifvmode \IgnorePar\fi \EndP
        \HCode{</div></div>}}
\Configure{ShowCode}
   {\HCode{\ifx \:test\:minus  \else </div>\fi
       \html:src<div class="body"><span class="ShowCode"
             |<no wrap|> 
     >\html:invisible}%
     \nobreak
   }
   {\special{t4ht=\html:src</span>%
           \ifx \:test\:minus \else
              </div><div class="tail">\html:BackTitle\fi}}
   {\HCode{<br\xml:empty>\html:invisible}}
   {\HCode{<i>}}
   {\HCode{</i>}}
   {\HCode{\string&nbsp;}}
\Css{div.ShowCode{background:\#EEEEEE; border: 1px white solid;}} 
\Css{div.ShowCode div.head{background:\#E0E0E0;}} 
\Css{div.ShowCode div.tail{background:\#E0E0E0;}} 
>>>


\<lynx separator\><<<
\ifx \par:end\empty \HCode{<p></p>}\fi
>>>




%%%%%%%%%%%%%%%%%%%%%%%%%%%%%%%%%%%%%%%%%%%%%%%%%%%%%%%%%%%%%%%%
\chapter{Shared}
%%%%%%%%%%%%%%%%%%%%%%%%%%%%%%%%%%%%%%%%%%%%%%%%%%%%%%%%%%%%%%%%


\<par del\><<<
!*?: >>>


\<tag of Tag\><<<
 cw:>>>

\<tail\><<<
tail>>>

\<addr for Tag and Ref of Sec\><<<
\xdef\:cursec{|<section html addr|>}%
>>>






\<redefine Configure\><<<
\let\:tempd|=\Configure
\def\Configure#1#2{%
   \:CheckOption{#1}\if:Option \def\:tempc{#2}\fi}
>>>

\<recall Configure\><<<
\let\Configure|=\:tempd
>>>


\<user's configuration files\><<<
\openin15=tex4ht.usr \ifeof15 \else \closein15 
   \input tex4ht.usr
\fi
>>>







\<save catcodes\><<<
\expandafter\edef\csname :RestoreCatcodes\endcsname{%
   \expandafter\ifx \csname :RestoreCatcodes\endcsname\relax\else
      \csname :RestoreCatcodes\endcsname \fi
   \catcode`\noexpand :|=\the\catcode`:%
   \ifnum \the\catcode`\#=6 \else
      \catcode`\noexpand \#|=\the\catcode`\#\fi
   \let\expandafter\noexpand\csname :RestoreCatcodes\endcsname|=
                                   \noexpand\UnDefcS}
\catcode`\:|=11  \catcode`\#|=6 
>>>

%%%%%%%%%%%%%%%%%%%%%%%%%%%%%%%%%%%%%
\begin{thebibliography}{9}
%%%%%%%%%%%%%%%%%%%%%%%%%%%%%%%%%%%%%
\bibitem{val}
\url{http://htmlhelp.com/tools/validator/index.html.en}
\end{thebibliography}



\endinput
