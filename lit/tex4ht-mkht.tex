% $Id$
% (1) latex tex4ht-mkht      
% (2) perl -c mk4ht.perl                (checking correctness)
% (3) latex mkht-scripts.4ht
%
% Copyright 2009-2023 TeX Users Group
% Copyright 1997-2009 Eitan M. Gurari
% Released under LPPL 1.3c+.
% See tex4ht-cpright.tex for license text.

\documentclass{article}
   \usepackage{url}
\def\CodeDel{{<<<-}{->>>}}
\ifx \HCode\UnDef
   \input tex4ht.sty
   \Preamble{xhtml}
   \input ProTex.sty     
   \AlProTex{4ht,<<<>>>,?,title,list}
   \AtBeginDocument{\EndPreamble}
\else
   \Configure{ProTex}{4ht,<<<>>>,?,title,list,[[]]}
\fi

\begin{document}

%%%%%%%%%%%%%%%%%%%%% definitions %%%%%%%%%%%%%%%%%%%%%%%%%%%%%

% $Id$
% A few common TeX definitions for literate sources.  Not installed in runtime.
% 
% Copyright 2009-2017 TeX Users Group
% Copyright 1996-2009 Eitan M. Gurari
%
% This work may be distributed and/or modified under the
% conditions of the LaTeX Project Public License, either
% version 1.3c of this license or (at your option) any
% later version. The latest version of this license is in
%   http://www.latex-project.org/lppl.txt
% and version 1.3c or later is part of all distributions
% of LaTeX version 2005/12/01 or later.
%
% This work has the LPPL maintenance status "maintained".
%
% The Current Maintainer of this work
% is the TeX4ht Project <http://tug.org/tex4ht>.
% 
% If you modify this program, changing the 
% version identification would be appreciated.

\newcount\tmpcnt  \tmpcnt\time  \divide\tmpcnt  60
\edef\temp{\the\tmpcnt}
\multiply\tmpcnt  -60 \advance\tmpcnt  \time

\edef\version{\the\year-\ifnum \month<10 0\fi
  \the\month-\ifnum \day<10 0\fi\the\day
   -\ifnum \temp<10 0\fi \temp
   :\ifnum \tmpcnt<10 0\fi\the\tmpcnt}

% a fixed-string version that can be enabled for debugging.
%\edef\versionDebug{000-00-00-00:00}
%\let\version\versionDebug

% #1 is the first year for Eitan.  The last year is always 2009.  RIP.
\def\CopyYear.#1.{#1-2009}

% command for write to terminal and the log file
% this version is used in the .4ht files build
% identical command is defined also in tex4ht-sty.tex, 
% it is used in TeX document compilation
\def\writesixteen#1{\immediate\write1616{#1}}


\def\mkhtNote{

}
\newtoks\ourtoks

\expandafter\ourtoks\expandafter{\mkhtNote}

%%%%%%%%%%%%%%%%%%%%% end definitions %%%%%%%%%%%%%%%%%%%%%%%%%



%%%%%%%%%%%%%%%%%%
\section{TeX Dialects}
%%%%%%%%%%%%%%%%%%



%%%%%%%%%%%%%%%%%%%%%%%%%%%%%%%%%%%%%%%%%%%%%%%%%%%%%%%%%%%%%%%%%%%%%%%%
\subsection{\LaTeX}
%%%%%%%%%%%%%%%%%%%%%%%%%%%%%%%%%%%%%%%%%%%%%%%%%%%%%%%%%%%%%%%%%%%%%%%%



\<def script\><<<-
%%%%%%%%%%%%%%%%%%%%%%%%%%%%%%%%%
\section{LaTeX}
%%%%%%%%%%%%%%%%%%%%%%%%%%%%%%%%%


\<htlatex ...\><<<
|<unix line|>|<windows line|>
        |<latex|>
        |<latex|>
        |<latex|>
        |<base|>tex4ht |<dir ch|>|<@|>1  |<III|>
        |<base|>t4ht |<dir ch|>|<@|>1 |<IV|>

>>>


\<latex\><<<
latex |<@|>5 |<'|>|<definitions|>|<get parameters|>|empty>>>

\<get parameters\><<<
\HCode |<'|>|<@|>2|<'|>.a.b.c.\input |<'|> |<@|>1>>>


\<definitions\><<<
\makeatletter|<get 2nd arg|>|<2e + 209|>\makeatother>>>

\<get 2nd arg\><<<
\def\HCode{\futurelet\HCode\HChar}\def\HChar{|<double quotes??|>}>>>

\<double quotes??\><<<
\ifx"\HCode|<remove qts|>\expandafter\HCode\else\expandafter\Link\fi>>>

\<remove qts\><<<
\def\HCode"##1"{\Link##1}>>>


\<2e + 209\><<<
\def\Link#1.a.b.c.{|<2e|>|<209|>}>>>

\<2e\><<<
\g@addto@macro\@documentclasshook{|<require package|>}>>>

\<209\><<<
|<save|>\def\documentstyle{|<recall|>|<options|>|<209+tex4ht|>}>>>

\<require package\><<<
\RequirePackage[#1|<II|>|<,html|>]{tex4ht}>>>


\<options\><<<
|<def tex4ht|>{#1|<II|>|<,html|>}>>>


\<209+tex4ht\><<<
|<209+tex4ht+...|>\@ifnextchar[{\HCode}{\documentstyle[tex4ht]}>>>

\<209+tex4ht+...\><<<
\def\HCode####1{\documentstyle[tex4ht,}>>>


\<save\><<<
\let\HCode\documentstyle>>>

\<recall\><<<
\let\documentstyle\HCode>>>



\<htlatex 2e...\><<<
|<unix line|>|<windows line|>|<latex2e|>
        |<latex2e|>
        |<latex2e|>
        |<base|>tex4ht |<dir ch|>|<@|>1  |<III|>
        |<base|>t4ht |<dir ch|>|<@|>1 |<IV|>

>>>




\<latex2e\><<<
latex |<@|>5 |<'|>|<definitions2e|>|<get parameters|>|empty>>>

\<definitions2e\><<<
\makeatletter|<get 2nd arg|>|<2e only|>\makeatother>>>


\<2e only\><<<
\def\Link#1.a.b.c.{|<2e|>}>>>





\<htlatex 209...\><<<
|<unix line|>|<windows line|>|<latex209|>
        |<latex209|>
        |<latex209|>
        |<base|>tex4ht |<dir ch|>|<@|>1  |<III|>
        |<base|>t4ht |<dir ch|>|<@|>1 |<IV|>

>>>




\<latex209\><<<
latex  |<@|>5 |<'|>|<definitions209|>|<get parameters|>|empty>>>

\<definitions209\><<<
\makeatletter|<get 2nd arg|>|<209 only|>\makeatother>>>


\<209 only\><<<
\def\Link#1.a.b.c.{|<209|>}>>>

->>>


\<add script\><<<-
\OutputCodE\<\pref latex\latex.\ext\>   \script{\pref latex}{\ext}  %
\expandafter\let\csname\pref latex\latex\endcsname=\UnDef
%
->>>


%%%%%%%%%%%%%%%%%%%%%%%%%%%%%%%%%%%%%%%%%%%%%%%%%%%%%%%%%%%%%%%%%%%%%%%%
\subsection{\TeX}
%%%%%%%%%%%%%%%%%%%%%%%%%%%%%%%%%%%%%%%%%%%%%%%%%%%%%%%%%%%%%%%%%%%%%%%%



\<def script\><<<-

%%%%%%%%%%%%%%%%%%%%%%%%%%%%%%%%%
\section{TeX}
%%%%%%%%%%%%%%%%%%%%%%%%%%%%%%%%%


\<httex...\><<<
|<unix line|>|<windows line|>|<tex|>
        |<tex|>
        |<tex|>
        |<base|>tex4ht |<dir ch|>|<@|>1 |<III|>
        |<base|>t4ht |<dir ch|>|<@|>1 |<IV|>

>>>

\<tex\><<<
etex  |<@|>5 |<'|>|<tex definitions|>|<get parameters|>|empty>>>


\<tex definitions\><<<
|<set hook|>|<get 2nd arg|>|empty>>>

\<set hook\><<<
\def\Link#1.a.b.c.{|<def tex4ht|>{|<options|>\input tex4ht.sty }}>>>


\<def tex4ht\><<<
\expandafter\def\csname tex4ht\endcsname>>>

->>>


\<add script\><<<-
\<\pref tex.\ext\><<<
\<httex...\>
>>> %
%
\OutputCodE\<\pref tex.\ext\>    \script{\pref tex}{\ext}  %
%
->>>

%%%%%%%%%%%%%%%%%%%%%%%%%%%%%%%%%%%%%%%%%%%%%%%%%%%%%%%%%%%%%%%%%%%%%%%%
\subsection{Texinfo}
%%%%%%%%%%%%%%%%%%%%%%%%%%%%%%%%%%%%%%%%%%%%%%%%%%%%%%%%%%%%%%%%%%%%%%%%


\<def script\><<<-
%%%%%%%%%%%%%%%%%%%%
\section{Texinfo}
%%%%%%%%%%%%%%%%%%%%

The loading of tex4ht takes place at the @rm command at the end of the
texinfo.tex file.

\<httexi...\><<<
|<unix line|>|<windows line|>|<texinfo|>
        |<texinfo|>
        |<texinfo|>
        |<base|>tex4ht |<dir ch|>|<@|>1 |<III|>
        |<base|>t4ht |<dir ch|>|<@|>1 |<IV|>
|<Rem|>        texindex ??

>>>

\<texinfo\><<<
etex |<@|>5 |<'|>|<texinfo definitions|>|<get parameters|>|empty>>>


\<texinfo definitions\><<<
|<tex definitions|>\let\svrm=\rm\def\rm{\svrm|<tex4ht into texinfo|>}>>>

\<tex4ht into texinfo\><<<
\ifx\c\comment\def\rm{\let\rm=\svrm|<load into texinfo|>}\expandafter\rm\fi>>>

\<load into texinfo\><<<
|<change cats|>\csname tex4ht\endcsname|<restore cats|>|empty>>>

\<change cats\><<<
\catcode`\@=12\catcode`\\=0 >>>

\<restore cats\><<<
\catcode`\@=0\catcode`\\=13 >>>
->>>


\<add script\><<<-
\<\pref texi.\ext\><<<
\<httexi...\>
>>> %
%
\OutputCodE\<\pref texi.\ext\>   \script{\pref texi}{\ext}  %
%
->>>


%%%%%%%%%%%%%%%%%%%%%%%%%%%%%%%%%%%%%%%%%%%%%%%%%%%%%%%%%%%%%%%%%%%%%%%%
\subsection{XeTeX}
%%%%%%%%%%%%%%%%%%%%%%%%%%%%%%%%%%%%%%%%%%%%%%%%%%%%%%%%%%%%%%%%%%%%%%%%

\<def script\><<<-

%%%%%%%%%%%%%%%%%%%%%%%%%%%%%%%%%
\section{XeTeX}
%%%%%%%%%%%%%%%%%%%%%%%%%%%%%%%%%


\<htxetex...\><<<
|<unix line|>|<windows line|>|<xetex|>
        |<xetex|>
        |<xetex|>
        |<base|>tex4ht -.xdv |<dir ch|>|<@|>1 |<III|>
        |<base|>t4ht -.xdv |<dir ch|>|<@|>1 |<IV|>

>>>

\<xetex\><<<
xetex -no-pdf |<@|>5 |<'|>|<tex definitions|>|<get parameters|>|empty>>>
->>>


\<add script\><<<-
\<\pref xetex.\ext\><<<
\<htxetex...\>
>>> %
%
\OutputCodE\<\pref xetex.\ext\>    \script{\pref xetex}{\ext}  %
%
->>>



\<def script\><<<-
%%%%%%%%%%%%%%%%%%%%%%%%%%%%%%%%%
\section{XeLaTeX}
%%%%%%%%%%%%%%%%%%%%%%%%%%%%%%%%%


\<htxelatex...\><<<
|<unix line|>|<windows line|>|<xelatex|>
        |<xelatex|>
        |<xelatex|>
        |<base|>tex4ht -.xdv |<dir ch|>|<@|>1  |<III|>
        |<base|>t4ht -.xdv |<dir ch|>|<@|>1 |<IV|>

>>>


\<xelatex\><<<
xelatex -no-pdf |<@|>5 |<'|>|<definitions|>|<get parameters|>|empty>>>
->>>

\<add script\><<<-
\<\pref xelatex.\ext\><<<
\<htxelatex...\>
>>> %
%
\OutputCodE\<\pref xelatex.\ext\>    \script{\pref xelatex}{\ext}  %
%
->>>


%%%%%%%%%%%%%%%%%%%%%%%%%%%%%%%%%%%%%%%%%%%%%%%%%%%%%%%%%%%%%%%%%%%%%%%%
\subsection{Mex}
%%%%%%%%%%%%%%%%%%%%%%%%%%%%%%%%%%%%%%%%%%%%%%%%%%%%%%%%%%%%%%%%%%%%%%%%


\<def script\><<<-

%%%%%%%%%%%%%%%%%%%%%%%%%%%%%%%%%
\section{MeX}
%%%%%%%%%%%%%%%%%%%%%%%%%%%%%%%%%

\<htmex...\><<<
|<unix line|>|<windows line|>|<mex|>
        |<mex|>
        |<mex|>
        |<base|>tex4ht |<dir ch|>|<@|>1 |<III|>
        |<base|>t4ht |<dir ch|>|<@|>1 |<IV|>

>>>

\<mex\><<<
mex |<@|>5 |<'|>|<tex definitions|>|<get parameters|>|empty>>>

->>>


\<add script\><<<-
\<\pref mex.\ext\><<<
\<htmex...\>
>>> %
%
\OutputCodE\<\pref mex.\ext\>    \script{\pref mex}{\ext}  %
%
->>>

% %%%%%%%%%%%%%%%%%%%%%%%%%%%%%%%%%%%%%%%%%%%%%%%%%%%%%%%%%%%%%%%%%%%%%%%%
% \subsection{Mex}
% %%%%%%%%%%%%%%%%%%%%%%%%%%%%%%%%%%%%%%%%%%%%%%%%%%%%%%%%%%%%%%%%%%%%%%%%
% 
% 
% \<def script\><<<-
% 
% %%%%%%%%%%%%%%%%%%%%%%%%%%%%%%%%%
% \subsection{MeX il2-pl}
% %%%%%%%%%%%%%%%%%%%%%%%%%%%%%%%%%
% 
% \<htmex-pl...\><<<
%         |<unix line|>|<windows line|>|<mex-pl|>
%         |<mex-pl|>
%         |<mex-pl|>
%         |<base|>tex4ht |<dir ch|>|<@|>1 |<III|>
%         |<base|>t4ht |<dir ch|>|<@|>1 |<IV|>
% 
% >>>
% 
% \<mex-pl\><<<
% mex-pl -translate-file=il2-pl |<'|>|<tex definitions|>|<get parameters|>|empty>>>
% 
% ->>>
% 
% 
% \<add script\><<<-
% \<\pref mex-pl.\ext\><<<
% \<htmex-pl...\>
% >>> %
% %
% \OutputCodE\<\pref mex-pl.\ext\>    \script{\pref mex-pl}{\ext}  %
% %
% ->>>



%%%%%%%%%%%%%%%%%%
\section{Markup Dialects}
%%%%%%%%%%%%%%%%%%

%%%%%%%%%%%%%
\subsection{HTML: ht}
%%%%%%%%%%%%%


WARNING: Don't include `-cvalidatehtml' in the htlatex script as it will
be required by every script called with mk4ht.  For instance, `mk4ht
dblatex file'.

\<unix scripts\><<<-
\Ii{,html}
\Iii{-i~/tex4ht.dir/texmf/tex4ht/ht-fonts/#1}
\Iv{#1   ## -d~/WWW/temp/ -m644  }

\make{ht}
->>>

\<ms scripts\><<<-
\Ii{,html}
\Iii{-i/tex4ht/ht-fonts/#1 
     -ewin32/tex4ht.env}
\Iv{#1 -ewin32/tex4ht.env  }

\make{ht}
->>>

The first line is for `ht ...  ...' scripts.

\<perl options\><<<-
 "",     "ht",        "",         "", "", "-cvalidatehtml",
 "ht",   "htlatex",   "latex",    "", "", "-cvalidatehtml",
 "ht",   "httex",     "tex",      "", "", "-cvalidatehtml",
 "ht",   "httexi",    "texi",     "", "", "-cvalidatehtml",
 "ht",   "htxetex",   "xetex",    "", "", "-cvalidatehtml",
 "ht",   "htxelatex", "xelatex",  "", "", "-cvalidatehtml",
->>>


\<unix scripts\><<<-
\<ht.unix\><<<
|<unix line|>|<windows line|>$1 $2 
        $1 $2 
        $1 $2 
        tex4ht $2 
        t4ht $2  $3 
>>>
\OutputCodE\<ht.unix\>     \script{ht}{unix}  %
->>>


\<ms scripts\><<<-
\<ht.bat\><<<
|<windows line|>
        %1 %2  
        %1 %2  
        %1 %2  
        tex4ht %2  
        t4ht %2  %3  
>>>
\OutputCodE\<ht.bat\>    \script{ht}{bat}  %
->>>


%%%%%%%%%%%%%
\subsection{XHTML: xh}
%%%%%%%%%%%%%

\<unix scripts\><<<-
\Ii{,xhtml}
\Iii{-i~/tex4ht.dir/texmf/tex4ht/ht-fonts/#1}
\Iv{#1  -cvalidate   ## -d~/WWW/temp/ -m644 }
\make{xh}
->>>

\<ms scripts\><<<-
\Ii{,xhtml}
\Iii{-i/tex4ht/ht-fonts/#1 
     -ewin32/tex4ht.env}
\Iv{#1 -ewin32/tex4ht.env -cvalidate }
\make{xh}
->>>



\<perl options\><<<-
 "xh", "xhlatex",  "latex",   "xhtml", "", "-cvalidate",
 "xh", "xhtex",    "tex",     "xhtml", "", "-cvalidate",
 "xh", "xhtexi",   "texi",    "xhtml", "", "-cvalidate",
 "xh", "xhxelatex", "xelatex", "xhtml", "", "-cvalidate",
 "xh", "xhxetex",  "xetex",   "xhtml", "", "-cvalidate",
->>>



%%%%%%%%%%%%%
\subsection{Unicode XHTML: uxh}
%%%%%%%%%%%%%

\<unix scripts\><<<-
\Ii{,xhtml,uni-html4}
\Iii{-i~/tex4ht.dir/texmf/tex4ht/ht-fonts/#1 -cunihtf}
\Iv{#1 -cvalidate  ## -d~/WWW/temp/ -m644 }
\make{uxh}
->>>

\<ms scripts\><<<-
\Ii{,xhtml,uni-html4}
\Iii{-i/tex4ht/ht-fonts/#1 
     -ewin32/tex4ht.env
      -cunihtf}
\Iv{#1 -ewin32/tex4ht.env -cvalidate }
\make{uxh}
->>>

\<perl options\><<<-
 "uxh", "uxhlatex",  "latex",   "xhtml,uni-html4", " -cunihtf", "-cvalidate",
 "uxh", "uxhtex",    "tex",     "xhtml,uni-html4", " -cunihtf", "-cvalidate",
 "uxh", "uxhtexi",   "texi",    "xhtml,uni-html4", " -cunihtf", "-cvalidate",
 "uxh", "uxhxelatex",  "xelatex",   "xhtml,uni-html4", " -cunihtf", "-cvalidate",
 "uxh", "uxhxetex",    "xetex",     "xhtml,uni-html4", " -cunihtf", "-cvalidate",
->>>

%%%%%%%%%%%%%
\subsection{XHTML+MathML: xhm}
%%%%%%%%%%%%%

\<unix scripts\><<<-
\Ii{,xhtml,mathml}
\Iii{-i~/tex4ht.dir/texmf/tex4ht/ht-fonts/#1 -cunihtf}
\Iv{#1 -cvalidate   ## -d~/WWW/temp/ -m644 }
\make{xhm}
->>>

\<ms scripts\><<<-
\Ii{,xhtml,mathml}
\Iii{-i/tex4ht/ht-fonts/#1
     -ewin32/tex4ht.env -cunihtf}
\Iv{#1 -ewin32/tex4ht.env -cvalidate }
\make{xhm}
->>>


\<perl options\><<<-
 "xhm", "xhmlatex", "latex",  "xhtml,mathml", " -cunihtf", "-cvalidate",
 "xhm", "xhmtex",   "tex",    "xhtml,mathml", " -cunihtf", "-cvalidate",
 "xhm", "xhmtexi",  "texi",   "xhtml,mathml", " -cunihtf", "-cvalidate",
 "xhm", "xhmxelatex", "xelatex",  "xhtml,mathml", " -cunihtf", "-cvalidate",
 "xhm", "xhmxetex",   "xetex",    "xhtml,mathml", " -cunihtf", "-cvalidate",
->>>


%%%%%%%%%%%%%
\subsection{Mozilla XHTML+MathML: mz}
%%%%%%%%%%%%%

\<unix scripts\><<<-
\Ii{,xhtml,mozilla}
\Iii{-i~/tex4ht.dir/texmf/tex4ht/ht-fonts/#1 -cmozhtf}
\Iv{#1 -cvalidate   ## -d~/WWW/temp/ -m644 }
\make{mz}
->>>

\<ms scripts\><<<-
\Ii{,xhtml,mozilla}
\Iii{-i/tex4ht/ht-fonts/#1
     -ewin32/tex4ht.env  -cmozhtf}
\Iv{#1 -ewin32/tex4ht.env -cvalidate }
\make{mz}
->>>


\<perl options\><<<-
 "mz", "mzlatex",   "latex",   "xhtml,mozilla", " -cmozhtf",  "-cvalidate",
 "mz", "mztex",     "tex",     "xhtml,mozilla", " -cmozhtf",  "-cvalidate",
 "mz", "mztexi",    "texi",    "xhtml,mozilla", " -cmozhtf",  "-cvalidate",
 "mz", "mzxelatex", "xelatex",   "xhtml,mozilla", " -cmozhtf",  "-cvalidate",
 "mz", "mzxetex",   "xetex",     "xhtml,mozilla", " -cmozhtf",  "-cvalidate",
->>>


%%%%%%%%%%%%%
\subsection{OpenOffice: oo}
%%%%%%%%%%%%%

\<unix scripts\><<<-
\Ii{,xhtml,ooffice}
\Iii{-i~/tex4ht.dir/texmf/tex4ht/ht-fonts/#1 -cmozhtf}
\Iv{#1 -cooxtpipes -coo }
\make{oo}
->>>

\<ms scripts\><<<-
\Ii{,xhtml,ooffice}
\Iii{-i/tex4ht/ht-fonts/#1
     -ewin32/tex4ht.env -cmozhtf}
\Iv{#1 -cooxtpipes -coo -ewin32/tex4ht.env -cvalidate }
\make{oo}
->>>


\<perl options\><<<-
 "oo", "oolatex",   "latex",   "xhtml,ooffice", "ooffice/\! -cmozhtf",  "-cooxtpipes -coo",
 "oo", "ootex",     "tex",     "xhtml,ooffice", "ooffice/\! -cmozhtf",  "-cooxtpipes -coo",
 "oo", "ootexi",    "texi",    "xhtml,ooffice", "ooffice/\! -cmozhtf",  "-cooxtpipes -coo",
 "oo", "ooxelatex",   "xelatex",   "xhtml,ooffice", "ooffice/\! -cmozhtf",  "-cooxtpipes -coo",
 "oo", "ooxetex",     "xetex",     "xhtml,ooffice", "ooffice/\! -cmozhtf",  "-cooxtpipes -coo",
->>>





%%%%%%%%%%%%%
\subsection{EmacSpeak: es}
%%%%%%%%%%%%%

\<unix scripts\><<<-
\Ii{,xhtml,emspk}
\Iii{-i~/tex4ht.dir/texmf/tex4ht/ht-fonts/#1 -cemspkhtf -s4es}
\Iv{#1 -cemspk   ## -d~/WWW/temp/ -m644 }
\make{es}
->>>

\<ms scripts\><<<-
\Ii{,xhtml,emspk}
\Iii{-itex4ht/ht-fonts/#1
     -ewin32/tex4ht.env  -cemspkhtf -s4es}
\Iv{#1 -ewin32/tex4ht.env -cemspk }
\make{es}
->>>


\<perl options\><<<-
 "es", "eslatex",   "latex",   "xhtml,emspk", " -cemspkhtf -s4es",  "-cemspk",
 "es", "estex",     "tex",     "xhtml,emspk", " -cemspkhtf -s4es",  "-cemspk",
 "es", "estexi",    "texi",    "xhtml,emspk", " -cemspkhtf -s4es",  "-cemspk",
 "es", "esxelatex",   "xelatex",   "xhtml,emspk", " -cemspkhtf -s4es",  "-cemspk",
 "es", "esxetex",     "xetex",     "xhtml,emspk", " -cemspkhtf -s4es",  "-cemspk",
->>>





%%%%%%%%%%%%%
\subsection{JSML: js}
%%%%%%%%%%%%%

\<unix scripts\><<<-
\Ii{,xhtml,jsml}
\Iii{-i~/tex4ht.dir/texmf/tex4ht/ht-fonts/#1 -cjsmlhtf }
\Iv{#1 -cjsml   ## -d~/WWW/temp/ -m644 }
\make{js}
->>>

\<ms scripts\><<<-
\Ii{,xhtml,jsml}
\Iii{-i/tex4ht/ht-fonts/#1
     -ewin32/tex4ht.env  -cjsmlhtf }
\Iv{#1 -ewin32/tex4ht.env -cjsml }
\make{js}
->>>


\<perl options\><<<-
 "js", "jslatex",   "latex",   "xhtml,jsml", " -cjsmlhtf",  "-cjsml",
 "js", "jstex",     "tex",     "xhtml,jsml", " -cjsmlhtf",  "-cjsml",
 "js", "jstexi",    "texi",    "xhtml,jsml", " -cjsmlhtf",  "-cjsml",
 "js", "jsxelatex",   "xelatex",   "xhtml,jsml", " -cjsmlhtf",  "-cjsml",
 "js", "jsxetex",     "xetex",     "xhtml,jsml", " -cjsmlhtf",  "-cjsml",
->>>





%%%%%%%%%%%%%
\subsection{jsMath: jm}
%%%%%%%%%%%%%

\<unix scripts\><<<-
\Ii{,xhtml,jsmath}
\Iii{-i~/tex4ht.dir/texmf/tex4ht/ht-fonts/#1 -cmozhtf }
\Iv{#1 ## -d~/WWW/temp/ -m644 }
\make{jm}
->>>

\<ms scripts\><<<-
\Ii{,xhtml,jsmath}
\Iii{-i/tex4ht/ht-fonts/#1
     -ewin32/tex4ht.env  -cmozhtf }
\Iv{#1 -ewin32/tex4ht.env  }
\make{jm}
->>>


\<perl options\><<<-
 "jm", "jmlatex",   "latex",   "xhtml,jsmath", " -cmozhtf", "",
 "jm", "jmtex",     "tex",     "xhtml,jsmath", " -cmozhtf", "",
 "jm", "jmtexi",    "texi",    "xhtml,jsmath", " -cmozhtf", "",
 "jm", "jmxelatex",   "xelatex",   "xhtml,jsmath", " -cmozhtf", "",
 "jm", "jmxetex",     "xetex",     "xhtml,jsmath", " -cmozhtf", "",
->>>








%%%%%%%%%%%%%
\subsection{TEI: tei, teim}
%%%%%%%%%%%%%

\<unix scripts\><<<-
\Ii{,xhtml,tei}
\Iii{-i~/tex4ht.dir/texmf/tex4ht/ht-fonts/#1 -cunihtf}
\Iv{#1 -cvalidate   ## -d~/WWW/temp/ -m644 }
\make{tei}

\Ii{,xhtml,tei-mml}
\Iii{-i~/tex4ht.dir/texmf/tex4ht/ht-fonts/#1 -cunihtf}
\Iv{#1 -cvalidate   ## -d~/WWW/temp/ -m644 }
\make{teim}
->>>

\<ms scripts\><<<-
\Ii{,xhtml,tei}
\Iii{-i/tex4ht/ht-fonts/#1
     -ewin32/tex4ht.env -cunihtf}
\Iv{#1 -ewin32/tex4ht.env -cvalidate }
\make{tei}

\Ii{,xhtml,tei-mml}
\Iii{-i/tex4ht/ht-fonts/#1 
     -ewin32/tex4ht.env -cunihtf}
\Iv{#1 -ewin32/tex4ht.env -cvalidate }
\make{teim}
->>>


\<perl options\><<<-
 "tei",  "teilatex",  "latex",   "xhtml,tei",    " -cunihtf",  "-cvalidate",
 "tei",  "teitex",    "tex",     "xhtml,tei",    " -cunihtf",  "-cvalidate",
 "tei",  "teitexi",   "texi",    "xhtml,tei",    " -cunihtf",  "-cvalidate",
 "teim", "teimlatex", "latex",   "xhtml,tei-mml"," -cunihtf",  "-cvalidate",
 "teim", "teimtex",   "tex",     "xhtml,tei-mml"," -cunihtf",  "-cvalidate",
 "teim", "teimtexi",  "texi",    "xhtml,tei-mml"," -cunihtf",  "-cvalidate",
 "tei",  "teixelatex",  "xelatex",   "xhtml,tei",    " -cunihtf",  "-cvalidate",
 "tei",  "teixetex",    "xetex",     "xhtml,tei",    " -cunihtf",  "-cvalidate",
->>>


%%%%%%%%%%%%%
\subsection{DocBook: db, dbm}
%%%%%%%%%%%%%

\<unix scripts\><<<-
\Ii{,xhtml,docbook}
\Iii{-i~/tex4ht.dir/texmf/tex4ht/ht-fonts/#1 -cunihtf}
\Iv{#1    -cvalidate -cdocbk  ## -d~/WWW/temp/ -m644  }
\make{db}

\Ii{,xhtml,docbook-mml}
\Iii{-i~/tex4ht.dir/texmf/tex4ht/ht-fonts/#1 -cunihtf}
\Iv{#1  -cdocbk  ## -d~/WWW/temp/ -m644 }
\make{dbm}
->>>

\<ms scripts\><<<-
\Ii{,xhtml,docbook}
\Iii{-i/tex4ht/ht-fonts/#1
     -ewin32/tex4ht.env -cunihtf}
\Iv{#1 -ewin32/tex4ht.env -cvalidate -cdocbk }
\make{db}

\Ii{,xhtml,docbook-mml}
\Iii{-i/tex4ht/ht-fonts/#1
     -ewin32/tex4ht.env -cunihtf}
\Iv{#1 -ewin32/tex4ht.env -cvalidate -cdocbk }
\make{dbm}
->>>



\<perl options\><<<-
 "db",  "dblatex",   "latex",   "xhtml,docbook",     " -cunihtf",  "-cvalidate -cdocbk",
 "db",  "dbtex",     "tex",     "xhtml,docbook",     " -cunihtf",  "-cvalidate -cdocbk",
 "db",  "dbtexi",    "texi",    "xhtml,docbook",     " -cunihtf",  "-cvalidate -cdocbk",
 "dbm", "dbmlatex",  "latex",   "xhtml,docbook-mml", " -cunihtf",  "-cdocbk",
 "dbm", "dbmtex",    "tex",     "xhtml,docbook-mml", " -cunihtf",  "-cdocbk",
 "dbm", "dbmtexi",   "texi",    "xhtml,docbook-mml", " -cunihtf",  "-cdocbk",
 "db",  "dbxelatex",   "xelatex",   "xhtml,docbook",     " -cunihtf",  "-cvalidate -cdocbk",
 "db",  "dbxetex",     "xetex",     "xhtml,docbook",     " -cunihtf",  "-cvalidate -cdocbk",
->>>



%%%%%%%%%%%%%
\subsection{MS Word HTML: w}
%%%%%%%%%%%%%


\<unix scripts\><<<-
\Ii{,xhtml,word}
\Iii{-i~/tex4ht.dir/texmf/tex4ht/ht-fonts/#1  -csymhtf}
\Iv{#1    -cvalidate ## -d~/WWW/temp/ -m644 }
\make{w}
->>>

\<ms scripts\><<<-
\Ii{,xhtml,word}
\Iii{-i/tex4ht/ht-fonts/#1 
     -ewin32/tex4ht.env  -csymhtf}
\Iv{#1 -ewin32/tex4ht.env -cvalidate }
\make{w}
->>>




\<perl options\><<<-
 "w", "wlatex",   "latex",   "xhtml,word", " -csymhtf", "",
 "w", "wtex",     "tex",     "xhtml,word", " -csymhtf", "",
 "w", "wtexi",    "texi",    "xhtml,word", " -csymhtf", "",
 "w", "wxelatex",   "xelatex",   "xhtml,word", " -csymhtf", "",
 "w", "wxetex",     "xetex",     "xhtml,word", " -csymhtf", "",
->>>




%%%%%%%%%%%%%
\subsection{JavaHelp: jh}
%%%%%%%%%%%%%

\<unix scripts\><<<-
\Ii{,html,javahelp,xml,3.2,unicode}
\Iii{-i~/tex4ht.dir/texmf/tex4ht/ht-fonts/#1 -cmozhtf -u10}
\Iv{#1 -d$1-doc/   -cjavahelp -cvalidatehtml  -m644}
\make{jh}
->>>

\<ms scripts\><<<-
\Ii{,html,javahelp,xml,3.2,unicode}
\Iii{-i/tex4ht/ht-fonts/#1 
     -ewin32/tex4ht.env
      -cmozhtf -u10}
\Iv{#1 -ewin32/tex4ht.env -d%1-doc\ -cjavahelp }
\make{jh}
->>>


\<perl options\><<<-
 "jh", "jhlatex",  "latex",  "html,javahelp,xml,3.2,unicode", " -cmozhtf -u10", " -d$texFile-doc/ -cjavahelp -cvalidatehtml",
 "jh", "jhtex",    "tex",    "html,javahelp,xml,3.2,unicode", " -cmozhtf -u10", " -d$texFile-doc/ -cjavahelp -cvalidatehtml",
 "jh", "jhtexi",   "texi",   "html,javahelp,xml,3.2,unicode", " -cmozhtf -u10", " -d$texFile-doc/ -cjavahelp -cvalidatehtml",
 "jh", "jhxelatex",  "xelatex",  "html,javahelp,xml,3.2,unicode", " -cmozhtf -u10", " -d$texFile-doc/ -cjavahelp -cvalidatehtml",
 "jh", "jhxetex",    "xetex",    "html,javahelp,xml,3.2,unicode", " -cmozhtf -u10", " -d$texFile-doc/ -cjavahelp -cvalidatehtml",
->>>


\<unix scripts\><<<-
\Ii{,html,javahelp,xml,3.2,unicode,jh1.0}
\Iii{-i~/tex4ht.dir/texmf/tex4ht/ht-fonts/#1 -cmozhtf -u10}
\Iv{#1 -d$texFile-doc/  -cjavahelp1 -m644  }
\make{jh1}
->>>

\<ms scripts\><<<-
\Ii{,html,javahelp,xml,3.2,unicode,jh1.0}
\Iii{-i/tex4ht/ht-fonts/#1 
     -ewin32/tex4ht.env
      -cmozhtf -u10}
\Iv{#1 -ewin32/tex4ht.env -d%1-doc\ -cjavahelp1 }
\make{jh1}
->>>


\<perl options\><<<-
 "jh1", "jh1latex",  "latex",  "html,javahelp,xml,3.2,unicode,jh1.0", " -cmozhtf -u10", " -d$texFile-doc/ -cjavahelp",
 "jh1", "jh1tex",    "tex",    "html,javahelp,xml,3.2,unicode,jh1.0", " -cmozhtf -u10", " -d$texFile-doc/ -cjavahelp",
 "jh1", "jh1texi",   "texi",   "html,javahelp,xml,3.2,unicode,jh1.0", " -cmozhtf -u10", " -d$texFile-doc/ -cjavahelp",
 "jh1", "jh1xelatex",  "xelatex",  "html,javahelp,xml,3.2,unicode,jh1.0", " -cmozhtf -u10", " -d$texFile-doc/ -cjavahelp",
 "jh1", "jh1xetex",    "xetex",    "html,javahelp,xml,3.2,unicode,jh1.0", " -cmozhtf -u10", " -d$texFile-doc/ -cjavahelp",
->>>



The xml declaration at
\url{http://java.sun.com/products/javahelp/toc_1_0.dtd}
doesn't start at row/col=1 make it impossible to validate
against that file.


%%%%%%%%%%%%%%%%%%%%%%%%%%%%%%%%%%%%%%%%%%%%%%%%%%%%%%%%%%%%%%%%%%%%%%%%
\section{mkht.4ht}
%%%%%%%%%%%%%%%%%%%%%%%%%%%%%%%%%%%%%%%%%%%%%%%%%%%%%%%%%%%%%%%%%%%%%%%%

\<mkht\><<<-
% mkht.4ht (?version), generated from ?jobname.tex
% Copyright 2009-2023 TeX Users Group
% Copyright ?CopyYear.1997. Eitan M. Gurari
?<TeX4ht copyright?>
\immediate\write-1{version ?version}

\def\exit{\documentclass{article}\begin{document}\end{document}\endinput}
\bgroup
  \def\missing#1{\aftergroup\exit
     \immediate\write16{---------------------------- error 
        ---------------------------- 
        ^^JRequires #1ProTex.sty from 
        https://tug.org/tex4ht^^J%
       ---------------------------------------------------------------}}
  \openin15=ProTex.sty \ifeof15 \missing{}\else \closein15 \fi
  \openin15=AlProTex.sty \ifeof15 \missing{Al}\else \closein15 \fi
\egroup

\let\ScriptFileName\relax
\let\AddExtn\relax

\documentclass{article}

\ifx \HCode\UnDef
  \input tex4ht.sty
  \Preamble{xhtml}
  \input ProTex.sty
  \AlProTex{foo,<<<>>>,|,title,list}
  \begin{document}
  \EndPreamble
\else
  \Configure{ProTex}{foo,<<<>>>,|,title,list,[[]]}
  \begin{document}
\fi

\catcode`\:=11
\def\OutputCodE\<#1.#2\>{{\xdef\ScriptFileName{#1}
   \:DoName\def{#1}{\<#1.#2\>}
   \OutputCode[#2]\<#1\>}}
\def\winextn{bat}
\def\AddExtn{\ifx\ext\winextn .bat\fi}

\tableofcontents

?<def script?>

%%%%%%%%%%%%%%%%%%%%%%
\section{Script Components}
%%%%%%%%%%%%%%%%%%%%%%

\def\setcats{\catcode`\\=12 \catcode`\%=12 \catcode`\~=12
 \catcode`\_=12
}

\def\Remark{\bgroup \catcode`\#=12 \setcats \Rema}  \def\Rema#1{\egroup\def\Rem{#1}}
\def\Ii{\def\II}
\def\Iii{\bgroup \setcats \Iiicont}  \def\Iiicont#1{\egroup\def\III.##1.{#1}}
\def\Iv{\bgroup \setcats \Ivcont}    \def\Ivcont#1{\egroup\def\IV.##1.{#1}}
\def\Quote{\bgroup \setcats \Qcont}  \def\Qcont#1{\egroup\def\quote{#1}}
\def\Argchar{\bgroup \setcats \Acont}\def\Acont#1{\egroup\def\argchar{#1}}
\def\Dirchar{\bgroup \setcats \Dicont}\def\Dicont#1{\egroup\def\dirchar{#1}}
\def\Echooff{\bgroup \setcats \Dcont}\def\Dcont#1{\egroup\def\echoOff{#1}}
\def\Ext{\bgroup \setcats \Econt}    \def\Econt#1{\egroup\def\ext{#1}%
  ?<unix shebang line?>}
\def\Base{\bgroup \setcats \Bcont}   \def\Bcont#1{\egroup\def\base{#1}}
\def\Script{\def\script##1##2}  % #1 -- file    #2 -- extension   
\def\Options#1{\ifx \OPTIONS\UnDef \def\OPTIONS{#1}\fi}
\def\Latex#1{\ifx \latex\UnDef \def\latex{#1}\fi}

\let\Rem=\relax       
\let\II=\relax       
\let\III=\relax
\let\IV=\relax
\let\quote=\relax
\let\argchar=\relax
\let\dirchar=\relax
\let\echoOff=\relax
\let\ext=\relax
\ifx \script\UnDef \let\script=\relax \fi
\let\base=\relax
\let\firstln=\relax

\<Rem\><<<
|Rem>>>

\<II\><<<
|II>>>

\<III\><<<
|III.|<@|>3.>>>

\<IV\><<<
|IV.|<@|>4.>>>

\<'\><<<
|quote>>>

\<@\><<<
|argchar>>>

\<dir ch\><<<
|dirchar>>>

\<base\><<<
|base>>>

\<echo off\><<<
|echoOff>>>

\<unix line\><<<
|firstln#!/bin/sh
# stop at first error
set -e

# No interaction on the TeX runs is desirable.
# Simpler to do that here than on the individual commands;
# let's hope exec redirections are portable enough.
exec </dev/null
endfirstln
>>>

Despite the name, this macro is not for Windows. It is unconditionally
included in all the scripts. It should only be the copyright notice,
which somehow gets transformed for Unix. Thus we go to lots of trouble
to insert the at-echo off only for Windows.

\<windows line\><<<
|<echo off|>
?<MYcopyrightnotice?>
>>>


\def\IfExt#1,#2//{\def\next{#1}%
   \ifx \next\ext \def\next{\csname iftrue\endcsname}%
   \else \ifx \next\empty \def\next{\csname iffalse\endcsname}%
   \else \def\next{\IfExt#2,,//}\fi\fi \next}

\begingroup
\catcode`\:=11 \catcode`\^=7    \catcode`\^^M=13%
%
\gdef\make#1{%
  \expandafter\IfExt\OPTIONS,,//%
  %
   \subsection{#1}%
   \def\temp##1{\par{\tt\string##1: \meaning##1}}%
   \temp\quote
   \temp\argchar
   \temp\dirchar
   \temp\echoOff
   \temp\ext
   \temp\script
   \temp\base
   \temp\firstln
   \temp\latex
   \temp\Rem
   \temp\II
   \temp\III
   \temp\IV
   \def\pref{#1} %
   \<\pref latex\latex.\ext\><<<
   \<htlatex \latex...\>
   >>> %
   ?<add script?>%
  \fi
}                                      %
\endgroup

\ifx \JOBNAME\UnDefined
   ?<user's made scripts?>
   \expandafter\endinput
\fi

\input \JOBNAME

\end{document}
\endinput
->>>

\<user's made scripts\><<<-
\def\one{\Ii}
\def\two{\Iii}
\def\three{\Iv}
\Latex{}             % {}, {2e}, {209}
\def\temp{unix}\ifx\script\temp
   \Options{unix}   % {unix,bat}, {unix}, {bat}
   ?<unix setup?>
\else
   \def\temp{bat}\ifx\script\temp
      \Options{bat}
      ?<ms setup?>
   \else
      \bgroup
         \def\1{\space\space\space}
         \def\2{\space\space\space\space\space\space}
         \catcode`\%=12
         \catcode`\#=12
         \catcode`\~=12
         \catcode`\@=0
         \catcode`\\=12
      @immediate@write16{|-------------------------------------------------------------}
      @immediate@write16{|@2@1 A sample of a scripts generator file }
      @immediate@write16{|}
      @immediate@write16{|@1    Creates a foolatex script for commands}
      @immediate@write16{|}
      @immediate@write16{|@2        foolatex file}
      @immediate@write16{|}
      @immediate@write16{|@1    similar to htlatex for commands}
      @immediate@write16{|}
      @immediate@write16{|@2        htlatex file "html,...1..." " ...2..." "...3..."}
      @immediate@write16{|}
      @immediate@write16{|@1    with the arguments ...1..., ...2..., and ...3... }
      @immediate@write16{|@1    embedded in the new script. General usage}
      @immediate@write16{|}
      @immediate@write16{|@2        foolatex file "..." "..." "..."}
      @immediate@write16{|}
      @immediate@write16{|----------------- sample.tex --------------------------------}
      @immediate@write16{|% latex sample}
      @immediate@write16{|}
      @immediate@write16{| \def\script{bat} }
      @immediate@write16{| % \def\script{unix} }
      @immediate@write16{|}
      @immediate@write16{| \input ./mkht.4ht }
      @immediate@write16{|}
      @immediate@write16{| \one{,html,...1...}}
      @immediate@write16{| \two{-i/tex4ht/ht-fonts/#1 ...2...}    % bat }
      @immediate@write16{| % \two{-i~/tex4ht/texmf/tex4ht/ht-fonts/#1 ...2...}   % unix }
      @immediate@write16{| \three{#1 ...3...}}
      @immediate@write16{| \make{foo}}
      @immediate@write16{|}
      @immediate@write16{| \end{document} }
      @immediate@write16{-------------------------------------------------------------}
      @egroup
      \def\next{\csname fi\endcsname
                \csname fi\endcsname
                \csname fi\endcsname\end{document}}
\fi \fi
->>>

The script should have an explicit shebang (\#!/bin/sh) line for the
systems where csh-like shells are the foundation. If there is no
shebang line, the defaul shell (which can be anything) is used.


\<unix shebang line\><<<-
  \def\firstln{unix}\ifx \firstln\ext
           \def\firstln##1endfirstln{##1}%
  \else    \def\firstln##1endfirstln{}\fi
->>>





%%%%%%%%%%%%%%%%%%%%%%%%%%%%
\section{mkht-scripts.4ht}
%%%%%%%%%%%%%%%%%%%%%%%%%%%

\<mkht-scripts\><<<-
% mkht-scripts.4ht (?version), generated from ?jobname.tex
% Copyright 2009-2023 TeX Users Group
% Copyright ?CopyYear.2000. Eitan M. Gurari
?<TeX4ht copyright?>

\ifx \JOBNAME\UnDef
   \def\JOBNAME{mkht-scripts.4ht}
   \def\next{\input ./mkht.4ht  \endinput}
   \expandafter\next
\fi

\Options{unix,bat}   % {unix,bat}, {unix}, {bat}
\Latex{}             % {}, {2e}, {209}

%%%%%%%%%%%%%%%%%%
\section{Unix Scripts}
%%%%%%%%%%%%%%%%%%
?<unix setup?>
\ifOption{Needs}{%
\Script{%                        #1 -- file    #2 -- extension
   \Needs{"sed -e 's/\#\# -d/ -d/g' < #1.#2 >  /opt/cvr/gurari/tex4ht.dir/bin/solaris/#1"}%
   \Needs{"chmod 700 /opt/cvr/gurari/tex4ht.dir/bin/solaris/#1"}%
   \Needs{"mkdir -p mn.dir/ht-unix"}%
   \Needs{"cp #1.#2  mn.dir/ht-unix/#1"}%
}}{}
\Script{%                        #1 -- file    #2 -- extension
   \Needs{"mkdir -p /opt/cvr/gurari/tex4ht.dir/bin/ht"}%
   \Needs{"mkdir -p /opt/cvr/gurari/tex4ht.dir/bin/ht/unix"}%
   \Needs{"mv #1.#2  /opt/cvr/gurari/tex4ht.dir/bin/ht/unix/#1"}%
   \Needs{"chmod 744 /opt/cvr/gurari/tex4ht.dir/bin/ht/unix/#1"}%
}

?<unix scripts?>

%%%%%%%%%%%%%%%%%%
\section{MS Window Scripts}
%%%%%%%%%%%%%%%%%%
?<ms setup?>
\ifOption{Needs}{\Script{%
   \Needs{"mkdir -p mn.dir/ht-win32"}%
   \Needs{"cp #1.#2  mn.dir/ht-win32/."}%
}}{}
\Script{%                        #1 -- file    #2 -- extension
   \Needs{"mkdir -p /opt/cvr/gurari/tex4ht.dir/bin/ht/win32"}%
   \Needs{"mv #1.#2  /opt/cvr/gurari/tex4ht.dir/bin/ht/win32/#1.#2"}%
   \Needs{"chmod 744 /opt/cvr/gurari/tex4ht.dir/bin/ht/win32/#1.#2"}%
}

?<ms scripts?>
?<mkht note?>

\end{document}
->>>

\<unix setup\><<<-
%%%%%%%%%%
\Ext{unix}
%%%%%%%%%%
\Base{}
\Quote{'}
\Argchar{$}
\Dirchar{-f/}
\Echooff{}
\Remark{#}
->>>

\<ms setup\><<<-
%%%%%%%%%%%%%%%%%%
\Options{bat}
%%%%%%%%%%%%%%%%%%
%%%%%%%%%
\Ext{bat}
%%%%%%%%%
\Base{}
\Quote{}
\Argchar{%}
\Dirchar{}
\Echooff{@echo off}
\Remark{Rem }
->>>



The htlatex-oriented scripts have the following outline.

\begin{verbatim}
        latex  $5 $1
        latex  $5 $1
        latex  $5 $1   
        tex4ht -f/$1  -i~/tex4ht.dir/texmf/tex4ht/ht-fonts/$3  
        t4ht -f/$1 $4 
\end{verbatim}

The latex command may be introduced with a filename residing out of
the work directory. However, it produces the dvi code in the work
directory. Consequently, when activated by htlatex-oriented scripts,
the tex4ht.c and t4ht.c utilities need to look for the dvi code in the
work directory.  The -f prefix asks these utilities to ignore the
paths possibly appearing in \verb!$1!. A path is recognized by
determining whether the character immediately after \verb'-f' is a
directory indicator character `\verb+\+' or `\verb+/+'.


 



%%%%%%%%%%%%%%%%%%
\section{Perl Script}
%%%%%%%%%%%%%%%%%%


%%%%%%%%%%%%%
\subsection{Core}
%%%%%%%%%%%%%



\<mk4ht.perl\><<<-
#!/usr/bin/env perl
?<perl copyright notice?>
use strict;
$^W=1; # turn warning on
my $texFile = ''; 
if( @ARGV  ){ 
  my(@array) = split('\.',@ARGV[1]); 
  $texFile = @array[0]; 
} 
?<available options?>
?<help info?>
print "mk4ht (?version)\n";
if(  !@ARGV  ){
  print "improper command\n";
  showInstrucions(); exit(1);
}
my @command=("","","","","");
my $i=0;
my $j=0;
my $param;
?<mk4ht.cfg vars?>
foreach $param (@ARGV) {
  if( $i == 0 ){
    ?<load mk4ht.cfg?>
    for( $j=1; $j<$#options; $j+=6 ){
      if( $param eq $options[$j] ){
         if( $name ){
            ?<use cfg command?>            
         } else {
            ?<use cfg+built in command?>
         }
         ?<insert cfg extras?>
         last;
    } }
    if( $j>$#options ){ 
      if( $name ){ ?<use cfg command?> }
      else {
          print "improper command: $param \n";
          showInstrucions(); exit(1); 
    } }
  } elsif ( $i== 1 ) {
    $command[1] = $param;
  } elsif ( $i== 2 ) {
    if( $command[2] eq "" ){
      $command[2] = $param;
    } else {
      $command[2] = $param. "," . $command[2];
    }
  } elsif ( $i== 3 ) {
    $command[3] = $param . $command[3];
  } else {
    $command[4] = $param. " " .$command[4];
  }
  $i++;
}
my $cmd;
?<cmd = search command with kpsewhich?>
my $commando = $cmd . " "    . $command[1] . " \"".
            $command[2] . "\" \"". $command[3] . "\" \"" .
            $command[4] . "\"";
print "$commando\n";
my $rtrn;
if( $rtrn = system($commando) ){
   print "--- error --- failed to execute command\n";
} else {
   ?<clean files?>   
}
exit( $rtrn ); 
->>>


\<ensure initialized command[*]MOPE\><<<-
if( !$command[1] ){ $command[1] = ""; } 
if( !$command[2] ){ $command[2] = ""; } 
if( !$command[3] ){ $command[3] = ""; } 
if( !$command[4] ){ $command[4] = ""; } 
->>>



\<cmd = search command with kpsewhich\><<<-
open (KPSEA, "kpsewhich " . $command[0] . " |");
if ($cmd = <KPSEA>){ 
  $cmd =~ s/\s+$//; 
} else {
  $cmd = $command[0];
}
close KPSEA;
->>>


\<available options\><<<-
my @options = (
?<perl options?>
);
->>>


Safety guards:

\begin{itemize}
\item The `use strict'  should be employed when not
   dynamically exporting variables. 
\item Put `my' on local declarations.
\item The `W'  (?)

\end{itemize}


%%%%%%%%%%%%%
\subsection{Load Command}
%%%%%%%%%%%%%


\<use cfg command\><<<-
$command[0] = "ht".$compiler;
if( $tex    ){ $command[2] = $tex;    } else { $command[2] = ""; }
if( $tex4ht ){ $command[3] = $tex4ht; } else { $command[3] = ""; }
if( $t4ht   ){ $command[4] = $t4ht;   } else { $command[4] = ""; }
->>>


\<use cfg+built in command\><<<-
if( $options[$j-1] eq "" ){
  $command[0] = $options[$j];
} else {
  $command[0] = "ht".$options[$j+1];
}
if( $tex    ){ $command[2] = $tex;    } else { $command[2] = $options[$j+2]; }
if( $tex4ht ){ $command[3] = $tex4ht; } else { $command[3] = $options[$j+3]; }
if( $t4ht   ){ $command[4] = $t4ht;   } else { $command[4] = $options[$j+4]; }
->>>



\<insert cfg extras\><<<-
if( $texp   ){ $command[2] = $command[2] . "," . $texp; } 
if( $tex4htp){ $command[3] = $command[3] . " " . $tex4htp; } 
if( $t4htp  ){ $command[4] = $command[4] . " " . $t4htp; } 
->>>


%%%%%%%%%%%%%
\subsection{Commands from Configuration File}
%%%%%%%%%%%%%


A configuration file may contain records of the following kinds.

\begin{description}
\item[\#] Comment
\item[name = type]  Defines a ht*tex like command, and assocites to it 
the TeX compiler of the specified type.  Examples of TeX types: latex, tex,
texi.
\item[name.tex = options]  Command line options for the compilation under 
   the (la)tex compiler  
\item[name.tex4ht = options]  Command line options for  tex4ht.c
\item[name.t4ht = options]  Command line options for  t4ht.c
\end{description}


Each record should appear in a different line.
The variants `name.tex += options', `name.tex4ht += options',
`name.t4ht += options' ask to add the listed options to the base
values.





Example:


\begin{verbatim}
foohlatex         = latex
foohlatex.tex     = xhtml,uni-html4
foohlatex.tex4ht += -cunihtf
foohlatex.t4ht    = -cvalidate
htlatex.t4ht     += -d./
\end{verbatim}




\<load mk4ht.cfg\><<<-
?<inf = cfg handler?>
if( $inf ){
   print "(mk4ht cfg)\n"; 
   while(<$inf>) { 
     my($line) = $_; 
     chomp($line);      # remove eoln char 
     if ($line =~ m|\s*#.*|) {} 
     elsif($line =~ m|^\s*(\S*)\.(\S*)\s*\+=\s*(.*\S)\s*$|) { 
         if( ($param."tex4ht") eq ($1.$2) ){ 
              $tex4htp = $tex4htp . " " . $3; 
         } 
         elsif( ($param."t4ht") eq ($1.$2) ){ 
              $t4htp = $t4htp . " " . $3; 
         } 
         elsif( ($param."tex") eq ($1.$2) ){ 
              $texp = $texp . "," . $3; 
         } 
     } 
     elsif($line =~ m|^\s*(\S*)\.(\S*)\s*=\s*(.*\S)\s*$|) { 
         if( ($param."tex4ht") eq ($1.$2) ){ 
              $tex4ht = $3 . " "; 
              $tex4htp = ""; 
         } 
         elsif( ($param."t4ht") eq ($1.$2) ){ 
              $t4ht = $3 . " ";
              $t4htp = "";  
         } 
         elsif( ($param."tex") eq ($1.$2) ){ 
              $tex = $3 . ",";
              $texp = "";  
         } 
     }  
     elsif($line =~ m|^\s*(\S*)\s*=\s*(.*\S)\s*$|) { 
         if( $param eq $1 ){ 
              $name = $1; 
              $compiler = $2; 
         } 
     }
     ?<elsif get ext?>
     elsif ($line) { print "--- Error --- " . $line . "\n"; } 
   } 
   close $inf; 
}
->>>


\<inf = cfg handler\><<<-
my $inf; 
open $inf, "<mk4ht.cfg"  
or  
(  open $inf, "<.mk4ht"  
   or   
  ( 
     open $inf, "<" . $ENV{HOME} . "/mk4ht.cfg"  
     or  
     ( 
        open $inf, "<" . $ENV{HOME} . "/.mk4ht"  
        or $inf = "" 
) )  ) 
; 
->>>

\<mk4ht.cfg vars\><<<-
my $name;
my $compiler;
my $tex;
my $tex4ht;
my $t4ht;
my $texp;
my $tex4htp = "";
my $t4htp = "";
->>>




%%%%%%%%%%%%%
\subsection{Cleaning Temporary files}
%%%%%%%%%%%%%

Requested in the mk4ht.cfg file through records of the following form

\begin{description}
\item[clean ext1 ext2 ...]
The extension names of the files to be removed.

\item[clean.name ext1 ext2 ...]

The `name' refers to the ht*tex like command in use.

\end{description}






Example:


\begin{verbatim}
clean             dvi idv
clean.foohlatex   lg 
clean.htlatex     lg tmp
\end{verbatim}



\<mk4ht.cfg vars\><<<-
my @ext;
->>>

\<clean files\><<<-
my $file;
my $ext;
opendir(DIR,".") ;
while ($file = readdir(DIR) ){
  if(index($file,$texFile) == 0 ){
    foreach $ext(@ext){
    if (index($file,$ext,length($file)-length($ext)) != -1){
      if( stat($file)){
         unlink($file);
         print  "Deleted: ". $file . "\n";
} } } } }
closedir(DIR);
->>>




\<elsif get ext\><<<-
elsif($line =~ m|^\s*clean\s+(.+)|){
   my(@array) = split(' ',$1);
   push(@ext,@array);
} elsif($line =~ m|^\s*clean\.(\S+)\s+(.+)|){
   if( $1 eq @ARGV[0] ){
      my(@array) = split(' ',$2);
      push(@ext,@array);
}  }
->>>



   
%%%%%%%%%%%%%
\subsection{Info}
%%%%%%%%%%%%%




\<help info\><<<-
sub showInstrucions(){
  print " option1:  mk4ht #1 \"#2\" \"#3\" \"#4\" \"#5\"\n";
  print " \n";
  print "    #1: htlatex, xhlatex, mzlatex, oolatex, dblatex, dbmlatex,\n";
  print "        jhlatex, eslatex, teilatex, teimlatex, uxhlatex,  \n";
  print "        wlatex, xhmlatex\n";
  print " \n";
  print "        also 'tex', 'texi', 'xetex', and 'xelatex'\n";
  print "        instead of 'latex'\n";
  print " \n";
  print "    #2: file name\n";
  print "    #3: optional arguments for latex/tex/... \n";
  print "    #4: optional arguments for tex4ht.c\n";
  print "    #5: optional arguments for t4ht.c\n";
  print " \n";
  print " option2:  mk4ht ht #2 #3 \"#4\" \"#5\"\n";
  print " \n";
  print "    #1: ht\n";
  print "    #2: latex, tex\n";
  print "    #3: file name\n";
  print "    #4: optional arguments for tex4ht.c\n";
  print "    #5: optional arguments for t4ht.c\n";
  print " \n";
  print " Within the program, in column three of the options\n";
  print " variable, the  requests for the commands \"latex\",\n";
  print " \"tex\", etc. can be replaced with other equivalent\n";
  print " commands (e.g., \"tex -fmt=latex\").\n";

  ?<help cfg info?>

  ?<clean info?>
}
->>>






\<help cfg info\><<<-
print "--------------------------------------------------------------------------\n";
print "           Private configuration file: mk4ht.cfg\n";
print "--------------------------------------------------------------------------\n";
print "\n";
print "A private configuration file mk4ht.cfg or .mk4ht may be placed at the\n";
print "work or home directory, to update existing commands and introduce new\n";
print "ones. The configuration file may contain records of the following\n";
print "kinds.\n";
print "\n";
print "   #  Comment\n";
print "    \n";
print "   name = type\n";
print "          Defines a ht*tex like command, and assocites to it the \n";
print "          TeX compiler of the specified type. Examples of TeX \n";
print "          types: latex, tex, texi, xetex, xelatex.\n";
print "    \n";
print "   name.tex = options\n";
print "          Command line options for the compilation under\n";
print "          the (la)tex compiler\n";
print "      \n";
print "   name.tex4ht = options\n";
print "          Command line options for tex4ht.c \n";
print "    \n";
print "   name.t4ht = options\n";
print "          Command line options for t4ht.c\n";
print "    \n";
print "Each record should appear in a different line.  Variants\n";
print "`name.tex += options', `name.tex4ht += options',\n";
print "`name.t4ht += options' of the above records are also allowed.\n";
print "They append the listed options to the base values.\n";
print "\n";
print "Example:\n";
print "\n";
print "   foohlatex         = latex\n";
print "   foohlatex.tex     = xhtml,uni-html4\n";
print "   foohlatex.tex4ht += -cunihtf\n";
print "   foohlatex.t4ht    = -cvalidate\n";
print "   htlatex.t4ht     += -d./\n";
->>>




\<clean info\><<<-
print "--------------------------------------------------------------------------\n";
print "           Deleting files\n";
print "--------------------------------------------------------------------------\n";
print "\n";
print "The configuration file mk4ht.cfg may also contain requests for\n";
print "removing files created in the work directory during the compilation.\n";
print "The requests are to be made through records of the following forms.\n";
print "\n";
print "   clean ext1 ext2 ...\n";
print "     The extensions of the file name to be removed.\n";
print "   \n";
print "   clean.name ext1 ext2 ...\n";
print "     Conditional request. The `name' refers to the ht*tex \n";
print "     like command in use.\n";
print "\n";
print "Example:\n";
print "  clean             dvi idv\n";
print "  clean.foohlatex   lg \n";
print "  clean.htlatex     lg tmp\n";
->>>





%%%%%%%%%%%%%%%%%%
\section{Info}
%%%%%%%%%%%%%%%%%%



\<mkht note\><<<-
\immediate\write16{--------------------------------------------------------}
\immediate\write16{* Compile mkht-scripts.4ht with latex to get the 
                                                              full scripts.}
\immediate\write16{* For shorter latex2e scripts, compile a file whose 
                                                                content is:}
\immediate\write16{\space\space\space\space\def\string\latex{2e}
                                      \string\input\space mkht-scripts.4ht }
\immediate\write16{* For shorter latex209 scripts, compile a file whose 
                                                                content is:}
\immediate\write16{\space\space\space\space\def\string\latex{209}
                                      \string\input\space mkht-scripts.4ht }
\immediate\write16{* Remove the extension .unix from the file names        }
\immediate\write16{--------------------------------------------------------}
->>>

% The - at the end of the first line prevents us from using tex4ht-cpright.tex.
\<TeX4ht copyright\><<<-
%
% This work may be distributed and/or modified under the
% conditions of the LaTeX Project Public License, either
% version 1.3c of this license or (at your option) any
% later version. The latest version of this license is in
%   http://www.latex-project.org/lppl.txt
% and version 1.3c or later is part of all distributions
% of LaTeX version 2005/12/01 or later.
%
% This work has the LPPL maintenance status "maintained".
%
% The Current Maintainer of this work
% is the TeX4ht Project <https://tug.org/tex4ht>.
% 
% If you modify this program, changing the 
% version identification would be appreciated.
%
\immediate\write-1{version \ifx \JOBNAME\UnDefined ?version\else |version\fi}
->>>

\<perl copyright notice\><<<-
# mk4ht (?version), generated from ?jobname.tex
# Copyright 2009-2020 TeX Users Group
# Copyright ?CopyYear.2003. Eitan M. Gurari
#
# This work may be distributed and/or modified under the
# conditions of the LaTeX Project Public License, either
# version 1.3 of this license or (at your option) any
# later version. The latest version of this license is in
#   http://www.latex-project.org/lppl.txt
# and version 1.3 or later is part of all distributions
# of LaTeX version 2003/12/01 or later.
#
# This work has the LPPL maintenance status "maintained".
#
# The Current Maintainer of this work
# is the TeX4ht Project <https://tug.org/tex4ht>.
#
# If you modify this file, changing the
# version identification be appreciated.
->>>

% used in the generated scripts.
\<MYcopyrightnotice\><<<-
|Rem |ScriptFileName|AddExtn (?version), generated from ?jobname.tex
|Rem Copyright 2009-2020 TeX Users Group
|Rem Copyright ?CopyYear.2003. Eitan M. Gurari
|Rem
|Rem This work may be distributed and/or modified under the
|Rem conditions of the LaTeX Project Public License, either
|Rem version 1.3 of this license or (at your option) any
|Rem later version. The latest version of this license is in
|Rem   http://www.latex-project.org/lppl.txt
|Rem and version 1.3 or later is part of all distributions
|Rem of LaTeX version 2003/12/01 or later.
|Rem
|Rem This work has the LPPL maintenance status "maintained".
|Rem
|Rem The Current Maintainer of this work
|Rem is the TeX4ht Project <https://tug.org/tex4ht>.
|Rem
|Rem If you modify this file, changing the
|Rem version identification be appreciated.
->>>


%%%%%%%%%%%%%%%%%%%%%%%%%%%%%%%%%%%%%%%%

\OutputCode\<mkht\>
\OutputCode\<mkht-scripts\>

\OutputCodE\<mk4ht.perl\>
%"mv mk4ht.perl mk4ht"


   \Needs{"mkdir -p         /opt/cvr/gurari/tex4ht.dir/bin/ht/perl"}%
   \Needs{"cp mk4ht.perl /opt/cvr/gurari/tex4ht.dir/bin/ht/perl/mk4ht.perl"}%
   \Needs{"chmod 744     /opt/cvr/gurari/tex4ht.dir/bin/ht/perl/mk4ht.perl"}%


\immediate\write16{--------------------------------------------------------}
\immediate\write16{Compile mkht-scripts.4ht with latex to get the
                                                              full scripts.}
\immediate\write16{Use the option `Needs' to automatically move the
    files to private directories.}
\immediate\write16{--------------------------------------------------------}

\end{document}
\endinput
