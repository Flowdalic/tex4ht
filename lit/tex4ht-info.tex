% $Id$
% Compile with the command `ht latex tex4ht-info'
% or
% compile 3 times: latex tex4ht-info   
%           or   xhlatex tex4ht-info "html,3,sections+"
% Copyright 2009-2017 TeX Users Group
% Copyright 2000-2009 Eitan M. Gurari
% Released under LPPL 1.3c+.
% See tex4ht-cpright.tex for license text.

\ifx \HTML\UnDef
   \def\HTML{info4ht}
   \def\CONFIG{\jobname}
   \def\MAKETITLE{\author{Eitan M. Gurari}}
   \def\next{\input mktex4ht.4ht   \endinput}
   \expandafter\next
\fi


% $Id$
% Common TeX definitions used only in the *-info.tex literate sources.
% Not installed.
% 
% Copyright 2009-2017 TeX Users Group
% Copyright 1996-2009 Eitan M. Gurari
% Released under LPPL 1.3c+.
% See tex4ht-cpright.tex for license text.

\expandafter\ifx \csname YES\HTML\endcsname\relax
    % begin comment. 21/07/2016 (dg)
    %   on first run \infoIVht expects \ConfigureHinput
    %   ( \def\infoIVht#1\ConfigureHinput{..} )
    %   so we feed it with "\ConfigureHinput" (no expansion here, merely
    %   a delimiter); the rest is slurped until the "//".
    % end
    \def\CleanComment{[0]\ConfigureHinput\id:gobble}
\else
    \let\saveCd=\<
    \def\<{\edef\FIRST{\the\inputlineno}\let\<\saveCd \saveCd}
    %
    % Eitan's commented-out definition started like this (and doesn't work):
    %\def\CleanComment#1tex4ht-info#2#3#4.#5>#6//{[\eatIV#4%     <--jobname
    %
    % Eitan's active definition started like this:
    %\def\CleanComment#1tex4ht-info#2#3#4.#5>#6//{[#4%     <--jobname
    %
    % However, that didn't work either.  #4 is not the jobname. 
    % The arguments when running htlatex tex4ht-info-mml.tex are these:
    % %#1<-
    %\CleanComment #1tex4ht-info#2#3#4.#5>#6//->
    %[\if ,\ifnum \FIRST =#6 .\else ,0\fi
    %#1<-
    %#2<--
    %#3<-m
    %#4<-ml
    %#5<-html#QPrTx1"\<infomml\
    %#6<-92\ifx \CodeId \:gobbleii \else ...\fi 
    %
    % As a result, when running  mzlatex hello.tex xhtml,info  there was
    % an error on the first line of infomml.4ht, which looked like this:
    % \ifx\infoIVht\UnDeF\def\infoIVht#1//{}\fi\infoIVht[ml0]28...//
    % That "ml" is not a number, so \ifnum fails.  This only happens
    % mzlatex and the info option, not htlatex.  We don't understand.
    %
    % This version avoids the spurious "ml" but mzlatex hello.tex still
    % fails, trying to process the \ConfigureHinput blocks as text.
    % Changing the bracketed number in infomml.4ht to small values seems
    % to make it pass, but can't see how to generate it.  The number
    % after the brackets (#6) changes also.
    % 
    % Since all this is only about the info option with mzlatex,
    % just leaving it failing for now.  Other things to do.
    \def\CleanComment#1tex4ht-info#2#3#4.#5>#6//{[1\empty %
                  \if,\ifnum \FIRST=#6 .\else ,0\fi\fi]#6//}
    \def\eatIV#1#2#3#4{}
\fi

\Comment{

\string\ifx\string\infoIVht\string\UnDeF\string\def\string\infoIVht#1//{}\string\fi\string\infoIVht\CleanComment}{//

}

\def\>>>#1<<<{\bgroup\csname no:catcodes\endcsname0{255}{12}%
   \csname no:catcodes\endcsname{13}{13}{13}% ^^M
   \def\temp##1>>>{\egroup
      \expandafter \def\csname #1\endcsname{##1}}\temp}

% $Id$
% A few common TeX definitions for literate sources.  Not installed in runtime.
% 
% Copyright 2009-2017 TeX Users Group
% Copyright 1996-2009 Eitan M. Gurari
%
% This work may be distributed and/or modified under the
% conditions of the LaTeX Project Public License, either
% version 1.3c of this license or (at your option) any
% later version. The latest version of this license is in
%   http://www.latex-project.org/lppl.txt
% and version 1.3c or later is part of all distributions
% of LaTeX version 2005/12/01 or later.
%
% This work has the LPPL maintenance status "maintained".
%
% The Current Maintainer of this work
% is the TeX4ht Project <http://tug.org/tex4ht>.
% 
% If you modify this program, changing the 
% version identification would be appreciated.

\newcount\tmpcnt  \tmpcnt\time  \divide\tmpcnt  60
\edef\temp{\the\tmpcnt}
\multiply\tmpcnt  -60 \advance\tmpcnt  \time

\edef\version{\the\year-\ifnum \month<10 0\fi
  \the\month-\ifnum \day<10 0\fi\the\day
   -\ifnum \temp<10 0\fi \temp
   :\ifnum \tmpcnt<10 0\fi\the\tmpcnt}

% a fixed-string version that can be enabled for debugging.
%\edef\versionDebug{000-00-00-00:00}
%\let\version\versionDebug

% #1 is the first year for Eitan.  The last year is always 2009.  RIP.
\def\CopyYear.#1.{#1-2009}

% command for write to terminal and the log file
% this version is used in the .4ht files build
% identical command is defined also in tex4ht-sty.tex, 
% it is used in TeX document compilation
\def\writesixteen#1{\immediate\write1616{#1}}



%%%%%%%%%%%%%%%%%%%%%%%%%%%%%%%%%%%%%%%%%%%%%%%%%%%%%%%%%%%%%%%%%%%%%%%%
\chapter{INFO}
%%%%%%%%%%%%%%%%%%%%%%%%%%%%%%%%%%%%%%%%%%%%%%%%%%%%%%%%%%%%%%%%%%%%%%%%


\<info4ht\><<<
%%%%%%%%%%%%%%%%%%%%%%%%%%%%%%%%%%%%%%%%%%%%%%%%%%%%%%%%%%  
% info4ht.4ht                           |version %
% Copyright (C) |CopyYear.2000.       Eitan M. Gurari         %
%                                                        %
% This program can redistributed and/or modified under   %
% the terms of the LaTeX Project Public License          %
% Distributed from CTAN archives in directory            %
% macros/latex/base/lppl.txt; either version 1 of the    %
% License, or (at your option) any later version.        %
%                                                        %
% If you modify this program your changing its signature %
% with a directive of the following form will be         %
% appreciated.                                           %
%            \message{signature}                         %
%                                                        %
%                              gurari@cse.ohio-state.edu %
%                  http://www.cse.ohio-state.edu/~gurari %
%%%%%%%%%%%%%%%%%%%%%%%%%%%%%%%%%%%%%%%%%%%%%%%%%%%%%%%%%%
\immediate\write-1{version |version}
{          \catcode`\@=0 \catcode`\\=11 @relax
  @gdef@infoIVht[#1]#2//{%
    @ifnum #1>1
      @def@infoIVht[##1]##2//{%
        @ifnum ##1>1 @ifnum ##1<#1
           @bgroup 
             @no:catcodes0{255}{11}%
             @no:catcodes{91}{91}{12}% [
             @no:catcodes{47}{47}{12}% /
             @newlinechar13 %   
             @long@def@infoIVht####1\ifx\infoIVht####2infoIVht[####3//{%
               @def@infoIVht{******************************************}%
               @immediate@write-1{@infoIVht}%
               @immediate@write-1{****** @csname :Hin@endcsname.4ht}%
               @immediate@write-1{@infoIVht}%
               @bgroup
                @def@infoIVht{~~~~~~~~~~~~~~~~~~~~~~~~~~~~~~~~~~~~~~~~~*}%
                @let~=@space   @immediate@write-1{@infoIVht}%
               @egroup   
               @immediate@write-1{####1}%
               @bgroup
                @def@infoIVht{~~~~~~~~~~~~~~~~~~~~~~~~~~~~~~~~~~~~~~~~~*}%
                @let~=@space   @immediate@write-1{@infoIVht}%
               @egroup
               @immediate@write-1{@infoIVht}%
             @egroup}%
           @expandafter@expandafter@expandafter@infoIVht
     @fi@fi }%
  @fi }
}
>>>











\chapter{The Code}


\section{tex4ht}
                
\<configure info4ht Preamble\><<<
\Configure{PROLOG}.........1

   #1 Comma separated list of hooks to appear before HTML.

      Each hook E is declared to be configurable by an
      instruction of the form \NewConfigure{E}{1}

      A star '*' prefix calls for accumulative configurations

   Example:

      \Configure{PROLOG}{VERSION,DOCTYPE,*XML-STYLESHEET}
      \Configure{VERSION}
         {\HCode{<?xml version="1.0"?>}}

\Configure{ext}............1

   #1: default extension name for target files  (recorded in \:html)

   Can also be requested through a command line option ext=...

\Preamble...................0

   Records the list of the requested options.  Defined upon entering
   the environment \Preamble{...}....\EndPreamble, to replace the 
   earlier version of \Preamble.

\ifOption ................. 3

   #1  Argument to be checked  wheteher it is a given option.
   #2  True part
   #3  False part
>>>


                
\<configure info4ht tex4ht\><<<
Wrapper for the Document
------------------------

\Configure{DOCTYPE}.........1
\Configure{HTML}............2
\Configure{HEAD}............2
\Configure{@HEAD}...........1
\Configure{BODY}............2
\Configure{TITLE+}..........1
\Configure{TITLE}...........2
\Configure{@TITLE}..........1
\Configure{Preamble}........2

    <DOCTYPE>
    <HTML 1>
      <HEAD 1>
         <TITLE 1>
            <@TITLE>
            <TITLE+>
         <TITLE 2>
         <@HEAD>
      <HEAD 2>
      <BODY 1>
      ......
      <BODY 2>
    <HTML 2>
    
The \Configure{@HEAD}{...} command is additive, concatenating the
content of all of its appearances.  An empty parameter requests
the cancellation of the earlier contributions.  

For instance,

  \Configure{@HEAD}{A}
  \Configure{@HEAD}{}
  \Configure{@HEAD}{B}
  \Configure{@HEAD}{C}

contributes `BC'.

The \Configure{TITLE+} provides the content for the title,
\Configure{TITLE} sets the envelop, and \Configure{@TITLE} acts as a
hook for introducing localized configurations. As is the case for
\Configure{@HEAD}, the contribution of \Configure{@TITLE} is also
additive.

These configurations should be introduced early enough in the
compilation. For instance, in the case of LaTeX, between \Preamble
and \begin{document} of a local configuration file.

           \Preamble
             %%% here %%%
           \begin{document}
             ...
           \EndPreamble


\Configure{@BODY}...........1
\Configure{@/BODY}..........1

   Variants of \Configure{@HEAD} which contribute their content,
   respectively, after <body> and before </body>.

\Configure{CutAtTITLE+}.....1
\Configure{HPageTITLE+}.....1

   #1 an insertion just before the content of <TITLE>;

   If #1 is a one parametric macro, it gets the title content for
   an argument.


Support for Sectioning Commands
-------------------------------

\Configure{unit-name} ......................4

   #1 start
   #2 end
   #3 before title
   #4 after title

   Example:

        \Configure{section}
           {\HCode{<section>}}    {\HCode{</section>}}
           {\HCode{<title>}}      {\HCode{</title>}}


\ConfigureMark{unit-name}...................1

   Defines a macro \<<unit-name>>HMark to hold the given argument.  
   Upon entering the unit, \TitleMark gets the content of this macro.
   
   Some built-in configurations of TeX4ht require an argument for the
   \<<unit-name>>HMark commands. For safety, these commands should
   always be followed by a, possiblely empty, argument.  The argument
   should be a separator between the title mark and its content.

   Example: 

        \Configure{section}
           {}{}
           {\HCode{<h3>}\TitleMark\space}    {\HCode{</h3>}}
        \ConfigureMark{section}{\thesection}


\Configure{toTocLink}.......................2

   Each unit title contains a \Link{...}{...}...\EndLink command.
   The first argument of \Link points to the first table of contents
   referencing the title. The second argument provides an anchor
   for references to the title (mainly from tables of contents). 
   
   The package option `section+' requests the inclusion of the
   title within the anchor.  Without this option, the link command
   resides between the title mark and its content.

   The \Configure{toTocLink} command is provided for configuring
   the \Link and \EndLink  instructions.  In the default setting,
   when the `sections+' option is not activated, the \Link 
   command is altered to replace its first argument with an empty
   argument.  

     Example:

          \Configure{toTocLink}
             {\Link}
             {\ifx \TitleMark\sectionHMark
                \Picture[\up]{haut.jpg align="right"}%
                \EndLink 
                \TitleMark\space
              \else \EndLink \fi
             }
          \def\up{[up]}

\Configure{toToc}...........................2

    #1  unit type
    #2  desired contents type  (if empty, `unit type' is assumed)
     
  Example: \Configure{toToc}{chapter}{likechapter}
           Introduces chapter as likechapter into toc

    #1  empty: stop adding entries of `unit type' to toc
            @: add entries of `unit type' to toc
            ?: resume mode in effect before the last stop
    #2  unit type

  Example: \Configure{toToc}{}{chapter}
           \chapter{...}
           \Configure{toToc}{@}{chapter}
 
\Configure{writetoc}.........................1

    #1  Configuration material for the insertion instruction.  
        New configurations are added to those request earlier
        by the command.  An empty argument cancels the earlier
        contributions.

\NoLink.......................1

   Ignore option `section+' for sections of type #1   

\TitleCount

   Count of entries submitted to the toc file

\Configure{NoSection}.........2

   Insertions around the parameters of sectioning commands, applied when 
   the parameters are not used to create titles for the divisions.


\CutAt{#1,#2,#3,...}

    #1           section type to be placed in a separate web page
    #2,#3,...    end delimiting section types, other than #1, for
                 the web pages
    A `+' before #1 requests hypertext buttons for the web pages

    Examples:  

          \CutAt{mychapter,myappendix,mypart}
          \CutAt{+myappendix,mychapter,mypart}
 
    Cut points at arbitrary points can be introduced by introducing section-like
    commands in a manner similar to

         \NewSection\mysection{} 
         \CutAt{mysection} 


\Configure{+CutAt}.................................3

     #1 sectioning type
     #2 before
     #3 after

     Requests delimiters for the \CutAt buttons of the specified
     sectioning type

    Example:   \Configure{+CutAt}{mysection}{[}{]}         

\PauseCutAt{#1}
\ContCutAt{#1}

    #1  section type

\Configure{CutAt-filename} ........................1

   A 2-parameter hook for tailoring section-based filenames.  
   The section type is available through #1. The section title
   is accessible through #2.
    
   Example:   \Configure{CutAt-filename}{\NextFile{#1-#2.html}}


Tables of Contents
------------------

Created from the entries collected in the previous compilation within
a jobname.4tc file.

\ConfigureToc{unit-name} ......................4

   #1 before unit number
   #2 before content
   #3 before page number
   #4 at end

   * Empty arguments request the omission of the corresponding field.

   * \TocCount  Specifies the entry count withing the jobname.4tc file.

   * \TitleCount Count of entries submitted to the toc file

   * An alternative to \ConfigureToc{unit-name}:

      \def\toc<unit-name>#1#2#3{<before unit number>#1<before content>#2%
                             <before page number>#3<at end>}


   Example:

        \ConfigureToc{section}
           {}
           {\Picture[*]{pic.jpg width="13"  height="13"}~}
           {}
           {\HCode{<br />}}
   

\Configure{TocLink}..................4

   Configures the link offered in the third arguments of \ConfigureToc

   Example:   \Configure{TocLink}{\Link{#2}{#3}#4\EndLink}   

\TocAt{#1,#2,#3,...}

    #1           section type for which local tables of contents
                 \Toc#1 are requested
    #2,#3,...    sectioning types to be included in the tables of 
                 contents

    The non-leading arguments may be preceded by slashes '/', in
    which cases the arguments specify end points for the tables.

    The default setting requests automatic insertion of the local
    tables immediately after the sectioning heads.

    A star `*' character may be introduced, between the  \TocAt and
    the left brace, to request the appearances of the tables of
    contents at the end of the units' prefaces.

    A hyphen `-' character, on the other hand, disables the automatic
    insertions of the local tables.

    In case of a single argument, the command removes the 
    existing definition of \Toc#1.

    Example:  
      \TocAt{mychapter,mysection,mysubsection,/myappendix,/mypart}
      \TocAt-{mysection,mysubsection,/mylikesection}
      \section{...}...\Tocmysection

    The definition  of the local table of contents can be redefined
    within \csname Toc#1\endcsname.

    Example:

       \TocAt{section} 
       \def\Tocsection{\TableOfContents[section]} 
 
       \Css{div.sectionTOCS { 
                           width : 30\%;  
                           float : right;  
                      text-align : left;  
                  vertical-align : top;  
                     margin-left : 1em;  
                       font-size : 85\%; 
                background-color : \#DDDDDD; 
           }} 
      
    Example: Table of content before the section title.

       \Configure{section}{}{}  
          {\Tocsection \let\saveTocsection=\Tocsection 
           \def\Tocsection{\let\Tocsection=\saveTocsection}% 
           \ifvmode \IgnorePar\fi \EndP\IgnorePar  
           \HCode{<h3 class="sectionHead">}\TitleMark\space\HtmlParOff}  
          {\HCode{</h3>}\HtmlParOn\ShowPar \IgnoreIndent \par}  


\Configure{TocAt}......................2
\Configure{TocAt*}.....................2

   #1 before the tables of contents
   #2 after the tables of contents


Navigation Links for Sectioning Divisions
-----------------------------------------

\Configure{crosslinks}.....................8

   #1  left delimiter
   #2  right delimiter
   #3  next 
   #4  previous
   #5  previous-tail
   #6  front
   #7  tail
   #8  up
   
  The content to be displayed in the pointers

\Configure{crosslinks*}.................1--7

  Links to be included and their order. Available 
  options: next, prev, prevtail, tail, front, up.
  The last argument must be empty.

  Default:
      
      \Configure{crosslinks*}
         {next}
         {prev}
         {prevtail}
         {tail} {front}
         {up}
         {}

\Configure{crosslinks+}.....................4
   
   #1  before top menu
   #2  after top menu
   #3  before bottom menu
   #4  after bottom menu

  The top cross links are omitted, if both #1 and #2 are empty.
  The bottom cross links are omitted, if both #3 and #4 are empty.

\Configure{next}.....................1
    #1  the anchor of the next button of the front page.
      
    Default: The value provided in \Configure{crosslinks}

\Configure{next+}.............................2

   #1  before the next button of the front page, when the `next'
       option is active.
   #2  after the button

    Default: The values provided in \Configure{crosslinks}

\Configure{crosslinks:next}..................1
\Configure{crosslinks:prev}..................1
\Configure{crosslinks:prevtail}..............1
\Configure{crosslinks:tail}..................1
\Configure{crosslinks:front}.................1
\Configure{crosslinks:up}....................1

   #1 local configurations for the delimiters and hooks

\Configure{crosslinks-}.....................2
   
   Asks to show linkless buttons with the following insertions.
   
      #1  before 
      #2  after
   
   The default values are used, if both #1 and #2 are empty

   Examples:

       \Configure{crosslinks-}{}{} 

       \Configure{crosslinks-} 
           {\HCode{<span class="hidden">}[} 
           {]\HCode{</span>} } 
       \Css{span.hidden {visibility:hidden;}}

Paragraphs
----------

\Configure{HtmlPar}..........4

   #1 content at the start non-indented paragraphs
   #2 content at the start indented paragraphs
   #3 insertion into \EndP, at the start of non-indented paragraphs
   #4 insertion into \EndP, at the start of indented paragraphs

   \HtmlParOff
   \HtmlParOn

   \IgnorePar     Asks to ignore the next paragraph
   \ShowPar       Asks to take into account the following paragraphs

   \IgnoreIndent  asks to ignore indentation in the next paragraph
   \ShowIndent    asks to check indentation in the following paragraphs

   \SaveEndP      Saves the content of \EndP, and sets it to empty content
   \RecallEndP

   \SaveHtmlPar
   \RecallHtmlPar


  Example:
     \Configure{@BODY}
        {\ifvmode \IgnorePar\fi \EndP
         \HCode{<div>}\par\ShowPar}
     \Configure{@/BODY}
        {\ifvmode \IgnorePar\fi \EndP
         \HCode{</div>}}


Cross-Linking
-------------

\Link[@1 @2]{@3}{@4}...\EndLink

  Creates

     <a href="@1#@3" name="@4" @2>...</a>   

  * When @1 is empty, tex4ht will derive its value automatically.
    The derived value will be the file name containing the target @3.

  * \Link may be followed by `-', if tex4ht needs not automatically
    determine (for other \link commands) the file containing @4. 
    In the present of such a flag, tex4ht can spare a definition of
    one macro.

  * The component [@1 @2] is optional. If omitted, @1 and @2 are
    assumed to be empty

  * The href attribute is omitted when @1 and @3 are empty

  * The name attribute is omitted when @4 is empty

  Examples:

      \Link{a}{}...\Endlink .....  \Link{}{b}...\EndLink
      \Link[http://foo  id="fooo"]{a}{b}...\EndLink

\Configure{Link}..............4

   Configures \Link...\EndLink so that

     #1 replaces `a'
     #2 replaces `href='
     #3 replaces `name='
     #4 replaces `#'.  If empty, the older value remains in effect.

  Examples:

   \Configure{Link}{a}{href=}{name=}{}   
   \Configure{Link}{ref}{target=}{id=}{\empty}   

\Configure{?Link}..............1

   #1 insertion before broken links

   To help with debugging 

\LinkCommand...................1 <= i <= 6

   Creates a \Link-like command

   #1   tag name
   #2   href-like attribute
   #3   name-like attribute
   #4   insertion
   #5   /, if empty element
   #6  replacement for #  (ignored if absent)

  Example:
  
    \LinkCommand\JSLink{a,\noexpand\jsref,name}
    \def\jsref="#1"{href="javascript:window.open('#1')"}
  
    \JSLink{a}{}xx\EndJSLink    
    \Link{}{a}\EndLink       % or \JSLink{}{a}\EndJSLink

\Configure{XrefFile}.....................1

   #1 names cross-references of files (appends #1 to `)F' and `)Q'
      entries of the .xref files). Applicable mainly implicitly 
      within \Link commands

\Tag.....................................2

   #1  label
   #2  content

\Ref.....................................1
\LikeRef.................................1

   #1  label
   
   \Tag and \Ref are tex4ht.sty commands introduced cross-referencing
   content through .xref auxiliary files.  

   \LikeRef is a variant of \Ref which doesn't verify whether the 
   labels exit.  It is mainly used in \Link and \edef environments.

\ifTag ..................................3

   #1  quetioned tag
   #2  true part
   #3  false part

\LoadRef-[prefix]+{filename.ext}{pattern}

   Load the named xref-type file

   .xref      optional--`.xref' is assume for a default
   +          optional-- asks \Ref and \LikeRef commands
              to use expanded tags `filename::tag', instead of just `tag'
   [prefix]   optional--asks just for tags starting with the 
              specified prefix. 
   -          optional--deletes the prefixes from the loaded tags
   {pattern}  to be included only when `[prefix]' or `+' are included.
              States how tags are to be addressed, with the parameter
              symbol `#1'  specifying the loaded part.

   Example:

        % a.tex
        \LoadRef-[to:]{b}{from:#1}      \Ref{from:filename}
                                        \LikeRef{from:filename}

        % b.tex
        \Tag{to:filename}{\FileName}

   Example:

        \LoadRef-[)F]{file}{)Ffoo##1}
        \LoadRef-[)Q]{file}{)Qfoo##1}
        \Configure{XrefFile}{foo}   \Link...\EndLink

        \LoadRef{another-file}


Files
-----

\FileName    Holds the name of the current hypertext file
\FileNumber  Holds the internal number of the current hypertext file
\RefFileNumber...........................1
    #1  File number

    Provides the file name

\NextFile.................................1
    #1 Requested name for the next file

\Hinput{#1}     

    The command asks to load the configuration files associated
    with mark #1. 

\Hinclude[#1]{#2}

    The command associates configuration file #2 with mark #1.  If
    the mark is the star character `*', the configuration files is
    associated to all marks.  The command is applicable until the
    \Preamble command is processed

    For instance,  \Hinclude[*]{html4.4ht}....\Hinput{latex}

\Hinclude{#1}{#2}  
    
    The command is applicable while the \Preamble command is 
    processed. Its purpose is to load *4ht hook files within
    the fragments of code specified in #1.

    For instance, \Hinclude{\input plain.4ht}{plain}  


Fonts
-----

\Configure{htf}...............................9

    #1         label (integer 0--255) 
    #2         delimiter (a character not appearing in #3,...,#9)
           even label             odd label
    #3       start opening tag      start empty tag
    #4       name                   alt
    #5       size                   name
    #6       mag                    size
    #7                              mag
    #8       end the tag            ord
    #9       closing tag            end the tag

    The htf fonts assign a content and a label to each symbol (possibly
    followed by a comment).  For instance,
    
        'e'    '1'    epsilon
        'z'    '3'    zeta   
    
    An even label asks that the content itself will be used for the
    symbol, and an odd label asks that the symbol will be represented by a
    bitmap.  In the later case,  the content serves as a substitution for
    browsers which don't exhibit bitmaps.
    
    The \Configure{htf}... command provides label-dependent wrappers to
    chosen representations.
    
    If they are not empty, `mag' and `ord' must be c-type 
    patterns for integer arguments, and `name' and `size' 
    should be a patterns for strings.  The `mag' entry is 
    ignored for fonts of the default dimension. Together
    they specify a attribute-value format, mainly for references
    in the css code.

  Examples:

     \Configure{htf}{0}{+}{<span\Hnewline
        class="}{\%s}{-\%s}{x-x-\%d}{}{">}{</span>}
     \Configure{htf}{1}{+}{<img\Hnewline
        src="}{" alt="}{" class="}{\%s}{-\%d}{x-x-\%x}{" />}

\Configure{htf-attr}....................... 2

     #1  c-pattern for the font name and size
     #2  c-pattern for font magnification

     Specify the format of the selectors within the css files.

   Example:
     \Configure{htf-attr}{.\%s-%s}{--\%s}

\Configure{htf-css}....................... 2
    
    #1  font name or label 
    #2  css entry

    A variant of the \Css command. If #1 is a font name, 
    the contribution replaces the one given within the
    htf font definition. If #1 is a label for an entry
    of a htf font, the contribution is added to the css
    file.  The contribution is offered, only when the
    font is in use.

  Example:

   \Configure{htf-css}{4}{.small-caps{font-variant: small-caps;}}  

    
Bitmaps
-------

\Configure{Picture}....................... #1

  #1  Extension name for bitmap files of dvi pictures,
      stored in \PictExt

  Default: \Configure{Picture}{.png}
  
  The extension names of bitmap files of glyphs of htf fonts may be
  determined within a g-entry in the environment file tex4ht.env, or a
  g-flag of the tex4ht.c utility.

\Configure{Picture-alt}......................1

  #1  alt value for \Picture+{...}  and \Picture*{...} 

\Configure{Picture+}.........................2
\Configure{Picture*}.........................2

  #1  before the dvi picture code
  #2  after the dvi picture code

  Typically, the plus `+' variant is introduced as an inline
  contribution into paragraphs, and the star `*' variant as an
  independent block between paragraphs.

\Configure{PictureAlt}........................2
\Configure{PictureAlt*+}......................2
\Configure{PictureAlt*+[]}....................2

  #1 definitions before alt 
  #2 definitions after alt

 Apply to \Picture{...}, \Picture*+{...}, and \Picture*+[...]{...}

\Configure{PictureAlt}........................1
\Configure{PictureAlt*+}......................1
\Configure{PictureAlt*+[]}....................1

  #1 definition for attributes (introduced through 
     a parameter named `#1')

   Apply to \Picture{...}, \Picture*+{...}, and \Picture*+[...]{...}

\Configure{IMG}...............................5

  #1 before file name
  #2 between file name and alt
  #3 close alt for  \Picture without * or +
  #4 close alt for  \Picture with * and +
  #5 right delimiter

  Example:

     \Configure{IMG}
        {\ht:special{t4ht=<img src="}}
        {\ht:special{t4ht=" alt="}}
        {" }
        {\ht:special{t4ht=" }}
        {\ht:special{t4ht=/>}}

\NextPictureFile.............................1

   Requests a file name for the next created picture.

\PictureFile.............................0

   Records the filename of the most recent created picture.

Math
----

\Configure{$}................................2
\Configure{$$}...............................2
\Configure{DviMath}..........................2

\DviMath ... \EndDviMath
\MathClass ... \EndMathClass
\PicMath ... \EndPicMath
\DisplayMath ... \EndDispalyMath

   Example:

     \Configure{$} {\Tg<math>\DviMath} {\EndDviMath\Tg</math>} {}



\Configure{PicMath}..........................4

   Example:

       \Configure{PicMath}{}{}{}{ class="math" }

     \Configure{()}{\protect\PicMath$}{$\protect\EndPicMath}

\Configure{SUB}..............................2
\Configure{SUP}..............................2
\Configure{SUBSUP}...........................3
\Configure{SUPSUB}...........................3
\Configure{SUB/SUP}..........................6

\Configure{putSUB}...........................1
\Configure{putSUP}...........................1

     #1 the code to be used for realizing subscripts and postcripts

\Configure{afterSUB}.........................2

     #1 look ahead token after subscript
     #2 the code to be used for realizing subscripts having #1 for
        lookahead token

\Configure{over}.............................2
\Configure{atop}.............................2
\Configure{above}............................2
\Configure{overwithdelims}...................2
\Configure{atopwithdelims}...................2
\Configure{abovewithdelims}..................2

   #1 before \over, \atop, \above
             \overwithdelims, \atopwithdelims, \abovewithdelims
   #2 after  \over, \atop, \above <dimension>
             \overwithdelims <del1> <del2>
             \atopwithdelims <del1> <del2>
             \abovewithdelims <del1> <del2> <dimension>

   Example:

     \Configure{over}
         {\Send{GROUP}{0}{[before]}[before-rule]}
         {[before-argument]\Send{EndGROUP}{0}{[after]}}


\Configure{MathClass}........................5

   #1  class number
          0: mathord, 1: mathop, 2: mathbin, 3: mathrel,
          4: mathopen, 5: mathclose, 6: mathpunc
   #2  delimiter
   #3  before
   #4  after
   #5  characters
   
   Extra support:
   
      \PauseMathClass
      \EndPauseMathClass
      \NewMathClass<new control sequence>  (7, 8, ...)

\Configure{FormulaClass}.....................4

   #1  class number
          0: mathord, 1: mathop, 2: mathbin, 3: mathrel,
          4: mathopen, 5: mathclose, 6: mathpunc
   #2  a character not in #3 and #4
   #3  before
   #4  after

   If #2 is empty, the formula gets the same marking as a 
   single character of the specified type

\Configure{FormulaClass*}....................4
   
   Like the previous case, but allow marking in the
   nested content.

\Configure{MathDelimiters}...................2
  
   #1  left
   #2  right

\Configure{mathbin*}.........................4
\Configure{mathclose*}.......................4
\Configure{mathop*}..........................4
\Configure{mathopen*}........................4
\Configure{mathord*}.........................4
\Configure{mathpunct*}.......................4
\Configure{mathrel*}.........................4

   #1  a character not presented in #2#3#4
   #2  code before
   #3  code after
   #4  possible definitions for successive cases

  Example:
     \Configure{mathop*}{*}{}{}
        {\Configure{mathop}{*}{<mo>}{</mo>}{}}
     \mathop{\overline{x \mathop{op} y}} \limits^{a=3}

\Configure{mathbin}..........................4
\Configure{mathclose}........................4
\Configure{mathopen}.........................4
\Configure{mathop}...........................4
\Configure{mathord}..........................4
\Configure{mathpunct}........................4
\Configure{mathrel}..........................4

Variants of the above group, requesting to supress nested marks.

\Configure{nolimits}.........................1

\MathSymbol

AtBeginDocument
---------------

\Configure{AtBeginDocument}..................2

    #1  before the corresponding hook of latex
    #2  after

   Insertions are accumulative, and can be erased by providing
   two empty arguments

Other Hooks
-----------

  \Configure{HChar}...................1

    #1  a character

    The \HChar{i} instruction inserts the character code i with the 
    font information of character #1, when i is positive. If i is
    negative, the font info is not included.


\Configure{Canvas}
\Configure{ExitHPage}
\Configure{LinkHPage}......................1
\Configure{FontCss}
\Configure{HVerbatim+}
\Configure{MiniHalign}
\Configure{Needs-}
\Configure{Needs}


\Configure{TraceTables}
\Configure{edit}
\Configure{halignTB}
\Configure{halignTD}
\Configure{halign}
\Configure{hooks}
\Configure{moveright}

\Configure{noalign-}
\Configure{pic-halign}

\Configure{accent}
\Configure{mathaccent}
\Configure{accented}
\Configure{accenting}


Back-end Specials
-----------------
                                                    insertions
                                                    ----------
  =    \special{t4ht=...content...}
           Insert the specified content to the html output, under 
           edef mode of processing, and without using the mapping
           of the htf fonts.  Used in \HCode{...}.
  @    \special{t4ht@...integer...}
           Insert the absolute value as character code to the output.
           Positve values ask the insertion to be included in place
           of the next chracter, together with the font information
           of that character.
                                                    files
                                                    -----
  >    \special{t4ht>...file-name...}
           Open a new file, if needed, and direct future output
           to the specified file.  Used in \File{...}.
  <    \special{t4ht<...file-name...}
           Close the specified file.  If it is the current file,
           activate the youngest file. Used in \EndFile{...}.
  >*   \special{t4ht*>...file-name...}
           Declare the file to be the oldest.
       \special{t4ht*>}
           Reactivate the file that activated the current file.
  *<   \special{t4ht*<file}
           Input file (with no processing)
  +    \special{t4ht++file-name}...dvi...\special{t4ht+}
           Pipe the dvi code into a dvi page in the secondary dvi file
           `jobname.idv'.  Used by \Picture{...}, e.g., for requesting 
           gif's.
  +    \special{t4ht+embeded-specials within idv}
  .    \special{t4ht.ext}
           Change default ext of root file
  @D   \special{t4ht@D....} Writes the content, augmented with a
           loc stamp, to the .lg file.  The locations stamp consists
           a byte-address in a named output file.

                                                    character maps
                                                    --------------
  !    \special{t4ht!...optional-parameters....}...dvi...\special{t4ht!}
           Create an approximated character map for the dvi code.
           Used in \Picture{...}, e.g., for ALT of IMG
  ||    \special{t4ht||}...\special{t4ht||}
           Use the non-pictorial characters of the htf fonts.
           Used for character maps of \Picture{....}
  @    \special{t4ht@-}....\special{t4ht@-}
           Remove left margin from character map.  Used in \Picture{...}.

                                                    character settings
                                                    ------------------
  @    \special{t4ht@@}....\special{t4ht@@}
           Insert the character codes, instead of their mappings through
           the htf fonts.  Used in \JavaScript...
  @    \special{t4ht@...integer...}, \special{t4ht@-...integer...}
           Introduce the character code into the output.
           Used by \HChar{...} and \HChar{-...}. The earlier one
           also inherites the current font info.
  @    \special{t4ht@+...string...}
           Replace the character code introduced by the next character
           with the specified string.  The decoration of the character 
           code is inherited, when the string is not empty. The string
           might include character codes by enclosing them between braces.  
  @    \special{t4ht@*...string...}
           A variant \special{t4ht@+...string...} that inserts the content
           after the character instead of replacing it.
  @    \special{t4ht@(}
           Ignore spaces
  @    \special{t4ht@)} 
           End ignore spaces
  @    \special{t4ht@[}
           Ignore chs and spaces
  @    \special{t4ht@]}
           End ignore chs and spaces
  @    \special{t4ht@[...}...\special{t4ht@]...}\special{t4ht@?...}
           Ignore chs and spaces, if the specials  have the above
           syntax on identical strings.
  @    \special{t4ht@!}
           Get the last ignored spaces (none, if from previous lines).
  @    \special{t4ht@_....}
           Output character for rulers. Empty string is also allowed.
  @    \special{t4ht@.''''}
           Output for line break characters (empty
                                         content resets the default).
  @    \special{t4ht@,''''}
           Output for space characters (empty content resets the default).

                                                    dvi tracing
                                                    -----------
  @    \special{t4ht@%X}...\special{t4ht@%x}
           Request dvi tracing.

               X  x
               P  p      groups
               C  c      characters
               H  h      horizontal spaces
               V  v      vertical spaces
               R  r      rulers
         
       \special{t4ht@%%X*...open-del....*...close-del....}
       \special{t4ht@%%x*...open-del....*...close-del....}
           Tailor dvi tracing

  @    \special{t4ht@/}  
           On/off tracing of specials.
  @    \special{t4ht@e...} 
           String for tracing errors into the output.
  ;    \special{t4ht;....}         
           Decorations for htf characters (e.g., css)
              8    pause
              9    end pause
              |... pattern
              =... show font name of char
              %... show font size of char
              ,    don't report next htf class to lg
              -    set default font info
              +    unset default font info
  ^    \special{t4ht^i}$symbols$\special{t4ht^}}' 
           Requests math class i for the listed math symbols.
           
           Tex assignes class numbers 0--7 to the atoms of math
           formulas: 0--ordinary symbol, 1--large operator, 2--binary
           operation, 3--relational operation, 4--math delimiter,
           5--right delimiter, 6--punctuation mark, and 7--adjustable.
           TeX4ht adds classes  8 and 9, while using 
           class 7 independently
       \special{t4ht^}
           on/off processing delimiters
       \special{t4ht^-} 
           pause processing delimeters
       \special{t4ht^+} 
           continue processing delimeters
       \special{t4ht^i} 
           on/off processing delimiters of class i   
       \special{t4ht^i*...*...}
           configure delimiters for class i. * can be any 
           character distinguished for the group.
       \special{t4ht^i(}
           put delimiters of class i on next group
       \special{t4ht^i)}
           As before, but ignore the delimeters within the sub-group.
       \special{t4ht^)*...*...}
           put the specified delimiters on next group.
           Ignore delimeters within the group.
       \special{t4ht^<*...*...}
           put the specified delimiters on the next group.
           Don't ignore delimeters within the group.
                                                    dvi arithmetic
                                                    --------------
  :    \special{t4ht:....}
           Dvi-mode arithmetics.  
             :+...  increment by 1( define, if not defined)
             :-...  decrement by 1
             :>...  push current value
             :<...  pop current value
             :!...  display current value
             :|...  display top value

                                                    messages to lg file
                                                    -------------------
  +    \special{t4ht+@...message...}
           Send message to the lg file.  Used in the \Needs{...} command. 
  @    \special{t4ht@D....}
           Send message to the lg file, together with location and file
           stamp.

                                                    positional code
                                                    -------------------
  "    \special{t4ht"}
           Start/end positional env
       \special{t4ht"* before-all * after-all ...** before-char 
                     * after-char * rect 
                     *%A*%B*%C*%D*%E
                     * optional
           Configure positioned code
             * before-all
             * after-all      %...
             ** before-char  %x %y
             * after-char    
             * rect          %x1 %y1 ...
             * x,x1-coefficients %A(x) + %B     
             * y1-coefficients %C(y1 - %E(height)) + %D
             * y-coefficients  %C(y) + %D
             * optional: 1, 2

               %x1 %y1 %length %thickness          default
               %x1 %y1 %x1+length %y1+thickness    1
               %x1 %y1 %x1+length %y1 %thickness   2

           A-magnification, B-displacement
           C-magnification, D-displacement, 
           E- origin (0: top, 0.5: mid, 1: bot)

           The %...'s should be c-type templates (e.g., "%.2f"; "%.0f"
              gives an integer)

           Multiple after-all templates are allowed. The leading 
             character is a code specifying the dimension type.
                x           min x
                X           max x
                y           min y
                Y           max y
                d           dx
                D           dy
                otherwise   a string with no values
           The delimiter `*' can be substitued by another character.


  ~    \special{t4ht~...}
           Grouped-base two-way delivery for content created by
           inline commands like \over.  

       \special{t4ht~}...\special{t4ht~}   on/off

       ~<i...  send forward to the start of the group nested
                                                at relative level i.
       ~>i...  send forward to the end of the group nested
                                                at relative level i.
               i=0, current group


       ~<*...  send back to start of previous token / group.
       \special{t4ht~<)}...\special{t4ht~<(}
           activate / deactivate  back token / group submissions
       \special{~<[}...\special{t4ht~<]}'  
           hide region from back submissions over token / group
       \special{t4ht~<-} ... \special{t4ht~<+}
           latex back token / group

       \special{t4ht\string~!...path...<...content}
           insertion at the start of the group reached by the path
       \special{t4ht\string~!...path...>...content}  
           insertion at the end of the group reached by the path
       \special{t4ht\string~!...path.../} 
           ignore content within the group reached by the path
       \special{t4ht\string~!...path...-}       
           ignore rulers from the group reached by the path
           until the start of the next group
        A path may consist only of `e' and `s' characters for,
        respectively, entering and skipping groups

  *!   \special{t4ht*! system command}
           System call
  *^   accent specials
        t text accent #1#2#1#3#1#4#1#5#1#6 pattern
                      empty                insertion point
        m math accent #1#2#1#3#1#4#1#5#1   pattern
                      empty                insertion point
        a accented    #1#2#1#3#1#4#1#5#1#6
        i             #1#2#1#3#1#1
  *@   halign specials

\HCode...............................1

   A wrapper for \special{t4ht=...}.  

   The sharp symbol # may be accessed indirectly through the command.

\Hnewline............................0

   Requests new lines within specials
>>>

%%%%%%%%%%%%%%%
\section{latex}
%%%%%%%%%%%%%%%





\>>>VBorder<<<   
\Configure{VBorder}...................4

    Break points, when scanning the pattern of column desriptions, at

    #1  at start of pattern
    #2  at |
    #3  at a non-@ entry
    #4  at a @ entry

  Applies to \begin{tabular / array}{...pattern...} 

    \ar:cnt    index of entry in pattern
    \ch:class  records the current alignment type: -,<,>,p,...
    \HColAlign produces the \Configure{halignTD} contribution 
               for the current alignment type
    \HColWidth holds the width of the current p column

\Configure{HBorder}..................10
   
   hline:
    #1  insert at start of row (e.g., <tr>)
    #2  insert at each cell    (e.g., <td><hr/></td>)
    #3  insert at end of row   (e.g., </tr>)     

   cline:
    #4  insert at start of row         (e.g., <tr>)
    #5  insert at each `extra' cell    (e.g., <td></td>)
    #6  insert at each cell            (e.g., <td><hr/></td>)
    #7  insert at end of row           (e.g., </tr>)     

   vspace:
    #8  insert at start of row (e.g., <tr>)
    #9  insert at each cell    (e.g., <td>&nbsp;</td>)
    #10 insert at end of row   (e.g., </tr>)     

   The contributions are collected into \HBorder.  (The \InitHBorder
   clears \HBorder.)

\Configure{putHBorder}...............1
    #1 Specifies how \HBorder is to be used.

   Example: \Configure{putHBorder}{\HCode{\HBorder}}

>>>
                
\<configure info4ht latex\><<<
Sectioning
----------

\Configure{@sec @ssect}


Tables of Contents
------------------


\Configure{tableofcontents}........................5
   
   #1 before
   #2 at end
   #3 after
   #4 at indented paragraph break
   #5 at non-indented paragraph break
   
   The \tableofcontents command may be followed by a comma separated
   list of sectioning unit names to be included in the table of
   contents.  The list should be enclosed within square brackets.
   Alternatively, a command of the form \TableOfContents[...] might
   be used.

Lists
-----

\ConfigureList.....................5

   #1   type of list (e.g., itemize, description, enumerate, 
                            list, trivlist)
   #2   before list
   #3   after  list
   #4   before label 
   #5   after label

   \DeleteMark   removes latex's label; to be placed at the end of #4
   \AnchorLabel  defines an anchor for \label in current item; to
                 be placed in #5


Tables
------

\Configure{tabular}...................6
\Configure{array}.....................6

    #1   before table         #2   after table
    #3   before row           #4   after row
    #5   before cell          #6   after cell

    \HRow         current row number
    \HCol         current column number
    \HMultispan   number of cells covered by the current cell
    \ar:cnt       number of columns in the table

  NOTE: Table require a number of compilations that depends
        on the number of columns.

  Example

   \Configure{tabular}  
       {\HCode{<table>}}  
       {\HCode{</table>}}  
       {\HCode{<tr class="row-\HRow">}}  
       {\HCode{</tr>}}  
       {\HCode{<td  
               \ifnum \HMultispan>1 colspan="\HMultispan"\fi >}} 
       {\HCode{</td>}} 
  
|VBorder

\Configure{halignTD}..................2 + 2i + {}

  interpretation for character codes referenced in \HAlign

  e.g.,

   \Configure{halignTD}
   {}{}
   {<}{\HCode{style="text-align:left"}}
   {-}{\HCode{style="text-align:center"}}
   {>}{\HCode{style="text-align:right"}}
   {^}{\HCode{style="vertical-align:top"}}
   {=}{\HCode{style="vertical-align:baseline"}}
   {||}{\HCode{style="vertical-align:middle"}}
   {_}{\HCode{style="vertical-align:bottom"}}
   {p}{\HCode{style="text-align:left"}}
   {}

   \halignTD can be used in td elements to extract the alignment.
   It recieves information from \halignTB.

\Configure{halignTB}..................2
   
   delimiters for \halignTB{tabular}

   Example
       \Configure{halignTB}{\HCode{<table }}{\HCode{>}} 

\Configure{tabbing}[mag]..................4

   #1 before each line
   #2 after each line
   #3 before each entry
   #4 after each entry

   [mag] optional parameter specifying the magnification desired
         for the dimensions.  When offered, the other parameters
         have no effect if all of them are assigned empty arguments

   \TabType   \` or \relax
   \TabWidth  Provides the entry width; 0 at trailing entry that is 
              not flushed rightward


Cross References
----------------

\Configure{ref}.......................3

  #1   \Link-type command
  #2   \EndLink-type command
  #3   anchor (the system anchor is 
               reachable through the parameter name #1)
  \RefArg   Holds the argument of \ref

  If #1 is empty, the hyper links are ignored
  If #3 is empty, the anchor is the one provided by the system

  Example:

    \Configure{ref}{\Link}{\EndLink}{{\bf #1}}

\Configure{pageref}...................3

  #1   before
  #2   after
  #3   anchor (system anchor, if parameter is empty)

\Configure{newlabel}..................2

  #1   address for hyperlink (\cur:th \:currentlabel, if empty)
  #2   anchor (the system anchor is 
               reachable through the parameter name #1)

\Configure{@newlabel}.................1

   #1  modifications to the newlabel environment

\Configure{newlabel-ref}..............1

   #1  an intermediate link command for the aux file  
       (Configured by \Configure{ref}...)

   The default for #1 is \rEfLiNK

\Configure{cite}......................4

  #1   before
  #2   after
  #3   \Link-type command
  #4   \EndLink-type command

  If #3 is empty, the hyper links are ignored.

\Configure{bibitem}...................2

  #1   \Link-type command
  #2   \EndLink-type command

\Configure{bibcite}...................1

  #1  configurations for content transfered by bibitem to the aux file

  Example:    
       \Configure{bibcite}
                 {\def\hookrightarrow{\string\hookrightarrow}}
       \bibitem[$\hookrightarrow$...]{...}


\LoadLabels[#1]{#2}....................

  [#1]   optional group name
  #2     aux file name, without the extension
  
  Loads labels of another file, under the specified group name

\RefLabel.............................2

  #1     group name (for separating files and labels from
                     different sources)
  #2     label
  
  A variant of \ref for loading labels produced for other files

  Example:
    file1.tex:  \label{foo}

    file2.tex:  \LoadLabels[x]{file1}
                \RefLabel{x}{foo}

\SkipRefstepAnchor.....................0

    No \Link anchor for next \refstepcounter

\ShowRefstepAnchor.....................0
\AutoRefstepAnchor.....................0


Bibliography on bibtex2 option: 

  \Configure{bibliography2}........................ 4

     #1 before anchor
     #2 anchor
     #3 after anchor
     #4 link attributes

    Example:
    
       \Configure{bibliography2}
          {\bgroup ~~[\Configure{Link}{a}{target="x"  href=}{ name=}{}}
          {more} {]\egroup} 

  \Configure{bibitem2}..............................3

     #1 at start of bibitem 
     #2 at end of bibitem
     #3 separator after label

  \Configure{bibliographystyle2}....................1
   
     #1 an empty argument asks for the same style as the 
        normal aux file (still bibtex may produce different
        output).

Note: Option `bibtex2' requires compilation
      of `\jobname j.aux' with bibtex.

Captions
--------

refcaption

  An option for \Preamble, requesting anchors at \caption. The default
  setting sends them back to the start of the floating environment.


Theorems
--------

\Configure{newtheorem} ......................3

  #1 before theorem
  #2 between title and body
  #3 after theorem

Math
----     
\Configure{()}...............................2
\Configure{[]}...............................2

   Example:

     \Configure{()}{\protect\PicMath$}{$\protect\EndPicMath}
     \Configure{[]} {\Tg<display>\DviMath$$} {$$\EndDviMath\Tg</display>}

\Configure{equation}.........................3

   #1    at start
   #2    between the equation and its numbering
   #3    at end

  Examples:

      \Configure{equation}
           {\IgnorePar\EndP\bgroup \Configure{HtmlPar}{}{}{}{}%
                    \HCode{<table class="equation"><tr><td>}\IgnorePar
           }
           {\HCode{</td><td class="eq-number">}}
           {\HCode{</td></tr></table>}\egroup}

      %%%%%%%%%%%%%%%%%%%%%%%%%%%%%%%%%%%%%%%%%%%%%

      \Configure{equation}
         {\IgnorePar\EndP \bgroup \Configure{$$}{}{}{}%
          \Configure{@math}{display="inline"}\DviMath
                  \HCode{<mtable class="equation"><mtr><mtd>}\IgnorePar
         }
         {\IgnorePar\HCode{</mtd><mtd class="eq-number">}}
         {\HCode{</mtd></mtr></mtable>}\EndDviMath\egroup}


\Configure{frac}.............................4
\Configure{sqrtsign}.........................2

\Configure{mbox}.............................2

Environments of latex 
---------------------

  \ConfigureEnv{...}.........................4

     #1 environment name
     #2 before env
     #3 after env
     #4 before underlying list
     #5 after underlying list

    #2 and #3 are ignore when they are both empty as well as
    when there is no underlying list


   array
   center
   flushleft
   flushright
   minipage
   tabbing
   tabular
   verbatim*
   verbatim


   \Configure{@begin}........................2

      #1 environment name
      #2 insertion before the environment
     
   Example:
      \Configure{@begin}{theindex}{\section*{\indexname}} 


Verbatim
--------

\Configure{verbatim}......................2

   #1 at start of line
   #2 space character

\Configure{verb}..........................2
   
   #1 before
   #2 after

\Configure{obeylines}.....................3

   #1 before
   #2 at start of line
   #3 after

\ScriptEnv................................3

  Introduces a verbatim environent

   #1 name
   #2 before
   #3 after

  A `-' immediately after \begin{...} designate as an escape symbol
  the character following the dash

  Example:

     \ScriptEnv{foo}
       {\HCode{<myscript>}\NoFonts\hfill\break }
       {\EndNoFonts \HCode{</myscript>}}
     
     \begin{foo}
     ....
     ....
     \end{foo}

     \begin{foo}-|
     ....
     ....
     \end{foo}


Fonts
-----

\Configure{texttt}........................2
\Configure{textit}........................2
\Configure{textrm}........................2
\Configure{textup}........................2
\Configure{textsl}........................2
\Configure{textsf}........................2
\Configure{textbf}........................2
\Configure{textsc}........................2
\Configure{emph}..........................2

  #1  before content
  #2  after content


Footnotes
---------

\Configure{footnotetext}..................3

    #1 before footnote
    #2 between mark and content
    #3 after footnote

   \FNnum     footnote number

\Configure{footnotemark}..................2

    #1 before
    #2 after

Pictures
--------

\Configure{picture}.......................2

    #1  before
    #2  after

Other Hooks
-----------

\Configure{ }.........................1

   #1 representation for non-breaking space ch

\Configure{hline}.....................1

\Configure{hspace} ...................3

   \tmp:dim      register holding the size
   #1            before the space
   #2            after the space
   #3            after #1 (\tmp:dim mod 6em  copies)

\Configure{vspace} ...................1

   #1 the size of space is prvided in a parameter nmaed `#1'

   Example:

      \Configure{vspace}
      {\ifhmode
         \HCode{<br />}%
         \ifdim #1>1ex \HCode{<br />}\fi
       \fi
      }

\Configure{fbox} .................................. 2

   Examples:

        \Configure{fbox}
          {\HCode{<div class="fbox">}\bgroup \fboxrule=0pt}
          {\egroup\HCode{</div>}}
        \Css{div.fbox {border: 1pt solid black;}}
        
        \Configure{fbox}
          {\HCode{<table cellspacing="0pt"
              border="1"><tr><td>}\bgroup \fboxrule=0pt}
          {\egroup\HCode{</td></tr></table>}}

\Configure{'} ..................................... 3

   #1 at entry to math prime environment
   #2 at exit 
   #3 content of \prime
 
\Configure{float}....................................
   #1 optional, to appear within brakects [ and ].  An anchor for
      the links preceeding the float, when option refcaption is
      not active
   #2 Insertion before the links
   #3 at start
   #4 at end

\Configure{textcircled}.............................2n+1
   2i'th     replaced       i=1,...,n
   2i+1'st   replacement
   2n+'nd    empty (terminator)


\Configure{add accent}{#1:#2}{#3}{#4}...{}{}
   #1  encoding
   #2  font number
   #3  character under font
   #4  replacement
   
   Applies to accents that reach \add@accent

  Example:

     \Configure{add accent}{OT4:18}  
       {E}{\add:acc{00C8}} 
       {e}{\add:acc{00E8}} 
       {}{}  
  

    

\Configure{//[]}
\Configure{AfterTitle}
\Configure{HAccent}
\Configure{InsertTitle}
\Configure{accents}
\Configure{accent}
\Configure{centercr}
\Configure{centerline}

\Configure{displaylines}
\Configure{framebox}

\Configure{leftline}
\Configure{marginpar}
\Configure{mathaccent}
\Configure{newline}
\Configure{oalign}

\Configure{overline}

\Configure{rightline}
\Configure{stackrel}
\Configure{tt}
\Configure{underline}
\Configure{thanks}....................2

>>>


%%%%%%%%%%%%%%%
\section{book}
%%%%%%%%%%%%%%
             

\>>>maketitleInfo<<<
Title Page
----------
\Configure{maketitle}.....................4

   #1 start of maketitle
   #2 end of maketitle
   #3 before title
   #4 after title

\Configure{thanks author date and}........8

   #1  before thanks
   #2  after thanks
   #3  before author
   #4  after author
   #5  before date
   #6  after date
   #7  representation of `and'
   #8  line breaks (= end of rows, for an embedded tabular environment)
>>>

\>>>thebibliographyInfo<<<
Bibliography
------------

\ConfigureList{thebibliography}......4

   #1   before list
   #2   after  list
   #3   before label 
   #4   after label

   \DeleteMark   removes latex's label; to be placed at the end of #3
   \AnchorLabel  defines an anchor for \label in current item; to
                 be placed in #4

\Configure{cite}        see the
\Configure{bibitem}     latex section
>>>





\>>>tableofcontentsStr<<<
\Configure{tableofcontents*}.......................1

    #1  A non-empty  parameter asks to implicitly introduce
        a \tableofcontents upon reaching the first sectioning
        command, if no \tableofcontents command is encountered
        earlier. The parameter assumes a colon-separated list
        of sectioning types to be included in the output
        of \tableofcontents.  Starred variants of sectioning
        types should be referenced with the `like' prefix.

        An empty parameter cancels earlier requests for implicit
        calls to \tableofcontents (e.g., embedded within requests
        to package options 1, 2, 3, 4)

   Example:

      \Configure{tableofcontents*}{part,likepart,%
           section,likesection,subsection,likesubsection}

\contentsname

   A LaTeX macro stating the title for a table of contents division.
>>>

\>>>theindex<<<
\Configure{theindex} ..........................9

     #1    before-env
     #2    after-env
     #3    before-item
     #4    after-item
     #5    before-subitem
     #6    after-subitem
     #7    before-subsubitem
     #8    after-subsubitem
     #9    at-indexspace

    Example:
    
       \Configure{theindex}
          {\HCode{<ul class="theindex">}\global\let\IndexSpace=\empty}
          {\HCode{</ul>}}
          {\HCode{<li \IndexSpace>}\global\let\IndexSpace=\empty}
                                         {\HCode{</li>\Hnewline}}
          {\HCode{<li>}\ \ \ \ }         {\HCode{</li>\Hnewline}}
          {\HCode{<li>}\ \ \ \ \ \ \ \ } {\HCode{</li>\Hnewline}}
          {\global\def\IndexSpace{class="indexspace"}}
       
       \Css{.indexspace{margin-top:1em;}}
       
    The links are indirectly requested in the idx files within 
    \beforeentry macros. For instance, a file try.tex
    
       \documentclass{article}
          \makeindex
       \begin{document}
       
       \section{xx}
       
       \index{a1}  x
       \index{a2}  x
       \index{a2}  x
       \index{b1}  x
       \index{b2}  x
       \index{b3}  x
       
       \input \jobname.ind
       
       \end{document}
    
    produces a file try.idx of the form
    
       \beforeentry{try.html}{dx1-1001}{}
       \indexentry{a1}{1}
       \beforeentry{try.html}{dx1-1002}{}
       \indexentry{a2}{1}
       \beforeentry{try.html}{dx1-1003}{}
       \indexentry{a2}{1}
       \beforeentry{try.html}{dx1-1004}{}
       \indexentry{b1}{1}
       \beforeentry{try.html}{dx1-1005}{}
       \indexentry{b2}{1}
       \beforeentry{try.html}{dx1-1006}{}
       \indexentry{b3}{1}
    
    where each pair 
    
        \beforeentry{A}{B}{}\indexetry{C}{D} 
    
    represents an entry of the form 
    
        \indexentry{\Link[A]{B}{}C\EndLink}{D}
          
    The makeindex utility ignores the \beforeentry records.  To compensate
    for that, one needs to pre-process the idx file which is introduced to
    the makeindex utility and/or post-process the output of the utility.

    A script consisting of two instructions similar to

      tex '\def\filename{{try}{idx}{4dx}{ind}} \input idxmake.4ht'
      makeindex -o try.ind try.4dx

    instead of

      makeindex -o try.ind try.idx

    should do the job.

    On some platforms, the quotation marks ' should be
    replaced by double quotation marks " or eliminated.

\Configure{makeindex} ..........................1

    The default setting, requests consecutive numbers for the
    pointers in the indexes.  The current command provides the
    means to configure the pointers to other values.

    Example: \Configure{makeindex}{Sec \arabic{section}}
>>>



  
\<configure info4ht book\><<<
|maketitleInfo

Sectioning
----------

\Configure{part}
\Configure{section}
\Configure{subsection}
\Configure{subsubsection}
\Configure{paragraph}
\Configure{subparagraph}

\Configure{likepart}
\Configure{likechapter}
\Configure{likesection}
\Configure{likesubsection}
\Configure{likesubsubsection}
\Configure{likeparagraph}
\Configure{likesubparagraph}
|thebibliographyInfo


\ConfigureToc
 
 lof, lot, appendix, chapter, likechapter, likeparagraph, likepart,
 likesection, likesubparagraph, likesubsection, likesubsubsection,
 paragraph, part, section, subparagraph, subsection, subsubsection


\Configure{appendixTITLE+}........1
\Configure{chapterTITLE+}.........1
\Configure{partTITLE+}............1
\Configure{sectionTITLE+}.........1
\Configure{subsectionTITLE+}......1
\Configure{subsubsectionTITLE+}...1

   #1  the content of <TITLE>

   The insertion overrides the one offered by \Configure{CutAtTITLE+}
   for the given section type (the `like' counterparts acn also be
   configured).

   The sectioning title content can be accessed through the parameter #1.

   Example:

     \Configure{chapterTITLE+}{Synodos - #1}

\Configure{endpart}................1
\Configure{endchapter}.............1
\Configure{endsection}.............1
\Configure{endsubsection}..........1
\Configure{endsubsubsection}.......1
\Configure{endparagraph}...........1
\Configure{endsubparagraph}........1
\Configure{endappendix}............1
\Configure{endlikepart}............1
\Configure{endlikechapter}.........1
\Configure{endlikesection}.........1
\Configure{endlikesubsection}......1
\Configure{endlikesubsubsection}...1
\Configure{endlikeparagraph}.......1
\Configure{endlikesubparagraph}....1

   #1  a comma separated list specifying the end
       points for the configured logical unit

   e.g., \Configure{endsection}
              {likesection,chapter,likechapter,appendix,part,likepart}



|tableofcontentsStr


Index (\theindex)
-----------------

|theindex


Environments of book:
---------------------

  \ConfigureEnv{...}.........................4

   description
   figure
   figure*
   quotation
   quote
   table
   table*
   thebibliography
   titlepage
   verse

\Configure{listof}

\Configure{appendix}
\Configure{caption}
\Configure{chapter}
>>>









\section{report}
                
\<configure info4ht report\><<<
|maketitleInfo
|thebibliographyInfo


\ConfigureList{description}%

\ConfigureToc
  
  lof, lot, appendix, chapter, likechapter, likeparagraph, likepart,
  likesection, likesubparagraph, likesubsection, likesubsubsection,
  paragraph, part, section, subparagraph, subsection, subsubsection


\Configure{appendixTITLE+}
\Configure{chapterTITLE+}
\Configure{partTITLE+}
\Configure{sectionTITLE+}
\Configure{subsectionTITLE+}
\Configure{subsubsectionTITLE+}

   #1 an insertion just before the content of <TITLE>;

   The insertion overrides the one offered by \Configure{CutAtTITLE+}
   for the given section type (the `like' counterparts acn also be
   configured).

   If #1 is a one parametric macro, it gets the title content for
   an argument.


\Configure{endpart}................1
\Configure{endchapter}.............1
\Configure{endsection}.............1
\Configure{endsubsection}..........1
\Configure{endsubsubsection}.......1
\Configure{endparagraph}...........1
\Configure{endsubparagraph}........1
\Configure{endappendix}............1
\Configure{endlikepart}............1
\Configure{endlikechapter}.........1
\Configure{endlikesection}.........1
\Configure{endlikesubsection}......1
\Configure{endlikesubsubsection}...1
\Configure{endlikeparagraph}.......1
\Configure{endlikesubparagraph}....1

   #1  a comma separated list specifying the end
       points for the configured logical unit

   e.g., \Configure{endsection}
              {likesection,chapter,likechapter,appendix,part,likepart}


|tableofcontentsStr


\Configure{appendix}................4
\Configure{chapter}
\Configure{likechapter}
\Configure{likeparagraph}
\Configure{likepart}
\Configure{likesection}
\Configure{likesubparagraph}
\Configure{likesubsection}
\Configure{likesubsubsection}
\Configure{paragraph}
\Configure{part}
\Configure{section}
\Configure{subparagraph}
\Configure{subsection}
\Configure{subsubsection}


|theindex

Environments of report:
----------------------

  \ConfigureEnv{...}.........................4

   abstract
   description
   figure
   figure*
   quotation
   quote
   table
   table*
   thebibliography
   titlepage
   verse

\Configure{listof}
\Configure{thanks author date and}
\Configure{caption}
>>>

\section{article}
                
\<configure info4ht article\><<<
|maketitleInfo

Sectioning Commands
-------------------
\Configure{part}...................4
\Configure{section}................4
\Configure{subsection}.............4
\Configure{subsubsection}..........4
\Configure{paragraph}..............4
\Configure{subparagraph}...........4

   #1 before division
   #2 after division
   #3 before title
   #4 after title

\Configure{likepart}...............4
\Configure{likesection}............4
\Configure{likesubsection}.........4
\Configure{likesubsubsection}......4
\Configure{likeparagraph}..........4
\Configure{likesubparagraph}.......4

   starred versions of the sectioning commands

\Configure{endpart}................1
\Configure{endsection}.............1
\Configure{endsubsection}..........1
\Configure{endsubsubsection}.......1
\Configure{endparagraph}...........1
\Configure{endsubparagraph}........1
\Configure{endlikepart}............1
\Configure{endlikesection}.........1
\Configure{endlikesubsection}......1
\Configure{endlikesubsubsection}...1
\Configure{endlikeparagraph}.......1
\Configure{endlikesubparagraph}....1

   #1  a comma separated list specifying the end
       points for the configured logical unit

   e.g., \Configure{endsection}
              {likesection,chapter,likechapter,appendix,part,likepart}

\Configure{partTITLE+}
\Configure{sectionTITLE+}
\Configure{subsectionTITLE+}
\Configure{subsubsectionTITLE+}


   #1 an insertion just before the content of <TITLE>;

   The insertion overrides the one offered by \Configure{CutAtTITLE+}
   for the given section type (the `like' counterparts acn also be
   configured).

   If #1 is a one parametric macro, it gets the title content for
   an argument.


|thebibliographyInfo

Tables of Content
-----------------
\ConfigureToc
  
  lof, lot, appendix, chapter, likechapter, likeparagraph, likepart,
  likesection, likesubparagraph, likesubsection, likesubsubsection,
  paragraph, part, section, subparagraph, subsection, subsubsection,

|tableofcontentsStr


Captions
--------

\Configure{caption}...............4
   
   #1   before number          #2    after number
   #3   before title           #4    after title

Indexes
-------

|theindex

Environments of article:
------------------------

  \ConfigureEnv{...}.........................4

   abstract
   description
   figure
   figure*
   quotation
   quote
   table
   table*
   thebibliography
   titlepage
   verse

Other Hooks
-----------
\Configure{listof}
>>>

\section{fontmath}
                
\<configure info4ht fontmath\><<<
\Configure{mathbf}........................2
\Configure{mathit}........................2
\Configure{mathrm}........................2
\Configure{mathsf}........................2
\Configure{mathtt}........................2

  #1  before content
  #2  after content

\Configure{overbrace}.................3
\Configure{underbrace}................3
>>>


\section{graphics}
                
\<configure info4ht graphics\><<<
\Configure{graphics}...............2
   
    #1  before \includegraphics
    #2  after \includegraphics


    Examples:

       \Configure{graphics}
          {\Picture+[PIC]{ class="graphics"}}
          {\EndPicture }

       \Configure{graphics}
         {\bgroup
             \Configure{IMG}
                {\ht:special{t4ht=<img src="}}
                {\ht:special{t4ht=" alt="}}
                {" }
                {\ht:special{t4ht=" }}
                {}%
          \Picture+[PIC]{}}
         {\EndPicture
             \def\temp{.pstex}\expandafter\ifx 
                              \csname Gin@ext\endcsname\temp
                                         \HCode{ width="75\%" }\fi
             \HCode{ />}%
          \egroup}


\Configure{graphics*}..............2
   
    #1  extension name
    #2  insertion

    \Gin@base (file name), \Gin@ext, \Gin@req@width, \Gin@req@height,
    \noBoundingBox (defined iff bounding box is unknown)

    Allows to configure tex4ht for graphics files named in
    the \includegraphics macro, based on the type of the files.

    An empty insertion #2 cancels previous requests for the 
    specified extension.

    Example:

       \Configure{graphics*}
         {jpg}
         {\Picture[pict]{\csname Gin@base\endcsname.jpg}}

       \Configure{graphics*}
         {wmf}
         {\Needs{"convert \csname Gin@base\endcsname.wmf
                          \csname Gin@base\endcsname.gif"}%
          \Picture[pict]{\csname Gin@base\endcsname.gif
                      width="\expandafter\the\csname
                                Gin@req@width\endcsname"
                     height="\expandafter\the\csname
                                Gin@req@height\endcsname"}%
         }

       \Configure{graphics*}
         {eps}
         {\openin15=\csname Gin@base\endcsname\PictExt\relax
          \ifeof15
             \Needs{"convert \csname Gin@base\endcsname.eps
                             \csname Gin@base\endcsname\PictExt"}%
          \fi
          \closein15
          \Picture[pict]{\csname Gin@base\endcsname\PictExt}%
         }

  Note: Arguments of the \includegraphics command such as angle and
    scale in

       \includegraphics[angle=-90,scale=0.5]{fig.eps}

    are not known to the given figure (e.g., to fig.eps). To be
    taken into account, the scripts should handle the transformations
    they request (e.g., in \csname Grot@angle\endcsname, 
    \csname Gscale@x\endcsname, \csname Gscale@y\endcsname)
>>>




\section{babel}
                
\<configure info4ht babel\><<<
\Configure{accent}
\Configure{quotedblbase}
\Configure{quotesinglbase}

>>>

\section{plain}
                
\<configure info4ht plain\><<<

\Configure{ }
\Configure{HAccent}
\Configure{TableOfContents}
\Configure{accents}
\Configure{accent}
\Configure{beginsection}
\Configure{centerline}
\Configure{displaylines}
\Configure{eqalignno}
\Configure{eqalign}
\Configure{insert}
\Configure{item}
\Configure{leftline}
\Configure{leqalignno}
\Configure{line}
\Configure{mathaccent}
\Configure{narrower}
\Configure{noalign}
\Configure{obeylines}
\Configure{overline}
\Configure{proclaim}
\Configure{rightline}
\Configure{settabs}
\Configure{underline}
\Configure{vfootnote}
>>>



\section{amsfonts}


\<configure info4ht amsfonts\><<<
\Configure{mathbb}........................2
\Configure{mathfrak}......................2

  #1  before content
  #2  after content

>>>



\section{amsmath}

% \Configure{dbinom}..................4
% \Configure{dfrac}..................4
% \Configure{frac}..................4
% \Configure{binom}..................4
                
\<configure info4ht amsmath\><<<
\Configure{tmspace}...................1
    
  \mathglue   amount of space in math units
  \textspace  amount of space in points

  Example:
    \Configure{tmspace}{\mskip\mathglue}


\Configure{boxed}.....................2
\Configure{equations}.................2
\Configure{equation}..................3
\Configure{gather*}...................6
\Configure{gather}....................6
\Configure{genfrac}...................6
\Configure{measure@}..................1
\Configure{multline*}.................4
\Configure{multline}..................6

    #5  before the label
    #6  after the label
\Configure{overset}...................2
\Configure{smallmatrix}...............6
\Configure{split}.....................6
\Configure{subarray}..................4
\Configure{substack}..................2
\Configure{underset}..................2
\Configure{xleftarrow}................2
\Configure{xrightarrow}...............2
>>>




\section{amstex1}
                
\<configure info4ht amstex1\><<<
\Configure{eqn}
\Configure{gather}


Environments of amstex1:

   Vmatrix
   align*
   alignat*
   alignat
   alignedat
   aligned
   aligned
   align
   bmatrix
   equation*
   equation
   gather*
   gathered
   gather
   matrix
   multline*
   multline
   pmatrix
   vmatrix
   xalignat*
   xalignat
   xxalignat

>>>

\section{amsart}
                
\<configure info4ht amsart\><<<
|tableofcontentsStr
|maketitleInfo

\ConfigureList{thebibliography}%
\Configure{HtmlPar}
\Configure{HtmlPar}
\Configure{Needs}
\Configure{abstract}
\Configure{authors}
\Configure{caption}
\Configure{date}
\Configure{endlikeparagraph}
\Configure{endlikepart}
\Configure{endlikesection}
\Configure{endlikesubparagraph}
\Configure{endlikesubsection}
\Configure{endlikesubsubsection}
\Configure{endparagraph}
\Configure{endpart}
\Configure{endsection}
\Configure{endsubparagraph}
\Configure{endsubsection}
\Configure{endsubsubsection}
\Configure{keywords}
\Configure{likeparagraph}
\Configure{likepart}
\Configure{likesection}
\Configure{likesubparagraph}
\Configure{likesubsection}
\Configure{likesubsubsection}
\Configure{paragraph}
\Configure{partTITLE+}
\Configure{part}
\Configure{sectionTITLE+}
\Configure{section}
\Configure{setdate}
\Configure{subjclass}
\Configure{submaketitle}
\Configure{subparagraph}
\Configure{subsectionTITLE+}
\Configure{subsection}
\Configure{subsubsectionTITLE+}
\Configure{subsubsection}
\Configure{thanks author date and}
\Configure{thanks}
\Configure{title}

Environments of amsart:

   abstract
   abstract
   picture
   picture
   thebibliography

  \ConfigureToc
  
  likeparagraph likepart, likesection, likesubparagraph,
  likesubsection, likesubsubsection, paragraph, part, section,
  subparagraph, subsection, subsubsection,

>>>


\section{amsbook}
                
\<configure info4ht amsbook\><<<
|tableofcontentsStr
|maketitleInfo

\ConfigureList{thebibliography}%
\Configure{HtmlPar}
\Configure{Needs}
\Configure{abstract}
\Configure{addresses}
\Configure{authors}
\Configure{caption}
\Configure{date}
\Configure{endlikeparagraph}
\Configure{endlikepart}
\Configure{endlikesection}
\Configure{endlikesubparagraph}
\Configure{endlikesubsection}
\Configure{endlikesubsubsection}
\Configure{endparagraph}
\Configure{endpart}
\Configure{endsection}
\Configure{endsubparagraph}
\Configure{endsubsection}
\Configure{endsubsubsection}
\Configure{keywords}
\Configure{likeparagraph}
\Configure{likepart}
\Configure{likesection}
\Configure{likesubparagraph}
\Configure{likesubsection}
\Configure{likesubsubsection}
\Configure{paragraph}
\Configure{partTITLE+}
\Configure{part}
\Configure{sectionTITLE+}
\Configure{section}
\Configure{subjclass}
\Configure{subparagraph}
\Configure{subsectionTITLE+}
\Configure{subsection}
\Configure{subsubsectionTITLE+}
\Configure{subsubsection}
\Configure{thanks author date and}
\Configure{thanks}
\Configure{title}
\Configure{translators}

Environments of amsbook:

   abstract
   picture
   thebibliography

  \ConfigureToc
  
  appendix, chapter, likechapter, likeparagraph, likepart,
  likesection, likesubparagraph, likesubsection, likesubsubsection,
  paragraph, part, section, subparagraph, subsection, subsubsection

>>>

\section{amsproc}
                
\<configure info4ht amsproc\><<<
|tableofcontentsStr
|maketitleInfo

\ConfigureList{thebibliography}%
\Configure{HtmlPar}
\Configure{Needs}
\Configure{abstract}
\Configure{authors}
\Configure{caption}
\Configure{date}
\Configure{endlikeparagraph}
\Configure{endlikepart}
\Configure{endlikesection}
\Configure{endlikesubparagraph}
\Configure{endlikesubsection}
\Configure{endlikesubsubsection}
\Configure{endparagraph}
\Configure{endpart}
\Configure{endsection}
\Configure{endsubparagraph}
\Configure{endsubsection}
\Configure{endsubsubsection}
\Configure{keywords}
\Configure{likeparagraph}
\Configure{likepart}
\Configure{likesection}
\Configure{likesubparagraph}
\Configure{likesubsection}
\Configure{likesubsubsection}
\Configure{paragraph}
\Configure{partTITLE+}
\Configure{part}
\Configure{sectionTITLE+}
\Configure{section}
\Configure{subparagraph}
\Configure{subsectionTITLE+}
\Configure{subsection}
\Configure{subsubsectionTITLE+}
\Configure{subsubsection}
\Configure{thanks author date and}
\Configure{thanks}
\Configure{title}

Environments of amsproc:

   abstract
   picture
   thebibliography

  \ConfigureToc
  
  likeparagraph, likepart, likesection, likesubparagraph,
  likesubsection, likesubsubsection, paragraph, part, section,
  subparagraph, subsection, subsubsection
>>>


\section{amsthm}
                
\<configure info4ht amsthm\><<<
Environments of amsthm:

   proof
>>>




\section{amsppt}
                
\<configure info4ht amsppt\><<<
  \ConfigureToc

    head,  specialhead, subhead, subsubhead

\Configure{HtmlPar}
\Configure{Refs}
\Configure{abstract}
\Configure{affil}
\Configure{author}
\Configure{block}
\Configure{book}
\Configure{caption}
\Configure{date}
\Configure{footnote}
\Configure{head}
\Configure{itemitem}
\Configure{keyformat}
\Configure{keywords}
\Configure{paper}
\Configure{ref}
\Configure{roster}
\Configure{specialhead}
\Configure{subhead}
\Configure{subjclass}
\Configure{subsubhead}
\Configure{thanks}
\Configure{title}
\Configure{vol}

>>>

%%%%%%%%%%%%%%%%%%%%%%%%%
\section{listings}
%%%%%%%%%%%%%%%%%%%%%%%%%
                
\<configure info4ht listings\><<<
\Configure{lst@Kerm}...........................2
    
   #1  width of character
   #2  width of inter character space

   In LaTeX, Default: \Configure{lst@Kern}{0.499em}{0.1em}

\Configure{listings}...........................4

     #1 start environment
     #2 end environment
     #3 before line label
     #4 between line label and content

\Configure{lstinputlisting}....................2
\Configure{lstinline}..........................2

     #1 start environment
     #2 end environment
>>>





\section{psfig}
                
\<configure info4ht psfig\><<<
\Configure{psfig}

>>>

\section{epsfig}
                
\<configure info4ht epsfig\><<<
\Configure{epsfig}...........................#1
    
   #1  before
   #2  after

   In LaTeX, epsfig reduces to graphicsx

   Example:

    \Configure{epsfig}
       {\Configure{graphics}
          {} {\xdef\foo{width="\the\Gin@req@width"
              height="\the\Gin@req@height"}}%
        \Picture+[epsfig]{}}
       {\EndPicture\immediate\write16{..... \foo}}
>>>

\section{xy}
                
\<configure info4ht xy\><<<
\Configure{xypic}
\Configure{Xy}
>>>

\section{amscd}
                
\<configure info4ht amscd\><<<

Environments of amscd:

   CD
>>>


\section{array}
                
\<configure info4ht array\><<<
\Configure{tabular}
\Configure{array}

|VBorder

\Configure{@{}}{}


Environments of array:

   array
   tabular
>>>

\section{minitoc}
                
\<configure info4ht minitoc\><<<
\Configure{minilof}
\Configure{minilot}
\Configure{minitoc}
\Configure{partlof}
\Configure{partlot}
\Configure{parttoc}
\Configure{sectlof}
\Configure{sectlot}
\Configure{secttoc}
\Configure{tableofcontents}

>>>


\section{index}
                
\<configure info4ht index\><<<
\Configure{NoSection}

>>>





\section{ntheorem}
                
\<configure info4ht ntheorem\><<<
Environments of ntheorem:

   Anmerkung
   Beispiel
   Bemerkung
   Beweis
   Corollary
   Definition
   Example
   Korollar
   Lemma
   Proof
   Proposition
   Remark
   Satz
   Theorem
   anmerkung
   beispiel
   bemerkung
   beweis
   corollary
   definition
   example
   korollar
   lemma
   proof
   proposition
   remark
   satz
   theorem


>>>


\section{longtable}
                
\<configure info4ht longtable\><<<
\Configure{longtable}.............................6

    #1   before table         #2   after table
    #3   before row           #4   after row
    #5   before cell          #6   after cell
>>>

\section{fancyvrb}
                
\<configure info4ht fancyvrb\><<<
\Configure{fancyvrbcolor}
\Configure{fancyvrbframe}
\Configure{fancyvrb}

Environments of fancyvrb:

   verbatim*
   verbatim

>>>

                \section{moreverb}


\<configure info4ht moreverb\><<<
\Configure{verbatimtab}......................2

   #1 at start of line
   #2 space character

Environments of moreverb:

   boxedverbatim*
   boxedverbatim*
   boxedverbatim
   boxedverbatim
   verbatimtab
>>>


\section{fancybox}
                
\<configure info4ht fancybox\><<<
\Configure{Ovalbox}
\Configure{doublebox}
\Configure{ovalbox}
\Configure{shadowbox}

>>>

\section{color}
                
\<configure info4ht color\><<<
\Configure{HColor}......................2
       
      #1 color name
      #2 color value

  Examples:  \Configure{HColor}{blue}{\#0000FF}
             \Configure{HColor}{light}{rgb(122,251,255)}

\Configure{colorbox}
\Configure{color}
\Configure{textcolor}

>>>

\section{colortbl}
                
\<configure info4ht colortbl\><<<
\Configure{@classz}
\Configure{rowcolor}

>>>

\section{alltt}
                
\<configure info4ht alltt\><<<
Environments of alltt:

   alltt
>>>




\section{url}
                
\<configure info4ht url\><<<
\Configure{url}

>>>

\section{hyperref}
                
\<configure info4ht hyperref\><<<
A trick like the following one allows a compilation to choose 
between a pdf and a html mode automatically.
 
   \ifx \HCode\UnDef  \usepackage[pdf]{hyperref}  
   \else              \usepackage[tex4ht]{hyperref}  \fi 
   
The trick assumes the compilations for html are invoked by
htlatex-like commands.

\Configure{::#1}
\Configure{::action}
\Configure{::default}
\Configure{::menulength}
\Configure{::method}
\Configure{::name}
\Configure{::value}
\Configure{CheckBox::checked}
\Configure{CheckBox::}
\Configure{ChoiceMenu::combo}
\Configure{ChoiceMenu::popdown}
\Configure{ChoiceMenu::radio}
\Configure{ChoiceMenu::}
\Configure{Form}
\Configure{PushButton::}
\Configure{Reset::}
\Configure{Submit::}
\Configure{TextField::default}
\Configure{TextField::multiline}
\Configure{TextField::password}
\Configure{TextField::width}
\Configure{TextField::}
\Configure{combo::default}
\Configure{multiline::value}
\Configure{multiline::width}
\Configure{radio::default}

>>>


\section{CJK}
                
\<configure info4ht CJK\><<<
\Configure{charset}

>>>

\section{vanilla}
                
\<configure info4ht vanilla\><<<
  \ConfigureToc

    heading,  subheading

\Configure{aligned}
\Configure{align}
\Configure{author}
\Configure{demo}
\Configure{heading}
\Configure{matrix}
\Configure{subheading}
\Configure{title}

>>>






\section{emulateapj}
                
\<configure info4ht emulateapj\><<<
\Configure{affil}
\Configure{author}
\Configure{keywords}
\Configure{section}
\Configure{slugcomment}
\Configure{subjectheadings}
\Configure{submitted}
\Configure{subsection}
\Configure{subtitle}
\Configure{title}

Environments of emulateapj:

   references
>>>

\section{aa}
                
\<configure info4ht aa\><<<
|maketitleInfo

\Configure{HtmlPar}
\Configure{caption}
\Configure{endlikeparagraph}
\Configure{endlikepart}
\Configure{endlikesection}
\Configure{endlikesubparagraph}
\Configure{endlikesubsection}
\Configure{endlikesubsubsection}
\Configure{endparagraph}
\Configure{endpart}
\Configure{endsection}
\Configure{endsubparagraph}
\Configure{endsubsection}
\Configure{endsubsubsection}
\Configure{likeparagraph}
\Configure{likepart}
\Configure{likesection}
\Configure{likesubparagraph}
\Configure{likesubsection}
\Configure{likesubsubsection}
\Configure{makeheadbox}
\Configure{paragraph}
\Configure{partTITLE+}
\Configure{part}
\Configure{sectionTITLE+}
\Configure{section}
\Configure{subparagraph}
\Configure{subsectionTITLE+}
\Configure{subsection}
\Configure{subsubsectionTITLE+}
\Configure{subsubsection}
\Configure{subsubsection}
\Configure{subtitle institute}
\Configure{thanks author date and}

Environments of aa:

   abstract
   picture
   thebibliography
>>>

\section{pictex}
                
\<configure info4ht pictex\><<<
\Configure{pictex}

>>>




\section{diagram}
                
\<configure info4ht pb-diagram\><<<
Environments of pb-diagram:

   diagram
>>>



\section{ltugboat}
                
\<configure info4ht ltugboat\><<<
Environments of ltugboat:

   quote
>>>


\section{doc}
                
\<configure info4ht doc\><<<
Environments of doc:

   macrocode
>>>





\section{lineno}
                
\<configure info4ht doc\><<<
\Configure{lineno}...................2

   Asks lineno.sty to introduce #1\LineNumber #2, instead of \LineNumber,
   into the output
>>>






\section{elsart}
                
\<configure info4ht elsart\><<<
\Configure{abstract}
\Configure{keyword}

Environments of elsart:

   frontmatter
>>>


%%%%%%%%%%%%%%%%%%
\section{ProTex}
%%%%%%%%%%%%%%%%%%

\<configure info4ht ProTex\><<<
\Configure{ShowCode}...................6

   #1  start body
   #2  end body
   #3  everypar
   #4  open comment
   #5  close comment
   #6  space character
>>>


\section{seminar}
                
\<configure info4ht seminar\><<<

Environments of seminar:

   slide
>>>


\section{slides}
                
\<configure info4ht slides\><<<
\Configure{slidename}....................................1
\CutAt{slide}.............................................0

\ConfigureList{description}%

Environments of slides:

   note
   quotation
   quote
   slide
   titlepage
   verse
>>>


\section{slidesec}
                
\<configure info4ht slidesec\><<<
    \ConfigureToc

          slidesection
>>>

\section{web}
                
\<configure info4ht web\><<<
\Configure{maketitle}
\Configure{webuniversity}
\Configure{webversion}

>>>

\section{exerquiz}
                
\<configure info4ht exerquiz\><<<
\ConfigureList{questions}%
\Configure{@HEAD}
\Configure{Form}
\Configure{ReturnTo}
\Configure{TextField::}
\Configure{javascript}
\Configure{quiz*}
\Configure{quiz}
\Configure{shortquiz}

Environments of exerquiz:

   shortquiz
>>>

\section{foils}
                
\<configure info4ht foils\><<<
    \ConfigureToc

         foilheads

\Configure{foilheads}

|tableofcontentsStr

Environments of foils:

   Corollary*
   Corollary
   Definition
   Lemma
   Proof
   Proposition
   Theorem
   thebibliography

>>>



\section{th4}
                
\<configure info4ht th4\><<<
\Configure{JavaScript}............2

   Configures the environments

      \JavaScript
         ....
      \EndJavaScript

   and

      \javascript{...}

   These environments are activated by the `javascript' option.

  Default:

   \Configure{JavaScript}
      {\HCode{<script type="text/JavaScript" ><!--\Hnewline}}
      {\HCode{//-->\Hnewline </script>}}



    \ConfigureToc

        Chapter,  LikeSection, Section, SubSection

\Configure{Appendix}
\Configure{Chapter}
\Configure{Columns}
\Configure{Columns}
\Configure{DList}
\Configure{HTable}
\Configure{Item}
\Configure{LikeChapter}
\Configure{LikeSection}
\Configure{OList}
\Configure{Part}
\Configure{Section}
\Configure{SubSection}
\Configure{UList}
\Configure{Verbatim}
\Configure{buttonList+}
\Configure{buttonList}
\Configure{centerline}
\Configure{index}

Frames
------

When the package option `frames-' or `frames' is up, 
TeX4ht introduces the following commands for defining frames.

   \Frame[#1]{#2}

       A variant of the \Link[#1]{#2} portion of the
       \Link[#1]{#2}{#3}#4\EndLink, for specifying a frame.

   \HorFrames[#1]{#2}, \VerFrames[#1]{#2}

     #1 attributes  ([#1] is optional)
     #2 list of dimensions 

      A container partitioned, respectively, horizontally or
      vertically. Each partition may hold a sub-container or
      a frame.

      The list of dimensions determine the cardinality of the
      partition, and the dimension of each partition.  A dimension
      can be specified absolutely by pixels (\HorFrames{80,130,50}),
      relatively by percentage (\HorFrames{20\%,30\%,50\%}), and
      relatively through ratio (\HorFrames{2*,3*,5*}).

   \EndFrames

      An end delimiter for a container

   \NoFrames

       When used, it should be placed before the outer-most
       \EndFrames command.  The region \NoFrames...\EndNoFrames
       provides the content for browsers which do not support frames.

Tailoring configurations for frames is a little tricky, but the
job is simpler when the configurations don't deal with the root
file.

Example 1:
..........

     % Requires the command line option `frames-'
     \documentclass{article}
     
     \begin{document}
     
     \begingroup
        \Configure{@HEAD}{}
        \Configure{BODY}{}{}
        \Configure{DOCTYPE}
             {\HCode{<!DOCTYPE html
                  PUBLIC "-//W3C//DTD HTML 4.01 Frameset//EN"
                          "http://www.w3.org/TR/html4/frameset.dtd">}}
        \Configure{HEAD}{}{}
        \Configure{HTML}
            {\HCode{<html><head><title></title></head>}}
            {\HCode{</html>}}
        \Configure{HtmlPar}{}{}{}{}
        \Configure{TITLE}{}{}
     
        \FileStream+{\jobname f.html}
           \HorFrames{*,*}
              \Frame[\jobname.html]{}
              \VerFrames{3*,*,*}
                 \Frame[http://www.tug.org]{}
                 \Frame[\RefFile{z} ]{}
                 \Frame[ name="y"]{w}
              \EndFrames
           \EndPreamble
           \NoFrames
               a comment
           \EndFrames
        \EndFileStream{\jobname f.html}
     \endgroup
     
     
     \HPage[z]{}
     \bgroup
       \Configure{Link}{a}{target="y"  href=}{ name=}{}
       \let\contentsname=\empty   \tableofcontents
     \egroup
     \EndHPage{z}
     
     \tableofcontents
     
     \section{First} Is first.
     
     \Link{}{w}\EndLink
     \section{Second} Is Second.
     
     \end{document}

Example 1a (allows also compilations for non-html output):
..........................................................
     % src.tex
     \documentclass{article}
     \begin{document}

       \tableofcontents

       \section{First} Is first.
     
       \ifx \Link\undefined \else
          \Link{}{w}\EndLink
       \fi
       \section{Second} Is Second.
     
     \end{document}

     % src.cfg
     \Preamble{html,frames-}
     \begin{document}
     \EndPreamble
       \begingroup
          \Configure{@HEAD}{}
          \Configure{BODY}{}{}
          \Configure{DOCTYPE}{}
          \Configure{HEAD}{}{}
          \Configure{HTML}{}{}
          \Configure{HtmlPar}{}{}{}{}
          \Configure{TITLE}{}{}
       
          \FileStream+{\jobname f.html}
             \HorFrames{*,*}
                \Frame[\jobname.html]{}
                \VerFrames{3*,*,*}
                   \Frame[http://www.tug.org]{}
                   \Frame[\RefFile{z} ]{}
                   \Frame[ name="y"]{w}
                \EndFrames
          \EndPreamble
             \NoFrames
                 a comment
             \EndFrames
          \EndFileStream{\jobname f.html}
       \endgroup
       
       \HPage[z]{}
       \bgroup
         \Configure{Link}{a}{target="y"  href=}{ name=}{}
         \let\contentsname=\empty   \tableofcontents
       \egroup
       \EndHPage{z}
     \endinput

Example 2:
..........

         % Source .tex file

         \documentclass{article}
           \usepackage{verbatim}
         \begin{document}
         
         \section{The Main File: \jobname.tex}
           \verbatiminput{\jobname.tex}
         \section{The Configurations File: \jobname.cfg}
           \verbatiminput{\jobname.cfg}
         \section{The Command Line}
           htlatex \jobname\space "\jobname"
         
         \NextFile{page.html}\HPage{}
           Hello!
         \EndHPage{}
         
         \HPage{}\Link{}{anchor}\EndLink
           \Link[http://www.ctan.org]{}{}ctan\EndLink
         \EndHPage{}
         
         \end{document}

         % configurations .cfg file

         \Preamble{frames}
         
         \Configure{frames}
           {\VerFrames{*,5*}
              \Frame[ name="tex4ht-menu" ]{tex4ht-toc}
              \HorFrames{*,3*}
                \VerFrames{6*,*,*}
                  \Frame[http://www.tug.org]{}
                  \Frame[page.html]{}
                  \Frame{anchor}
                \EndFrames
                \Frame[ name="tex4ht-main" ]{tex4ht-body}
              \EndFrames
            }
            {\let\contentsname=\empty \tableofcontents}
         
         \begin{document}
         \EndPreamble

Example 3:
..........
 
         % file.tex 
         \documentclass{book}  
         \begin{document}  
           \chapter{Title}   
              Body.  
           \chapter{Another Title}   
              \Link{}{init-body}\EndLink   
              Another body.  
         \end{document}  
          
         % file.cfg 
         \Preamble{html,frames,3,info}  
           \begin{document}  
               \Configure{frames}   
                  {\HorFrames[rows="*"]{3*,*}   
                     \Frame[ name="body" frameborder="0" ]{init-body}   
                     \Frame[ name="menu" frameborder="0" ]{init-toc}  
                  }  
                  {\Configure{Link}{a}{target="body"  href=}{ id=}{}   
                   \let\contentsname=\empty    
                   \tableofcontents  
                   \Link{}{init-toc}\EndLink  
                  }  
         \EndPreamble  
         \endinput  



The arguments`init-body' and `init-toc' are labels connecting reference points 
\Frame[...]{label} with anchors \Link{}{label}\EndLink.  Their objective 
is to associate initial web pages with the frames. 
 
The name attribute values `body' and `menu' provide 
identifiers to the frames.  They offer the option to dynamically load 
new pages into the frames through hypertext references targeting those 
values. 

>>>




\section{hebtex}
                
\<configure info4ht hebtex\><<<
\Configure{arabtext}
\Configure{chireq}
\Configure{cholem}
\Configure{cholem}
\Configure{chpatach}
\Configure{chqames}
\Configure{chsegol}
\Configure{dagesh}
\Configure{meteg}
\Configure{patachf}
\Configure{patach}
\Configure{qameschat}
\Configure{qames}
\Configure{qibbus}
\Configure{rdot}
\Configure{segol}
\Configure{sere}
\Configure{shindot}
\Configure{shwa}
\Configure{sindot}

>>>




%%%%%%%%%%%%%%%%%%%%
\section{endnotes}
%%%%%%%%%%%%%%%%%%%%

\<configure info4ht endnotes\><<<
\Configure{theendnotes}......................2
\Configure{enoteformat}......................2
\Configure{makeenmark}.......................2

  Provide markups for the mentioned entities.

\endnoteN

  A built-in counter for end notes.
>>>


%%%%%%%%%%%%%%%%%%%%
\section{float}
%%%%%%%%%%%%%%%%%%%%

\<configure info4ht float\><<<
New float environments can be configured with the \ConfigureEnv
instruction. For instance,
   
   \newfloat{myfloat}{htb}{}[section]
   
   \ConfigureEnv{myfloat}
     {BEFORE MYFLOAT}  {AFTER MYFLOAT}
     {}{}
   
   \begin{myfloat}
    body
    \caption{title}
   \end{myfloat}
>>>


%%%%%%%%%%%%%%%%%%%%
\section{mktex4ht}
%%%%%%%%%%%%%%%%%%%%




\<configure info4ht mktex4ht\><<<
Purpose
-------

A package to help creating 4ht files for configuring seeded hooks.  

A sample self-explanatory template file make-4ht.tex is shown at the
end. The sample file can be retrieved in ascii form from the file
info4ht.4ht.

General Information
-------------------

To get a better understanding of the structural meaning of some of the
available hooks

a. Compile the mktex4ht.4ht file with the `htlatex mktex4ht.4ht' 
   command (or, `xhlatex mktex4ht.4ht' for XHTML output).

   Visit the links in the index of the outcome `mktex4ht.html' file.

b. Compile your sorce file with a command similar to

     htlatex filename "html,info" "" "-p"
    
   and check the source file.

c. Compile your sorce file with a command similar to

     htlatex filename "html,0.0,hooks" "" "-p"
    
   and view the outcome with a html browser.

d. Take a look at the *.4ht configuration files mentioned in tex4ht.4ht.

Setting up a calling command 
----------------------------

i.  Create a tex4ht.usr file, and introduce there \Configure command(s)
    similar to those found in tex4ht.4ht.  For instance,

       \Configure{foo}{%
          \Hinclude[*]{myhtml.4ht}%
          \Hinclude[*]{mymath.4ht}%
       }
     
    Choose arbitrary name in the first argument, and include *.4ht
    configuration files of your liking in the second argument.
    
    You might want to temporarily include also the file html0.4ht, as a
    first entry, while constructing new *.4ht configuration files. For
    instance,
     
       \Configure{foo}{%
          \Hinclude[*]{html0.4ht}% 
          \Hinclude[*]{myhtml.4ht}%
          \Hinclude[*]{mymath.4ht}%
       }
       
ii.  A configuration setting may be loaded by mentioning its name
     within the option list of the htlatex command. For instance,
      
        htlatex filename "xhtml,foo" "unicode/!" "-p"
     
iii. Alternatively, a htlatex-like command can be tailored to
     automatically load a configuration setting. That can be done
     by including  the name in the argument list of
     \csname tex4ht\endcsname,  within the script of the command. For
     instance, replace `docbook' with `foo' in the script of dblatex.


Defining a New Configuration File with mktex4ht.4ht 
---------------------------------------------------

1. Define a parameter-less macro \HTML to hold a comma separeted list
   of the output file names. Place the definition at the top of the
   source document.  For instance

         \def\HTML{myhtml,mymath}

   request the files myhtml.4ht and mymath.4ht.

2. Define a parameter-less macro \CONFIG to hold the name of the file
   containing the contributed configurations. Place the definition at
   the top of the source document.

3. The configurations contributed to output file YYYY.4ht for style
   XXXX should be provided within code fragments of the form

        \<configure XXXX YYYY\><<<
          .......
        >>|empty>

   Arbitrary valid LaTeX source code may be inserted around the code
   fragments.

    Example:

        Configurations for class book.cls:

        \<configure myhtml book\><<<
        \Configure{chapter} {...}{...}{...}{...}
        \Configure{section} {...}{...}{...}{...}
        >>|empty>

        Configurations for class article.cls:
  
        \<configure mymath article\><<<
        \Configure{section} {...}{...}{...}{...}
        >>|empty>

        The hooks are provided in book.4ht and article.4ht.

4. Define a parameter-less macro \MAKETITLE with contribution for
   the \maketitle command (e.g., \def\MAKETITLE{\title{my name}}). Place
   the definition at the top of the source document.

5. A file may reload itself, if it starts with preamble similar to
   the following one.

               \ifx \HTML\UnDef
                  \def\HTML{...}
                  \def\CONFIG{\jobname}
                  \def\MAKETITLE{\author{...}}
                  \def\next{\input mktex4ht.4ht   \endinput}
                  \expandafter\next
               \fi


A Template File make-4ht.tex 
-----------------------------

%--------------------- start of template -----------------
%%%%%%%%%%%%%%%%%%%%%%%%%%%%%%%%%%%%%%%%%%%%%%%%%%%%%%%%%%%%%%%%
%
%  make-4ht.4ht                         Eitan M. Gurari
%
%  A sample file with instructions for creating 4ht 
%  configuration files through literate programming.
%
%%%%%%%%%%%%%%%%%%%%%%%%%%%%%%%%%%%%%%%%%%%%%%%%%%%%%%%%%%%%%%%%
%
% *  Compile this file twice with the command `latex make-4ht' 
%    (or with `htlatex make-4ht')
%
% *  Inspect the outcome files sample.4ht and example.4ht
%
%%%%%%%%%%%%%%%%%%%%%%%%%%%%%%%%%%%%%%%%%%%%%%%%%%%%%%%%%%%%%%%%
%
% *  Change `sample,example' in (1) below to a comma-separated list
%    of your choice.  Each entry in the list stands for a name
%    of a 4ht configuration file to be created.  The current list 
%    requests the configuration files `sample.4ht' and `example.4ht'.
%
% *  Change `my title' in (2), and `my name' in (3), to a
%    content of your choice
%
% *  Go to (4)
%
%%%%%%%%%%%%%%%%%% load style files %%%%%%%%%%%%%%%%%%%%%%%%%%

\ifx \HTML\UnDef
   \def\HTML{sample,example}                     %<------------ (1)     
   \def\CONFIG{\jobname}
   \def\MAKETITLE{\title{my title}%              %<------------ (2)
        \author{my name}}                        %<------------ (3)
   \def\next{\input mktex4ht.4ht  \endinput}
   \expandafter\next
\fi

%%%%%%%%%%%%%%%%%%%%% useful definitions %%%%%%%%%%%%%%%%%%%%%

\newcount\tmpcnt  \tmpcnt\time  \divide\tmpcnt  60
\edef\temp{\the\tmpcnt}
\multiply\tmpcnt  -60 \advance\tmpcnt  \time

\edef\version{\the\year-\ifnum \month<10 0\fi
  \the\month-\ifnum \day<10 0\fi\the\day
   -\ifnum \temp<10 0\fi \temp
   :\ifnum \tmpcnt<10 0\fi\the\tmpcnt}

\def\CopyYear.#1.{%
   \ifnum #1=\year #1\space\space\space\space\space\space
    \else          #1--\the\year\fi
}


%<--------------------------------------------------------------- (4)
%
% *  The code for the configuration files is to be written within
%    fragments of the form
%
%        \<fragment name\><<<
%        fragment content
%        >>|empty>
%
% *  References from code fragments to other code fragments should
%    take the form
%
%        ||<fragment name||>
%
% *  The character || is treated as an escape character 
%    within the code fragments.  References to the
%    character should take the form
%
%        ||||
%
% *  Insertions outside the code fragments are ignored in the
%    configuration files, and they should abide to the latex 
%    conventions.
%
% *  Change all the prose and code fragments below to meet
%    your needs.  Make sure to follow the given instructions.

%%%%%%%%%%%%%%%%%%%%%%%%%%%%%%%%%%%%%%%%%%%%%%%%%%%%%%%%%%%%%%%%%%%%%%%%
\chapter{Root Code}
%%%%%%%%%%%%%%%%%%%%%%%%%%%%%%%%%%%%%%%%%%%%%%%%%%%%%%%%%%%%%%%%%%%%%%%%

%<--------------------------------------------------------------- (5)
%
% *  For each entry in the list of (1), you may have arbitrary many
%    code fragments named by the entry.
%
% *  The order of fragments is relevant only among those having
%    identical names.
%


\<sample\><<<
%%%%%%%%%%%%%%%%%%%%%%%%%%%%%%%%%%%%%%%%%%%%%%%%%%%%%%%%%  
% sample.4ht                           ||version %
||<copyright statement||>
>>|empty>


\<example\><<<
%%%%%%%%%%%%%%%%%%%%%%%%%%%%%%%%%%%%%%%%%%%%%%%%%%%%%%%%%  
% example.4ht                          ||version %
||<copyright statement||>
>>|empty>


\<copyright statement\><<<
% Copyright (C) ||CopyYear.2000.              my name         % 
%                                                       %
% My copyright statement                                %
%                                                       %
%%%%%%%%%%%%%%%%%%%%%%%%%%%%%%%%%%%%%%%%%%%%%%%%%%%%%%%%%
\immediate\write-1{version ||version}
>>|empty>


%<--------------------------------------------------------------- (6)
%  
% *  Each configuration of tex4ht requires a base 4ht configuration
%    file containing the following `Hinclude' code.  

% *  The supplied  html4.4ht, html32.4ht, html0.4ht, tei.4t, 
%    and docbook.4ht files already include such code.
%
% *  In a compilation of a file, exactly one 4ht configuration
%    file should load  `Hinclude' code.   
% 
% *  Either remove the following three code fragments, or replace 
%    `sample' in their titles with an entry from (1)
%

\<configure sample tex4ht\><<<
\if:latex  ||<Hinclude latex||>
\else      ||<Hinclude plain||>  \fi
>>|empty>

\<configure sample plain\><<<
||<Hinclude plain lib||>
||<Hinclude plain + latex lib||>
>>|empty>

\<configure sample latex\><<<
||<Hinclude latex lib||>
||<Hinclude plain + latex lib||>
>>|empty>

\endinput
%--------------------- end of template -------------------
>>>



%%%%%%%%%%%%%%%%%%%%%%%%%%%%%%%%%%%%%%%%%%%%%%%%%%%%%%%%%%%%%%%%%%%%%%%%
\chapter{Segment Code}
%%%%%%%%%%%%%%%%%%%%%%%%%%%%%%%%%%%%%%%%%%%%%%%%%%%%%%%%%%%%%%%%%%%%%%%%

%<--------------------------------------------------------------- (7)
%
% *  A 4ht configuration file is made up of segments which
%    correspond to classes and styles of tex, latex, amslatex,
%    etc.  Have a look at these code segments in html4.4ht .
%
% *  You can get additional information about the possible
%    configurations in the different segments by compiling
%    your file with the `info' switch on.  For instance
%
%           htlatex foo "html,info"
%
% *  Code for segment `xxxx' in file `yyyy.4ht' is introduced
%    through fragments named `configure  yyyy xxxx'. That is,
%    through fragments of the form
%
%           \<configure yyyy xxxx\><<<
%                ..........
%           >>|empty>
%


\<configure sample tex4ht\><<<    
....tex4ht segment in sample.4ht.....
>>|empty>

\<configure sample latex\><<< 
....latex segment in sample.4ht.....
||<more code||>
>>|empty>




\<configure example tex4ht\><<<    
....tex4ht segment in example.4ht.....
>>|empty>

\<configure example latex\><<< 
....latex segment in example.4ht.....
>>|empty>


\<more code\><<<
....additional code....
>>|empty>

%<--------------------------------------------------------------- (8)
%
% *  Compile this file twice with the command `latex filename' 
%    (or with `ht latex filename')
%
% *  If you don't have a file named tex4ht.usr, introduce such a file
%
% *  Insert into tex4ht.usr a file along the following lines, which
%    invokes  *.4ht configuration files
%
%            \Configure{my4hts}{%
%               \Hinclude[*]{html4.4ht}%
%               \Hinclude[*]{html4-math.4ht}%
%               \Hinclude[*]{myflavor.4ht}%
%            }
%
%    For additional examples inspect, but do not change, the 
%    tex4ht.4ht file.
%
% *  Invoke the compilations of you latex files with commands similar to
%
%            htlatex foo "html,my4hts"
%


\endinput
>>>






\endinput




