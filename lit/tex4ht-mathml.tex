% $Id$$
% compile 3 times: latex tex4ht-mathltx
%           or   xhlatex tex4ht-mathltx "html,3,sections+"
%
% Copyright 2009-2014 TeX Users Group
% Copyright 1999-2009 Eitan M. Gurari
% Released under LPPL 1.3c+.
% See tex4ht-cpright.tex for license text.

\ifx \HTML\UnDef
   \def\HTML{mathml,html-mml} 
   \def\CONFIG{\jobname}
   \def\MAKETITLE{\author{Eitan M. Gurari}}
   \def\next{\input mktex4ht.4ht  \endinput}
   \expandafter\next
\fi

% $Id$
% A more few common TeX definitions for literate sources.  Not installed
% in runtime.  These are only used in a few files, compared to those in
% common.tex.  Do not know if any harm would come from including them always.
% 
% Copyright 2009, 2010 TeX Users Group
% Copyright 1996--2009 Eitan M. Gurari
%
% This work may be distributed and/or modified under the
% conditions of the LaTeX Project Public License, either
% version 1.3c of this license or (at your option) any
% later version. The latest version of this license is in
%   http://www.latex-project.org/lppl.txt
% and version 1.3c or later is part of all distributions
% of LaTeX version 2005/12/01 or later.
%
% This work has the LPPL maintenance status "maintained".
%
% The Current Maintainer of this work
% is the TeX4ht Project <http://tug.org/tex4ht>.
% 
% If you modify this program, changing the 
% version identification would be appreciated.

\let\AltxModifyShowCode=\ModifyShowCode
\def\ModifyShowCode{%
   \def\by{by}\def\={=}\AltxModifyShowCode}

\let\pReModifyOutputCode=\ModifyOutputCode
\def\ModifyOutputCode{%
   \def\by{}\def\={}%
   \pReModifyOutputCode}

% $Id$
% A few common TeX definitions for literate sources.  Not installed in runtime.
% 
% Copyright 2009-2017 TeX Users Group
% Copyright 1996-2009 Eitan M. Gurari
%
% This work may be distributed and/or modified under the
% conditions of the LaTeX Project Public License, either
% version 1.3c of this license or (at your option) any
% later version. The latest version of this license is in
%   http://www.latex-project.org/lppl.txt
% and version 1.3c or later is part of all distributions
% of LaTeX version 2005/12/01 or later.
%
% This work has the LPPL maintenance status "maintained".
%
% The Current Maintainer of this work
% is the TeX4ht Project <http://tug.org/tex4ht>.
% 
% If you modify this program, changing the 
% version identification would be appreciated.

\newcount\tmpcnt  \tmpcnt\time  \divide\tmpcnt  60
\edef\temp{\the\tmpcnt}
\multiply\tmpcnt  -60 \advance\tmpcnt  \time

\edef\version{\the\year-\ifnum \month<10 0\fi
  \the\month-\ifnum \day<10 0\fi\the\day
   -\ifnum \temp<10 0\fi \temp
   :\ifnum \tmpcnt<10 0\fi\the\tmpcnt}

% a fixed-string version that can be enabled for debugging.
%\edef\versionDebug{000-00-00-00:00}
%\let\version\versionDebug

% #1 is the first year for Eitan.  The last year is always 2009.  RIP.
\def\CopyYear.#1.{#1-2009}

% command for write to terminal and the log file
% this version is used in the .4ht files build
% identical command is defined also in tex4ht-sty.tex, 
% it is used in TeX document compilation
\def\writesixteen#1{\immediate\write1616{#1}}

\<TeX4ht copyright\><<<
%
% This work may be distributed and/or modified under the
% conditions of the LaTeX Project Public License, either
% version 1.3c of this license or (at your option) any
% later version. The latest version of this license is in
%   http://www.latex-project.org/lppl.txt
% and version 1.3c or later is part of all distributions
% of LaTeX version 2005/12/01 or later.
%
% This work has the LPPL maintenance status "maintained".
%
% The Current Maintainer of this work
% is the TeX4ht Project <http://tug.org/tex4ht>.
% 
% If you modify this program, changing the 
% version identification would be appreciated.
>>>


%%%%%%%%%%%%%%%%%%%%%%%%%%%%%%%%%%%%%%%%%%%%%%%%%%%%%%%%%%%%%%%%
\chapter{Constructs}
%%%%%%%%%%%%%%%%%%%%%%%%%%%%%%%%%%%%%%%%%%%%%%%%%%%%%%%%%%%%%%%%

%%%%%%%%%%%%%%%%%%%%%%%%%%%%%%%%%%%%%%
\section{The equation Environment}
%%%%%%%%%%%%%%%%%%%%%%%%%%%%%%%%%%%%%%


%%%%%%%%%%%%%%%%%%%%%%%%%%%%%%%%%%%%%%
\subsection{LaTeX}
%%%%%%%%%%%%%%%%%%%%%%%%%%%%%%%%%%%%%%


\<mathml latex\><<<
\let\A:equation|=\a:equation
\let\B:equation|=\b:equation
\let\C:equation|=\c:equation
\Configure{equation}
  {\A:equation|<math display in equation|>}
  {\B:equation}{\C:equation}
>>>

\<configure html-mml latex\><<<
\Configure{equation}
  {\IgnorePar|<inline end-par comment|>\EndP
   \HCode{<table class="equation"><tr><td>}\IgnorePar
     |<math display in equation|>}
  {\ifvmode \IgnorePar\fi\EndP\HCode{</td><td class="eq-no">}}
  {\HCode{</td></tr></table>}|<try inline par|>}
\Css{.equation td{text-align:center; vertical-align:middle; }}
\Css{td.eq-no{ width:5\%; }}
\Css{table.equation { width:100\%; }   }
>>>

\<math display in equation\><<<
\Configure{$$}
   {\Configure{@math}{|<block display|>%
                       \mml:class="equation"}\IgnorePar\EndP\DviMath}
   {\EndDviMath\EndP}
   {\DisplayMathtrue}%
>>>



\<block display\><<<
%\a:mathml
 display="block"
>>>


\<inline display\><<<
%\a:mathml
 display="inline"
>>>



%%%%%%%%%%%%%%%%%%%%%%%%%%%%%%%%%%%%%%
\subsection{AmsMath}
%%%%%%%%%%%%%%%%%%%%%%%%%%%%%%%%%%%%%%



\<amsmath mathml\><<<
\Configure{equation}
  {\Configure{gather}
     {\HCode{<\a:mathml mtable\Hnewline
        \mml:class="equation"><\a:mathml mtr><\a:mathml mtd>}\IgnorePar }
     {\HCode{</\a:mathml mtd></\a:mathml mtr></\a:mathml mtable>}}
     {}
     {}
     {\ifnum\HCol=2 \IgnorePar
         \HCode{</\a:mathml mtd><\a:mathml mtd>}\fi }
     {}
  }{}{}
\Css{.equation-star td{text-align:center; vertical-align:middle; }}
\Css{table.equation-star { width:100\%; }   }
>>>



\<amsmath html-mml pre 2000\><<<
\Configure{equation}
  {\ifvmode \IgnorePar\fi \EndP
   \HCode{<table class="equation"><tr><td>}%
   |<amsmath equation pre 2000|>%
   \IgnorePar
  }
  {\ifvmode \IgnorePar\fi\EndP \HCode{</td><td class="eq-no">}}
  {\ifvmode \IgnorePar\fi\EndP \HCode{</td></tr></table>}%
   \IgnoreIndent\par}
\Configure{equation*}
  {\ifvmode \IgnorePar\fi \EndP
   \HCode{<table class="equation-star"><tr><td>}%
   |<amsmath equation pre 2000|>%
   \IgnorePar
  }
  {\ifvmode \IgnorePar\fi\EndP \HCode{</td></tr></table>}%
   \IgnoreIndent\par}
>>>


\<amsmath html-mml jan 2000\><<<
\Configure{equation}
  {\ifvmode \IgnorePar\fi \EndP
   \HCode{<table class="equation"><tr><td>}%
   |<amsmath equation jan 2000|>%
   \IgnorePar
  }
  {\ifvmode \IgnorePar\fi\EndP 
   \HCode{</td><td class="eq-no">}%
  }
  {\ifvmode \IgnorePar\fi\EndP
   \HCode{</td></tr></table>}%
   \IgnoreIndent\par}
\Configure{equation*}
  {\ifvmode \IgnorePar\fi \EndP
   \HCode{<table class="equation-star"><tr><td>}%
   |<amsmath equation jan 2000|>%
   \IgnorePar
  }
  {\ifvmode \IgnorePar\fi\EndP
   \HCode{</td></tr></table>}%
   \IgnoreIndent\par}
>>>


\<amsmath html-mml july 2000\><<<
\Configure{equation}
  {\ifvmode \IgnorePar\fi\EndP
   \HCode{<table class="equation"><tr><td>}%
   |<amsmath equation july 2000|>%
   |<math display in equation|>%
  }
  {\ifvmode \IgnorePar\fi\EndP
   \HCode{</td><td class="eq-no">}%
   }
  {\ifvmode \IgnorePar\fi\EndP \HCode{</td></tr></table>}%
   \ShowPar\IgnoreIndent\par}
\Configure{equation*}
  {\ifvmode \IgnorePar\fi\EndP
   \HCode{<table class="equation-star"><tr><td>}%
   |<amsmath equation july 2000|>%
   |<math display in equation|>%
  }
  {\ifvmode \IgnorePar\fi\EndP \HCode{</td></tr></table>}%
   \ShowPar\IgnoreIndent\par}
>>>

%      <\a:mathml math\Hnewline
%        \csname a:math-xmlns\endcsname \Hnewline display="inline">

% \HCode{</\a:mathml math>}%
   


\<amsmath equation jan 2000\><<<
|<amsmath equation july 2000|>%
>>>







\<amsmath equation july 2000\><<<
\Configure{gather}
     {\HCode{<\a:mathml mtable\Hnewline
        \mml:class="equation"><\a:mathml mtr><\a:mathml mtd>}\IgnorePar 
         |<math display in equation|>}
     {\HCode{</\a:mathml mtd></\a:mathml mtr></\a:mathml mtable>}}
     {}
     {}
     {\ifnum\HCol=2 \IgnorePar
         \HCode{</\a:mathml mtd><\a:mathml mtd>}\fi }
     {}
>>>

\<amsmath equation pre 2000\><<<
\Configure{$$}{}{}{}%
\Configure{gather}
  {\Configure{$$}
    {\Configure{@math}{\mml:class="equation"}\IgnorePar\EndP\DviMath}
    {\EndDviMath\EndP}%
    {}%
   \IgnorePar
  }{}{}{}{}{}
>>>









%%%%%%%%%%%%%%%%%%%%%%%%%%%%%%%%%%%%%%%%%%%%%%%%%%%%%%%%%%%%%%%%
\chapter{MathML}
%%%%%%%%%%%%%%%%%%%%%%%%%%%%%%%%%%%%%%%%%%%%%%%%%%%%%%%%%%%%%%%%
%%%%%%%%%%%%%%%%%
\section{Front Matters}
%%%%%%%%%%%%%%%%%



%%%%%%%%%%%%%
\subsection{DTD}
%%%%%%%%%%%%%

\<configure mathml Preamble\><<<  
\Configure{PROLOG}{VERSION,DOCTYPE,*XML-STYLESHEET}
\:CheckOption{no-DOCTYPE} \if:Option
   \Configure{PROLOG}{VERSION,*XML-STYLESHEET}
\fi
>>>


% \def\mathml:DOCTYPE{|<xhtml+mathml dtd|>}

\<configure html-mml tex4ht\><<<
\:CheckOption{frames}  \if:Option 
      \Configure{frames-altDOCTYPE}{|<mathml dtd|>}
   \else       
      |<no frames DOCTYPE|>
\fi
>>>



\<no frames DOCTYPE\><<<
\def\:temp{xhtml}
\ifx \:DTD\:temp
   |<mathml dtd|>
\fi
>>>

% 

\<xhtml+mathml dtd\><<<
PUBLIC
     "-//W3C//DTD XHTML 1.1 plus MathML 2.0//|<dtd lang|>"\Hnewline
     "http://www.w3.org/Math/DTD/mathml2/xhtml-math11-f.dtd"
>>>

\<mathml dtd\><<<
\Configure{DOCTYPE}{\HCode
  {<!DOCTYPE html |<xhtml+mathml dtd|>%
   >\Hnewline
}}
>>>

\<frames dtd\><<<
\Configure{DOCTYPE}
   {\HCode{<!DOCTYPE html PUBLIC
          "-//W3C//DTD XHTML 1.0 Frameset//|<dtd lang|>"
    \Hnewline
     "http://www.w3.org/TR/xhtml1/DTD/xhtml1-frameset.dtd">\Hnewline}}
>>>





DTD to be used.

\begin{description}



\item [http://www.w3.org/TR/MathML2/dtd/xhtml-math11-f.dtd]

Declares a  FPI (Formal Public Identifier)
``-//W3C//DTD XHTML 1.1 plus MathML 2.0//EN'' for XHTML+MathML. 
The DTD is problematic
since it contains a processing instruction
\verb!<?IS10744:arch xhtml ...>!  with a colon in the name. XML 
parsers don't allow it when namespaces are enabled. It carries 
revision id `xhtml-1,v 1.1 2000/10/31 19:52:09 davidc Exp'.

\item[http://www.w3.org/Math/DTD/mathml2/xhtml-math11-f.dtd]

Suggested by the MathML 
specification
 http://www.w3.org/TR/2003/REC-MathML2-20031021/appendixa.html.  However, the
DTD incorporates also SVG.  It shows revision id
`xhtml-math11-f.dtd,v 1.16 2003/11/04 13:14:35 davidc Exp'

\item [others]

See SVG specifications
\end{description}

The DTD 
a   "-//W3C//DTD XHTML 1.1 plus MathML 
2.0//EN" is claimed: 





\<dtd lang\><<<
\expandafter
\ifx \csname a:dtd-lang\endcsname\relax EN\else
  \csname a:dtd-lang\endcsname
\fi
>>>


\<xmlns\><<<
xmlns="http://www.w3.org/1999/xhtml"
>>>









%%%%%%%%%%%%%
\subsection{???????????}
%%%%%%%%%%%%%




\<configure mathml Preamble\><<<
\ifx \a:mathml\:UnDef
   \NewConfigure{mathml}{1}
   \Configure{mathml}{}
\fi 
\def\mml:class{\a:mathml class}
>>>


\<plain,latex mathml\><<<
|<nolimit functions|>
|<limit functions|>
>>>


\<\><<<
\def\mml:prefix#1{%
   \Configure{mathop}{*}{<\a:mathml mo #1\Hnewline>}{</\a:mathml mo>}%
    {\Configure{mathop}{*}{<\a:mathml mo>}{</\a:mathml mo>}{}}}
\def\mml:prefix#1{}
\def\limit:csname{|%form="prefix" |%\mml:class="csname-limit"}
\def\nolimit:csname{|%form="prefix" |%\mml:class="csname-nolimit"}
>>>


\<nolimit functionsNO\><<<
\def\:tempb#1{%
  \def\:tempc##11->##2///{\if :##2:%
    \pend:def#1{\mml:prefix \nolimit:csname}\fi}%
  \expandafter\:tempc\meaning#11->///%
}
\:tempb\arccos
\:tempb\arcsin
\:tempb\arctan
\:tempb\arg
\:tempb\cosh
\:tempb\cos
\:tempb\coth
\:tempb\cot
\:tempb\csc
\:tempb\deg
\:tempb\dim
\:tempb\exp
\:tempb\hom
\:tempb\ker
\:tempb\lg
\:tempb\ln
\:tempb\log
\:tempb\sec
\:tempb\sinh
\:tempb\sin
\:tempb\tanh
\:tempb\tan
|<lim inf/sup|>
>>>

\<limit functionsNO\><<<
\def\:tempb#1{%
  \def\:tempc##11->##2///{\if :##2:%
    \pend:def#1{\mml:prefix \limit:csname}\fi}%
  \expandafter\:tempc\meaning#11->///%
}
\:tempb\Pr
\:tempb\det
\:tempb\gcd
\:tempb\inf
\:tempb\lim
\:tempb\max
\:tempb\min
\:tempb\sup
>>>

\<lim inf/supNO\><<<
\def\:tempc{\mml:prefix\limit:csname \mathop{\operator@font lim inf}}
\HLet\liminf\:tempc
\def\:tempc{\mml:prefix\limit:csname \mathop{\operator@font lim sup}}
\HLet\limsup\:tempc
>>>


\<plain,latex mathml\><<<
\let\A:displaylines|=\a:displaylines
\let\B:displaylines|=\b:displaylines
\let\C:displaylines|=\c:displaylines
\let\D:displaylines|=\d:displaylines
\Configure{displaylines}
   {\ifmmode
     \def\A:displaylines{\HCode{<\a:mathml mtable 
                                   \mml:class="displaylines">}}%
     \def\B:displaylines{\HCode{</\a:mathml mtable>}}%  
     \def\C:displaylines{\HCode{<\a:mathml mtr\Hnewline
                              \mml:class="displaylines"><\a:mathml mtd>}}%
     \def\D:displaylines{\HCode{</\a:mathml mtd></\a:mathml mtr>}}%  
    \fi 
    \A:displaylines} 
   {\B:displaylines} {\C:displaylines}{\D:displaylines}
>>>



\<mathml m:env\><<<
\def\m:env#1{\HCode{<\a:mathml mtable
  \mml:class="#1"><\a:mathml mtr><\a:mathml mtd>}%
   \def\endm:env{\HCode
      {</\a:mathml mtd></\a:mathml mtr></\a:mathml mtable>}}}
>>>



\<amstex.tex mathml\><<<
\Configure{pmatrix}
   {\bgroup\m:env{pmatrix}}
   {\endm:env\egroup}
\Configure{multline}
   {\HCode{<\a:mathml mtable\mml:class="multline">}} 
   {\HCode{</\a:mathml mtable>}}
   {\HCode{<\a:mathml mtr\Hnewline\mml:class="multline"><\a:mathml mtd>}}
   {\HCode{</\a:mathml mtd></\a:mathml mtr>}}
\Configure{smallmatrix}
   {\HCode{<\a:mathml mtable\mml:class="smallmatrix">}}  
   {\HCode{</\a:mathml mtable>}}
   {\HCode{<\a:mathml mtr\Hnewline\mml:class="smallmatrix">}} 
   {\HCode{</\a:mathml mtr>}}
   {\HCode{<\a:mathml mtd\Hnewline\mml:class="smallmatrix">}}
   {\HCode{</\a:mathml mtd>}}
\Configure{bmatrix}
   {\bgroup\m:env{bmatrix}}
   {\endm:env\egroup}
\Configure{vmatrix}{\bgroup\m:env{vmatrix}}{\endm:env\egroup}
\Configure{Vmatrix}{\bgroup\m:env{Vmatrix}}{\endm:env\egroup}
\Configure{aligned}
   {\HCode{<\a:mathml mtable\mml:class="aligned">}} 
   {\HCode{</\a:mathml mtable>}}
   {\HCode{<\a:mathml mtr\Hnewline\mml:class="aligned">}}
   {\HCode{</\a:mathml mtr>}}
   {\HCode{<\a:mathml mtd\Hnewline\mml:class="aligned">}}
   {\HCode{</\a:mathml mtd>}}
\Configure{xalignat}
   {\HCode{<\a:mathml mtable\mml:class="xalignat">}}
   {\HCode{</\a:mathml mtable>}}
   {\HCode{<\a:mathml mtr\Hnewline \mml:class="xalignat">}}
   {\HCode{</\a:mathml mtr>}}
   {\HCode{<\a:mathml mtd\Hnewline  \mml:class="xalignat">}}
   {\HCode{</\a:mathml mtd>}}
\Configure{xxalignat}
   {\HCode{<\a:mathml mtable  \mml:class="xxalignat">}}
   {\HCode{</\a:mathml mtable>}}
   {\HCode{<\a:mathml mtr\Hnewline  \mml:class="xxalignat">}}
   {\HCode{</\a:mathml mtr>}}
   {\HCode{<\a:mathml mtd\Hnewline  \mml:class="xxalignat">}}
   {\HCode{</\a:mathml mtd>}}
\Configure{alignedat}
   {\HCode{<\a:mathml mtable \mml:class="alignedat">}} 
   {\HCode{</\a:mathml mtable>}}
   {\HCode{<\a:mathml mtr\Hnewline  \mml:class="alignedat">}}
   {\HCode{</\a:mathml mtr>}}
   {\HCode{<\a:mathml mtd\Hnewline \mml:class="alignedat">}}
   {\HCode{</\a:mathml mtd>}}
\Configure{gathered}
   {\HCode{<\a:mathml mtable \mml:class="gathered">}}
   {\HCode{</\a:mathml mtable>}}
   {\HCode{<\a:mathml mtr\Hnewline \mml:class="gathered">}}
   {\HCode{</\a:mathml mtr>}}
   {\HCode{<\a:mathml mtd\Hnewline \mml:class="gathered">}}
   {\HCode{</\a:mathml mtd>}}
>>>


\<mathml\><<<
% mathml.4ht (|version), generated from |jobname.tex
% Copyright 2009-2014 TeX Users Group
% Copyright |CopyYear.1999. Eitan M. Gurari
|<TeX4ht copywrite|>
>>>

\<html-mml\><<<
% html-mml.4ht (|version), generated from |jobname.tex
% Copyright 2009-2014 TeX Users Group
% Copyright |CopyYear.1999. Eitan M. Gurari
|<TeX4ht copywrite|>
>>>




\<configure mathml tex4ht\><<<
|<tex compiler mathml|>
|<sub, sup, subsup|>            
|<limits sub, sup, subsup|>     
>>>










\<html4 font decorations\><<<
\Configure{htf}{0}{+}{<span\Hnewline
   class="}{\%s}{-\%s}{x-x-\%d}{}{">}{</span>}%
\Configure{htf}{1}{+}{<img\Hnewline
   src="}{" alt="}{" class="}{\%s}{-\%d}{x-x-\%x}{"\xml:empty>}%
\Configure{htf}{3}{+}{<img\Hnewline                                
   src="}{" alt="}{" class="\%s-}{\%s}{-\%d}{x-x-\%x}%
   {" style="text-align:middle"\xml:empty>}%
\Configure{htf}{4}{+}{<small\Hnewline
   class="}{}{}{}{}{small-caps">}{</small>}%
\Configure{htf}{6}{+}{<u\Hnewline
   class="}{}{}{}{}{underline">}{</u>}%
\Configure{htf}{8}{+}
   {<sup class="htf"><strong>}{}{}{}{}{}{</strong></sup>}%
\Configure{htf}{10}{+}
   {<span class="htf-cmbx">}{}{}{}{}{}{</span>}%
\Configure{htf}{12}{+}
   {<span\Hnewline class="htf-calligraphy">}{}{}{}{}{}{</span>}%
>>>

\<no html4 font decorations\><<<
\Configure{htf}{0}{+}{}{}{}{}{}{}{}%
\Configure{htf}{1}{+}{}{}{}{}{}{}{}%
\Configure{htf}{3}{+}{}{}{}{}{}{}{}%
\Configure{htf}{4}{+}{}{}{}{}{}{}{}%
\Configure{htf}{6}{+}{}{}{}{}{}{}{}%
\Configure{htf}{8}{+}{}{}{}{}{}{}{}%
\Configure{htf}{10}{+}{}{}{}{}{}{}{}%
\Configure{htf}{12}{+}{}{}{}{}{}{}{}%
>>>





\<configure mathml tex4ht\><<<
\def\mml:htfsym{\Configure
   {htf}{254}{+}{<\a:mathml mglyph\Hnewline fontfamily="}{\%s"}{ 
   index="\%s"}{ mathsize="\%d\%\%"}{}{ alt="}{"/>}}
\def\ext:htfsym{\Configure{htf}{254}{+}{}{}{}{}{}{}{}}
>>>


\<configure mathml-mml tex4ht\><<<
\append:def\mml:htfsym{|<no html4 font decorations|>}
\append:def\ext:htfsym{|<html4 font decorations|>}
>>>


We reserve code 254 (and 255) for unknown symbols.



\<configure mathml latex\><<<
\:CheckOption{core}\if:Option \expandafter\endinput \fi
|<mathml latex|>
|<plain,latex mathml|>
>>>

\<configure mathml fontmath\><<<
|<math fonts utilities|>
|<set mathcal class|>
|<set mathcal symbols|>
|<mathml plain,fontmath|>
>>>

\<configure mathml plain\><<<
\:CheckOption{core}\if:Option \expandafter\endinput \fi
|<mathml plain|>
|<mathml plain,fontmath|>
|<plain,latex mathml|>
>>>

\<mathml plain,fontmath\><<<
|<rlbabel everypar|>
|<plain,fontmath math classes|>
>>>

\<configure mathml amsmath\><<<
|<arr amsmath, amstex1|>
|<amsmath mathml|>
|<amsmath symbols|>
>>>

\<configure mathml amsmath\><<<
\Configure{intertext}
    {\IgnorePar \ifinalign@   
        \HCode{\ifnum \HRow>0 </\a:mathml mtd></\a:mathml mtr>\fi
          <\a:mathml mtr><\a:mathml mtd 
          columnspan="\ifnum \HRow=0 15\else \HCol\fi"
                                            \mml:class="intertext">}%
     \else \ifingather@
        \HCode{<\a:mathml mtr><\a:mathml mtd columnspan="\HCol"
                                            \mml:class="intertext">}%
     \fi\fi
     \AA:text{intertext}%
    }
    {\BB:text \ifvmode \IgnorePar \fi \EndP\ifinalign@ 
        \HCode{</\a:mathml mtd></\a:mathml mtr>%
           \ifnum \HRow>0 <\a:mathml mtr><\a:mathml mtd>\fi}%
     \else \ifingather@
         \HCode{</\a:mathml mtd></\a:mathml mtr>}%
     \fi\fi
    }
>>>



\<configure mathml amsmath\><<<
\Configure{maketag}
  {\ifmathml 
     \ifmtext \expandafter\expandafter\expandafter\no:maketag\fi
   \else \expandafter\no:maketag\fi
   \AA:text{maketag}\hbox\bgroup\:gobbleIII :@:}
  {\egroup\BB:text }
\def\no:maketag#1:@:#2\b:maketag{\hbox{#2}}
>>>


\<configure mathml amstext\><<<
\Configure{text}{\AA:text{text}}{\BB:text}
>>>


\<amsmath symbols\><<<
\def\:tempc{\DOTSI\protect\:idotsint}
\HLet\idotsint|=\:tempc
\def\:idotsint{%
  \mathop{\intop\cdotp\cdotp\cdotp\intop }\ilimits@
}
>>>

\<configure mathml amsopn\><<<
|<amsopn / amstex.sty mathml|>
>>>

\<configure mathml amstex1\><<<
% amstex.sty
|<arr amsmath, amstex1|>
|<amstex.sty mathml|>
|<amsopn / amstex.sty mathml|>
  >>>

\<configure mathml amstex\><<<
% amstex.tex
|<mathml m:env|>
|<amstex.tex mathml|>
>>>

  \<configure mathml vanilla\><<<
      |<vanilla mathml|>
  >>>

     \<plain,latex mathml\><<<
\let\mathml:on|=\relax
\Configure{ }
   {\ifmathml \ifmtext \space \else\Tg<\a:mathml mspace 
    width="1em" \mml:class="nbsp" />\fi
    \else \:nbsp\fi} 
>>>


\<mathml latex\><<<
\Configure{hspace}
  {\ifmathml\ifmtext\else \Tg<\a:mathml mspace width="\the\tmp:dim"/>\fi\fi}{}{ }
>>>




\<mathml latex\><<<
\Configure{romannumeral}
   {\ifmathml\ifmtext\else \HCode{<\a:mathml mrow  \mml:class="romannumeral">}\fi\fi}
   {\ifmathml\ifmtext\else \HCode{</\a:mathml mrow>}\fi\fi}
>>>

\<configure html-mml latex\><<<
\Configure{romannumeral}
   {\ifmtext\else
      \HCode{<\ifmathml \a:mathml mrow \else span \fi \mml:class="romannumeral">}\fi
   }
   {\ifmtext\else \HCode{</\ifmathml \a:mathml mrow\else span\fi>}\fi }
>>>


\<configure html-mml latex\><<<
\Configure{hspace}
  {\ifmathml \ifmtext \else \Tg<\a:mathml mspace width="\the\tmp:dim"/>\fi\fi}
  {}{\ifmathml  \else \:nbsp\fi}
>>>


  \<configure html-mml plain\><<<
|<plain,latex html-mml|>
>>>


\<configure html-mml latex\><<<
|<plain,latex html-mml|>
>>>

\<plain,latex mathml\><<<
\:CheckOption{new-accents}     \if:Option \else
  \Configure{accent}\widetilde\widetilde{{}{}} 
  {}
  {\HCode{<\a:mathml mover\Hnewline accent="false"><\a:mathml mrow\Hnewline>}#2\HCode{</\a:mathml mrow>}%
    \:widetilde:over  \HCode{</\a:mathml mover>}}
  \Configure{accent}\widehat\widehat{{}{}} 
  {}
  {\HCode{<\a:mathml mover\Hnewline accent="false"><\a:mathml mrow\Hnewline>}#2\HCode{</\a:mathml mrow>}%
    \:widehat:over  \HCode{</\a:mathml mover>}}
\fi
>>>

\<plain,latex html-mml\><<<
\:CheckOption{new-accents}     \if:Option \else
   \ifx \mathml:on\:UnDef
      \Configure{accents}
        {\ifmmode \expandafter\mathord\fi
         {|<temp hcode accents|>%
         \ht:special{t4ht@[}\if!#1!x\else#2\fi\ht:special{t4ht@]}}}
        {\ifmmode 
             \Tg<\a:mathml mrow
                \mml:class="accent#1">\mathord{#3}\Tg</\a:mathml mrow>%
          \else \HCode{<span class="accent#1">}#3\HCode{</span>}%
          \fi
        }%
   \fi
\fi
>>>



     \<plain,latex mathml\><<<
\:CheckOption{new-accents}     \if:Option \else
   \ifx \mathml:on\:UnDef
      \Configure{accents}
        {\ifmmode \expandafter\mathord\fi
         {|<temp hcode accents|>%
          \ht:special{t4ht@[}\if!#1!x\else#2\fi\ht:special{t4ht@]}}}
        {\Tg<\a:mathml mrow
                \mml:class="accent#1">\mathord{#3}\Tg</\a:mathml mrow>%
        }%
   \fi
\fi
>>>







\<temp hcode accents\><<<
\leavevmode 
\if !#1!\ht:special{t4ht@+\string&{35}x#2;}%
\else \expandafter \ifx\csname U#2#1\endcsname\relax
    \ht:special{t4ht@+\string&{35}x#2;}%
\else
    \ht:special{t4ht@+\string&{35}x\csname U#2#1\endcsname;}%
\fi \fi x%
>>>


The special replaces the `x' character.



\<plain,latex mathml\><<<
\let\mml:TeX|=\TeX
\def\TeX{\ifmathml 
   \HCode{<\a:mathml mtext
      \mml:class="TeX">TeX</\a:mathml mtext>}\else \mml:TeX\fi}
>>>

%%%%%%%%%%%%%%%%%%%%%%%%%
\subsection{TeX Engine}
%%%%%%%%%%%%%%%%%%%%%%%%%

\<sub, sup, subsup\><<<
\Configure{SUB}
   {\Send{BACK}
     {<\a:mathml msub><\a:mathml mrow\Hnewline>}%
     \Tg</\a:mathml mrow>\Tg<\a:mathml mrow\Hnewline>\trap:base}
   {\Tg</\a:mathml mrow>\TG</\a:mathml msub\Hnewline>}
\Configure{SUP}
   {\Send{BACK}
      {<\a:mathml msup><\a:mathml mrow\Hnewline>}%
      \Tg</\a:mathml mrow>\Tg<\a:mathml mrow\Hnewline>\trap:base}
   {\Tg</\a:mathml mrow>\TG</\a:mathml msup\Hnewline>}
\Configure{SUBSUP}
   {\Send{BACK}
      {<\a:mathml msubsup><\a:mathml mrow\Hnewline>}%
      \Tg</\a:mathml mrow>\Tg<\a:mathml mrow\Hnewline>\trap:base}
   {\Tg</\a:mathml mrow>\Tg<\a:mathml mrow\Hnewline>\trap:base}
   {\Tg</\a:mathml mrow>\TG</\a:mathml msubsup\Hnewline>}
\def\trap:base{{\HCode{}}}
>>>

The \verb'\trap:base' is to catch empty bases of exponents like, e.g.,
in \verb'$a^{^b}$'.


\<mathml Link\><<<
\LinkCommand\mtxt:link{\a:mathml mtext,%
   xlink:type="simple"
   xmlns:xlink="http://www.w3.org/1999/xlink" 
   xlink:href,
   id, \mml:class="label",,}
\LinkCommand\msp:link{\a:mathml mstyle,%
   xlink:type="simple"
   xmlns:xlink="http://www.w3.org/1999/xlink"
   xlink:href,
   id, \mml:class="label",,}
\let\M:Link|=\Link
\def\Link{\ifmathml
     \ifmtext |<end mtext space|>\HCode{</\a:mathml mtext>}%
              \expandafter\expandafter\expandafter\mtxt:link
     \else    \expandafter\expandafter\expandafter\msp:link\fi
   \else \expandafter\M:Link\fi}
\let\M:EndLink|=\EndLink
\edef\:temp{\catcode`\#=\the\catcode`\#}
\catcode`\#=12
   \def\EndLink{\ifmathml 
      \ifmtext    |<end mtext space|>%
          \Endmtxt:link \HCode{<\a:mathml mtext\Hnewline 
                  \mml:class="endlabel">}|<start sharp mtext space|>%
           \else  \HCode{</\a:mathml mstyle><!--endlabel-->}\fi
       \else \expandafter\M:EndLink\fi}  
\:temp
>>>

For some unclear reasom Mozilla requests mtext the link.

\begin{verbatim} <mrow xlink:type="simple"
xmlns:xlink="http://www.w3.org/1999/xlink" xlink:href="..."><mtext>
...</mtext></mrow> \end{verbatim}




\<tex compiler mathml\><<<
\Configure{DviMath}
    {\a:DviMathML \mathmltrue
       \ifOption{mml-fonts}{}{|%\mml:htfsym|%\NoFonts}%
       \IgnoreRule \HCode{<!--l.
       \the\inputlineno-->}\Tg<\a:mathml math\Hnewline
           \csname a:math-xmlns\endcsname \Hnewline \a:@math>\a:math
       |<sv dvimath par|>\IgnorePar}
    {\Tg</\a:mathml math>\EndIgnoreRule
     \ifOption{mml-fonts}{}{|%\ext:htfsym|%\EndNoFonts}%
     \mathmlfalse \b:DviMathML |<recall dvimath par|>}
\NewConfigure{DviMathML}{2}
\NewConfigure{math-xmlns}{1}
\Configure{math-xmlns}
  { xmlns="http://www.w3.org/1998/Math/MathML" }
>>>


\<configure html-mml tex4ht\><<<
\Configure{DviMath}
    {\a:DviMathML \mathmltrue
       \ifOption{mml-fonts}{|<mml fonts|>}{|%\mml:htfsym|%\NoFonts}%
       \IgnoreRule \HCode{<!--l.
       \the\inputlineno-->}\Tg<\a:mathml math\Hnewline
           \csname a:math-xmlns\endcsname \Hnewline \a:@math>\a:math
       |<sv dvimath par|>\IgnorePar}
    {\Tg</\a:mathml math>\EndIgnoreRule
     \ifOption{mml-fonts}{|<html fonts|>}{|%\ext:htfsym|%\EndNoFonts}%
     \mathmlfalse \b:DviMathML |<recall dvimath par|>}
\Css{\a:mathml math { text-indent: 0em; }}
>>>

\<html fonts\><<<
\Configure{htf}{0}{+}{<span\Hnewline
   class="}{\%s}{-\%s}{x-x-\%d}{}{">}{</span>}%
>>>

\<mml fonts\><<<
\Configure{htf}{0}{+}{<mstyle\Hnewline
   class="}{\%s}{-\%s}{x-x-\%d}{}{">}{</mstyle>}%
>>>

\<tex compiler mathml\><<<
\newif\ifmathml \mathmlfalse
\newif\ifmtext \mtextfalse
|<mathml Link|>
\NewConfigure{math}{1}
\NewConfigure{@math}{1}
\def\IgnoreRule{\ht:special{t4ht@\string_}}
\def\EndIgnoreRule{\ht:special{t4ht@\string_\string_}}
\Configure{$$} 
   {\Configure{@math}{|<block display|>}\IgnorePar
    |<inline end-par comment|>\EndP\DviMath}
   {\EndDviMath\ShowPar\par{\csname HCondtrue\endcsname\noindent}}
   {\DisplayMathtrue}
\Configure{$}
   {\Configure{@math}{|<inline display|>}\DviMath}
   {\EndDviMath}{}
\newif\ifDisplayMath  \DisplayMathfalse
>>>


The \verb+\sqrt+ fanction produces a structure of the following form.

\begin{verbatim}
PUSH
   PUSH
      ......sqrt et al symbols, possibly with PUSH-POP......
   POP
   PUSH
      W3:       655361
      DOWN3:    -1020474
      PUT_RULE: height: 26213
                length: 1091179
      DOWN3:    1020474
      PUSH
         .....body.........
      POP
   POP
POP
\end{verbatim}



\<tex compiler mathml\><<<
\Configure{radical}
  {\Tg<\a:mathml msqrt>%
   \special{t4ht\string~!e<<\a:mathml mrow>}%
   \special{t4ht\string~!e></\a:mathml mrow></\a:mathml msqrt>}%
   \special{t4ht\string~!ee/}%
   \special{t4ht\string~!ese-}%
  }
>>>

The following fails when the root symbol is made up
from multiple characters.

\begin{verbatim}
\Configure{radical}{\Tg<\a:mathml msqrt>%
   \Send{EndGROUP}{2}{}%
   \Send{GROUP}{3}{<\a:mathml mrow>}%
   \Send{EndGROUP}{3}%
        {\tg</\a:mathml mrow>\tg</\a:mathml msqrt>}\special{t4ht@+}}
\end{verbatim}




\<inline end-par comment\><<<
\ifvmode\else \HCode{<!--tex4ht:inline-->}\fi
>>>


\<?\><<<
\def\MathRow#1{%
   \Configure{\expandafter\:gobble\string#1*}{*}%
      {<\a:mathml mrow\Hnewline 
         \mml:class="\expandafter\:gobble\string#1">}{</\a:mathml mrow>}%
      {\Configure{\expandafter\:gobble\string#1}{}{}{}{}}#1}%
>>>


\<recall dvimath par\><<<
\sv:ignore
>>>

\<sv dvimath par\><<<
\edef\sv:ignore{\if:nopar  
    \noexpand\IgnorePar\else \noexpand\ShowPar\fi}%
>>>


The \verb'\MathRow' requests a \verb'<\a:mathml mrow\Hnewline>...</\a:mathml mrow>', instead of the contributions
of \verb'\mathop', \verb'\mathrel',...., for the next parameter.


\<tex compiler mathml\><<<
\def\:MT{\ifmathml \a:mathml m\fi}
\Configure{halignTB}{\HCode{<\:MT table }\halignTBL}{\HCode{>}}
\Configure{halignTBL}
   {t}{}
   {b}{}
   {c}{}
   {}
\def\t:HA{\HCode{</\:MT table>}}
\def\R:HA{\HCode{<\:MT tr \Hnewline style="vertical-align:baseline">}}
\def\r:HA{\HCode{</\:MT tr>}}
\def\D:HA{|<td save EndP|>\HCode{<\:MT td 
                        \ifnum \HMultispan>1 colspan="\HMultispan"\fi}%
   \halignTD \HCode{\Hnewline>}}
\def\d:HA{\HCode{</\:MT td>}|<td recall EndP|>}
>>>

\<configure html-mml tex4ht\><<< 
\Configure{halignTBL}
   {t}{\ifmathml \HCode{ align="baseline 1" style="|<border lines|>" }\fi}
   {b}{\ifmathml \HCode{ align="baseline -1" style="|<border lines|>" }\fi}
   {c}{\ifmathml \HCode{ align="axis" style="|<border lines|>" }\fi}
   {}
>>>


%%%%%%%%%%%%%%%%%%%%%%%%
\subsection{latex.ltx}
%%%%%%%%%%%%%%%%


MathML: 
\begin{verbatim}
Authors wishing to encode whitespace characters at the start
or end of the content of a token, or in sequences other than a single
space, without having them ignored, must use &nbsp; or other
`whitespace' non-marking entities as described in Section 6.2.4
[Non-Marking Characters].
\end{verbatim}

\<start mtext space\><<<
\ht:special{t4ht@,&\#x00A0;}%
>>>

\<start sharp mtext space\><<<
\ht:special{t4ht@,&#x00A0;}%
>>>

\<end mtext space\><<<
\ht:special{t4ht@,}%
>>>


\<configure html-mml latex\><<<
\Configure{newline}
  {\ifmtext |<end mtext space|>\HCode{</\a:mathml mtext><\a:mathml 
                                        mtext \mml:class="newline">}%
            |<start mtext space|>%
   \else \Tg<br\Hnewline class="newline" />\fi}
>>>

\<mathml latex\><<<
\Configure{newline}
  {|<end mtext space|>\HCode{</\a:mathml mtext\Hnewline><\a:mathml
                         mtext \mml:class="newline">}|<start mtext space|>}
>>>



\<mathml latex\><<<
\Configure{[]}
  {\Configure{@math}{|<block display|>}\csname a:mathml[]\endcsname
   \DviMath\Tg<\a:mathml mrow\Hnewline>$$\DisplayMathtrue }
  {$$\Tg</\a:mathml mrow>\EndDviMath\csname b:mathml[]\endcsname}
\Configure{()}
  {\Configure{@math}{|<inline display|>}\csname a:mathml()\endcsname
   \DviMath\Tg<\a:mathml mrow\Hnewline>$}
  {$\Tg</\a:mathml mrow>\EndDviMath\csname b:mathml()\endcsname}
\NewConfigure{mathml[]}{2}
\NewConfigure{mathml()}{2}
|<config array|>
>>>

\<configure html-mml latex\><<<
\Configure{mathml[]}
  {\edef\MathPar{\ifvmode par-\fi}\IgnorePar\endgraf\EndP
    \HCode{<div class="\MathPar math-display">}}
  {\HCode{</div>}|<try inline par|>}
\Css{div.math-display, div.par-math-display{text-align:center;}}
>>>




\<configure mathml rlbabel\><<<
\Configure{[]}
  {\Configure{@math}{|<block display|>}\csname a:mathml[]\endcsname
   \DviMath\Tg<\a:mathml mrow\Hnewline>$$\DisplayMathtrue }
  {$$\Tg</\a:mathml mrow>\EndDviMath\csname b:mathml[]\endcsname
   \if@rl\else\beginR\fi }
>>>





\<configure mathml array\><<<
|<config array|>
>>>

\<config array\><<<
\Configure{halignTB<>}{array}{\HCode{\Hnewline
   \a:mathml equalrows="false"
   \a:mathml columnlines="\columnlines"
   \a:mathml equalcolumns="false" \a:mathml class="array"}<>\HAlign}
\Configure{array} 
   {\halignTB{array}}
   {\ifvmode \IgnorePar\fi |<insert Clr|>\Tg</\a:mathml mtable>}
   {\Tg<\a:mathml mtr>}{\Tg</\a:mathml mtr>}
   {|<mtd array|>}
   {\Tg</\a:mathml mtd>}
\ConfigureEnv{array}
   {\bgroup\Configure{$}{}{}{}} 
   {\egroup}
   {}{}
>>>




\<configure html-mml latex\><<<
\ConfigureEnv{array}
   {\PushStack\Col:Marg\AllColMargins
    \bgroup\Configure{$}{}{}{}} 
   {\egroup \PopStack\Col:Marg\AllColMargins}
   {}{}
>>>

\<configure html-mml array\><<<
\ConfigureEnv{array}
   {\PushStack\Col:Marg\AllColMargins
    \bgroup\Configure{$}{}{}{}} 
   {\egroup \PopStack\Col:Marg\AllColMargins}
   {}{}
>>>




\<mtd array\><<<
\HCode{<\a:mathml mtd\Hnewline \mml:class="array" 
   \ifnum 1<\HMultispan columnspan="\HMultispan"\fi}%
   \halignTD    \HCode{>}%
>>>

\<mtd matrix\><<<
\HCode{<\a:mathml mtd\Hnewline \mml:class="matrix" 
   \ifnum 1<\HMultispan columnspan="\HMultispan"\fi}%
   \halignTD    \HCode{>}%
>>>




\<insert Clr\><<<
\expandafter\tmp:toks\expandafter{\Clr}%
\HCode{<!--\the\tmp:toks-->}%
>>>





\<mathml latex\><<<
\def\:MM{m}
\Configure{HBorder}
   {<\a:mathml mtr\Hnewline \mml:class="hline">} 
   {|<mml hline|>}
   {</\a:mathml mtr>}
   {<\a:mathml mtr\Hnewline \mml:class="cline">}
   {<\a:mathml mtd></\a:mathml mtd>}
   {|<mml hline|>}
   {</\a:mathml mtr>}
   {<\a:mathml mtr\Hnewline
         \mml:class="vspace" style="font-size:\HBorderspace">}
   {<\a:mathml mtd\Hnewline></\a:mathml mtd>}
   {</\a:mathml mtr>}
>>>

\<configure html-mml latex\><<<
\def\:MM{\ifmathml \a:mathml m\fi}
\Configure{HBorder}
   {<\:MM tr\Hnewline 
       \ifmathml \expandafter\mml:class\else class\fi="hline">} 
   {\ifmathml|<mml hline|>\else
            <td><hr\xml:empty></td>\fi} 
   {</\:MM tr>}
   {<\:MM tr\Hnewline
       \ifmathml \expandafter\mml:class\else class\fi ="cline">}
   {<\:MM td></\:MM td>}
   {\ifmathml|<mml hline|>\else
            <td><hr\xml:empty></td>\fi}
   {</\:MM tr>}
   {<\:MM tr\Hnewline
         \ifmathml \expandafter\mml:class\else  class\fi="vspace"
         style="font-size:\HBorderspace">}
   {<\:MM td\Hnewline>\ifmathml\else\string&\#x00A0;\fi
      </\:MM td>}
   {</\:MM tr>}
\Configure{halign}
   {\EndP
    \PushStack\Table:Stck\TableNo 
    \gHAdvance\Next:TableNo by 1  
    \global\let\TableNo=\Next:TableNo 
    \halignTB{halign}}
   {\HCode{</\:MT table>}\PopStack\Table:Stck\TableNo}
   {\HCode{<\:MT tr \Hnewline}\halignTR
               \HCode{ id="TBL-\TableNo-\HRow-">}}
   {\HCode{</\:MT tr>}}
   {|<td save EndP|> \HCode{<td 
               \ifnum \HMultispan>1 colspan="\HMultispan"\fi}%
   \halignTD \HCode{\Hnewline 
               id="TBL-\TableNo-\HRow-\HCol">}\ShowPar\par}
   {\ifvmode \IgnorePar \fi \EndP
               \HCode{</\:MT td>}|<td recall EndP|>}
>>>


\<configure html-mml latex\><<<
\Css{mtr.hline mtd{ 
  border-bottom:black solid 1px;  
  padding-top:2px; 
  padding-bottom:0em; 
}} 
\Css{mtr.hline mtd mo{ 
  display:none 
}}
>>>

\<td save EndP\><<<
\SaveEndP 
>>>

\<td recall EndP\><<<
\RecallEndP 
>>>


\<mml hline\><<<
<\a:mathml mtd><mo> \string&\#x0332; </mo></\a:mathml mtd>
>>>




\<mathml latex\><<<
|<mathml fonts|>
\let\A:fbox=\a:fbox
\let\B:fbox=\b:fbox
\let\OLDn:fbox:\n:fbox:
\def\OLDf:box{\ifx \EndPicture \:UnDef 
    \expandafter \OLDn:fbox: \else\expandafter \o:fbox: \fi}
\def\:temp{\relax\ifmmode
    \Configure{fbox}{\leavevmode\Configure{HtmlPar}{}{}{}{}\IgnoreRule
                     \AA:text{fbox}}{\BB:text\EndIgnoreRule}\else
    \Configure{fbox}{\leavevmode
                     \IgnoreRule\A:fbox}{\B:fbox\EndIgnoreRule}\fi
    \OLDf:box}
\HLet\fbox\:temp
\let\A:mbox|=\a:mbox
\let\B:mbox|=\b:mbox
\let\OLD:mbox|=\mbox
\def\mbox{\relax\ifmmode \Configure{mbox}{\AA:text{mbox}}{\BB:text}\else
                   \Configure{mbox}{\A:mbox}{\B:mbox}\fi\OLD:mbox}
|<mathml text util|>
>>>


\<mathml text util\><<<
\NewConfigure{@mtext}{1}
\Configure{@mtext}{ }
\def\AA:text#1{|<config start mtext|>}
\def\BB:text{|<config end mtext|>}
>>>


\<config start mtext\><<<
\mtexttrue
\HCode{<\a:mathml mstyle\Hnewline \mml:class="#1"><\a:mathml
   mtext \a:@mtext>}|<start mtext space|>\PauseMathClass
   |<config math within mtext|>%
>>>


\<config end mtext\><<<
\mtextfalse\EndPauseMathClass|<end mtext space|>\HCode
     {</\a:mathml mtext></\a:mathml mstyle>}%
>>>

\<config math within mtext\><<<
\Configure{$}%
  {\EndPauseMathClass \mtextfalse
      |<end mtext space|>\HCode{</\a:mathml mtext><\a:mathml
                                mstyle\Hnewline \mml:class="math">}}%
  {\HCode{</\a:mathml mstyle><\a:mathml
       mtext \a:@mtext>}|<start mtext space|>\PauseMathClass
   \mtexttrue}%
  {}%
>>>

\<config math within mtext\><<<
\Configure{()}%
  {\EndPauseMathClass \mtextfalse
      |<end mtext space|>\HCode{</\a:mathml mtext><\a:mathml
                                mstyle\Hnewline \mml:class="math">}$}%
  {$\HCode{</\a:mathml mstyle><\a:mathml
       mtext \a:@mtext>}|<start mtext space|>\PauseMathClass
   \mtexttrue}%
  {}%
>>>



%  \Configure{$}
%     {\Configure{@math}{|<inline display|>}\DviMath}
%     {\EndDviMath}{}
%  
%  \Configure{()}
%    {\Configure{@math}{|<inline display|>}\csname a:mathml()\endcsname
%     \DviMath
%      \Tg<\a:mathml mrow\Hnewline>$}
%    {$\Tg</\a:mathml mrow>%
%      \EndDviMath\csname b:mathml()\endcsname}



\begin{verbatim}
\documentclass{article}
\begin{document}
$  \mbox{is right-continuous at } -r  $
\end{document}

I see the problematic space between

    is right-continuous at

and

    -r

but I don't understand it.  The mathml code currently has there

    <\a:mathml mrow><\a:mathml mtext>is right-continuous at</\a:mathml mtext></\a:mathml mrow>
    <\a:mathml mo>&#x2212;</\a:mathml mo><\a:mathml mi>r</\a:mathml mi>

Replacing mrow with mstyle doesn't make a difference. Removing
the mrow solves the problem of the extra space.

I introduced the above changes to the mathml and mozilla-mathml
configurations, respectively.
\end{verbatim}





\<amstex.tex mathml\><<<
|<mathml text util|>
\Configure{text}{\AA:text{text}}{\BB:text}
\Configure{format}
   {\HCode{<\a:mathml mtable \mml:class="format">}}
   {\HCode{</\a:mathml mtable>}}
   {\HCode{<\a:mathml mtr\Hnewline \mml:class="format">}}
   {\HCode{</\a:mathml mtr>}}
   {\HCode{<\a:mathml mtd\Hnewline \mml:class="format">}}
   {\HCode{</\a:mathml mtd>}}
>>>



\<amsmath mathml\><<<
\Configure{above}
   {\Send{GROUP}{0}{<\a:mathml mfrac 
                      linethickness="\the\tmp:dim"><\a:mathml mrow>}%
    \HCode{</\a:mathml mrow>}\o:@@over:\:gobbleII}
   {\HCode{<\a:mathml 
      mrow>}\Send{EndGROUP}{0}{</\a:mathml mrow></\a:mathml mfrac>}}
\def\overwithdelims:III#1#2#3{%
   \def\:temp{#2#3}\def\:tempa{..}\ifx \:temp\:tempa
      \Send{GROUP}{0}{<\a:mathml mfrac linethickness="0"><\a:mathml mrow>}%
      \HCode{</\a:mathml mrow>}\o:over:
      \HCode{<\a:mathml mrow>}\Send{EndGROUP}{0}{</\a:mathml mrow></\a:mathml mfrac>}
   \else
      \Send{GROUP}{0}{<\a:mathml mfenced separators=""\Hnewline
          open="|<lr del 2|>" close="|<lr del 3|>"><\a:mathml mfrac
          linethickness="0"><\a:mathml mrow>}%
      \HCode{</\a:mathml mrow>}\o:@@over:
      \HCode{<\a:mathml mrow>}\Send{EndGROUP}{0}{</\a:mathml
                                    mrow></\a:mathml mfrac></\a:mathml mfenced>}
   \fi
 }
>>>


\<configure mathml tex4ht\><<<
\Configure{over}
   {\Send{GROUP}{0}
      {<\a:mathml mfrac><\a:mathml mrow>}\HCode{</\a:mathml mrow>}}
   {\HCode{<\a:mathml mrow>}\Send{EndGROUP}{0}
      {</\a:mathml mrow></\a:mathml mfrac>}}
\Configure{atop}
   {\Send{GROUP}{0}{<\a:mathml mfrac linethickness="0"><\a:mathml mrow>}\HCode{</\a:mathml mrow>}}
   {\HCode{<\a:mathml mrow>}\Send{EndGROUP}{0}{</\a:mathml mrow></\a:mathml mfrac>}}
\Configure{above}
   {\Send{GROUP}{0}{<\a:mathml mfrac linethickness="\the\tmp:dim"><\a:mathml mrow>}%
    \HCode{</\a:mathml mrow>}\o:over:\:gobbleII}
   {\HCode{<\a:mathml mrow>}\Send{EndGROUP}{0}{</\a:mathml mrow></\a:mathml mfrac>}}
>>>

\<configure mathml tex4ht\><<<
\Configure{abovewithdelims}
   {\abovewithdelims:III}
   {}
\def\abovewithdelims:III#1#2#3{%
   \def\:temp{#2#3}\def\:tempa{..}\ifx \:temp\:tempa
      \Send{GROUP}{0}{<\a:mathml mfrac linethickness="\the\tmp:dim"><\a:mathml mrow>}%
      \HCode{</\a:mathml mrow>}\o:over:
      \HCode{<\a:mathml mrow>}\Send{EndGROUP}{0}{</\a:mathml mrow></\a:mathml mfrac>}
   \else
      \Send{GROUP}{0}{<\a:mathml mfenced separators=""\Hnewline
          open="|<lr del 2|>" close="|<lr del 3|>"><\a:mathml mfrac
          linethickness="\the\tmp:dim"><\a:mathml mrow>}%
      \HCode{</\a:mathml mrow>}\o:over:
      \HCode{<\a:mathml mrow>}\Send{EndGROUP}{0}{</\a:mathml 
                                mrow></\a:mathml mfrac></\a:mathml mfenced>}
   \fi  
   \tmp:dim
 }
\Configure{atopwithdelims}
   {\atopwithdelims:III}
   {}
\def\atopwithdelims:III#1#2#3{%
   \def\:temp{#2#3}\def\:tempa{..}\ifx \:temp\:tempa
      \Send{GROUP}{0}{<\a:mathml mfrac linethickness="0"><\a:mathml mrow>}%
      \HCode{</\a:mathml mrow>}\o:atop:
      \HCode{<\a:mathml mrow>}\Send{EndGROUP}{0}{</\a:mathml mrow></\a:mathml mfrac>}
   \else
      \Send{GROUP}{0}{<\a:mathml mfenced separators=""\Hnewline
          open="|<lr del 2|>" close="|<lr del 3|>"><\a:mathml mfrac
          linethickness="0"><\a:mathml mrow>}%
      \HCode{</\a:mathml mrow>}\o:atop:
      \HCode{<\a:mathml mrow>}\Send{EndGROUP}{0}{</\a:mathml
                              mrow></\a:mathml mfrac></\a:mathml mfenced>}
   \fi
 }
\Configure{overwithdelims}
   {\overwithdelims:III}
   {}
\def\overwithdelims:III#1#2#3{%
   \def\:temp{#2#3}\def\:tempa{..}\ifx \:temp\:tempa
      \Send{GROUP}{0}{<\a:mathml mfrac><\a:mathml mrow>}%
      \HCode{</\a:mathml mrow>}\o:over:
      \HCode{<\a:mathml mrow>}\Send{EndGROUP}{0}{</\a:mathml mrow></\a:mathml mfrac>}
   \else
      \Send{GROUP}{0}{<\a:mathml mfenced separators=""\Hnewline
          open="|<lr del 2|>" close="|<lr del 3|>"><\a:mathml mfrac><\a:mathml mrow>}%
      \HCode{</\a:mathml mrow>}\o:over:
      \HCode{<\a:mathml mrow>}\Send{EndGROUP}{0}{</\a:mathml 
                               mrow></\a:mathml mfrac></\a:mathml mfenced>}
   \fi
 }
>>>



\<configure mathml amsmath\><<<
\Configure{abovewithdelims}
   {\abovewithdelims:III}
   {}
\def\abovewithdelims:III#1#2#3{%
   \def\:temp{#2#3}\def\:tempa{..}\ifx \:temp\:tempa
      \Send{GROUP}{0}{<\a:mathml mfrac linethickness="\the\tmp:dim"><\a:mathml mrow>}%
      \HCode{</\a:mathml mrow>}\o:@@over:
      \HCode{<\a:mathml mrow>}\Send{EndGROUP}{0}{</\a:mathml mrow></\a:mathml mfrac>}
   \else
      \Send{GROUP}{0}{<\a:mathml mfenced separators=""\Hnewline
          open="|<lr del 2|>" close="|<lr del 3|>"><\a:mathml mfrac
          linethickness="\the\tmp:dim"><\a:mathml mrow>}%
      \HCode{</\a:mathml mrow>}\o:@@over:
      \HCode{<\a:mathml mrow>}\Send{EndGROUP}{0}{</\a:mathml
                                  mrow></\a:mathml mfrac></\a:mathml mfenced>}
   \fi  
   \tmp:dim
 }
\Configure{atopwithdelims}
   {\atopwithdelims:III}
   {}
\def\atopwithdelims:III#1#2#3{%
   \def\:temp{#2#3}\def\:tempa{..}\ifx \:temp\:tempa
      \Send{GROUP}{0}{<\a:mathml mfrac linethickness="0"><\a:mathml mrow>}%
      \HCode{</\a:mathml mrow>}\o:@@atop:
      \HCode{<\a:mathml mrow>}\Send{EndGROUP}{0}{</\a:mathml mrow></\a:mathml mfrac>}
   \else
      \Send{GROUP}{0}{<\a:mathml mfenced separators=""\Hnewline
          open="|<lr del 2|>" close="|<lr del 3|>"><\a:mathml mfrac
          linethickness="0"><\a:mathml mrow>}%
      \HCode{</\a:mathml mrow>}\o:@@atop:
      \HCode{<\a:mathml mrow>}\Send{EndGROUP}{0}{</\a:mathml
                               mrow></\a:mathml mfrac></\a:mathml mfenced>}
   \fi
 }
\Configure{overwithdelims}
   {\overwithdelims:III}
   {}
\def\overwithdelims:III#1#2#3{%
   \def\:temp{#2#3}\def\:tempa{..}\ifx \:temp\:tempa
      \Send{GROUP}{0}{<\a:mathml mfrac><\a:mathml mrow>}%
      \HCode{</\a:mathml mrow>}\o:@@over:
      \HCode{<\a:mathml mrow>}\Send{EndGROUP}{0}{</\a:mathml mrow></\a:mathml mfrac>}
   \else
      \Send{GROUP}{0}{<\a:mathml mfenced separators=""\Hnewline
          open="|<lr del 2|>" close="|<lr del 3|>"><\a:mathml mfrac><\a:mathml mrow>}%
      \HCode{</\a:mathml mrow>}\o:@@over:
      \HCode{<\a:mathml mrow>}\Send{EndGROUP}{0}{</\a:mathml
                                 mrow></\a:mathml mfrac></\a:mathml mfenced>}
   \fi
 }
>>>



\<mathml latex\><<<
\Configure{frac}
   {\Tg<\a:mathml mfrac>\Tg<\a:mathml mrow\Hnewline>}
   {\Tg</\a:mathml mrow>\Configure{over}{}{}}
   {\Tg<\a:mathml mrow\Hnewline>}
   {\Tg</\a:mathml mrow>\Tg</\a:mathml mfrac>}
>>>



\<configure html-mml amsmath\><<<
\ifx \mathdisplay@@pop\:UnDef
   \ifx \emdf@La\:UnDef
      |<amsmath html-mml pre 2000|>
   \else
      |<amsmath html-mml jan 2000|>
   \fi
\else
   |<amsmath html-mml july 2000|>
\fi
>>>


% \Configure{quad}{{\ifmathml \ifmtext \o:quad:\else
%    \fi\else \o:quad: \fi}}
%

\<amsmath mathml\><<<
\Configure{tmspace}
   {\ifdim \textspace>0.2em
       \HCode{<\a:mathml mspace 
                         width="\textspace" \mml:class="tmspace"/>}%
    \else \HCode{<\a:mathml mspace 
                            width="0.3em" \mml:class="thinspace"/>}\fi
   }
>>>





\<try inline par\><<<
\ShowPar\par{\HCondtrue\noindent}%
>>>


% \Configure{math}{}

\<mathml fonts\><<<
\Configure{texttt}
   {\ifmathml \Configure{@mtext}{\mml:class="texttt"
         mathvariant="monospace" }\fi}
   {}%
\Configure{textit}
   {\ifmathml \Configure{@mtext}{\mml:class="textit"
         mathvariant="italic" }\fi}
   {}%
\Configure{textrm}
   {\ifmathml \Configure{@mtext}{\mml:class="textrm"
         mathvariant="normal" }\fi}
   {}%
\Configure{textup}
   {\ifmathml \Configure{@mtext}{\mml:class="textup"
         mathvariant="normal" }\fi}
   {}%
\Configure{textsl}
   {\ifmathml \Configure{@mtext}{\mml:class="textsl"
         mathvariant="italic" }\fi}
   {}%
\Configure{textsf}
   {\ifmathml \Configure{@mtext}{\mml:class="textsf"
         mathvariant="sans-serif" }\fi}
   {}%
\Configure{textbf}
   {\ifmathml \Configure{@mtext}{\mml:class="textbf"
         mathvariant="bold" }\fi}
   {}%
\Configure{textsc}
   {\ifmathml \Configure{@mtext}{\mml:class="textsc"
         mathvariant="normal" }\fi}
   {}%
\Configure{emph}
   {\ifmathml \Configure{@mtext}{\mml:class="emph"
         mathvariant="italic" }\fi}
   {}%
>>>




\<configure html-mml latex\><<<
\:CheckOption{fonts} \if:Option
\Configure{texttt}
     {\ifmathml \Configure{@mtext}{\mml:class="texttt" 
             mathvariant="monospace" }%
      \else     \HCode{<span class="texttt">}\fi}
     {\ifmathml\else \HCode{</span>}\fi}
  \Configure{textit}
     {\ifmathml \Configure{@mtext}{\mml:class="textit" 
             mathvariant="italic" }%
      \else     \HCode{<span class="textit">}\fi}
     {\ifmathml\else \HCode{</span>}\fi}
  \Configure{textrm}
     {\ifmathml \Configure{@mtext}{ \mml:class="textrm" 
             mathvariant="normal" }%
      \else     \HCode{<span class="textrm">}\fi}
     {\ifmathml\else \HCode{</span>}\fi}
  \Configure{textup}
     {\ifmathml \Configure{@mtext}{ \mml:class="textup" 
             mathvariant="normal" }%
      \else     \HCode{<span class="textup">}\fi}
     {\ifmathml\else \HCode{</span>}\fi}
  \Configure{textsl}
     {\ifmathml \Configure{@mtext}{ \mml:class="textst" 
             mathvariant="italic" }%
      \else     \HCode{<span class="textsl">}\fi}
     {\ifmathml\else \HCode{</span>}\fi}
  \Configure{textsf}
     {\ifmathml \Configure{@mtext}{ \mml:class="textsf" 
             mathvariant="sans-serif" }%
      \else     \HCode{<span class="textsf">}\fi}
     {\ifmathml\else \HCode{</span>}\fi}
  \Configure{textbf}
     {\ifmathml \Configure{@mtext}{ \mml:class="textbf" 
             mathvariant="bold" }%
      \else     \HCode{<span class="textbf">}\fi}
     {\ifmathml\else \HCode{</span>}\fi}
  \Configure{textsc}
     {\ifmathml \Configure{@mtext}{ \mml:class="textsc" 
             mathvariant="normal" }%
      \else     \HCode{<span class="textsc">}\fi}
     {\ifmathml\else \HCode{</span>}\fi}
  \Configure{emph}
     {\ifmathml \Configure{@mtext}{ \mml:class="emph" 
             mathvariant="italic" }%
      \else     \HCode{<span class="emph">}\fi}
     {\ifmathml\else \HCode{</span>}\fi}
\fi
>>>

The \verb+\bgroup...\egroup+ is to handle cases like
\verb+\nathbf{a}^x+.


\<configure mathml fontmath\><<<
\def\bold:version{bold}
\Configure{mathbf}%
  {\bgroup \Tg<\a:mathml mstyle 
             mathvariant="bold">}%
  {\Tg</\a:mathml mstyle>\egroup }%
\Configure{mathit}%
  {\bgroup \Tg<\a:mathml mstyle  mathvariant="\ifx 
                 \bold:version\math@version bold-\fi italic">}%
  {\Tg</\a:mathml mstyle>\egroup }%
\Configure{mathrm}%
  {\bgroup \Tg<\a:mathml mstyle 
             mathvariant="\ifx 
                 \bold:version\math@version bold\else normal\fi">}%
  {\Tg</\a:mathml mstyle>\egroup }%
\Configure{mathsf}%
  {\bgroup \Tg<\a:mathml mstyle  mathvariant="\ifx
            \bold:version\math@version bold-\fi sans-serif">}%
  {\Tg</\a:mathml mstyle>\egroup }%
\Configure{mathtt}%
  {\bgroup \Tg<\a:mathml mstyle mathvariant="monospace">}%
  {\Tg</\a:mathml mstyle>\egroup }%
>>>

\<configure mathml fontmath\><<<
\Css{ 
   math mstyle[mathvariant="bold"]  mi {  
      font-weight: bold;  
      font-style: normal;  
   }  
}
\Css{ 
   math mstyle[mathvariant="normal"]  mi {  
      font-weight: normal;  
      font-style: normal;  
   }  
} 
>>>

\<configure mathml amsfonts\><<<  
\Configure{mathbb}{}{}
\Configure{mathfrak}{}{}
>>>



Used to be

\begin{verbatim}
   \Configure{mathrm}{\Tg<\a:mathml mi \mml:class="mathrm">\PauseMathClass}%
                     {\EndPauseMathClass\Tg</\a:mathml mi>}%
   \Configure{mathbf}{\Tg<\a:mathml mi \mml:class="mathbf"                
             mathvariant="bold">\PauseMathClass}%
                     {\EndPauseMathClass\Tg</\a:mathml mi>}%
\end{verbatim}


\<configure html-mml latex\><<<
\Css{math .texttt { font-family: monospace; }}
\Css{math .textit { font-style: italic; }}
\Css{math .textsl { font-style: oblique; }}
\Css{math .textsf { font-family: sans-serif; }}
\Css{math .textbf { font-weight: bold; }}
>>>





\<plain,latex mathml\><<<
\Configure{afterSUB}{'}{\def\SUB:prime{\SUB:SUP}}
\Configure{'}
   {\let\:primes=\empty}
   {\relax \csname SUB:prime\endcsname\sp{%
      \global\let\SUB:prime=\:UnDef
      \def\:tempa{\prime\prime}\ifx \:tempa\:primes
              \mathop{\HCode{\string&\#x2033;}}\else
      \def\:tempa{\prime\prime\prime}\ifx \:tempa\:primes
              \mathop{\HCode{\string&\#x2034;}}\else
              \:primes \fi\fi}%
   }
   {\let\prime=\relax \xdef\:primes{\:primes\prime}}
>>>

LaTeX places the primes within superscript, and mathml seems to follow
suit (maybe because of the use of latex fonts in mozilla). However,
unicode maps already seem to assume raised representation for the
primes.

%%%%%%%%%%%%%%%%%%%%%%%%%%
\subsection{plain.sty}
%%%%%%%%%%%%%%%%%%%%%%%%%%




\<mathml plain\><<<
 \Configure{eqalign}
   {\HCode{<\a:mathml mtable \mml:class="eqalign">}}
   {\HCode{</\a:mathml mtable>}}
   {\HCode{<\a:mathml mtr \Hnewline \mml:class="eqalign">}}
   {\IgnorePar\HCode{</\a:mathml mtr>}}
   {\HCode{\ifnum \HCol=3 <\a:mathml mtd\Hnewline
       \mml:class="eqalign3"></\a:mathml mtd>\fi
       <\a:mathml mtd \Hnewline  \mml:class="eqalign\HCol">}}
   {\HCode{</\a:mathml mtd>}}
>>>

%%%%%%%%%%%%%%%%%%%%%%%%%%%%
\subsection{Palin + LaTeX}
%%%%%%%%%%%%%%%%%%%%%%%%%%%%

The default \verb'\left' and \verb'\right' in their default definition
with tex produce multi-part delimiters, from cmex, on large
subformulas. Hence, the `'.' below is needed.

\<plain,latex mathml\><<<
\def\:temp#1{\a:left{#1}\o:left:.\b:left{#1}}   \HLet\left|=\:temp
\def\:temp#1{\a:right{#1}\o:right:.\b:right{#1}}\HLet\right|=\:temp
\Configure{left}
   {}
   {\HCode{<\a:mathml mfenced separators=""\Hnewline 
       open="|<lr del 1|>"\space }\bgroup 
                              \HCode{><\a:mathml mrow>}\bgroup}
\Configure{right}
   {\egroup\Send{GROUP}{0}{ close="|<lr del 1|>"\space}\egroup}
   {\HCode{</\a:mathml mrow></\a:mathml mfenced>}}
\def\def:DEL#1{\def\:DEL{#1}}
\def\check:DEL#1#2{\def\:tempa{#1}\ifx \:tempa\:DEL \def\:DEL{#2}\fi}
\NewConfigure{mfenced}[2]{\append:defI\def:DEL{\check:DEL{#1}{#2}}}
|</bigl, /bigr,... for mathml|>
\:CheckOption{new-accents}     \if:Option \else
   |<config mathml accents|>
\fi
\def\mathinner#1{\Tg<\a:mathml mrow
   \mml:class="mathinner">#1\Tg</\a:mathml mrow>}
>>>


\<configure mathml etex4ht\><<<
\def\:tempc#1{\a:middle\mathrel{\left:right#1\:EMPTY}\b:middle} 
\HLet\middle=\:tempc 
\NewConfigure{middle}{2}
\Configure{middle} 
   {\bgroup 
        \Configure{mathrel}{*} 
          {<mo \mml:class="MathClass-rel" 
                     fence="true" stretchy="true" >} 
          {</mo>} 
        {\Configure{mathrel}{}{}{}{}}% 
        } 
        {\egroup} 
>>>


The \verb+\root ...\of {...}+  creates the following structure.

\begin{verbatim}
PUSH
   ...root value...
POP
PUSH
   PUSH
      PUSH
         ...root sign characters with PUSH-POP...
      POP
      PUSH
         RIGHT4:   15213546
         DOWN3:    -1553322
         PUT_RULE: height: 26213
                   length: 1197667
         DOWN3:    1553322
         PUSH
            ...root content...
         POP
      POP
   POP
POP
\end{verbatim}



\<plain,latex mathml\><<<
\Configure{root}{\rev:root}{}{}
\def\rev:root #1\b:root#2\c:root{
   \Tg<\a:mathml mroot>{\Tg<\a:mathml mrow\Hnewline>#2\Tg</\a:mathml mrow>}%
             {\Tg<\a:mathml mrow\Hnewline>#1\Tg</\a:mathml mrow>}\Tg</\a:mathml mroot>}
\def\AccentMath#1{\relax\ifmmode #1\else $#1$\fi}
>>>

\<lr del 1\><<<
\expandafter
\ifx \csname \string#1:lr del\endcsname\relax #1\else 
     \csname \string#1:lr del\endcsname\fi
>>>

\<lr del 2\><<<
\expandafter
\ifx \csname \string#2:lr del\endcsname\relax #2\else 
     \csname \string#2:lr del\endcsname\fi
>>>


\<lr del 3\><<<
\expandafter
\ifx \csname \string#3:lr del\endcsname\relax #3\else 
     \csname \string#3:lr del\endcsname\fi
>>>


\<plain,latex mathml\><<<
\NewConfigure{mfenced}[2]{\expandafter
   \def\csname \string#1:lr del\endcsname{#2}}
|<plain,latex mfenced|>
>>>

\<amsmath mathml\><<<
\Configure{mfenced}{\lvert}{\string&\#x007C;} |%vertical line|%
\Configure{mfenced}{\rvert}{\string&\#x007C;} |%double vertical line|%
\Configure{mfenced}{\lVert}{\string&\#x2016;}
\Configure{mfenced}{\rVert}{\string&\#x2016;}
>>>

\<amstex.sty mathml\><<<
\Configure{mfenced}{\lvert}{\string&\#x007C;}
\Configure{mfenced}{\rvert}{\string&\#x007C;}
\Configure{mfenced}{\lVert}{\string&\#x2016;}
\Configure{mfenced}{\rVert}{\string&\#x2016;}
>>>





\<plain,latex mfenced\><<<
\Configure{mfenced}{\||}{\string&\#x2225;}
\Configure{mfenced}{.}{}
\Configure{mfenced}{\langle}{\string&\#x27E8;} |%lang|%
\Configure{mfenced}{\rangle}{\string&\#x27E9;} |%rang|%
\Configure{mfenced}{\vert}{||}
\edef\:temp{\noexpand\Configure{mfenced}{\noexpand\{}{\string{}}\:temp
\edef\:temp{\noexpand\Configure{mfenced}{\noexpand\}}{\string}}}\:temp
\edef\:temp{\noexpand
   \Configure{mfenced}{\noexpand\lbrace}{\string{}}\:temp
\edef\:temp{\noexpand
   \Configure{mfenced}{\noexpand\rbrace}{\string}}}\:temp
\Configure{mfenced}{<}{\string&\#x003C;} |%lt|%
\Configure{mfenced}{>}{\string&\#x003E;} |%gt|%
\Configure{mfenced}{\lceil}{\string&\#x2308;} 
\Configure{mfenced}{\rceil}{\string&\#x2309;} 
\Configure{mfenced}{\lfloor}{\string&\#x230A;} 
\Configure{mfenced}{\rfloor}{\string&\#x230B;} 
\Configure{mfenced}{\Vert}{\string&\#x2225;}
\Configure{mfenced}{\rgroup}{)}
\Configure{mfenced}{\lgroup}{(}
\Configure{mfenced}{\bracevert}{||}
\Configure{mfenced}{\bracevert}{||}
\Configure{mfenced}{\downarrow}{\string&\#x2193;}
\Configure{mfenced}{\uparrow}{\string&\#x2191;}
\Configure{mfenced}{\Uparrow}{\string&\#x21D1;}
\Configure{mfenced}{\updownarrow}{\string&\#x2195;}
\Configure{mfenced}{\Updownarrow}{\string&\#x21D5;}
\Configure{mfenced}{\Downarrow}{\string&\#x21D3;}
\Configure{mfenced}{\backslash}{\string&\#x2216;}
\Configure{mfenced}{\rmoustache}{\string&\#x25DD;}
\Configure{mfenced}{\lmoustache}{\string&\#x25DC;}
\Configure{mfenced}{\Arrowvert}{\string&\#x2225;}
\Configure{mfenced}{\arrowvert}{\string&\#x007C;}
>>>

%    \def\:tempa{\{}\ifx \:tempa\:DEL  \let\:DEL\lbrc: \else
%    \def\:tempa{\}}\ifx \:tempa\:DEL  \let\:DEL\rbrc: \else
%    \def\:tempa{<}\ifx \:tempa\:DEL  \def\:DEL{\string&lt;}\else
%    \def\:tempa{>}\ifx \:tempa\:DEL  \def\:DEL{\string&gt;}%
%    \fi\fi\fi\fi }
% \edef\lbrc:{\string{}     \edef\rbrc:{\string}}

% \HCode{\string&#2#1;}%

%%%%%%%%%%%%%%%%%%%%%
\subsection{Amsmath}
%%%%%%%%%%%%%%%%%%%%%


\<configure html-mml amsmath\><<<
\Configure{@begin}
  {split}
  {\let\tagform@=\:gobble }
>>>

\<amsmath mathml\><<<
\Configure{split}
   {\HCode{<\a:mathml mtable \Hnewline
           \a:mathml  columnalign="right left"
           \mml:class="split">}}
   {\ifvmode\IgnorePar\fi \HCode{</\a:mathml mtable>}}
   {\HCode{<\a:mathml mtr \mml:class="split-mtr">}} 
   {\HCode{</\a:mathml mtr>}}
   {\HCode{<\a:mathml mtd\Hnewline \mml:class="split-mtd">}}
   {\HCode{</\a:mathml mtd>}}
>>>



\<amsmath mathml\><<<
\Configure{multline}
   {\HCode{<\a:mathml mtable\Hnewline  \mml:class="multline">}}
   {\HCode{</\a:mathml mtable>}}
   {\HCode{<\a:mathml mtr>%
           <\a:mathml mtd\Hnewline \mml:class="multline">}}
   {\HCode{</\a:mathml mtd></\a:mathml mtr>}} 
   {\HCode{</\a:mathml mtd><\a:mathml mtd>}} 
   {}
\Configure{multline*}
   {\HCode{<\a:mathml mtable\Hnewline \mml:class="multline-star">}}   
   {\HCode{</\a:mathml mtable>}}
   {\HCode{<\a:mathml mtr><\a:mathml mtd\Hnewline  
                                        \mml:class="multline-star">}} 
   {\HCode{</\a:mathml mtd></\a:mathml mtr>}} 
\Configure{gather}
   {\HCode{<\a:mathml mtable\Hnewline \mml:class="gather">}}
   {\HCode{</\a:mathml mtable>}}
   {\HCode{<\a:mathml mtr>}}    {\HCode{</\a:mathml mtr>}} 
   {\HCode{\Hnewline<\a:mathml mtd>}}   {\HCode{</\a:mathml mtd>}}
\Configure{gather*}
   {\HCode{<\a:mathml mtable\Hnewline \mml:class="gather-star">}}
   {\HCode{</\a:mathml mtable>}}
   {\HCode{<\a:mathml mtr>}}    {\HCode{</\a:mathml mtr>}} 
   {\HCode{\Hnewline<\a:mathml mtd>}}   {\HCode{</\a:mathml mtd>}}
\Configure{subarray}
   {\HCode{<\a:mathml mtable \mml:class="subarray\if c\Hlcr-c\fi" 
      \a:mathml rowspacing="0" 
      \if c\Hlcr \a:mathml columnalign="center"\fi>}}
   {\HCode{</\a:mathml mtable>}}
   {\HCode{<\a:mathml mtr><\a:mathml mtd>}} 
   {\HCode{</\a:mathml mtd></\a:mathml mtr>}}
\Configure{smallmatrix}
   {\HCode{<\a:mathml mtable>}}  
   {\HCode{</\a:mathml mtable>}}
   {\HCode{<\a:mathml mtr\Hnewline \mml:class="smallmatrix">}} 
   {\HCode{</\a:mathml mtr>}} 
   {\HCode{\Hnewline<\a:mathml mtd>}} 
   {\HCode{</\a:mathml mtd>}}
\def\:tempc#1#2{%
   \ConfigureEnv{#1}
     {\def\end:al:env{}\ifmmode\else 
       \ifvmode\else \def\end:al:env{\par\ShowPar\IgnoreIndent}\fi\fi}
     {\end:al:env }
     {}{}%
   \Configure{#1}
   {\IgnorePar
    \ifvmode \EndP  \else \EndP  \HCode{<!--tex4ht:inline-->}\fi    
    \ifmathml
       \Tg<\a:mathml mrow>\def\end:mathalign{\Tg</\a:mathml mrow>}%
    \else
       \Configure{@math}{|<block display|>}\DviMath
       \def\end:mathalign{\EndDviMath \DisplayMathfalse}%
       \DisplayMathtrue
    \fi
    \HCode{<\a:mathml mtable\Hnewline
                \a:mathml columnalign="left" \mml:class="#2">}}
   {\HCode{</\a:mathml mtable>}\end:mathalign }
   {\HCode{<\a:mathml mtr>}}    {\HCode{</\a:mathml mtr>}}
   {\HCode{<\a:mathml mtd\Hnewline
      |<rightward align col|>%
      \mml:class="align-\ifnum \HCol>\maxfields@ label\else 
         \ifodd\HCol\space  odd\else     even\fi 
       \fi">}}
   {\HCode{|<extra align col sep|></\a:mathml mtd>}}}
\:tempc{align}{align}
\:tempc{alignat}{alignat}
\:tempc{xalignat}{xalignat}
\:tempc{xxalignat}{xxalignat}
\:tempc{flalign}{flalign}
\:tempc{align*}{align-star}
\:tempc{alignat*}{alignat-star}
\:tempc{xalignat*}{xalignat-star}
\:tempc{flalign*}{flalign-star}
>>>



\<amsmath mathml\><<<
\Configure{gathered}
   {\HCode{<\a:mathml mtable\Hnewline \mml:class="gathered">}}
   {\HCode{</\a:mathml mtable>}}
   {\HCode{<\a:mathml mtr>}}    {\HCode{</\a:mathml mtr>}} 
   {\HCode{\Hnewline<\a:mathml mtd>}}   {\HCode{</\a:mathml mtd>}}
>>>



\<configure html-mml amsmath\><<<
\Css{mtd.align-odd{margin-left:2em;   text-align:right;}}
\Css{mtd.align-even{margin-right:2em; text-align:left;}}
>>>

\<extra align col sep\><<<
\ifodd\HCol\space\else
       <\a:mathml mspace width="2em"/>\fi
>>>

\<rightward align col\><<<
\ifodd\HCol \a:mathml  columnalign="right" \fi
>>>

Setting \verb+#1+ below within mrow is problematic in the sense that
arrows don't stretch over multi-characters content \verb+#2+.
A \verb+accent="true"+ move the arrow lower, but also distribute the
characters further apart.

\<under arrow\><<<
\Tg<\a:mathml munder>%
\Tg<\a:mathml mrow\Hnewline>#2\Tg</\a:mathml mrow>%
#1%
\Tg</\a:mathml munder>%
>>>

\<over arrow\><<<
\Tg<\a:mathml mover class="overarrow">%
\Tg<\a:mathml mrow\Hnewline>{#2#3}\Tg</\a:mathml mrow>%
#1%
\Tg</\a:mathml mover>%
>>>




% \Tg<\a:mathml mrow\Hnewline>#1\Tg</\a:mathml mrow>%
% \Tg<\a:mathml mrow>#1\Tg</\a:mathml mrow>%


\<arr amsmath, amstex1\><<<
\Configure{underarrow@}
   {\Tg<\a:mathml munder>\Tg<\a:mathml mrow\Hnewline>}
   {\Tg</\a:mathml mrow>\Tg<\a:mathml mrow\Hnewline>}
   {\Tg</\a:mathml mrow>\Tg</\a:mathml munder>}
\Configure{overarrow@}
   {\Tg<\a:mathml mover accent="true">\Tg<\a:mathml mrow\Hnewline>}
   {\Tg</\a:mathml mrow>\Tg<\a:mathml mrow>}
   {\Tg</\a:mathml mrow>\Tg</\a:mathml mover>}
\def\:tempc#1#2#3{\a:overarrow@
      \hbox{$\m@th#2#3$}\b:overarrow@
      \hbox{#1#2}\c:overarrow@
   }
\HLet\overarrow@\:tempc
>>>





\<configure mathml amstex1\><<<
\ifx \o:overrightarrow:\:UnDef\else     
   \HRestore\overrightarrow
   \HRestore\overleftarrow
   \HRestore\overleftrightarrow
\fi
>>>



\<amsmath html-mml jan 2000NO\><<<
\HRestore\overrightarrow
\HRestore\overleftarrow
\HRestore\overleftrightarrow
>>>

\<amsmath html-mml july 2000NO\><<<
\HRestore\overrightarrow
\HRestore\overleftarrow
\HRestore\overleftrightarrow
>>>

%   \HRestore\underrightarrow
%   \HRestore\underleftarrow
%   \HRestore\underleftrightarrow




\<\><<<
\def\:tempc{\mathpalette{\overarrow@\rightarrow}}
\HLet\overrightarrow=\:tempc
\def\:tempc{\mathpalette{\overarrow@\leftarrow}}
\HLet\overleftarrow=\:tempc
\def\:tempc{\mathpalette{\overarrow@\leftrightarrow}}
\HLet\overleftrightarrow=\:tempc

\def\:tempc{\mathpalette{\underarrow@\rightarrow}}
\HLet\underrightarrow=\:tempc
\def\:tempc{\mathpalette{\underarrow@\leftarrow}}
\HLet\underleftarrow=\:tempc
\def\:tempc{\mathpalette{\underarrow@\leftrightarrow}}
\HLet\underleftrightarrow=\:tempc
>>>


Replace with
\Link[http://www.unicode.org/charts/PDF/U27F0.pdf]{}{}stretching
arrows\EndLink{} once they work. It was \verb+\rightarrow+ (U-2192)
before going to \verb+\HCode{&\#x20D7;}+. 
(... suggested
to replace it with  
the diacritical one: \verb+&#x20D7+. I t solves two problems.  The arrow is
lower and the the spacing between a subscript and a vector is
reduced.)

\<mathml plain,fontmath\><<<
\Configure{overrightarrow}
  {\leavevmode\hbox\bgroup$\Tg<\a:mathml mover class="overrightarrow">%
   \Tg<\a:mathml mrow\Hnewline>\:gobble}
  {\Tg</\a:mathml mrow>\mathop{\HCode{&\#x20D7;}}%
   \Tg</\a:mathml mover>$\egroup}
\Configure{overleftarrow}
  {\leavevmode\hbox\bgroup$\Tg<\a:mathml mover class="overleftarrow">%
   \Tg<\a:mathml mrow\Hnewline>\:gobble}
  {\Tg</\a:mathml mrow>\mathop{\HCode{&\#x20D6;}}%
   \Tg</\a:mathml mover>$\egroup}
>>>


\<amsmath mathml\><<<
\Configure{underrightarrow}
  {\Tg<\a:mathml munder>\Tg<\a:mathml mrow\Hnewline>\:gobble}
  {\Tg</\a:mathml mrow>\mathop{\rightarrow}\Tg</\a:mathml munder>}
\Configure{underleftarrow}
  {\Tg<\a:mathml munder>\Tg<\a:mathml mrow\Hnewline>\:gobble}
  {\Tg</\a:mathml mrow>\mathop{\leftarrow}\Tg</\a:mathml munder>}
>>>


\<amsmath mathml\><<<
\Configure{overleftrightarrow}
  {\Tg<\a:mathml mover class="overleftrightarrow">\Tg<\a:mathml mrow\Hnewline>\:gobble}
  {\Tg</\a:mathml mrow>\mathop{\HCode{&\#x20E1;}}\Tg</\a:mathml mover>}
\Configure{underleftrightarrow}
  {\Tg<\a:mathml munder>\Tg<\a:mathml mrow\Hnewline>\:gobble}
  {\Tg</\a:mathml mrow>\mathop{\leftrightarrow}\Tg</\a:mathml munder>}
>>>






\<\><<<
\def\underarrow:#1#2#3{|<under arrow|>}
\def\overarrow:#1#2#3{|<over arrow|>}
\def\:tempc{\mathpalette{\overarrow:\rightarrow}}
\HLet\overrightarrow=\:tempc
\def\:tempc{\mathpalette{\overarrow:\leftarrow}}
\HLet\overleftarrow=\:tempc
\def\:tempc{\mathpalette{\overarrow:\leftrightarrow}}
\HLet\overleftrightarrow=\:tempc
>>>

\<\><<<
\def\:tempc{\mathpalette{\underarrow:\rightarrow}}
\HLet\underrightarrow=\:tempc
\def\:tempc{\mathpalette{\underarrow:\leftarrow}}
\HLet\underleftarrow=\:tempc
\def\:tempc{\mathpalette{\underarrow:\leftrightarrow}}
\HLet\underleftrightarrow=\:tempc
>>>



\<amsmath mathml\><<<
|<mathml m:env|>
\renewcommand{\xrightarrow}[2][]{%
   \Tg<\a:mathml munderover>%
     \rightarrow
     \Tg<\a:mathml mrow\Hnewline>#1\Tg</\a:mathml mrow>%
     \Tg<\a:mathml mrow\Hnewline>#2\Tg</\a:mathml mrow>%                     
   \Tg</\a:mathml munderover>}%
\renewcommand{\xleftarrow}[2][]{%
   \Tg<\a:mathml munderover>%
      \leftarrow
      \Tg<\a:mathml mrow\Hnewline>#1\Tg</\a:mathml mrow>%
      \Tg<\a:mathml mrow\Hnewline>#2\Tg</\a:mathml mrow>%
   \Tg</\a:mathml munderover>}%

\def\overset#1#2{{%
      \Tg<\a:mathml mover class="overset">\Tg<\a:mathml
                              mrow\Hnewline>#2\Tg</\a:mathml mrow>%
                \Tg<\a:mathml mrow\Hnewline>#1\Tg</\a:mathml mrow>%
      \Tg</\a:mathml mover>}}%
\def\underset#1#2{{%
      \Tg<\a:mathml munder>\Tg<\a:mathml mrow\Hnewline>#2\Tg</\a:mathml mrow>%
                 \Tg<\a:mathml mrow\Hnewline>#1\Tg</\a:mathml mrow>%
      \Tg</\a:mathml munder>}}%

\def\sideset#1#2#3{{\Tg<\a:mathml mrow \mml:class="sideset"\Hnewline
        >{\HCode{}}#1{\HCode{}#3}#2\Tg</\a:mathml mrow>}}
\ConfigureEnv{eqxample}{}{}{}{}
\ConfigureEnv{gather*}{}{}{}{}
\ConfigureEnv{gathered}{}{}{}{}
\ConfigureEnv{matrix}{}{}{}{}
>>>

 The \verb'\HCode{}' in \verb'\sideset' is for catching superscripts and subscripts



% \renewcommand{\hdotsfor}[1]{%
%   \ifx[#1\@xp\shdots@for\else\hdots@for\@ne{#1}\fi}
% \def\shdots@for#1]{\hdots@for{#1}}
% \def\hdots@for#1#2{\multicolumn{#2}c%
%   {\m@th\dotsspace@1.5mu\mkern-#1\dotsspace@
%    \xleaders\hbox{$\m@th\mkern#1\dotsspace@.\mkern#1\dotsspace@$}%
%            \hfill
%    \mkern-#1\dotsspace@}%
%    }



\<amsmath symbols\><<<
\def\:tempc{\HCode{<\a:mathml mspace width="0.2em"/>}\mathbin
  {mod}\HCode{<\a:mathml mspace width="0.2em"/>}}
\HLet\bmod\:tempc
\def\:tempc#1{\allowbreak
  \HCode{<\a:mathml mspace width="0.3em"/>}(#1)}
\HLet\pod=\:tempc
\def\:tempc#1{\pod{\mathop{mod}\HCode{<\a:mathml mspace width="0.3em"/>}#1}}
\HLet\pmod=\:tempc
\def\:tempc#1{\allowbreak
   \HCode{<\a:mathml mspace width="0.3em"/>}\mathop
      {mod}\HCode{<\a:mathml mspace width="0.3em"/>}#1}
\HLet\mod=\:tempc
>>>


%%%%%%%%%%%%%%%%%%%%
\subsection{amsopn}
%%%%%%%%%%%%%%%%%%%%


\marginpar{remove the following def in liu of xtpipes}

\<amsopn / amstex.sty mathml\><<<
\def\:tempc{{\underrightarrow\lim}}  \HLet\varinjlim\:tempc
\def\:tempc{{\underleftarrow\lim}}   \HLet\varprojlim\:tempc
\def\:tempc{{\overline\lim}}         \HLet\varlimsup\:tempc
\def\:tempc{{\underline\lim}}        \HLet\varliminf\:tempc
>>>


\<amsopn / amstex.sty mathml\><<<
\def\:tempc#1#2#3{\hbox{{\mtexttrue
   \HCode{<\a:mathml mrow\Hnewline
       \mml:class="text"><\a:mathml mtext 
              \a:@mtext>}|<start mtext space|>\PauseMathClass
   \Configure{$}{\EndPauseMathClass\mtextfalse
      |<end mtext space|>\HCode{</\a:mathml mtext><\a:mathml
              mrow\Hnewline \mml:class="math">}}
        {\HCode{</\a:mathml mrow><\a:mathml mtext
           \a:@mtext>}|<start mtext space|>\mtexttrue\PauseMathClass}{#1}
   \let\f@size#2\selectfont #3}%
   \mtextfalse \EndPauseMathClass |<end mtext space|>\HCode
      {</\a:mathml mtext></\a:mathml mrow>}}}
\HLet\textdef@\:tempc
>>>


The original definitions for the following macros contain small spaces
\verb+\,+ in the names, and the spaces imply improper mathml outcome
of the form \verb+<mo>...<mspace.../>...</mo>+.


\<configure mathml amsopn\><<<
\def\:tempc{\qopname\relax m{lim inf}}
\HLet\liminf=\:tempc
\def\:tempc{\qopname\relax m{lim sup}}
\HLet\limsup=\:tempc
\def\:tempc{\qopname\relax m{proj lim}} 
\HLet\projlim=\:tempc
\def\:tempc{\qopname\relax m{inj lim}}
\HLet\injlim=\:tempc
>>>





Why originally the accents are defined within a group? (knuth answer
this in the texbook.)

\<plain,latex mathml\><<<
\let\A:oalign|=\a:oalign
\let\B:oalign|=\b:oalign
\Configure{oalign}
   {\ifmmode \Configure{$}{\Tg<\a:mathml mrow\Hnewline>}%
                          {\Tg</\a:mathml mrow>}{}%
             \let\halign|=\TeXhalign
             \Tg<\a:mathml mrow\Hnewline \mml:class="oalign">\else\A:oalign
    \fi}
   {\ifmmode \Tg</\a:mathml mrow>\else\B:oalign\fi}
>>>


\<configure mathml amsopn\><<<
\Configure{qopname}
   {\Configure{mathop}{*}
      {<\a:mathml mo \mml:class="qopname">}
      {</\a:mathml mo>}
      {}}
\Css{mo.qopname + mi{margin-left:0.3em}}
\Css{mi + mo.qopname{margin-left:0.3em}}
>>>

%%%%%%%%%%%%%%%%%
\subsection{mathtools}
%%%%%%%%%%%%%%%%%%

\<configure mathml mathtools\><<<
\Configure{MT_gathered_env} 
   {\HCode{<\a:mathml mtable\Hnewline \mml:class="MT-gathered-env">}} 
   {\HCode{</\a:mathml mtable>}} 
   {\HCode{<\a:mathml mtr>}}    {\HCode{</\a:mathml mtr>}} 
   {\HCode{\Hnewline<\a:mathml mtd>}}   {\HCode{</\a:mathml mtd>}} 
\ConfigureEnv{MT_gathered_env}{}{}{}{} 
>>>

\<configure mathml mathtools\><<<
\ConfigureEnv{gathered}
    {%
     \Configure{MT_gathered_env}
        {\HCode{<\a:mathml mtable\Hnewline 
                          \mml:class="gathered" columnalign="center" >}}
        {\HCode{</\a:mathml mtable>}}
        {\HCode{<\a:mathml mtr>}}    {\HCode{</\a:mathml mtr>}}
        {\HCode{\Hnewline<\a:mathml mtd>}}   {\HCode{</\a:mathml mtd>}}%
    }
    {}{}{}
>>>

\<configure mathml mathtools\><<<
\ConfigureEnv{lgathered}
    {%
     \Configure{MT_gathered_env}
        {\HCode{<\a:mathml mtable\Hnewline 
                          \mml:class="lgathered" columnalign="left" >}}
        {\HCode{</\a:mathml mtable>}}
        {\HCode{<\a:mathml mtr>}}    {\HCode{</\a:mathml mtr>}}
        {\HCode{\Hnewline<\a:mathml mtd>}}   {\HCode{</\a:mathml mtd>}}%
    }
    {}{}{}
>>>


\<configure mathml mathtools\><<<
\ConfigureEnv{rgathered}
    {%
     \Configure{MT_gathered_env}
        {\HCode{<\a:mathml mtable\Hnewline 
                          \mml:class="rgathered" columnalign="right" >}}
        {\HCode{</\a:mathml mtable>}}
        {\HCode{<\a:mathml mtr>}}    {\HCode{</\a:mathml mtr>}}
        {\HCode{\Hnewline<\a:mathml mtd>}}   {\HCode{</\a:mathml mtd>}}%
    }
    {}{}{}
>>>


\<configure mathml mathtools\><<<
\ConfigureEnv{cases} 
   {\HCode{<\a:mathml mrow \mml:class="cases" >}} 
   {\HCode{</\a:mathml mrow>}} 
   {}{}
\Configure{dcases}
   {\HCode{<\a:mathml mtable\Hnewline \mml:class="dcases">}}
   {\HCode{</\a:mathml mtable>}}
   {\HCode{<\a:mathml mtr>}}    {\HCode{</\a:mathml mtr>}}
   {\HCode{\Hnewline<\a:mathml mtd columnalign="left">}}   
   {\HCode{</\a:mathml mtd>}}
\Configure{dcases*}
   {\HCode{<\a:mathml mtable\Hnewline \mml:class="dcases-star">}}
   {\HCode{</\a:mathml mtable>}}
   {\HCode{<\a:mathml mtr>}}    {\HCode{</\a:mathml mtr>}}
   {\HCode{\Hnewline<\a:mathml mtd columnalign="left">}%
    \ifnum\HCol=2
       \PauseMathClass
       \HCode{<\a:mathml mtext>}%
    \fi
   }
   {\ifnum\HCol=2 
       \EndPauseMathClass
       \HCode{</\a:mathml mtext>}%
    \fi
    \HCode{</\a:mathml mtd>}}
>>>

%%%%%%%%%%%%%%%%%%
\section{Gaus}
%%%%%%%%%%%%%%%%%%



\<configure mathml gauss\><<< 
\Configure{gmatrix} 
   {\HCode{<\a:mathml mtable \mml:class="gmatrix-\g@environment">}}  
   {\HCode{</\a:mathml mtable>}}  
   {\HCode{<\a:mathml mtr\Hnewline \mml:class="gmatrix">}}  
   {\HCode{</\a:mathml mtr>}}  
   {\HCode{\Hnewline<\a:mathml mtd>}}  
   {\HCode{</\a:mathml mtd>}}  
>>>

\<configure mathml gauss\><<< 
\Css{mtr.gmatrix mtd {margin-left:0em; margin-right:0em; 
                padding-left:0em; padding-right:0em; 
                text-indent:0pt; 
}}
\Css{mfenced mtable {margin-left:0em; margin-right:0em; 
                padding-left:0em; padding-right:0em; 
                text-indent:0pt; 
}} 
>>>

\<configure mathml gauss\><<< 
\Configure{rowops}
   {\HCode{<mtable class="rowops">
           <mtr><mtd columnspan="3"><mtext>Row ops:</mtext></mtd></mtr>
   }}
   {\HCode{</mtable>}}
\Configure{colops}
   {\HCode{<mtable class="colops">
           <mtr><mtd columnspan="3"><mtext>Col ops:</mtext></mtd></mtr>
   }}
   {\HCode{</mtable>}}
\Css{mtable.rowops, mtable.colops {
          margin-left:0.2em; 
          margin-right:0.2em; 
          font-size:75\%;
   }}
\Css{mtable.rowops mtr, mtable.colops mtr,
     mtable.rowops mtd, mtable.colops mtd 
   {
     margin:0em; padding:0em;
   }}
\Configure{add}
   {\HCode{<mtr><mtd>}\mbox{add }\HCode{</mtd><mtd>}}
   {,\HCode{</mtd><mtd>}}
   {\HCode{</mtd></mtr>}}
\Configure{mult}
   {\HCode{<mtr><mtd>}\mbox{mult }\HCode{</mtd><mtd>}}
   {,\HCode{</mtd><mtd>}}
   {\HCode{</mtd></mtr>}}
\Configure{swap}
   {\HCode{<mtr><mtd>}\mbox{swap }\HCode{</mtd><mtd>}}
   {,\HCode{</mtd><mtd>}}
   {\HCode{</mtd></mtr>}}
>>>

%%%%%%%%%%%%%%%%%%%%
\section{Eqnarray}
%%%%%%%%%%%%%%%%%%%%

Had `BASELINE' before `MIDDLE', but changed to conform with math
in page 252-- in intro to theory book.

\<mathml latex\><<<
\ConfigureEnv{eqnarray}
   {\ifvmode \IgnorePar \fi \EndP}
   {\ifvmode \IgnorePar \fi \EndP \ShowPar \par}
   {}{}
\Configure{eqnarray}
   {\HCode{<\a:mathml mtable\Hnewline \mml:class="eqnarray\ifx 
       \@seqncr\@eqncr\else-star\fi"
       \a:mathml columnalign="right center left" >}}%
   {\HCode{</\a:mathml mtable>}}
   {\HCode{<\a:mathml mtr>}}  {\HCode{</\a:mathml mtr>}}
   {\Tg<\a:mathml mtd\Hnewline \mml:class="eqnarray-\HCol">
      \ifnum \HCol=4
        \Tg<\a:mathml mtext 
             \mml:class="eqnarray">|<start mtext space|>\PauseMathClass 
        \mtexttrue
        |<config math within mtext|>%
      \fi}
   {\ifnum \HCol=4 \mtextfalse 
      \EndPauseMathClass|<end mtext space|>\Tg</\a:mathml mtext>\fi
      \Tg</\a:mathml mtd>}
>>>




\<configure html-mml latex\><<<
|<config multicolumn|>
>>>


\<configure html-mml array\><<<
|<config multicolumn|>
>>>


\<config multicolumn\><<<
\Configure{multicolumn}
   {\let\sv:VBorder\VBorder
    \let\VBorder\empty
    \Configure{VBorder}{}{}{}{}%
   }
   {\let\VBorder\sv:VBorder}   
   {\ifmmode \HCode{<\a:mathml mrow\Hnewline \mml:class="multicolumn}%
       {|<attributes of mml multicolumn|>\HColAlign}\HCode{">}\else
       \ifvmode\IgnorePar\fi
       \HCode{<div class="multicolumn" }\HColAlign\HCode{>}\fi }
   {\ifmmode  \HCode{</\a:mathml mrow>}\else
    \ifvmode\IgnorePar\fi \EndP\HCode{</div>}\fi }
\def\gob:sp#1{#1}
>>>

\<attributes of mml multicolumn\><<<
\Configure{halignTD} {}{}
   {<}{\HCode{-left}}
   {-}{\HCode{-center}}
   {>}{\HCode{-right}}
   {p}{\HCode{-left}}
   {}%
>>>


\<configure html-mml tex4ht\><<<
\Configure{halignTD} {}{}
   {<}{\ifmathml \HCode{ \a:mathml columnalign="left"}\else
        \HCode{ style="text-align:left; white-space:nowrap;"}\fi}
   {-}{\ifmathml \HCode{ \a:mathml columnalign="center"}\else
        \HCode{ style="text-align:center; white-space:nowrap;"}\fi}
   {>}{\ifmathml \HCode{ \a:mathml columnalign="right"}\else
        \HCode{ style="text-align:right; white-space:nowrap;"}\fi}
   {^}{\ifmathml \HCode{ \a:mathml rowalign="top"}\else
        \HCode{ style="vertical-align:top; white-space:nowrap;"}\fi}
   {=}{\ifmathml \HCode{ \a:mathml rowalign="baseline"}\else
        \HCode{ style="vertical-align:baseline; white-space:nowrap;"}\fi}
   {||}{\ifmathml \HCode{ \a:mathml rowalign="center"}\else
        \HCode{ style="vertical-align:middle; white-space:nowrap;"}\fi}
   {_}{\ifmathml \HCode{ \a:mathml rowalign="bottom"}\else
        \HCode{ style="vertical-align:bottom; white-space:nowrap;"}\fi}
   {p}{\ifmathml \HCode{ \a:mathml columnalign="left"}\else
        \HCode{ style="text-align:left;"}\fi}
   {}
>>>




\<configure mathml tex4ht\><<<
\Configure{halignTD} {}{}
   {<}{\HCode{ \a:mathml columnalign="left"}}
   {-}{\HCode{ \a:mathml columnalign="center"}}
   {>}{\HCode{ \a:mathml columnalign="right"}}
   {^}{\HCode{ \a:mathml rowalign="top"}}
   {=}{\HCode{ \a:mathml rowalign="baseline"}}
   {||}{\HCode{ \a:mathml rowalign="center"}}
   {_}{\HCode{ \a:mathml rowalign="bottom"}}
   {p}{\HCode{ \a:mathml columnalign="left"}}
   {}
>>>

%%%%%%%%%%%%%%%%%%%%%%%%%%
\section{Big, BIG, ....}
%%%%%%%%%%%%%%%%%%%%%%%%%

The 
\verb'\special{t4ht@[}...\special{t4ht@]}' gobble the enclosed stuff.
The external pair is provided as grouping mechanism for
sub/super-scripts cases like \verb'\bigl(...\bigr)^x' within dvimath
mode. The \verb'{\HCode{}}' is neded for creating content delimiters
\verb'.' delimiters like in \verb'$\bigl. a_b \bigr)$'; without that mathml 
gets something wrong there. 


\begin{verbatim}
 > There are some minor problems wit \bigl | corresponding to #x2223 in
 > Mozilla encoding
 > but I eliminted it by using \left | which turned to be a right solution.

The \big... macros decide the dimensions of the delimiters. As is the
case with \sum, also here I don't know how to transport the dimensions
to mozilla.

The \left...\right macros ask latex/mozilla to determine the
dimensions.
\end{verbatim}



\</bigl, /bigr,... for mathml\><<<
\NewConfigure{Biggl}{2}
\NewConfigure{Biggm}{2}
\NewConfigure{Biggr}{2}
\NewConfigure{Bigl}{2}
\NewConfigure{Bigm}{2}
\NewConfigure{Bigr}{2}
\NewConfigure{biggl}{2}
\NewConfigure{biggm}{2}
\NewConfigure{biggr}{2}
\NewConfigure{bigl}{2}
\NewConfigure{bigm}{2}
\NewConfigure{bigr}{2}
|<big common|>
\:CheckOption{mathml-}     \if:Option 
   |<big mathml-|>
\else
   |<big mathml|>
\fi
>>>

\<big mathml\><<<
\Configure{Biggl}
   {\bgroup
       \Configure{Bigg}{}{}{}{}%
       \Configure{mathopen}{*}
          {<mrow><mo \mml:class="MathClass-open" 
                     fence="true" mathsize="2.45em" >}
          {</mo><mrow>}
          {\Configure{mathopen}{}{}{}{}}%
   }
   {\egroup}
\Configure{Biggr}
   {\bgroup
       \Configure{Bigg}{}{}{}{}%
       \Configure{mathclose}{*}
          {</mrow><mo \mml:class="MathClass-close"
                     fence="true" mathsize="2.45em" >}
          {</mo></mrow>}
          {\Configure{mathclose}{}{}{}{}}%
   }
   {\egroup}
\Configure{Bigl}
   {\bgroup
       \Configure{Big}{}{}{}{}%
       \Configure{mathopen}{*}
          {<mrow><mo \mml:class="MathClass-open"
                     fence="true" mathsize="1.61em" >}
          {</mo><mrow>}
          {\Configure{mathopen}{}{}{}{}}%
   }
   {\egroup}
\Configure{Bigr}
   {\bgroup
       \Configure{Big}{}{}{}{}%
       \Configure{mathclose}{*}
          {</mrow><mo \mml:class="MathClass-close"
                     fence="true" mathsize="1.61em" >}
          {</mo></mrow>}
          {\Configure{mathclose}{}{}{}{}}%
   }
   {\egroup}
\Configure{biggl}
   {\bgroup
       \Configure{bigg}{}{}{}{}%
       \Configure{mathopen}{*}
          {<mrow><mo \mml:class="MathClass-open"
                     fence="true" mathsize="2.03em" >}
          {</mo><mrow>}
          {\Configure{mathopen}{}{}{}{}}%
   }
   {\egroup}
\Configure{biggr}
   {\bgroup
       \Configure{bigg}{}{}{}{}%
       \Configure{mathclose}{*}
          {</mrow><mo \mml:class="MathClass-close"
                     fence="true" mathsize="2.03em" >}
          {</mo></mrow>}
          {\Configure{mathclose}{}{}{}{}}%
   }
   {\egroup}
\Configure{bigl}
   {\bgroup
       \Configure{big}{}{}{}{}%
       \Configure{mathopen}{*}
          {<mrow><mo \mml:class="MathClass-open"
                     fence="true" mathsize="1.19em" >}
          {</mo><mrow>}
          {\Configure{mathopen}{}{}{}{}}%
   }
   {\egroup}
\Configure{bigr}
   {\bgroup
       \Configure{big}{}{}{}{}%
       \Configure{mathclose}{*}
          {</mrow><mo \mml:class="MathClass-close"
                     fence="true" mathsize="1.19em" >}
          {</mo></mrow>}
          {\Configure{mathclose}{}{}{}{}}%
   }
   {\egroup}
>>>




\<big mathml-\><<<
\Configure{Biggl}
   {\bgroup
       \Configure{Bigg}{}{}{}{}%
       \Configure{mathopen}{*}
          {<mo \mml:class="MathClass-open" 
                     fence="true" mathsize="2.45em" >}
          {</mo>}
          {\Configure{mathopen}{}{}{}{}}%
   }
   {\egroup}
\Configure{Biggr}
   {\bgroup
       \Configure{Bigg}{}{}{}{}%
       \Configure{mathclose}{*}
          {<mo \mml:class="MathClass-close"
                     fence="true" mathsize="2.45em" >}
          {</mo>}
          {\Configure{mathclose}{}{}{}{}}%
   }
   {\egroup}
\Configure{Bigl}
   {\bgroup
       \Configure{Big}{}{}{}{}%
       \Configure{mathopen}{*}
          {<mo \mml:class="MathClass-open"
                     fence="true" mathsize="1.61em" >}
          {</mo>}
          {\Configure{mathopen}{}{}{}{}}%
   }
   {\egroup}
\Configure{Bigr}
   {\bgroup
       \Configure{Big}{}{}{}{}%
       \Configure{mathclose}{*}
          {<mo \mml:class="MathClass-close"
                     fence="true" mathsize="1.61em" >}
          {</mo>}
          {\Configure{mathclose}{}{}{}{}}%
   }
   {\egroup}
\Configure{biggl}
   {\bgroup
       \Configure{bigg}{}{}{}{}%
       \Configure{mathopen}{*}
          {<mo \mml:class="MathClass-open"
                     fence="true" mathsize="2.03em" >}
          {</mo>}
          {\Configure{mathopen}{}{}{}{}}%
   }
   {\egroup}
\Configure{biggr}
   {\bgroup
       \Configure{bigg}{}{}{}{}%
       \Configure{mathclose}{*}
          {<mo \mml:class="MathClass-close"
                     fence="true" mathsize="2.03em" >}
          {</mo>}
          {\Configure{mathclose}{}{}{}{}}%
   }
   {\egroup}
\Configure{bigl}
   {\bgroup
       \Configure{big}{}{}{}{}%
       \Configure{mathopen}{*}
          {<mo \mml:class="MathClass-open"
                     fence="true" mathsize="1.19em" >}
          {</mo>}
          {\Configure{mathopen}{}{}{}{}}%
   }
   {\egroup}
\Configure{bigr}
   {\bgroup
       \Configure{big}{}{}{}{}%
       \Configure{mathclose}{*}
          {<mo \mml:class="MathClass-close"
                     fence="true" mathsize="1.19em" >}
          {</mo>}
          {\Configure{mathclose}{}{}{}{}}%
   }
   {\egroup}
>>>



\<big common\><<<
\Configure{Biggm}
   {\bgroup
       \Configure{Bigg}{}{}{}{}%
       \Configure{mathrel}{*}
          {<mo \mml:class="MathClass-rel"
                     fence="true" mathsize="2.45em" >}
          {</mo>}
          {\Configure{mathrel}{}{}{}{}}%
   }
   {\egroup}
\Configure{Bigm}
   {\bgroup
       \Configure{Big}{}{}{}{}%
       \Configure{mathrel}{*}
          {<mo \mml:class="MathClass-rel"
                     fence="true" mathsize="1.61em" >}
          {</mo>}
          {\Configure{mathrel}{}{}{}{}}%
   }
   {\egroup}
\Configure{biggm}
   {\bgroup
       \Configure{bigg}{}{}{}{}%
       \Configure{mathrel}{*}
          {<mo \mml:class="MathClass-rel"
                     fence="true" mathsize="2.03em" >}
          {</mo>}
          {\Configure{mathrel}{}{}{}{}}%
   }
   {\egroup}
\Configure{bigm}
   {\bgroup
       \Configure{big}{}{}{}{}%
       \Configure{mathrel}{*}
          {<mo \mml:class="MathClass-rel"
                     fence="true" mathsize="1.19em" >}
          {</mo>}
          {\Configure{mathrel}{}{}{}{}}%
   }
   {\egroup}
>>>



\</bigl, /bigr,... for mathml\><<<
\def\:tempc#1{\ht:special{t4ht@[}(\ht:special{t4ht@]}%
   \a:bigl\mathopen{\left:right\big{#1\:EMPTY}}\b:bigl\:EMPTY}
\HLet\bigl=\:tempc
\def\:tempc#1{\a:bigm\mathrel{\left:right\big{#1\:EMPTY}}\b:bigm}
\HLet\bigm=\:tempc
\def\:tempc#1{\:EMPTY
   \a:bigr\mathclose{\left:right\big{#1\:EMPTY}}\b:bigr%
   \ht:special{t4ht@[})\ht:special{t4ht@]}}
\HLet\bigr=\:tempc
\def\:tempc#1{\ht:special{t4ht@[}(\ht:special{t4ht@]}%
  \a:Bigl\mathopen{\left:right\Big{#1}\:EMPTY}\b:Bigl\:EMPTY}
\HLet\Bigl=\:tempc
\def\:tempc#1{\a:Bigm\mathrel{\left:right\Big{#1\:EMPTY}}\b:Bigm}
\HLet\Bigm=\:tempc
\def\:tempc#1{\:EMPTY
   \a:Bigr\mathclose{\left:right\Big{#1\:EMPTY}}\b:Bigr%
   \ht:special{t4ht@[})\ht:special{t4ht@]}}
\HLet\Bigr=\:tempc
\def\:tempc#1{\ht:special{t4ht@[}(\ht:special{t4ht@]}%
  \a:biggl\mathopen{\left:right\bigg{#1\:EMPTY}}\b:biggl\:EMPTY}
\HLet\biggl=\:tempc
\def\:tempc#1{\a:biggm\mathrel{\left:right\bigg{#1\:EMPTY}}\b:biggm}
\HLet\biggm=\:tempc
\def\:tempc#1{\:EMPTY
   \a:biggr\mathclose{\left:right\bigg{#1\:EMPTY}}\b:biggr%
   \ht:special{t4ht@[})\ht:special{t4ht@]}}
\HLet\biggr=\:tempc
\def\:tempc#1{\ht:special{t4ht@[}(\ht:special{t4ht@]}%
  \a:Biggl\mathopen{\left:right\Bigg{#1\:EMPTY}}\b:Biggl\:EMPTY}
\HLet\Biggl=\:tempc
\def\:tempc#1{\a:Biggm\mathrel{\left:right\Bigg{#1\:EMPTY}}\b:Biggm}
\HLet\Biggm=\:tempc
\def\:tempc#1{\:EMPTY
   \a:Biggr\mathclose{\left:right\Bigg{#1}\:EMPTY}\b:Biggr
   \ht:special{t4ht@[})\ht:special{t4ht@]}}
\HLet\Biggr=\:tempc
\def\:EMPTY{{\HCode{}}}
\def\left:right{%
   \let\left|=\o:left:
   \let\right|=\o:right:
   \Configure{left}{}{}\Configure{right}{}{}}
>>>

\verb+\bigl{.}+ et al produce empty para,etr , hence the \verb+\:EMPTY+ is a ompensation for such cases.

\<mathml plain,fontmath\><<<
\Configure{big}
   {}{}
   {\HCode{<\a:mathml mstyle mathsize="1.19em">}}  
   {\HCode{</\a:mathml mstyle>}}
\Configure{Big}
   {}{}
   {\HCode{<\a:mathml mstyle mathsize="1.61em">}}  
   {\HCode{</\a:mathml mstyle>}}
\Configure{bigg}
   {}{}
   {\HCode{<\a:mathml mstyle mathsize="2.03em">}}  
   {\HCode{</\a:mathml mstyle>}}
\Configure{Bigg}
   {}{}
   {\HCode{<\a:mathml mstyle mathsize="2.45em">}}  
   {\HCode{</\a:mathml mstyle>}}
>>>

% \R=8.5; \R*1.4; \WriteVal\R
% \R=11.5; \R*1.4; \WriteVal\R
% \R=14.5; \R*1.4; \WriteVal\R
% \R=17.5; \R*1.4; \WriteVal\R


\<mathml plain,fontmath\><<<
\def\:tempc#1{{\hbox{\a:big$\c:big
   \left#1\hbox{}\right.\n@space\d:big$\b:big}}}
\HLet\big=\:tempc
\def\:tempc#1{{\hbox{\a:Big$\c:Big
   \left#1\hbox{}\right.\n@space\d:Big$\b:Big}}}
\HLet\Big=\:tempc
\def\:tempc#1{{\hbox{\a:bigg$\c:bigg
   \left#1\hbox{}\right.\n@space\d:bigg$\b:bigg}}}
\HLet\bigg=\:tempc
\def\:tempc#1{{\hbox{\a:Bigg$\c:Bigg
   \left#1\hbox{}\right.\n@space\d:Bigg$\b:Bigg}}}
\HLet\Bigg=\:tempc
>>>

\<configure mathml amsbsy\><<<
\Configure{pmb} 
   {\ifmmode \HCode{<!--binary math rel, tripple hight-->}\fi} 
   {\ifmmode \HCode{<!--/binary math rel, tripple hight-->}\fi} 
>>>



%%%%%%%%%%%%%%%%%%%%%%%%%
\section{stackrel}
%%%%%%%%%%%%%%%%%%%%%%%%%

\<mathml latex\><<<
\def\:temp#1#2{{\a:stackrel{#1}\b:stackrel {#2}\c:stackrel}}   
\HLet\stackrel|=\:temp
\:CheckOption{stackrel-}     \if:Option
   \Configure{stackrel}
      {\mathrel\bgroup \HCode{<\a:mathml mglyph\Hnewline
          alt="\string\stackrel}\afterassignment\put:arg\tmp:toks}
      {\afterassignment\put:arg\tmp:toks}
      {\HCode{" />}\egroup}
   \def\put:arg{\HCode{{\the\tmp:toks}}}
\else
   \Configure{stackrel}
     {\HCode{<\a:mathml mover\Hnewline
             \mml:class="stackrel"><\a:mathml mrow\Hnewline>}\inv:i:iii}
     {\HCode{</\a:mathml mrow><\a:mathml
               mrow\Hnewline>}|<mathop nested within stackrel|>\mathop}
     {\HCode{</\a:mathml mrow></\a:mathml mover>}}
   \def\inv:i:iii#1#2#3{{#3}#2{#1}}
\fi
>>>

\<mathop nested within stackrel\><<<
\Configure{mathop*}{*}
         {<\a:mathml mrow>}
         {</\a:mathml mrow>}
         {\Configure{mathop}{}{}{}{}}%
>>>

%%%%%%%%%%%%%%%%%%%%%%%%%%%%%%
\subsection{fontmath.ltx}
%%%%%%%%%%%%%%%%%%%%%%%%%%%%%%


Accent versus underscript: \verb=\underbrace{x+y+z}=
   versus \verb=\underbrace{\strut x+y+z}=

\<mathml plain,fontmath\><<<
\Configure{overbrace}
   {\Configure{mathop*}{*}
         {<\a:mathml mover \a:mathml accent="false"><\a:mathml mrow>}
         {</\a:mathml mrow><\a:mathml mo>\string
               &\#xFE37;</\a:mathml mo></\a:mathml mover>}
         {\Configure{mathop}{}{}{}{}}%
   }
   {\let\o:overbrace:=\displaystyle}
   {}
\Configure{underbrace}
   {\Configure{mathop*}{*}
         {<\a:mathml munder
                 \a:mathml accentunder="false"><\a:mathml mrow>}
         {</\a:mathml mrow><\a:mathml mo>\string
               &\#xFE38;</\a:mathml mo></\a:mathml munder>}
         {\Configure{mathop}{}{}{}{}}%
   }
   {\let\o:underbrace:=\displaystyle}
   {}
>>>

\<\><<<
\def\:temp#1{%
   \Configure{mathop*}{*}{<\a:mathml mrow \mml:class="underbrace">}
         {</\a:mathml mrow>}%
         {\Configure{mathop}{}{}{}{}}%
   \mathop{\hbox{$\displaystyle{#1}$}}\limits}
\HLet\underbrace|=\:temp
\def\:temp#1{%
   \Configure{mathop*}{*}{<\a:mathml mrow \mml:class="overbrace">}
         {</\a:mathml mrow>}%
         {\Configure{mathop}{}{}{}{}}%
   \mathop{\hbox{$\displaystyle{#1}$}}\limits}
\HLet\overbrace|=\:temp
>>>


\<amsmath mathmlNO\><<<
\def\:temp#1{%
   \Configure{mathop*}{*}{<\a:mathml mrow \mml:class="underbrace">}
      {</\a:mathml mrow>}%
      {\Configure{mathop}{}{}{}{}}%
   \mathop{\vtop{$\displaystyle{#1}$}}\limits}
\HLet\underbrace|=\:temp
>>>

\<mathml plain,fontmath\><<<
\Configure{MathClass}{0}{*}{<\a:mathml mi\Hnewline>}{</\a:mathml mi>}{}
\Configure{MathClass}{1}{*}{<\a:mathml mo\Hnewline
             \mml:class="MathClass-op">}{</\a:mathml mo>}{}
\Configure{MathClass}{2}{*}{<\a:mathml mo\Hnewline
             \mml:class="MathClass-bin">}{</\a:mathml mo>}{}
\Configure{MathClass}{3}{*}{<\a:mathml mo\Hnewline
             \mml:class="MathClass-rel">}{</\a:mathml mo>}{}
\:CheckOption{mathml-}     \if:Option 
   \Configure{MathClass}{4}{*}{<\a:mathml mo\Hnewline
             \mml:class="MathClass-open">}
                           {</\a:mathml mo>}{}
   \Configure{MathClass}{5}{*}{<\a:mathml mo\Hnewline
             \mml:class="MathClass-close">}
                           {</\a:mathml mo>}{}
\else
   \Log:Note{for sources failing to produce
       clean mathml code, try the command line option `mathml-'}%
   \Configure{MathClass}{4}{*}{<\a:mathml mrow
                                     ><\a:mathml mo\Hnewline
             \mml:class="MathClass-open">}
                           {</\a:mathml mo><\a:mathml mrow>}{}
   \Configure{MathClass}{5}{*}{</\a:mathml mrow><\a:mathml mo\Hnewline
             \mml:class="MathClass-close">}
                           {</\a:mathml mo></\a:mathml mrow>}{}
\fi
\Configure{MathClass}{6}{*}{<\a:mathml mo\Hnewline
             \mml:class="MathClass-punc">}{</\a:mathml mo>}{}
\Configure{MathClass}{7}{*}{<\a:mathml mn>}{</\a:mathml mn>}
                           {0123456789}     
\ifx \mathordMathClass\:UnDef
  \NewMathClass\mathordMathClass
\fi
\Configure{MathClass}{\mathordMathClass}{*}{<\a:mathml mo\Hnewline
             \mml:class="MathClass-ord">}{</\a:mathml mo>}{} 
\Configure{FormulaClass}{0}{}{}{}
\Configure{FormulaClass}{1}{}{}{}
\Configure{FormulaClass}{2}{}{}{}
\Configure{FormulaClass}{3}{}{}{}
\Configure{FormulaClass}{4}{}{}{}
\Configure{FormulaClass}{5}{}{}{}
\Configure{FormulaClass}{6}{}{}{}
>>>


The following code is to insert spaces
for casses like \verb=$a \mathop{div} b$=
but not \verb=$\mathop{div} (\mathbf{F})$= 

\<mathml plain,fontmath\><<<
\Css{mo.MathClass-op + mi{margin-left:0.3em}}
\Css{mi + mo.MathClass-op{margin-left:0.3em}}
>>>




%%%%%%%%%%%%%%%%%%%%%%%%%%%%%%%%%%%%%%%%%%%
\section{Limits of Sub- snd Super- Scripts}
%%%%%%%%%%%%%%%%%%%%%%%%%%%%%%%%%%%%%%%%%%%


%%%%%%%%%%%%%%%%%%%%%%%%%
\subsection{Outline}
%%%%%%%%%%%%%%%%%%%%%%%%%


\<limits sub, sup, subsup\><<<
\Configure{nolimits}{\HCode{<!--nolimits-->}}
\Configure{limits}
   {|<limit scripts|>}
>>>


\<limit scripts\><<<
\let|<restore SUB|>=\:UnDef
\let|<restore SUP|>=\:UnDef
\ifx \:temp\sp |<conf limits SUP|>\else
    \ifx \:temp\sb |<conf limits SUB|>\fi
\fi
|<conf limits SUBSUP|>%  
>>>



\<limits sub, sup, subsup\><<<
\Configure{displaylimits}
   {\ifDisplayMath \expandafter\disp:limits\fi}
\def\disp:limits{|<limit scripts|>}  
>>>


%%%%%%%%%%%%%%%%%%%%%%%%%
\subsection{Simultaneously Sub- and Super- Scripts}
%%%%%%%%%%%%%%%%%%%%%%%%%


\<conf limits SUBSUP\><<<
|<save pre-limits SUBSUP|>%
\Configure{SUBSUP}
   {\ifx |<restore SUB|>\:UnDef\else  |<restore SUB|>\fi
    \ifx |<restore SUP|>\:UnDef\else  |<restore SUP|>\fi
    |<restore pre-limits SUBSUP|>%
    \Send{BACK}
      {<\a:mathml munderover \a:mathml accentunder="false" 
                             \a:mathml accent="false"><\a:mathml mrow
          \Hnewline>}%
    \Tg</\a:mathml mrow>\Tg<\a:mathml 
           mrow\Hnewline>\trap:base
   }
   {\Tg</\a:mathml mrow>\Tg<\a:mathml mrow\Hnewline>\trap:base}
   {\Tg</\a:mathml mrow>\TG</\a:mathml munderover\Hnewline>}
>>>


\<save pre-limits SUBSUP\><<<
\let|<restore SUBSUP|>=\SUBSUP:
>>>

\<restore pre-limits SUBSUP\><<<
\ifx |<restore SUBSUP|>\:UnDef \else
   \let\SUBSUP:=|<restore SUBSUP|>%
\fi
>>>

\<limits sub, sup, subsup\><<<
\ifx \a:putSUB\:UnDef \else
   \pend:def\a:putSUB{\ifx |<restore SUBSUP|>\:UnDef\else 
         \let\SUBSUP:=|<restore SUBSUP|>%
         \let|<restore SUBSUP|>=\:UnDef
      \fi}%
\fi
\ifx \a:putSUP\:UnDef \else
   \pend:def\a:putSUP{\ifx |<restore SUBSUP|>\:UnDef\else 
         \let\SUBSUP:=|<restore SUBSUP|>%
         \let|<restore SUBSUP|>=\:UnDef
      \fi}%
\fi
>>>


\<restore SUBSUP\><<<
\pl:SUBSUP
>>>

%%%%%%%%%%%%%%%%%%%%%%%%%
\subsection{Superscripts}
%%%%%%%%%%%%%%%%%%%%%%%%%



\<conf limits SUP\><<<
|<save pre-limits SUP|>%
\Configure{SUP}
   {|<open external limits SUP|>\bgroup
     |<restore pre-limits SUBSUP|>%
     |<conf second limits SUP|>\trap:base}  
   {}%
>>>

\<save pre-limits SUP\><<<
\def\pl:SUP{\Configure{SUP}}%
\expandafter\append:def
   \expandafter\pl:SUP\expandafter{\expandafter{\a:SUP}}%
\expandafter\append:def
   \expandafter\pl:SUP\expandafter{\expandafter{\b:SUP}}%
>>>

\<conf second limits SUP\><<<
\Configure{SUP}
   {|<open intermediate limits SUP|>\bgroup
    |<conf third limit SUP|>\trap:base}                 
   {\egroup|<close external limits SUP|>}%
>>>


\<conf third limit SUP\><<<
\Configure{SUP}
   {|<open internal limits SUP|>\bgroup \trap:base}
   {\egroup |<close internal limits SUP|>}% 
>>>


\<open external limits SUP\><<<
\Send{BACK}{<\a:mathml mover
             \mml:class="msup"><\a:mathml mrow\Hnewline>}%
\Tg</\a:mathml mrow>\Tg<\a:mathml mrow\Hnewline>%
>>>

\<close external limits SUP\><<<
\Tg</\a:mathml mrow>\TG</\a:mathml mover\Hnewline>\aftergroup|<restore SUP|>%
>>>

\<restore SUP\><<<
\pl:SUP  >>>

\<open intermediate limits SUP\><<<
\Send{BACK}{<\a:mathml msup><\a:mathml mrow\Hnewline>}%
\Tg</\a:mathml mrow>\Tg<\a:mathml mrow\Hnewline>%
>>>

\<open internal limits SUP\><<<
\Send{BACK}{<\a:mathml msup><\a:mathml mrow\Hnewline>}%
\Tg</\a:mathml mrow>\Tg<\a:mathml mrow\Hnewline>%
>>>

\<close internal limits SUP\><<<
\Tg</\a:mathml mrow>\TG</\a:mathml msup\Hnewline>%
>>>

%%%%%%%%%%%%%%%%%%%%%%%%%
\subsection{Subscripts}
%%%%%%%%%%%%%%%%%%%%%%%%%





\<conf limits SUB\><<<
|<save pre-limits SUB|>%
\Configure{SUB}
   {|<open external limits SUB|>\bgroup
     |<restore pre-limits SUBSUP|>%
     |<conf second limits SUB|>\trap:base}  
   {}%
>>>

\<save pre-limits SUB\><<<
\def\pl:SUB{\Configure{SUB}}%
\expandafter\append:def
   \expandafter\pl:SUB\expandafter{\expandafter{\a:SUB}}%
\expandafter\append:def
   \expandafter\pl:SUB\expandafter{\expandafter{\b:SUB}}%
>>>

\<conf second limits SUB\><<<
\Configure{SUB}
   {|<open intermediate limits SUB|>\bgroup
    |<conf third limit SUB|>\trap:base}                 
   {\egroup|<close external limits SUB|>}%
>>>


\<conf third limit SUB\><<<
\Configure{SUB}
   {|<open internal limits SUB|>\bgroup \trap:base}
   {\egroup |<close internal limits SUB|>}% 
>>>


\<open external limits SUB\><<<
\Send{BACK}{<\a:mathml munder 
             \mml:class="msub"><\a:mathml mrow\Hnewline>}%
\Tg</\a:mathml mrow>\Tg<\a:mathml mrow\Hnewline>%
>>>

\<close external limits SUB\><<<
\Tg</\a:mathml mrow>\TG</\a:mathml
              munder\Hnewline>\aftergroup|<restore SUB|>%
>>>

\<restore SUB\><<<
\pl:SUB >>>

\<open intermediate limits SUB\><<<
\Send{BACK}{<\a:mathml msub><\a:mathml mrow\Hnewline>}%
\Tg</\a:mathml mrow>\Tg<\a:mathml mrow\Hnewline>%
>>>

\<open internal limits SUB\><<<
\Send{BACK}{<\a:mathml msub><\a:mathml mrow\Hnewline>}%
\Tg</\a:mathml mrow>\Tg<\a:mathml mrow\Hnewline>%
>>>

\<close internal limits SUB\><<<
\Tg</\a:mathml mrow>\TG</\a:mathml msub\Hnewline>%
>>>

%%%%%%%%%%%%%
\subsection{Limits on Operators}
%%%%%%%%%%%%%


\<configure mathml latex\><<<
\def\:tempb#1{%
   \expandafter\let\expandafter\:tempc\csname #1\endcsname
   \append:def\:tempc{\limits}%
   \expandafter\HLet\csname #1\endcsname\:tempc
}
\:tempb{lim}
\:tempb{limsup}
\:tempb{liminf}
\:tempb{max}
\:tempb{min}
\:tempb{sup}
\:tempb{inf}
\:tempb{det}
\:tempb{Pr}
\:tempb{gcd}
>>>


%%%%%%%%%%%%%%%%%%%%%%%%%%%%%%%%%%%%%%%%%%%%%%%%%%%%%%%
\section{Sized Operations}
%%%%%%%%%%%%%%%%%%%%%%%%%%%%%%%%%%%%%%%%%%%%%%%%%%%%%%%

\<plain,latex mathml\><<<
\def\mathop:prefix#1{%
   \Configure{mathop}{*}{<\a:mathml mo #1\Hnewline>}{</\a:mathml mo>}%
    {\Configure{mathop}{*}{<\a:mathml mo>}{</\a:mathml mo>}{}}}
>>>

\<configure mathml amsmath\><<<
\def\:tempc{\ifDisplayMath\mathop:prefix{mathsize="big"}\fi\o:n:iint::}
\HLet\n:iint:\:tempc
\def\:tempc{\ifDisplayMath\mathop:prefix{mathsize="big"}\fi\o:n:iiint::}
\HLet\n:iiint:\:tempc
\def\:tempc{\ifDisplayMath\mathop:prefix{mathsize="big"}\fi\o:n:idotsint::}
\HLet\n:idotsint:\:tempc
>>>

\<configure mathml fontmath\><<<
\def\:tempd#1#2{%
   \expandafter\def\csname #1\endcsname{\mathchar"13#2}%
   \MathSymbol\mathop{#1}%
   \expandafter\def\csname #1:limits\endcsname{\expandafter
                    \:same \math:sym\mathop{#1}\limits}%
   \expandafter\def\csname #1:nolimits\endcsname{\expandafter
                  \:same \math:sym\mathop{#1}\nolimits}%
   \expandafter\edef\csname n:#1:\endcsname{\noexpand\ifDisplayMath
          \noexpand\expandafter \expandafter\noexpand \csname #1:limits\endcsname
          \noexpand\else \noexpand\expandafter
                         \expandafter\noexpand 
                         \csname #1:nolimits\endcsname\noexpand\fi}%
   \expandafter\pend:def\csname n:#1:\endcsname{\ifDisplayMath
          \mathop:prefix{mathsize="big"}\fi}%
}
\:tempd{bigcap}{54} 
\:tempd{bigcup}{53} 
\:tempd{bigodot}{4A} 
\:tempd{bigoplus}{4C} 
\:tempd{bigotimes}{4E} 
\:tempd{bigsqcup}{46} 
\:tempd{biguplus}{55} 
\:tempd{bigvee}{57} 
\:tempd{bigwedge}{56} 
\:tempd{coprod}{60} 
\:tempd{intop}{52} 
\:tempd{ointop}{48} 
\:tempd{prod}{51} 
\:tempd{sum}{50}
>>>




\<configure mathml amsmath\><<<
\ifx \ilimits@\o:nolimits:
  \let\ilimits@\nolimits
\fi
\ifx \ilimits@\o:nolimits:
  \let\slimits@\nolimits
\fi
>>>



\begin{description}
\item[sumlimits]
    (default) Place the subscripts and superscripts    of summation symbols above and below, in displayed equations. This option also affects other symbols of the same type---, , , , and so forth---but excluding integrals (see below).

\item[nosumlimits]
    Always place the subscripts and superscripts of summation-type symbols to the side, even in displayed equations.

\item[intlimits]
    Like sumlimits, but for integral  symbols.


\<configure mathml amsmath\><<<
\ifx \ilimits@\o:displaylimits: 
   \let\ilimits@\nolimits  
\fi 
>>>

We currently ignore cases of the following kinds.

\begin{verbatim}
\documentclass{article}  
    \usepackage[intlimits]{amsmath}  
\begin{document}  
  $\int_{0}$  
\end{document}  
\end{verbatim}


\item[nointlimits]
    (default) Opposite of intlimits.

\item[namelimits]
    (default) Like sumlimits, but for certain `operator names' such as , , , , , that traditionally have subscripts   placed underneath when they occur in a displayed equation.

\item[nonamelimits]
    Opposite of namelimits. 
\end{description}




\<\><<<
\pend:def\n:int:{\ifDisplayMath\mathop:prefix{mathsize="big"}\fi}
\def\sum{\mathchar"1350}
\MathSymbol\mathop{sum}
\def\sum:limits{\expandafter\:same \math:sym\mathop{sum}\limits}
\def\sum:nolimits{\expandafter\:same \math:sym\mathop{sum}\nolimits}
\def\n:sum:{\ifDisplayMath \expandafter\sum:limits\else 
                           \expandafter\sum:nolimits\fi}
\pend:def\n:sum:{\ifDisplayMath\mathop:prefix{mathsize="big"}\fi}
>>>

% %   \def\n:sum:{\expandafter\:same \math:sym\mathop{sum}\nolimits}


%%%%%%%%%%%%%%%%%%%%%%%%%%%%%%%%%%%%%%%%%%%%%%%%%%%%%%%
\section{Accent through `accents' Configurations}
%%%%%%%%%%%%%%%%%%%%%%%%%%%%%%%%%%%%%%%%%%%%%%%%%%%%%%%

Should rename `accents' to `combining' (characters).


Test file for accents

\begin{verbatim}
\documentclass{article}
\begin{document}

\def\showaccents#1#2#3#4#5#6{\cr\hline
   {\tt \ifx #1\relax\else \string#1\fi{} \string#2}                              & 
   \if !#3!\else $\ifx #1\relax\else#1{#3}\fi
                  \mbox{#2{#3}}$  {\bf #2{#3}}\fi &
   \if !#4!\else $\ifx #1\relax\else#1{#4}\fi
                  \mbox{#2{#4}}$  {\bf #2{#4}}\fi &
   \if !#5!\else $\ifx #1\relax\else#1{#5}\fi
                  \mbox{#2{#5}}$  {\bf #2{#5}}\fi 
   \if !#6!\else
        \if !#6!\else $\ifx #1\relax\else#1{#6}\fi
                       \mbox{#2{#6}}$  {\bf #2{#6}}\fi 
   \fi
}

\begin{tabular}{l|c|c|c}
& low unicode & high unicode & text combining accent \& mathml 
  \showaccents\dot\.{c}{a}{q}{X}
  \showaccents\breve\u{a}{}{c}{F}
  \showaccents\check\v{a}{h}{m}{F}
  \showaccents\bar\={a}{y}{c}{F}
  \showaccents\grave\`{a}{n}{c}{F}
  \showaccents\acute\'{a}{}{m}{F}
  \showaccents\hat\^{a}{z}{m}{F}
  \showaccents\tilde\~{a}{v}{m}{F}
  \showaccents\ddot\"{a}{h}{m}{C}
  \showaccents\mathring\r{a}{}{m}{F}
  \showaccents\relax\b{}{b}{c}{j}
  \showaccents\relax\c{c}{d}{a}{j}
  \showaccents\relax\d{a}{b}{c}{j}
  \showaccents\relax\H{o}{}{c}{F}
\end{tabular}

\end{document}
\end{verbatim}


|<temp indirect accents|>

\<config mathml accents\><<<
\Configure{accents}
  {\ifmmode \expandafter\mathord\fi
     {|<temp hcode accents|>%
   \ht:special{t4ht@[}\if!#1!x\else#2\fi
   \ht:special{t4ht@]}}}
  {\ifmathml
      \ifmtext |<accent in text mode|>%
      \else    \AccentMath{|<under over accents|>}\fi
   \else       |<accent in text mode|>\fi
  }
>>>

\<accent in text mode\><<<
\expandafter\ifx \csname :#1:over\endcsname\relax
   \expandafter\ifx \csname :#1:under\endcsname\relax
      \AccentMath{|<under over accents|>}%
   \else #3\csname :#1:under\endcsname \fi
\else #3\csname :#1:over\endcsname \fi
>>>

\<under over accents\><<<
\expandafter\ifx \csname :#1:over\endcsname\relax
  |<under accents|>%
\else \expandafter\ifx \csname :#1:under\endcsname\relax
  |<over accents|>%
\else
  |<both accents|>%
\fi \fi
>>>

\<under accents\><<<
\HCode{<\a:mathml munder\Hnewline 
   accent="true"><\a:mathml mrow\Hnewline>}#3\HCode{</\a:mathml mrow>}%
      \csname :#1:under\endcsname
   \HCode{</\a:mathml munder>}%
>>>

\<over accents\><<<
\HCode{<\a:mathml mover\Hnewline 
   accent="true"><\a:mathml mrow\Hnewline>}#3\HCode{</\a:mathml mrow>}%
   \csname :#1:over\endcsname
   \HCode{</\a:mathml mover>}%
>>>

\<both accents\><<<
\HCode{<\a:mathml munderover\Hnewline 
   accent="true"><\a:mathml mrow\Hnewline>}#3\HCode{</\a:mathml mrow>}%
   \csname :#1:under\endcsname
   \csname :#1:over\endcsname
   \HCode{</\a:mathml munderover>}%
>>>



%%%%%%%%%%%%%%%%%%%%%%%%%%%%%%%
\section{Underline and Overline}
%%%%%%%%%%%%%%%%%%%%%%%%%%%%%%%


\<configure html-mml latex\><<<
\Configure{overline}
   {\bgroup\ifmathml |<mathml overline|>\else
         \HCode{<span class="overline">}\fi}
   {\ifmathml |<end mathml overline|>\else \HCode{</span>}\fi\egroup}
\Configure{underline}
   {\bgroup
    \ifmathml |<mathml underline|>%
    \else |<nomath underline|>\HCode{<span class="underline">}\fi}
   {\ifmathml |<end mathml underline|>\else \HCode{</span>}\fi \egroup}
>>>


\<configure mathml ushort\><<<
\Configure{ushort}
   {\bgroup
    \expandafter\ushort:data
   }
   {\egroup}
\Configure{ushortw}
   {\bgroup
    \expandafter\ushort:data
   }
   {\egroup}
\Configure{ushortd}
   {\bgroup
    \expandafter\ushortd:data
   }
   {\egroup}
\Configure{ushortdw}
   {\bgroup
    \expandafter\ushortd:data
   }
   {\egroup}
\Configure{ushortdline}
   {\bgroup\expandafter\ushortdline:data }
   {\egroup}
\def\ushort:data#1#2{\underline{\:gobbleIII #2}}
\def\ushortd:data#1#2{\underline{\underline{\:gobbleIII #2}}}
\def\ushortdline:data#1#2{#1{#1{\:gobble #2}}}
>>>


\<mathml latex\><<<
|<und/ov-erline|>
>>>

\<mathml plain\><<<
|<und/ov-erline|>
>>>


\<und/ov-erline\><<<
\Configure{overline}
   {\bgroup |<mathml overline|>}
   {|<end mathml overline|>\egroup}
\Configure{underline}
   {\bgroup \ifmmode |<mathml underline|>\else 
     |<nomath underline|>\fi}
   {|<end mathml underline|>\egroup}
>>>

\<nomath underline\><<<
\expandafter\everymath
\expandafter{\expandafter\everymath
             \expandafter{\the\everymath}}%
\let\o::@underline:\o:@@underline:
\def\o:@@underline:{\let\o:@@underline:\o::@underline:}%
>>>











\<mathml overline\><<<
\HCode{<\a:mathml mover  accent="false"\Hnewline
\mml:class="mml-overline"><\a:mathml mrow>}%
>>>

\<end mathml overline\><<<
\HCode{</\a:mathml mrow><\a:mathml mo\Hnewline
 accent="true">&\#x00AF;</\a:mathml mo>%    
</\a:mathml mover>}%
>>>

\verb'&#x00AF;' is the code of \verb'&OverBar;'


\<mathml underline\><<<
\HCode{<\a:mathml munder \a:mathml accentunder="false"\Hnewline
\mml:class="mml-underline"><\a:mathml mrow>}% 
>>>

\<end mathml underline\><<<
\HCode{</\a:mathml mrow><\a:mathml mo\Hnewline
accent="true">&\#x0332;</\a:mathml mo></\a:mathml munder>}%
>>>


\verb'&#x0332;' is the code of \verb'&UnderBar;'





\<mathml plain\><<<
|<arr: amsmath, amstex1|>
>>>



\<configure mathml accents\><<<
\Configure{accentset}
   {\HCode{<mover accent="true" class="accentset"><mrow>}}
   {\HCode{</mrow><mrow> }}
   {\HCode{</mrow></mover>}} 
\Configure{underaccent}
   {\HCode{<munder class="underaccent"><mrow>}}
   {\HCode{</mrow><mrow> }}
   {\HCode{</mrow></munder>}} 
>>>






%%%%%%%%%%%%%%%%%%%%%%%%%%%%%%%%%%%
\section{Choose}
%%%%%%%%%%%%%%%%%%%%%%%%%%%%%%%%%%%%%

\<plain,latex mathmlNO\><<<
\Configure{choose}
   {\Send{GROUP}{0}{<\a:mathml mfrac linethickness="0" \mml:class="choose">\Hnewline
     <\a:mathml mrow \mml:class="choose-1">}\HCode{</\a:mathml mrow>}}
   {\HCode{\Hnewline<\a:mathml mrow \mml:class="choose-2">}
       \Send{EndGROUP}{0}{</\a:mathml mrow></\a:mathml mfrac>}}
>>>





%%%%%%%%%%%%%%%%%%%%%%%%%%%%
\section{Footnotes}
%%%%%%%%%%%%%%%%%%%%%%%%%%%%

\<configure html-mml latex\><<<
\Configure{footnote-mark}
   {\ifmathml \ifmtext \else
        \HCode{<\a:mathml mrow \mml:class="footnote-mark">}\fi
    \else \HCode{<span class="footnote-mark">}\fi }
   {\ifmathml \ifmtext \else \HCode{</\a:mathml mrow>}\fi
    \else \HCode{</span>}\fi }
\Configure{thank-mark}
   {\HCode{<\ifmathml \a:mathml mrow  \expandafter\mml:class\else span class\fi
                                   ="thank-mark">}}
   {\HCode{</\ifmathml \a:mathml mrow\else span\fi>}}  
>>>





The following code segments handle the different listed cases,
with possible exception of the last.

\begin{verbatim}
a\footnote{b c}

$a\footnote{b c}$

$a\footnote{\hbox{b c}}$

$a\footnote{\mbox{b c}}$

$a\mbox{\footnote{b c}}$
\end{verbatim}


We have the following hooks


\begin{verbatim}
\def\@footnotemark{...\a:@makefnmark\@makefnmark
                                           \b:@makefnmark...}
\long\def\@footnotetext#1{...\a:footnote-text\a:@makefntext
                \@makefntext{\b:@makefntext
                                   \a:@makefnbody #1\b:@makefnbody
                            }\c:@makefntext\b:footnote-text...}
\end{verbatim}





\<configure html-mml latex\><<<
\:CheckOption{fn-in}\if:Option  \else
   \expandafter\let\csname A:footnote-text\expandafter\endcsname
                     \csname a:footnote-text\endcsname
   \expandafter\let\csname B:footnote-text\expandafter\endcsname
                     \csname b:footnote-text\endcsname
   \NewConfigure{footnote-text}[2]{%
      \expandafter\def\csname a:footnote-text\endcsname{%
         #1%
         \ifmathml
             \HCode{<\a:mathml math\Hnewline
                \csname a:math-xmlns\endcsname \Hnewline \a:@math>%
                \ifmtext <\a:mathml mtext>\fi}
         \fi 
         \ifmathml  \fi
      }%
      \expandafter\def\csname b:footnote-text\endcsname{%
         \ifmathml  
            \HCode{\ifmtext</\a:mathml mtext>\fi </\a:mathml math>}%
         \fi 
         #2%
      }%
   }
   \Configure{footnote-text}
      {\csname A:footnote-text\endcsname}
      {\csname B:footnote-text\endcsname}
\fi
>>>



\<configure html-mml latex\><<<
\let\A:footnotetext=\a:footnotetext
\let\B:footnotetext=\b:footnotetext
\let\C:footnotetext=\c:footnotetext
\NewConfigure{footnotetext}[3]{%
   \def\a:footnotetext{%
      \ifmathml \ifmtext  |<end mtext space|>\fi\fi
      #1%
      \ifmathml \ifmtext  |<end mtext space|>\fi\fi
   }%
   \def\b:footnotetext{%
      \ifmathml \ifmtext  |<end mtext space|>\fi\fi
      #2%
      \ifmathml \ifmtext  |<end mtext space|>\fi\fi
   }%
   \def\c:footnotetext{%
      \ifmathml \ifmtext  |<end mtext space|>\fi\fi
      #3%
      \ifmathml \ifmtext  |<start mtext space|>\fi\fi
   }%
}
\Configure{footnotetext}
   {\A:footnotetext}
   {\B:footnotetext}
   {\C:footnotetext}
>>>

\<configure html-mml latex\><<<
\Configure{footnotebody}
   {\tmp:toks=}
   {\ifmathml \ifmmode $\mbox{$\the\tmp:toks$}$ 
              \else $\mbox{\the\tmp:toks}$\fi
    \else \the\tmp:toks\fi}
>>>


\<configure html-mml latex\><<<
\:CheckOption{mouseover} \if:Option
   \pend:def\js:on{\ifmathml \ifmtext  |<end mtext space|>\fi\fi}
   \append:def\js:on{\ifmathml \ifmtext  |<start mtext space|>\fi\fi}
\fi
>>>

%%%%%%%%%%%%%%%%%%%%%%%%%%%%
\section{Space Characters}
%%%%%%%%%%%%%%%%%%%%%%%%%%%%


\<plain,latex mathml\><<<
\def\:nbsp{\ifmtext |<unicode space|>\else\ifmathml
    \Tg<\a:mathml mspace />\else |<unicode space|>\fi\fi}
>>>

\<unicode space\><<<
\leavevmode\special{t4ht@+\string&{35}x00A0{59}}x%
>>>

% \HCode{\string&}\HChar{-35}\HCode{x00A0;}%


\<configure html-mml frenchb\><<<
\Configure{frenchb-nbsp}
   {\ifmtext |<unicode space|>\else\ifmathml
       \Tg<\a:mathml mspace class="frenchb-nbsp" />%
    \else 
      \ht:special{t4ht@?unhskip}\HCode{<span class="frenchb-thinspace">}%
      |<unicode space|>\HCode{</span>}%
    \fi\fi
    \setbox\tmp:bx=\hbox\bgroup}
   {\egroup}
>>>  


%%%%%%%%%%%%%%%%%%%%%%%%%%%%
\section{Matrices}
%%%%%%%%%%%%%%%%%%%%%%%%%%%%

%%%%%%%%%%%%%
\subsection{Aligned}
%%%%%%%%%%%%%



\<configure mathml amsmath\><<<
\Configure{aligned}  
   {\HCode{<\a:mathml mtable\Hnewline \mml:class="aligned">}}  
   {\HCode{</\a:mathml mtable>}}  
   {\HCode{<\a:mathml mtr\Hnewline>}}   {\HCode{</\a:mathml mtr>}}  
   {\HCode{<\a:mathml mtd 
       \a:mathml columnalign="\ifodd\HCol right\else left\fi" >}}
   {\HCode{</\a:mathml mtd>}} 
\Configure{alignedat}
   {\HCode{<\a:mathml mtable\Hnewline \mml:class="alignedat">}}
   {\HCode{</\a:mathml mtable>}}
   {\HCode{<\a:mathml mtr\Hnewline>}}   {\HCode{</\a:mathml mtr>}}
   {\HCode{<\a:mathml mtd 
       \a:mathml columnalign="\ifodd\HCol right\else left\fi" >}}
   {\HCode{</\a:mathml mtd>}}
>>>



%%%%%%%%%%%%%%%%%%%%
\subsection{Cases}
%%%%%%%%%%%%%%%%%

\<plain,latex mathml\><<<
\Configure{cases}
    {}{} {\Tg<\a:mathml mtable>}{\Tg</\a:mathml mtable>}
    {\Tg<\a:mathml mtr>}{\Tg</\a:mathml mtr>}
    {\Tg<\a:mathml mtd \mml:class="cases">}{\Tg</\a:mathml mtd>}
>>>


%%%%%%%%%%%%%%%%%%%%
\subsection{matrix}
%%%%%%%%%%%%%%%%%%%%

\<plain,latex mathml\><<<
\Configure{matrix}
   {\Tg<\a:mathml mtable 
        \a:mathml equalrows="false" \a:mathml equalcolumns="false"\Hnewline
          \mml:class="matrix">} 
   {\Tg</\a:mathml mtable>}
   {\Tg<\a:mathml mtr>}{\Tg</\a:mathml mtr>}
   {|<mtd matrix|>}{\Tg</\a:mathml mtd>}
>>>



%%%%%%%%%%%%%%%%%%%%
\subsection{pmatrix}
%%%%%%%%%%%%%%%%%%%%


The grouping below is handle the case that the matrix is a base of an exponent.


\<plain,latex mathml\><<<
\Configure{pmatrix}  {\bgroup\Tg<\a:mathml mrow\Hnewline
   \mml:class="pmatrix">}  {\Tg</\a:mathml mrow>\egroup}
>>>



The grouping below is handle the case that the matrix is a base of an
exponent. The \verb|\,| configurations is for facilitating the
catching of  left subscripts and superscripts.


\<plain,latex mathml\><<<
\Configure{quad}{{\ifmathml \ifmtext \o:quad:\else
   {\HCode{<\a:mathml mspace width="1em" \mml:class="quad"/>}}\fi
    \else \o:quad: \fi}}
\Configure{qquad}{{\ifmathml \ifmtext \o:qquad:\else
   {\HCode{<\a:mathml mspace width="2em" \mml:class="qquad"/>}}\fi
    \else \o:qquad: \fi}}
\Configure{,}
   {{\HCode{<\a:mathml mspace width="0em" \mml:class="thinspace"/>}}}
>>>


%%%%%%%%%%%%%
\subsection{Array Border Lines}
%%%%%%%%%%%%%


\<border lines\><<<
\border:lines
>>>

\<configure html-mml latex\><<<
\def\border:lines{%
  \ifnum \borderlines>0 
     \ifnum \borderlines>1 border-right:solid 1px black;\fi
     \ifodd \borderlines border-left:solid 1px black;\fi
  \fi
}
>>>




\<config array\><<<
\Configure{VBorder}
   {\HAssign\NewGroup = 0 
    |<init vborder|>}
   {|<vborder solid|>%
    \HAssign\NewGroup = 0}
   {\Advance:\NewGroup by 1 
    |<record cur vborder|>}
   {}
>>>



\<vborder html-mml latex/array\><<<
\Configure{VBorder}
   {\let\VBorder=\empty \let\AllColMargins=\empty
    \global\let\GROUPS=\empty \HAssign\NewGroup = 0 
    \gHAdvance\Next:TableNo by 1 \global\let\TableNo=\Next:TableNo
    |<init vborder|>%
   }
   {|<vborder solid|>%
    \xdef\VBorder{\VBorder\ifnum \NewGroup> 0 </colgroup>\fi}
    \HAssign\NewGroup = 0  \gdef\GROUPS{rules="groups"}}
   {\Advance:\NewGroup by 1
    |<record cur vborder|>% 
    \ifnum \NewGroup=1 \xdef\VBorder{\VBorder<colgroup 
        id="TBL-\TableNo-\ar:cnt g">}\fi      
    \xdef\VBorder{\VBorder<col\Hnewline id="TBL-\TableNo
       -\ar:cnt"\xml:empty>}\xdef\AllColMargins{\AllColMargins1}}
   {\xdef\AllColMargins{\AllColMargins 0}}
>>>



\<init vborder\><<<
\HAssign\borderlines = 0
\global\let\columnlines=\relax
>>>

\<vborder solid\><<<
\ifx \columnlines\relax \gHAssign\borderlines = 1
                      \else \gHAdvance\borderlines by 2 \fi
>>>


\<record cur vborder\><<<
\ifnum \borderlines>1 \gHAdvance\borderlines by -2 \fi
\ifx \columnlines\relax 
    \global\let\columnlines\empty
\else
    \ifnum \NewGroup=1 
       \xdef\columnlines{\columnlines\ifx\columnlines\empty
                                     \else \space\fi solid}%
    \else
       \xdef\columnlines{\columnlines\ifx\columnlines\empty
                                     \else \space\fi none}%
    \fi
\fi
>>>

\<configure html-mml array\><<<
|<vborder html-mml latex/array|>
>>>

\<configure html-mml latex\><<<
|<vborder html-mml latex/array|>
>>>





%%%%%%%%%%%%%%%%%%%%
\section{TeX}
%%%%%%%%%%%%%%%%%%%%


\<mathml plain\><<<
\Configure{eqalignno}
   {\HCode{<\a:mathml mtable \mml:class="eqalignno">}}
   {\HCode{</\a:mathml mtable>}}
   {\HCode{<\a:mathml mtr\Hnewline \mml:class="eqalignno">}}
   {\HCode{</\a:mathml mtr>}}
   {\HCode{<\a:mathml mtd \Hnewline \mml:class="eqalignno-\HCol">}}
   {\HCode{</\a:mathml mtd>}}
\def\:eqalignno:{\Configure{noalign}
  {\HCode{<\a:mathml mtr><\a:mathml mtd \mml:class="noalign" colspan="2">}}%
  {\HCode{</\a:mathml mtd></\a:mathml mtr>}}}
>>>


\<mathml plain\><<<
\Configure{leqalignno}
   {\HCode{<\a:mathml mtable \mml:class="leqalignno">}}
   {\HCode{</\a:mathml mtable>}}
   {\HCode{<\a:mathml mtr\Hnewline \mml:class="leqalignno">}}
   {\HCode{</\a:mathml mtr>}}
   {\HCode{<\a:mathml mtd \Hnewline \mml:class="leqalignno-\HCol">}}
   {\HCode{</\a:mathml mtd>}}
\def\:leqalignno:{\Configure{noalign}
  {\HCode{<\a:mathml mtr><\a:mathml mtd \mml:class="noalign" colspan="2">}}%
  {\HCode{</\a:mathml mtd></\a:mathml mtr>}}}
>>>

\section{Fractions}


         




\<amstex.sty mathml\><<<
|<config mathml amstex1|>
>>>


\<config mathml amstex1\><<<
\ConfigureEnv{aligned}{}{}{}{}
\Configure{aligned}
   {\HCode{<\a:mathml mtable\Hnewline \mml:class="aligned">}}
   {\HCode{</\a:mathml mtable>}}
   {\HCode{<\a:mathml mtr\Hnewline>}}   {\HCode{</\a:mathml mtr>}}
   {\HCode{<\a:mathml mtd>}}   {\HCode{</\a:mathml mtd>}}
>>>


\<amstex.tex mathml\><<<
\Configure{multilimits}
   {\HCode{<\a:mathml mtable\Hnewline \mml:class="multilimits">}}
   {\HCode{</\a:mathml mtable>}}
   {\HCode{<\a:mathml mtr><\a:mathml mtd\Hnewline>}}
   {\HCode{</\a:mathml mtd></\a:mathml mtr>}}
\Configure{smallmatrix}
   {\HCode{<\a:mathml mtable\Hnewline \mml:class="smallmatrix">}}
   {\HCode{</\a:mathml mtable>}}
   {\HCode{<\a:mathml mtr>}}
   {\HCode{</\a:mathml mtr>}} 
   {\HCode{<\a:mathml mtd\Hnewline>}}                 
   {\HCode{</\a:mathml mtd>}}
>>>

\<amstex.tex mathml\><<<
\Configure{matrix}
   {\HCode{<\a:mathml mtable\Hnewline \mml:class="matrix">}}
   {\HCode{</\a:mathml mtable>}}
   {\HCode{<\a:mathml mtr\Hnewline>}}   {\HCode{</\a:mathml mtr>}}
   {\HCode{|<mtd matrix|>}}   {\HCode{</\a:mathml mtd>}}
>>>

\<amstex.tex mathml\><<<
\Configure{pmatrix}{\Tg<\a:mathml mrow \mml:class="pmatrix">}
                   {\Tg</\a:mathml mrow>}
>>>

\<amstex.tex mathml\><<<
\Configure{align}
   {\HCode{<\a:mathml mtable\Hnewline \mml:class="align">}}
   {\HCode{</\a:mathml mtable>}}
   {\HCode{<\a:mathml mtr\Hnewline>}}   {\HCode{</\a:mathml mtr>}}
   {\HCode{<\a:mathml mtd>}}   {\HCode{</\a:mathml mtd>}}
\Configure{aligned}
   {\HCode{<\a:mathml mtable\Hnewline \mml:class="aligned">}}
   {\HCode{</\a:mathml mtable>}}
   {\HCode{<\a:mathml mtr\Hnewline>}}   {\HCode{</\a:mathml mtr>}}
   {\HCode{<\a:mathml mtd>}}   {\HCode{</\a:mathml mtd>}}
>>>

\<amstex.tex mathml\><<<
\Configure{cases}
   {\Tg<\a:mathml mrow \mml:class="cases">}
   {\Tg</\a:mathml mrow>}
\Configure{alignat}
   {\HCode{<\a:mathml mtable \mml:class="alignat">}}
   {\HCode{</\a:mathml mtable>}}
   {\HCode{<\a:mathml mtr\Hnewline \mml:class="alignat">}}
   {\HCode{</\a:mathml mtr>}}
   {\HCode{<\a:mathml mtd\Hnewline \mml:class="alignat">}}
   {\HCode{</\a:mathml mtd>}}
\Configure{gather}
   {\HCode{<\a:mathml mtable \mml:class="gather">}}
   {\HCode{</\a:mathml mtable>}}
   {\HCode{<\a:mathml mtr\Hnewline \mml:class="gather">}}
   {\HCode{</\a:mathml mtr>}}
   {\HCode{<\a:mathml mtd\Hnewline \mml:class="gather">}}
   {\HCode{</\a:mathml mtd>}}
>>>>

\<vanilla mathml\><<<
\Configure{align}
   {\EndP\HCode{<\a:mathml mtable\Hnewline \mml:class="align">}} 
   {\HCode{</\a:mathml mtable>}}
   {\HCode{<\a:mathml mtr>}}   {\HCode{</\a:mathml mtr>}}
   {\HCode{<\a:mathml mtd>}}   {\HCode{</\a:mathml mtd>}}
>>>







% {\mathchar"0403} % \square
% {\mathchar"0404} % \blacksquare
% {\mathchar"0406} % \lozenge
% {\mathchar"0407} % \blacklozenge
% {\mathchar"0438} % \backprime
% {\mathchar"0446} % \bigstar
% {\mathchar"0448} % \blacktriangledown
% {\mathchar"044E} % \blacktriangle
% {\mathchar"044F} % \triangledown
% {\mathchar"045D} % \measuredangle
% {\mathchar"045E} % \sphericalangle
% {\mathchar"0473} % \circledS
% {\mathchar"047B} % \complement
% {\mathchar"051E} % \diagup
% {\mathchar"051F} % \diagdown
% {\mathchar"053F} % \varnothing
% {\mathchar"0540} % \nexists
% {\mathchar"0560} % \Finv
% {\mathchar"0561} % \Game
% {\mathchar"0567} % \eth
% {\mathchar"0569} % \beth
% {\mathchar"056A} % \gimel
% {\mathchar"056B} % \daleth
% {\mathchar"057A} % \digamma
% {\mathchar"057B} % \varkappa
% {\mathchar"057C} % \Bbbk
% {\mathchar"057D} % \hslash







\<math fonts utilities\><<<
\ifx \tmp:bx\:UnDef \csname newbox\endcsname \tmp:bx \fi
\ifx \find:set:mgroup\:UnDef 
   \def\find:set:mgroup#1#2#3{\bgroup
     \def\use@mathgroup##1##2##3{\relax
       \tmp:cnt=##2%
       \xdef\:temp####1{\noexpand\Configure{MathClass}{#2}%
           {}{}{}{\mathchar"0\the\tmp:cnt ####1}}%
     }%
     \setbox\tmp:bx=\hbox{$#1$}\setbox\tmp:bx=\hbox{}% 
     \set:mgroup#3{}{}%
     \egroup
   }
\fi
\def\set:mgroup#1#2{\if :#1#2:\else
   \:temp{#1#2}\expandafter\set:mgroup \fi}
>>>



\<set mathcal class\><<<
\ifx \mathboldcalMathClass\:UnDef
  \NewMathClass\mathboldcalMathClass
\fi
\Configure{MathClass}{\mathboldcalMathClass}{*}
    {<\a:mathml mi\Hnewline
         mathvariant="bold-script">}{</\a:mathml mi>}{}  
>>>

\<set mathcal symbols\><<<
\expandafter\ifx\csname mv@bold\endcsname\relax  \else
\bgroup
\mathversion{bold}%
\find:set:mgroup{\mathcal{A}}{\mathboldcalMathClass}%
  {4142434445464748494A4B4C4D4E4F505152535455565758595A} |%A--Z|%
\egroup
\fi
>>>






\<configure mathml amssymb\><<<
\def\:tempa#1"#2#3///{%
   \edef\:next{\noexpand\Configure{MathClass}{\ifnum 0=#2
                                     \mathordMathClass \else #2\fi}}%
   \:next
}
% \def\:tempb#1{\:tempa#1///{}{}{}{#1}}
% \def\:tempc#1"#2//{\:tempb{\mathchar"#2}} 
% \def\:temp#1{\expandafter\:tempc\meaning#1//}
\def\:temp#1{\expandafter\:tempc\meaning#1//}
\def\:tempb#1{\:tempa#1///{}{}{}{#1}}
\def\:tempc#1"#2//{\:tempd#2...//}
\def\:tempd#1#2#3#4#5//{%
   \if #4.\:tempb{\mathchar"0#1#2#3}\else
          \:tempb{\mathchar"#1#2#3#4}\fi
}
\:temp \boxdot
\:temp \boxplus
\:temp \boxtimes
\:temp \centerdot
\:temp \boxminus
\:temp \veebar
\:temp \barwedge
\:temp \doublebarwedge
\:temp \Cup
\:temp \Cap
\:temp \curlywedge
\:temp \curlyvee
\:temp \leftthreetimes
\:temp \rightthreetimes
\:temp \dotplus
\:temp \intercal
\:temp \circledcirc
\:temp \circledast
\:temp \circleddash
\:temp \divideontimes
\:temp \lessdot
\:temp \gtrdot
\:temp \ltimes
\:temp \rtimes
\:temp \smallsetminus
\:temp \circlearrowright
\:temp \circlearrowleft
\:temp \leftrightharpoons
\:temp \Vdash
\:temp \Vvdash
\:temp \vDash
\:temp \twoheadrightarrow
\:temp \twoheadleftarrow
\:temp \leftleftarrows
\:temp \rightrightarrows
\:temp \upuparrows
\:temp \downdownarrows
\:temp \upharpoonright
\:temp \downharpoonright
\:temp \upharpoonleft
\:temp \downharpoonleft
\:temp \rightarrowtail
\:temp \leftarrowtail
\:temp \leftrightarrows
\:temp \rightleftarrows
\:temp \Lsh
\:temp \Rsh
\:temp \rightsquigarrow
\:temp \leftrightsquigarrow
\:temp \looparrowleft
\:temp \looparrowright
\:temp \circeq
\:temp \succsim
\:temp \gtrsim
\:temp \gtrapprox
\:temp \multimap
\:temp \therefore
\:temp \because
\:temp \doteqdot
\:temp \triangleq
\:temp \precsim
\:temp \lesssim
\:temp \lessapprox
\:temp \eqslantless
\:temp \eqslantgtr
\:temp \curlyeqprec
\:temp \curlyeqsucc
\:temp \preccurlyeq
\:temp \leqq
\:temp \leqslant
\:temp \lessgtr
\:temp \risingdotseq
\:temp \fallingdotseq
\:temp \succcurlyeq
\:temp \geqq
\:temp \geqslant
\:temp \gtrless
\:temp \vartriangleright
\:temp \vartriangleleft
\:temp \trianglerighteq
\:temp \trianglelefteq
\:temp \between
\:temp \blacktriangleright
\:temp \blacktriangleleft
\:temp \vartriangle
\:temp \eqcirc
\:temp \lesseqgtr
\:temp \gtreqless
\:temp \lesseqqgtr
\:temp \gtreqqless
\:temp \Rrightarrow
\:temp \Lleftarrow
\:temp \varpropto
\:temp \smallsmile
\:temp \smallfrown
\:temp \Subset
\:temp \Supset
\:temp \subseteqq
\:temp \supseteqq
\:temp \bumpeq
\:temp \Bumpeq
\:temp \lll
\:temp \ggg
\:temp \pitchfork
\:temp \backsim
\:temp \backsimeq
\:temp \lvertneqq
\:temp \gvertneqq
\:temp \nleq
\:temp \ngeq
\:temp \nless
\:temp \ngtr
\:temp \nprec
\:temp \nsucc
\:temp \lneqq
\:temp \gneqq
\:temp \nleqslant
\:temp \ngeqslant
\@ifpackageloaded{program}{}{\:temp \lneq}
\:temp \gneq
\:temp \npreceq
\:temp \nsucceq
\:temp \precnsim
\:temp \succnsim
\:temp \lnsim
\:temp \gnsim
\:temp \nleqq
\:temp \ngeqq
\:temp \precneqq
\:temp \succneqq
\:temp \precnapprox
\:temp \succnapprox
\:temp \lnapprox
\:temp \gnapprox
\:temp \nsim
\:temp \ncong
\:temp \varsubsetneq
\:temp \varsupsetneq
\:temp \nsubseteqq
\:temp \nsupseteqq
\:temp \subsetneqq
\:temp \supsetneqq
\:temp \varsubsetneqq
\:temp \varsupsetneqq
\:temp \subsetneq
\:temp \supsetneq
\:temp \nsubseteq
\:temp \nsupseteq
\:temp \nparallel
\:temp \nmid
\:temp \nshortmid
\:temp \nshortparallel
\:temp \nvdash
\:temp \nVdash
\:temp \nvDash
\:temp \nVDash
\:temp \ntrianglerighteq
\:temp \ntrianglelefteq
\:temp \ntriangleleft
\:temp \ntriangleright
\:temp \nleftarrow
\:temp \nrightarrow
\:temp \nLeftarrow
\:temp \nRightarrow
\:temp \nLeftrightarrow
\:temp \nleftrightarrow
\:temp \eqsim
\:temp \shortmid
\:temp \shortparallel
\:temp \thicksim
\:temp \thickapprox
\:temp \approxeq
\:temp \succapprox
\:temp \precapprox
\:temp \curvearrowleft
\:temp \curvearrowright
\:temp \backepsilon
\:temp \square
\:temp \blacksquare
\:temp \lozenge
\:temp \blacklozenge
\:temp \backprime
\:temp \bigstar
\:temp \blacktriangledown
\:temp \blacktriangle
\:temp \triangledown
\:temp \measuredangle
\:temp \sphericalangle
\:temp \circledS
\:temp \complement
\:temp \diagup
\:temp \diagdown
\:temp \varnothing
\:temp \nexists
\:temp \Finv
\:temp \Game
\:temp \eth
\:temp \beth
\:temp \gimel
\:temp \daleth
\:temp \digamma
\:temp \varkappa
\:temp \Bbbk
\:temp \hslash
>>>

Note that in the presence of bm.sty the symbols of amssymb change their
codings.

\begin{verbatim}
\documentclass{article}  
% \usepackage{bm}  
\usepackage{amsmath,amssymb}  
\begin{document}  
\hshow{boxdot} 
\begin{equation}  
2 
\end{equation}  
\end{document} 
\end{verbatim}


%%%%%%%%%%%%%%%%%%
\chapter{Loose Ends}
%%%%%%%%%%%%%%%%%%

%%%%%%%%%%%%%%%%%%
\section{Commutative Diagrams (amscd)}
%%%%%%%%%%%%%%%%%%



\<configure mathml amscd\><<<
\ConfigureEnv{CD}{}{}{}{}
\Configure{CD}    
   {\HCode{<\a:mathml mtable class="CD"\Hnewline>}}  
   {\HCode{</\a:mathml mtable>}}  
   {\HCode{<\a:mathml mtr>}}  
   {\HCode{</\a:mathml mtr>}}  
   {\HCode{<\a:mathml mtd\Hnewline>}}  
   {\HCode{</\a:mathml mtd>}}  
>>>




\<configure mathml amscd\><<<
\Configure{ @>}
  {\Configure{mathrel*}{*}   
          {<mrow \mml:class="rightarrow" >}   
          {</mrow>}   
          {}}
>>>


\<configure mathml amscd\><<<
\Configure{ @<}
  {\Configure{mathrel*}{*}   
          {<mrow \mml:class="leftarrow" >}   
          {</mrow>}   
          {}}
>>>

\<configure mathml amscd\><<<
\Configure{ @V}
   {\HCode{<mphantom>}
    \hbox{$\m@th\vcenter{\hbox{$\scriptstyle#2$}}$}%%
    \HCode{</mphantom>}}
   {\HCode{<mphantom>}
    \hbox{$\m@th\vcenter{\hbox{$\m@th\vcenter{\hbox 
                           {$\scriptstyle#1$}}$}}$}%
    \HCode{</mphantom>}}
>>>

Are there \verb+\llap+and  \verb+\rlap+ facilities in mathml?

%%%%%%%%%%%%%%%%%%
\section{Right-to-Left Babel}
%%%%%%%%%%%%%%%%%%



\<rlbabel everypar\><<<
\expandafter\ifx \csname o@everypar\endcsname\relax \else
\expandafter\ifx \csname ht:everypar\endcsname\relax
  \let\ht:everypar\o@everypar
\fi\fi
>>>






\<dir ltr/rtl\><<<
dir="\if@rl rtl\else ltr\fi"
>>>


\<configure html-mml rlbabel\><<<
\Configure{moreL}
  {\ifmathml |<ltr rlbabel mtext|>%
   \else
      \ifvmode \let\end:moreRL=\empty
      \else\HCode{<span dir="ltr" >}%
        \def\end:moreRL{\HCode{</span>}}%
      \fi
   \fi
  }
  {\end:moreRL}
\Configure{moreR}
  {\ifmathml |<rtl rlbabel mtext|>%
   \else
      \ifvmode \let\end:moreRL=\empty 
      \else\HCode{<span dir="rtl" >}\def\end:moreRL{\HCode{</span>}}\fi
   \fi
  }
  {\end:moreRL}
\Css{.rtl {direction: rtl}  
     .ltr {direction: ltr} }
>>>



\<ltr rlbabel mtext\><<<
\ifmtext
    \leavevmode\special{t4ht@+</\a:mathml mtext><\a:mathml mtext{32}\mml:class="ltr">}x%
    \def\end:moreRL{\HCode{</\a:mathml mtext><\a:mathml mtext>}}%
\else
   \let\end:moreRL=\empty
\fi
>>>



\<rtl rlbabel mtext\><<<
\ifmtext
    \leavevmode\special{t4ht@+</\a:mathml mtext><\a:mathml mtext{32}\mml:class="rtl">}x%
    \def\end:moreRL{\HCode{</\a:mathml mtext><\a:mathml mtext>}}%
\else
    \let\end:moreRL=\empty
\fi
>>>


%%%%%%%%%%%%%
\subsection{Tables}
%%%%%%%%%%%%%


%%%%%%%%%%%%%
\subsection{dcolumn}
%%%%%%%%%%%%%


\<configure mathml dcolumn\><<<
\Configure{dcolumn}  
  {\NoFonts\Configure{$}{}{}{}} 
  {\EndNoFonts} 
>>>




%%%%%%%%%%%%%%%%%%%%%%%%%%%%%%%%%%%%%%%%%%%%%%%%%%%%%%%%%%%%%%%%%%%%%%%%
\section{openmath}
%%%%%%%%%%%%%%%%%%%%%%%%%%%%%%%%%%%%%%%%%%%%%%%%%%%%%%%%%%%%%%%%%%%%%%%%
  

\<itemitem 0.0\><<<
\Configure{itemitem}{}{}{\par\leavevmode}{}
>>>

  


Used to have also a \verb'\special{t4ht@[}...\special{t4ht@]}',
where the specials asked to igore the enclosed
content. It is provided for getting `realistic' measurements in mathml
applets. a candidadte for delition.









%%%%%%%%%%%%%%%%%%%%%%%%%%%%%%%%%%%%%%%%%%%%%%%%%%%%%%%%%%%%%%%%
\chapter{Shared}
%%%%%%%%%%%%%%%%%%%%%%%%%%%%%%%%%%%%%%%%%%%%%%%%%%%%%%%%%%%%%%%%


\<par del\><<<
!*?: >>>


\<tag of Tag\><<<
 cw:>>>

\<tail\><<<
tail>>>

\<addr for Tag and Ref of Sec\><<<
\xdef\:cursec{|<section html addr|>}%
>>>


\<save catcodes\><<<
\expandafter\edef\csname :RestoreCatcodes\endcsname{%
   \expandafter\ifx \csname :RestoreCatcodes\endcsname\relax\else
      \csname :RestoreCatcodes\endcsname \fi
   \catcode`\noexpand :|=\the\catcode`:%
   \let\expandafter\noexpand\csname :RestoreCatcodes\endcsname|=
                                   \noexpand\UnDefcS}
\catcode`\:|=11 
>>>






%%%%%%%%%%%%%%%%%%
\section{Boxed}
%%%%%%%%%%%%%%%%%%

\<configure html-mml amsmath\><<<
\Configure{boxed}
   {\ifmathml \Tg<\a:mathml mrow \mml:class="boxed">%
    \else     \HCode{<span class="boxed">}\IgnorePar\fi}
   {\ifmathml \Tg</\a:mathml mrow>%
    \else \HCode{</span>}\fi}
\Css{.boxed{border: 1px solid black; padding-left:2px;
                                     padding-right:2px;}}
>>>

%%%%%%%%%%%%%%%%%%
\section{fontmath}
%%%%%%%%%%%%%%%%%%




\<fontmath mathord symbols as ?\><<<
\mathchar"3230 |%\prime|%  
\mathchar"3231 |%\infty|%  
\mathchar"323B |%\emptyset|%  
\mathchar"323C |%\Re|%  
\mathchar"323D |%\Im|%  
\mathchar"3240 |%\aleph|%  
>>>

\<fontmath mathord symbols as op\><<<
\mathchar"3234 |%\triangle|%  
\mathchar"3238 |%\forall|%  
\mathchar"3239 |%\exists|%  
\mathchar"323A |%\neg|%  
\mathchar"323E |%\top|%  
\mathchar"323F |%\bot|%  
\mathchar"3272 |%\nabla|%  
\mathchar"3278 |%\mathsection|%  
\mathchar"327B |%\mathparagraph|%  
\mathchar"327C |%\clubsuit|%  
\mathchar"327D |%\diamondsuit|%  
\mathchar"327E |%\heartsuit|%  
\mathchar"327F |%\spadesuit|%  
>>>



\subsection{1: Large Operators}

\begin{verbatim}
\mathchardef\coprod="1360
\mathchardef\bigvee="1357
\mathchardef\bigwedge="1356
\mathchardef\biguplus="1355
\mathchardef\bigcap="1354
\mathchardef\bigcup="1353
\mathchardef\intop="1352 \def\int{\intop\nolimits}
\mathchardef\prod="1351
\mathchardef\sum="1350
\mathchardef\bigotimes="134E
\mathchardef\bigoplus="134C
\mathchardef\bigodot="134A
\mathchardef\ointop="1348 \def\oint{\ointop\nolimits}
\mathchardef\bigsqcup="1346
\mathchardef\smallint="1273
\end{verbatim}

\<plain,fontmath math classes\><<<
\Configure{MathClass}{1}{}{}{}{
   \mathchar"1360
   \mathchar"1357
   \mathchar"1356
   \mathchar"1355
   \mathchar"1354
   \mathchar"1353
   \mathchar"1352
   \mathchar"1351
   \mathchar"1350
   \mathchar"134E
   \mathchar"134C
   \mathchar"134A
   \mathchar"1348
   \mathchar"1346
   \mathchar"1273
   |<fontmath mathord symbols as op|>%
}
>>>

%    \mathchar"230          |%math prime|%

A prime is typically  placed on letters and digits.


\subsection{2: Binary Operations}

\begin{verbatim}
\mathchardef\triangleleft="212F
\mathchardef\triangleright="212E
\mathchardef\bigtriangleup="2234
\mathchardef\bigtriangledown="2235
\mathchardef\wedge="225E \let\land=\wedge
\mathchardef\vee="225F \let\lor=\vee
\mathchardef\cap="225C
\mathchardef\cup="225B
\mathchardef\ddagger="227A
\mathchardef\dagger="2279
\mathchardef\sqcap="2275
\mathchardef\sqcup="2274
\mathchardef\uplus="225D
\mathchardef\amalg="2271
\mathchardef\diamond="2205
\mathchardef\bullet="220F
\mathchardef\wr="226F
\mathchardef\div="2204
\mathchardef\odot="220C
\mathchardef\oslash="220B
\mathchardef\otimes="220A
\mathchardef\ominus="2209
\mathchardef\oplus="2208
\mathchardef\mp="2207
\mathchardef\pm="2206
\mathchardef\circ="220E
\mathchardef\bigcirc="220D
\mathchardef\setminus="226E % for set difference A\setminus B
\mathchardef\cdot="2201
\mathchardef\ast="2203
\mathchardef\times="2202
\mathchardef\star="213F
\mathcode`\*="2203 % \ast
\mathcode`\+="202B
\mathcode`\-="2200
\end{verbatim}

\<plain,fontmath math classes\><<<
\Configure{MathClass}{2}{}{}{}{
*-+/
\mathchar"212F
\mathchar"212E
\mathchar"2234
\mathchar"2235
\mathchar"225E 
\mathchar"225F 
\mathchar"225C
\mathchar"225B
\mathchar"227A
\mathchar"2279
\mathchar"2275
\mathchar"2274
\mathchar"225D
\mathchar"2271
\mathchar"2205
\mathchar"220F
\mathchar"226F
\mathchar"2204
\mathchar"220C
\mathchar"220B
\mathchar"220A
\mathchar"2209
\mathchar"2208
\mathchar"2207
\mathchar"2206
\mathchar"220E
\mathchar"220D
\mathchar"226E 
\mathchar"2201
\mathchar"2203
\mathchar"2202
\mathchar"213F
\mathchar"6201 % cdot
}
>>>


\subsection{3: Relational Operations}

The catcode is needed because 303A is \verb':'.

\<plain,fontmath math classes\><<<
\Configure{MathClass}{3}{}{}{}{
   \mathchar"3128
   \mathchar"3129
   \mathchar"312A
   \mathchar"312B
   \mathchar"315E
   \mathchar"315F
   \mathchar"3210
   \mathchar"3211
   \mathchar"3212
   \mathchar"3213
   \mathchar"3214
   \mathchar"3215
   \mathchar"3216
   \mathchar"3217
   \mathchar"3218
   \mathchar"3219
   \mathchar"321A
   \mathchar"321B
   \mathchar"321C
   \mathchar"321D
   \mathchar"321E
   \mathchar"321F
   \mathchar"3220
   \mathchar"3221
   \mathchar"3224
   \mathchar"3227
   \mathchar"3232
   \mathchar"3233
   \mathchar"3236
   \mathchar"3237
   \mathchar"323F
   :=><
   \mathchar"322F
   \mathchar"3276
   \mathchar"3277
   \mathchar"326B
   \mathchar"326A
   \mathchar"3261
   \mathchar"3260
   \mathchar"3225
   \mathchar"3226
   \mathchar"322D
   \mathchar"322E
   \mathchar"322C
   \mathchar"3228
   \mathchar"3229 
}
>>>

%    \mathchar"327C  |%clubsuit|%
%    \mathchar"327D  |%diamondsuit|%
%    \mathchar"327E  |%heartsuit|%
%    \mathchar"327F  |%spadesuit|%



%   \mathchar"340A % rightleftharpoons
%   \mathchar"340B % leftrightharpoons
%

\begin{verbatim}
\mathcode`\>="313E
\mathcode`\<="313C
\mathcode`\=="303D
\mathcode`\:="303A
\mathchardef\leq="3214 \let\le=\leq
\mathchardef\geq="3215 \let\ge=\geq
\mathchardef\succ="321F
\mathchardef\prec="321E
\mathchardef\approx="3219
\mathchardef\succeq="3217
\mathchardef\preceq="3216
\mathchardef\supset="321B
\mathchardef\set="321A
\mathchardef\supseteq="3213
\mathchardef\seteq="3212
\mathchardef\in="3232
\mathchardef\ni="3233 \let\owns=\ni
\mathchardef\gg="321D
\mathchardef\ll="321C
\mathchardef\not="3236
\mathchardef\leftrightarrow="3224
\mathchardef\leftarrow="3220 \let\gets=\leftarrow
\mathchardef\rightarrow="3221 \let\to=\rightarrow
\mathchardef\mapstochar="3237 \def\mapsto{\mapstochar\rightarrow}
\mathchardef\sim="3218
\mathchardef\simeq="3227
\mathchardef\perp="323F
\mathchardef\equiv="3211
\mathchardef\asymp="3210
\mathchardef\smile="315E
\mathchardef\frown="315F
\mathchardef\leftharpoonup="3128
\mathchardef\leftharpoondown="3129
\mathchardef\rightharpoonup="312A
\mathchardef\rightharpoondown="312B
\mathchardef\propto="322F
\mathchardef\sqsubseteq="3276
\mathchardef\sqsupseteq="3277
\mathchardef\parallel="326B
\mathchardef\mid="326A
\mathchardef\dashv="3261
\mathchardef\vdash="3260
\mathchardef\nearrow="3225
\mathchardef\searrow="3226
\mathchardef\nwarrow="322D
\mathchardef\swarrow="322E
\mathchardef\Leftrightarrow="322C
\mathchardef\Leftarrow="3228
\mathchardef\Rightarrow="3229
\end{verbatim}


\subsection{4/5: Delimiters}

\begin{verbatim}
\mathcode`\(="4028
\mathcode`\)="5029
\mathcode`\[="405B
\mathcode`\]="505D
\mathcode`\{="4266
\mathcode`\}="5267
\delcode`\(="028300
\delcode`\)="029301
\delcode`\[="05B302
\delcode`\]="05D303
\def\lmoustache{\delimiter"437A340 } % top from (, bottom from )
\def\rmoustache{\delimiter"537B341 } % top from ), bottom from (
\def\lgroup{\delimiter"462833A } % extensible ( with sharper tips
\def\rgroup{\delimiter"562933B } % extensible ) with sharper tips
\def\backslash{\delimiter"26E30F } % for double coset G\backslash H
\def\rangle{\delimiter"526930B }
\def\langle{\delimiter"426830A }
\def\rbrace{\delimiter"5267309 } \let\}=\rbrace
\def\lbrace{\delimiter"4266308 } \let\{=\lbrace
\def\rceil{\delimiter"5265307 }
\def\lceil{\delimiter"4264306 }
\def\rfloor{\delimiter"5263305 }
\def\lfloor{\delimiter"4262304 }
\def\arrowvert{\delimiter"26A33C } % arrow without arrowheads
\def\Arrowvert{\delimiter"26B33D } % double arrow without arrowheads
\def\bracevert{\delimiter"77C33E } % the vertical bar that extends braces
\def\Vert{\delimiter"26B30D } \let\|=\Vert         How should these be treated?
\def\vert{\delimiter"26A30C }                       "   "       "    "   "
\def\uparrow{\delimiter"3222378 }
\def\downarrow{\delimiter"3223379 }
\def\updownarrow{\delimiter"326C33F }
\def\Uparrow{\delimiter"322A37E }
\def\Downarrow{\delimiter"322B37F }
\def\Updownarrow{\delimiter"326D377 }
\end{verbatim}

The comamnds \verb'\Configure{MathClass}{4}...'
and \verb'\Configure{MathClass}{5}...'
are for unmatched delimiters, and the comamnd
\verb'\Configure{MathDelimiters}{(}{)}' is for matched ones.


\<plain,fontmath math classes\><<<
\Configure{MathClass}{4}{}{}{}{}
\Configure{MathDelimiters}{(}{)}
\Configure{MathDelimiters}{[}{]}
\Configure{MathDelimiters}{\mathchar"4262}{\mathchar"5263}
\Configure{MathDelimiters}{\mathchar"4264}{\mathchar"5265}
\Configure{MathDelimiters}{\mathchar"4266}{\mathchar"5267}
\Configure{MathDelimiters}{\mathchar"4268}{\mathchar"5269}
\Configure{MathDelimiters}{\mathchar"4300}{\mathchar"5301}
\Configure{MathDelimiters}{\mathchar"4302}{\mathchar"5303}
\Configure{MathDelimiters}{\mathchar"4304}{\mathchar"5305}
\Configure{MathDelimiters}{\mathchar"4306}{\mathchar"5307}
\Configure{MathDelimiters}{\mathchar"4308}{\mathchar"5309} 
\Configure{MathDelimiters}{\mathchar"430A}{\mathchar"530B}
>>>

\begin{verbatim}
\mathcode`\?="503F
\end{verbatim}

%%%%%%%%%%%%%%%%%%%%%%%%%%%%%%%%%%%%%%%%
\subsection{6: Punctuation Marks}
%%%%%%%%%%%%%%%%%%%%%%%%%%%%%%%%%%%%%%%%

\begin{verbatim}
\mathcode`\;="603B
\mathcode`\,="613B
\end{verbatim}

\<plain,fontmath math classes\><<<
\Configure{MathClass}{6}{}{}{}{
\mathchar"613A % ldot
\mathchar"603A % colon
!?; ,
}
>>>

%%%%%%%%%%%%%%%%%%%%%%%%%%%%%%%%%
\subsection{Questions}
%%%%%%%%%%%%%%%%%%%%%%%%%%%%%%%%%

\begin{verbatim}


\delcode`\<="26830A
\delcode`\>="26930B
\delcode`\|="26A30C
\delcode`\\="26E30F

% N.B. { and } should NOT get delcodes; otherwise parameter grouping fails!

\def\mathhexbox#1#2#3{\leavevmode
  \hbox{$\m@th \mathchar"#1#2#3$}}
\def\dag{\mathhexbox279}
\def\ddag{\mathhexbox27A}
\def\S{\mathhexbox278}
\def\P{\mathhexbox27B}
\end{verbatim}

%%%%%%%%%%%%%%%%%%%%%%%%%%%%%%%%
\subsection{Type 4: Math Open}
%%%%%%%%%%%%%%%%%%%%%%%%%%%%%%%%


latex.ltx, fontmath.ltx, plain.tex



%%%%%%%%%%%%%%%%%%%%%%%%%%%%%%%%%%%%%%%%%%%%%%%%%%%%%%%%%%%%%%%%
\chapter{Packages}
%%%%%%%%%%%%%%%%%%%%%%%%%%%%%%%%%%%%%%%%%%%%%%%%%%%%%%%%%%%%%%%%

%%%%%%%%%%%%%%%%%%
\section{Small Ones}
%%%%%%%%%%%%%%%%%%



\<configure mathml sistyle\><<<
\Configure{thousandsep}
  {\ifmathml
     \ifmtext\else \Tg<\a:mathml mspace width="0.25em"/>\fi
   \fi}
>>>


%%%%%%%%%%%%%
\subsection{subeqnarray}
%%%%%%%%%%%%%

\<configure mathml subeqnarray\><<<
\Configure{subeqnarray}
   {\HCode{<\a:mathml mtable\Hnewline \mml:class="subeqnarray\ifx 
       \@seqncr\@eqncr\else-star\fi"
       \a:mathml columnalign="right center left" >}}%
   {\HCode{</\a:mathml mtable>}}
   {\HCode{<\a:mathml mtr>}}  {\HCode{</\a:mathml mtr>}}
   {\Tg<\a:mathml mtd\Hnewline \mml:class="subeqnarray-\HCol">
      \ifnum \HCol=4
        \Tg<\a:mathml mtext 
             \mml:class="subeqnarray">|<start mtext space|>\PauseMathClass 
        \mtexttrue
        |<config math within mtext|>%
      \fi}
   {\ifnum \HCol=4 \mtextfalse 
      \EndPauseMathClass|<end mtext space|>\Tg</\a:mathml mtext>\fi
      \Tg</\a:mathml mtd>}
>>>


%%%%%%%%%%%%%
\subsubsection{Bm}
%%%%%%%%%%%%%


\<configure mathml bm\><<<
\Configure{bm}
  {\ifmmode \ifmtext \else
     \Tg<\a:mathml mstyle 
             mathvariant="bold">%
     \expandafter\expandafter\expandafter\:gobble
   \fi\fi}
  {\ifmmode \ifmtext \else \Tg</\a:mathml mstyle>\fi\fi}
>>>


The bm commands produces bold printing by overprinting the characters
with small  shifting.


%%%%%%%%%%%%%%%%%%
\section{Color}
%%%%%%%%%%%%%%%%%%



\<configure html-mml color\><<<
\Configure{textcolor}
   {\gHAdvance\textcolorN by 1
    \Configure{color}%
      {\Configure{SetHColor}
           {\Css{.textcolor\textcolorN{color:\HColor}}}}%
    \HCode{<\ifmathml mstyle\else span\fi
           \space id="textcolor\textcolorN">}}%
   {\HCode{</\ifmathml mstyle\else span\fi>}}
>>>


\<configure html-mml color\><<<
\Configure{textcolor}
   {\ifmathml
       \Configure{color}%
         {\Configure{SetHColor}
              {\HCode{<mstyle mathcolor="\HColor">}}}%       
    \else
       \gHAdvance\textcolorN by 1
       \Configure{color}%
         {\Configure{SetHColor}
              {\Css{\#textcolor\textcolorN{color:\HColor}}}}%
       \HCode{<span id="textcolor\textcolorN">}%
    \fi
   }%
   {\HCode{</\ifmathml mstyle\else span\fi>}}
>>>




%%%%%%%%%%%%%%%%%%%%%%%%%%%%%%%%%%%%%%%%%%%%%%%%%%%%%%%%%%%%%%%%
\chapter{Notes}
%%%%%%%%%%%%%%%%%%%%%%%%%%%%%%%%%%%%%%%%%%%%%%%%%%%%%%%%%%%%%%%%

%%%%%%%%%%%%%%%%%%%%%%%%%%%%%%%%%%%%%%%%%%%%%%%%%%%%%%%%%%%%%%%%
\chapter{To Do}
%%%%%%%%%%%%%%%%%%%%%%%%%%%%%%%%%%%%%%%%%%%%%%%%%%%%%%%%%%%%%%%%

\begin{itemize}
\item Parametric commands as Superscripts or Subscripts:
    \verb+$E^\frac{1}{2}$+  (instead of \verb+$E^{\frac{1}{2}}$+).

\item
 Numbering is not right justified:

\begin{verbatim}
\documentclass{amsart}
 \begin{document}
 
 \begin{eqnarray} 
 A &=& B + C 
 \end{eqnarray} 
 
 \begin{align} 
 A &= B + C \qquad & N &= G + F 
 \end{align} 
 
 \begin{alignat}{2} 
 A &= B + C \qquad \qquad & N & = G + F 
 \end{alignat} 

\begin{gather}
A = B + C \tag{my equation} \\
B = \text{variable unknown} \notag \\
Y = \max X + \log 10
\end{gather}

\end{document}
\end{verbatim}


\item
\begin{verbatim}
\begin{align}
e &= a + b \\
\intertext{this should be located on the left browser hand margin}
a &= u
\end{align}

\end{verbatim}

\end{itemize}



\endinput
